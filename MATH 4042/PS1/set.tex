\documentclass[12pt,letterpaper]{article}
\usepackage{fullpage}
\usepackage[top=2cm, bottom=4.5cm, left=2.5cm, right=2.5cm]{geometry}
\usepackage{amsmath,amsthm,amsfonts,amssymb,amscd}
\usepackage{lastpage}
\usepackage{enumerate}
\usepackage{fancyhdr}
\usepackage{mathrsfs}
\usepackage{xcolor}
\usepackage{graphicx}
\usepackage{listings}
\usepackage{hyperref}
\usepackage{tikz}
\usepackage{relsize}
\usepackage{fancyvrb}
\usepackage{import}
\usetikzlibrary{shapes.geometric,fit}

\hypersetup{%
  colorlinks=true,
  linkcolor=blue,
  linkbordercolor={0 0 1}
}

\setlength{\parindent}{0.0in}
\setlength{\parskip}{0.05in}

\theoremstyle{definition}
\newtheorem*{statement}{Statement}
\newtheorem*{claim}{Claim}
\newtheorem*{theorem}{Theorem}
\newtheorem*{lemma}{Lemma}

\newcommand{\contra}{\Rightarrow\!\Leftarrow}
\newcommand{\R}{\mathbb{R}}
\newcommand{\F}{\mathbb{F}}
\newcommand{\Z}{\mathbb{Z}}
\newcommand{\Zeq}{\mathbb{Z}_{\geq 0}}
\newcommand{\Zg}{\mathbb{Z}_{>0}}
\newcommand{\Req}{\mathbb{R}_{\geq 0}}
\newcommand{\Rg}{\mathbb{R}_{>0}}
\newcommand{\N}{\mathbb{N}}
\newcommand{\Q}{\mathbb{Q}}
\newcommand{\C}{\mathbb{C}}

\newcommand{\incfig}[1] {%
    % \def\svgwidth{\columnwidth}
    \import{./figures/}{#1.pdf_tex}
}

\title{MATH 4042 HW 1}
\author{David Chen, dc3451}
\date{\today}

\begin{document}

\maketitle

\section*{Question 1}

\subsection*{a}

\begin{enumerate}
  \item We want first that $Y^{+}$ is an abelian group under addition. First, addition is associative:
        \begin{align*}
          ((n_{1}, a_{1}) + (n_{2}, a_{2})) + (n_{3}, a_{3}) &= (n_{1} + n_{2}, a_{1} + a_{2}) + (n_{3}, a_{3}) \\
                                                             &= ((n_{1} + n_{2}) + n_{3}, (a_{1} + a_{2}) + a_{3}) \\
                                                             &= (n_{1} + (n_{2} + n_{3}), a_{1} + (a_{2} + a_{3})) \\
                                                             &= (n_{1}, a_{1}) + ((n_{2}, a_{2}) + (n_{3}, a_{3}))
        \end{align*}
        and commutative:
        \[
        (n_{1}, a_{1}) + (n_{2}, a_{2}) = (n_{1} + n_{2}, a_{1} + a_{2}) = (n_{2} + n_{1}, a_{2} + a_{1}) = (n_{2}, a_{2}) + (n_{1}, a_{1})
        \]
        with identity $(0, 0)$ (where the latter 0 is the additive identity in $Y$)
        \[
        (n, a) + (0, 0) = (n + 0, a + 0) = (n, a)
        \]
        and inverses:
        \[
        (n, a) + (-n, -a) = (n - n, a - a) = (0, 0)
        \]
  \item Second, we have that multiplication is associative (note that due to the distributive property, $a_{1}(na_{2}) = a_{1}(a_{2} + a_{2} \dots + a_{2}) = a_{1}a_{2} + \dots a_{1}a_{2} = n(a_{1}a_{2})$ for $n \in \Z$):
        \begin{align*}
          (n_{1}, a_{1}) \cdot ((n_{2}, a_{2}) \cdot (n_{3}, a_{3})) &= (n_{1}, a_{1}) \cdot (n_{2}n_{3}, n_{2}a_{3} + n_{3}a_{2} + a_{2}a_{3}) \\
                                                                     &= (n_{1}n_{2}n_{3}, n_{1}(n_{2}a_{3} + n_{3}a_{2} + a_{2}a_{3}) + n_{2}n_{3}a_{1} + a_{1}(n_{2}a_{3} + n_{3}a_{2} + a_{2}a_{3})) \\
          &= (n_{1}n_{2}n_{3}, n_{2}n_{3}a_{1} + n_{1}n_{3}a_{2} + n_{1}n_{2}a_{3} + n_{1}a_{2}a_{3} + n_{2}a_{1}a_{3} + n_{3}a_{1}a_{2}) \\
          ((n_{1}, a_{1}) \cdot (n_{2}, a_{2})) \cdot (n_{3}, a_{3}) &= (n_{1}n_{2}, n_{1}a_{2} + n_{2}a_{1} + a_{1}a_{2}) \cdot (n_{3}, a_{3}) \\
                                                                     &= (n_{1}n_{2}n_{3}, n_{1}n_{2}a_{3} + n_{3}(n_{1}a_{2} + n_{2}a_{1} + a_{1}a_{2}) + a_{1}a_{2}(n_{1}a_{2} + n_{2}a_{1} + a_{1}a_{2})) \\
          &= (n_{1}n_{2}n_{3}, n_{2}n_{3}a_{1} + n_{1}n_{3}a_{2} + n_{1}n_{2}a_{3} + n_{1}a_{2}a_{3} + n_{2}a_{1}a_{3} + n_{3}a_{1}a_{2})
        \end{align*}
  \item Lastly, we check distributivity:
        \begin{align*}
          (n_{1}, a_{1}) \cdot ((n_{2}, a_{2}) + (n_{3}, a_{3})) &= (n_{1}, a_{1}) \cdot (n_{2} + n_{3}, a_{2} + a_{3}) \\
                                                                 &= (n_{1}(n_{2} + n_{3}), n_{1}(a_{2} + a_{3}) + (n_{2} + n_{3})a_{1} + a_{1}((a_{2} + a_{3}))) \\
                                                                 &= (n_{1}n_{2} + n_{1}n_{3}, n_{1}a_{2} + n_{2}a_{1} + a_{1}a_{2} + n_{1}a_{3} + n_{3}a_{1} + a_{1}a_{3}) \\
                                                                 &= ((n_{1}, a_{1}) \cdot (n_{2}, a_{2})) + ((n_{1}, a_{1}) \cdot (n_{3}, a_{3})) \\
          ((n_{1}, a_{1}) + (n_{2}, a_{2})) \cdot (n_{3}, a_{3})) &= (n_{1} + n_{2}, a_{1} + a_{2}) \cdot (n_{3}, a_{3}) \\
                                                                 &= ((n_{1} + n_{2})n_{3}, (n_{1} + n_{2})a_{3} + n_{3}(a_{1} + a_{2}) + (a_{1} + a_{2})a_{3}) \\
                                                                 &= (n_{1}n_{3} + n_{2}n_{3}, n_{1}a_{3} + n_{3}a_{1} + a_{1}a_{3} + n_{2}a_{3} + n_{3}a_{2} + a_{2}a_{3}) \\
                                                                 &= ((n_{1}, a_{1}) \cdot (n_{3}, a_{3})) + ((n_{2}, a_{2}) \cdot (n_{3}, a_{3}))
        \end{align*}
  \item To check the identity, we have
        \begin{align*}
          (n, a) \cdot (1, 0) &= (n1, n0 + 1a + a0) = (n, a) \\
          (1, 0) \cdot (n, a) &= (1n, 1a + 0n + 0a) = (n, a) \\
        \end{align*}
\end{enumerate}


\subsection*{b}

Note
\[
  (n, a) \cdot (0, 1) = (0, n \cdot 1 + a0 + a1) = (0, n + a)
\]
and
\[
  (0, 1) \cdot (n, a) = (0, 0a + n1 + 1a) = (0, n + a)
\]
where $n = 1 + 1 + \cdots + 1$ in $Y$.

Secondly, $(0,y_{1}) + (0, y_{2}) = (0, y_{1} + y_{2}) \in I_{Y}$, $(0, y) \in I_{Y} \implies (0, -y) \in I_{Y}$, and $(0, 0) \in I_{Y}$, so $I_{Y}$ is a subgroup under addition.

\subsection*{c}

We already have that since an ideal is a subgroup of $R$ (itself an abelian group with addition) under addition, it already is an abelian group under addition. Associativity follows from associativity of the multiplication of $R$, since for $x, y, z \in I$, associativity in $R$ gives $x(yz) = (xy)z$ in $R$, but since $I$ is an ideal, $xyz$ also lies in $I$. Similarly, distributivity also comes since in $R$, $x(y + z) = xy + xz$, and since $I$ an ideal, $xy, xz \in I$ and thus $xy + xz$ also lie in $I$; $(y+z)x = yx + zx$ also lies in $I$.

\subsection*{d}

We have that for $n \in \Z, p \in (x)$, $\phi: (n, p) \mapsto p + n$, where on the right $n = 1 + 1 + \cdots + 1$ in $\Z[x]$, is a isomorphism taking $(x)^{+} \rightarrow \Z[x]$. First, to see it is bijective, if $\phi((n_{1}, p_{1})) = \phi((n_{2}, p_{2}))$, then we have that $p_{1} + n_{1} = p_{2} + n_{2}$. However, since $p_{1} \in (x)$, we have that $p_{1} = \left(\sum_{i=0}^{n}a_{i}x^{i}\right)x$, and similarly for $p_{2}$. In particular, this means that $p_{1}, p_{2}$ have no (nonzero) terms of degree $0$, and since $n_{1}, n_{2}$ are both terms of degree 0, we have that $n_{1} = n_{2}$, which then immediately yields $p_{1} = p_{2}$, so $\phi$ is injective. For surjectivity, any polynomial $p \in \Z[x]$ where $p = \sum_{i=0}^{n}a_{i}x^{i}$ has preimage $(a_{0}, \sum_{i=1}^{n}a_{i}x^{i}) = (a_{0}, (\sum_{i=1}^{n}a_{i}x^{i-1})x)$. Checking that it is a homomorphism,
\[
  \phi((n_{1}, p_{1}) + (n_{2}, p_{2})) = \phi((n_{1} + n_{2}, p_{1}+  p_{2})) = p_{1} + p_{2} + n_{1} + n_{2} = \phi((n_{1}, p_{1})) + \phi((n_{2}, p_{2}))
\]
since addition commutes.
\begin{align*}
  \phi((n_{1}, p_{1}) \cdot (n_{2}, p_{2})) &= \phi((n_{1}n_{2}, n_{1}p_{2} + n_{2}p_{1} + p_{1}p_{2})) \\
                                            &= p_{1}p_{2} + n_{1}p_{2} + n_{2}p_{1} + n_{1}n_{2} \\
                                            &= (p_{1} + n_{1})(p_{2} + n_{2}) \\
                                            &= \phi((n_{1}, p_{1}))\phi((n_{2}, p_{2}))
\end{align*}
and lastly,
\[
  \phi((1, 0)) = 0 + 1 = 1
\]
so $\phi$ is a homomorphism.

\section*{Question 2}

\subsection*{a}

\begin{enumerate}
  \item Checking that this forms an abelian group under addition, we have that
        \[
        (a_{1}, a_{2}) + (b_{1}, b_{2}) = (a_{1} + a_{2}, b_{1} + b_{2}) = (a_{2} + a_{1}, b_{2} + b_{1}) = (a_{2}, b_{2}) + (a_{1}, b_{1})
        \]
        for commutativity and
        \begin{align*}
          (a_{1}, a_{2}) + ((b_{1}, b_{2}) + (c_{1}, c_{2})) &= (a_{1}, a_{2}) + (b_{1} + c_{1}, b_{2} + c_{2}) \\
                                                             &= (a_{1} + b_{1} + c_{1}, a_{2} + b_{2} + c_{2}) \\
                                                             &= (a_{1} + b_{1}, a_{2} + b_{2}) + (c_{1}, c_{2}) \\
                                                             &= ((a_{1}, a_{2}) + (b_{1}, b_{2})) + (c_{1}, c_{2})
        \end{align*}
        for associativity, with identity $(0, 0)$ (the zero element in the ring)
        \[
        (a_{1}, a_{2}) + (0, 0) = (a_{1}, a_{2})
        \]
        and inverses:
        \[
        (a_{1}, a_{2}) + (-a_{1}, -a_{2}) = (0, 0)
        \]
  \item Checking that multiplication is associative,
        \begin{align*}
          (a_{1},a_{2})((b_{1},b_{2})(c_{1},c_{2})) &= (a_{1},a_{2})(b_{1}c_{1}, b_{2},c_{2}) \\
                                                    &= (a_{1}b_{1}c_{1}, a_{2}b_{2}c_{2}) \\
                                                    &= (a_{1}b_{1}, a_{2}b_{2})(c_{1},c_{2}) \\
                                                    &= ((a_{1},a_{2})(b_{1},b_{2}))(c_{1},c_{2})
        \end{align*}
  \item Checking distributivity,
        \begin{align*}
          (a_{1},a_{2})((b_{1}, b_{2}) + (c_{1}, c_{2})) &= (a_{1},a_{2})(b_{1} + c_{1}, b_{2} + c_{2}) \\
                                                         &= (a_{1}(b_{1} + c_{1}), a_{2}(b_{2} + c_{2})) \\
                                                         &= (a_{1}b_{1} + a_{1}c_{1}, a_{2}b_{2} + a_{2}c_{2}) \\
                                                         &= (a_{1}b_{1}, a_{2}b_{2}) + (a_{1}c_{1}, a_{2}c_{2}) \\
                                                         &= (a_{1},a_{2})(b_{1},b_{2}) + (a_{1},a_{2})(c_{1},c_{2}) \\
          ((b_{1}, b_{2}) + (c_{1}, c_{2}))(a_{1},a_{2}) &= (b_{1} + c_{1}, b_{2} + c_{2})(a_{1},a_{2}) \\
                                                         &= ((b_{1} + c_{1})a_{1}, (b_{2} + c_{2})a_{2}) \\
                                                         &= (b_{1}a_{1} + c_{1}a_{1}, b_{2}a_{2} + c_{2}a_{2}) \\
                                                         &= (b_{1}a_{1}, b_{2}a_{2}) + (c_{1}a_{1}, c_{2}a_{2}) \\
                                                         &= (b_{1},b_{2})(a_{1},a_{2}) + (c_{1},c_{2})(a_{1},a_{2})
        \end{align*}
  \item Checking the identity to be $(1,1)$, we have
        \[
        (a_{1}, a_{2}) \cdot (1,1) = (a_{1}, a_{2})
        \]
        and
        \[
        (1,1) \cdot (a_{1}, a_{2}) = (a_{1}, a_{2})
        \]
\end{enumerate}

\subsection*{b}

To check that $I_{1} \times I_{2}$ is a subgroup, we have that $0 \in I_{1}, I_{2} \implies (0, 0) \in I_{1} \times I_{2}$, and also that $(a_{1}, a_{2}), (b_{1}, b_{2}) \in I_{1} \times I_{2} \implies (a_{1}, a_{2}) + (b_{1}, b_{2}) = (a_{1} + b_{1}, a_{2} + b_{2}) \in I_{1} \times I_{2}$, and $(a, b) \in I_{1} \times I_{2} \implies (-a, -b) \in I_{1} \times I_{2}$, so we have the identity, closure and inverses.

For multiplication, we get for $(r_{1}, r_{2}) \in R$, and $(a_{1}, a_{2}) \in I_{1} \times I_{2}$,
\[
  (r_{1}, r_{2})(a_{1}, a_{2}) = (r_{1}a_{1}, r_{2}a_{2})
\]
and since $a_{1}, a_{2}$ are members of ideals themselves, the product is in $I_{1} \times I_{2}$. Similarly,
\[
  (a_{1}, a_{2})(r_{1}, r_{2}) = (a_{1}r_{1}, a_{2}r_{2}) \in I_{1} \times I_{2}
\]

\subsection*{c}

Consider any ideal $I \in R_{1} \times R_{2}$ and the corresponding sets $I_{1} = \{a \mid \exists (a, \cdot) \in I\}$ and $I_{2} = \{b \mid \exists (\cdot, b) \in I\}$. Then, we want that $I = I_{1} \times I_{2}$.

Firstly, we have that $(a,b) \in I$ clearly implies $(a,b) \in I_{1} \times I_{2}$ by definition. Then, if we have some $(a, b) \in I_{1} \times I_{2}$, then there is some $(a, b') \in I$ and some $(a', b) \in I$ as well. Then, $(1, b)(a, b') = (a, bb') \in I$, and $(a', b)(1, b') = (a', bb') \in I$, and thus $(a, bb') - (a', bb') + (a', b) = (a,b) \in I$.

The last thing to show is that both $I_{1}, I_{2}$ are ideals. In particular, there is nothing significant about the first position, so showing it for $I_{1}$ is the same as for $I_{2}$. Then, picking any $r_{1} \in R_{1}$ and $a \in I_{1}$, we have that there is some $(a,b) \in I$ and $(r_{1}, 1)(a,b) = (r_{1}a, b) \in I \implies r_{1}a \in I$ and similarly $(a,b)(r_{1}, 1) = (ar_{1}, b) \in I \implies ar_{1} \in I$. Furthermore, $I$ contains the identity, since $(0, 0) \in I \implies 0 \in I_{1}$, $a \in I_{1} \implies (a,b) \in I \implies (-a, -b) \in I \implies -a \in I_{1}$ and $a_{1}, a_{2} \in I \implies (a_{1}, b_{1}), (a_{2}, b_{2}) \in I \implies (a_{1} + a_{2}, b_{1} + b_{2}) \in I \implies a_{1} + a_{2} \in I$ and we have closure, the identity, and inverses.

\subsection*{d}

I'm not sure why this problem is given as an isomorphism; these sets are (should be?) identical, which does give us a trivial isomorphism in the identity. If we have some $(r_{1}, r_{2}) \in (R_{1} \times R_{2})^{\times}$, then $\exists (r_{1}', r_{2}')$ such that $(r_{1},r_{2})(r_{1}', r_{2}') = (1,1) \implies r_{1}r_{1}' = 1$ and $r_{2}r_{2}' = 1$, so $r_{1} \in R_{1}^{\times}$ and $r_{2} \in R_{2}^{\times}$, and so $(R_{1} \times R_{2})^{\times} \subseteq R_{1}^{\times} \times R_{2}^{\times}$. Then, if we have some $r_{1} \in R_{1}^{\times}$ and $r_{2} \in R_{2}^{\times}$, there are $r_{1}^{-1}, r_{2}^{-1}$ units such that $r_{1}r_{1}^{-1} = 1$ and $r_{2}r_{2}^{-1} = 1$, so we have that $(r_{1}, r_{2})(r_{1}^{-1}, r_{2}^{-1}) = (1, 1) \implies (r_{1}, r_{2}) \in (R_{1} \times R_{2})^{\times}$, so $R_{1}^{\times} \times R_{2}^{\times} \subseteq (R_{1} \times R_{2})^{\times}$, so the two groups are equal.

\section*{Question 3}

Take $\phi: n \mapsto (n \text{ mod } 2, n \text{ mod } 3)$ (so for example $\phi(4) = (0, 1)$). That this is a bijection is immediate from the Chinese Remainder Theorem (since these are isomorphic as groups) but we can also check this directly:
\begin{align*}
  \phi(0) &= (0, 0) \\
  \phi(1) &= (1, 1) \\
  \phi(2) &= (0, 2) \\
  \phi(3) &= (1, 0) \\
  \phi(4) &= (0, 1) \\
  \phi(5) &= (1, 2) \\
\end{align*}
so we have that $\phi$ is bijective (and sends $1 \mapsto (1,1)$). That it is a ring homomorphism follows immediately from basic properties of modular arithmetic:
\[
  \phi(a + b) = (a + b \text{ mod } 2, a + b \text{ mod } 3) = (a \text{ mod } 2, a \text{ mod } 3) + (b \text{ mod } 2, b \text{ mod } 3) = \phi(a) + \phi(b)
\]
\[
  \phi(ab) = (ab \text{ mod } 2, ab \text{ mod } 3) = (a \text{ mod } 2, a \text{ mod } 3) \cdot (b \text{ mod } 2, b \text{ mod } 3) = \phi(a)\phi(b)
\]
\section*{Question 4}

\subsection*{a}

\[
  (1 - e)^{2} = 1^{2} - e1 - 1e + e^{2} = 1 - 2e + e = 1 - e
\]

\subsection*{b}

Consider
\[
  e(1 - e) = e - e^{2} = e - e = 0
\]
but if $e \neq 0, 1$, we have that $e \neq 0$ and $1 - e \neq 0$, so this is not an integral domain.

\subsection*{c}

Consider $(0, 1) \in R_{1} \times R_{2}$ which is neither the zero element $(0, 0)$ nor the multiplicative identity $(1, 1)$. Then, we have that $(0,1)^{2} = (0, 1)$, so this (as well as $(1, 0)$) is the element we are looking for.

\subsection*{d}

Suppose that $(a + bi)(c + di) = 0$, such that $ac - bd + (ad + bc)i = 0$ and in particular $ad + bc = 0$; note that this also yields
\[
  (a - bi)(c - di) = ac - bd - (ad + bc)i = 0
\]
so
\[
  (a + bi)(c + di)(a - bi)(c - di) = (a^{2} + b^{2})(c^{2} + d^{2}) = 0
\]
for $a,b,c,d \in \Z$. Then, one of $a^{2} + b^{2}$ and $c^{2} + d^{2}$ must vanish, and since $a^{2},b^{2},c^{2},d^{2} > 0$ for nonzero $a,b,c,d$, we have one of the pairs $a,b$ and $c,d$ must vanish, so one of $a + bi$ and $c + di$ must be 0. Thus, $\Z[i]$ must be an integral domain.

\section*{Question 5}

\subsection*{a}

Again, ideals are (from class) subgroups under addition in the ring that they live it, so it already satisfies that it is an abelian group under addition. Then, we still have for $x,y,z \in (e)$, the property that ideals absorb multiplication (and associativity in $R$) gives $x(yz) = (xy)z$ and that $xyz \in (e)$ as well; similarly, distributivity in $R$ again gives that $x(y + z) = xy + xz$ and $(y + z)x = yx + zx$, and that both are in $(e)$. Note that $(e)$ does not contain 1; however, we do have that for any element $re \in (e)$, $e(re) = (re)e = re^{2} = re$, so $e$ serves as the identity in $(e)$.

\subsection*{b}

We can show that $R \cong (e) \times (1 - e)$. In particular, the first part, since it is indenpendent of the choice of idempotent, shows that both $(e)$ and $(1 - e)$ are rings (with identities $e$ and $1 - e$ respectively). Then, consider the mapping $\phi: a \mapsto (ae, a(1 - e))$, such that
\[
  \phi(a + b) = ((a + b)e, (a + b)(1 - e)) = (ae + be, a(1-e) + b(1-e)) = (ae, a(1-e)) + (be, b(1-e)) = \phi(a) + \phi(b)
\]
\[
  \phi(ab) = ((ab)e, (ab)(1 - e)) = (abe^{2}, ab(1-e)^{2}) = (ae, a(1-e)) \cdot (be, b(1-e)) = \phi(a)\phi(b)
\]
\[
  \phi(1) = (e, (1 - e))
\]
which is the identity in $(e) \times (1-e)$ since we showed earlier the identity in those rings were $e$ and $1-e$ respectively.

In particular, this is surjective, since if we have some $re \in (e)$, $s(1 - e) \in (1 - e)$, then we get that
\[
  \phi(re + s(1 - e)) = ((re + s(1-e))e, (re + s(1-e))(1-e)) = (re^{2} + s(1-e)e, re(1-e) + s(1-e)^{2}) = (re, s(1-e))
\]
so we have a preimage for every element in $(e) \times (1 - e)$. Then, for injectivity, we can check the kernel of $\phi$ to be trivial: if $(ae, a(1-e)) = 0$, then $-ae = a(1 - e) = 0 \implies a = 0$.

\section*{Question 6}

For this problem, if we have some matrix $A$, let the entry in the $i^{th}$ row and $j^{th}$ column as $A_{ij}$.

\subsection*{a}

First, we need that $M_{n}(I)$ is a subgroup under addition; this is clear since matrix addition is just component-wise, so we have that $0 \in I \implies 0 \in M_{n}(I)$, $A \in M_{n}(I) \implies A_{ij} \in I \implies -A_{ij} \in I \implies -A \in M_{n}(I)$, and $A, B \in M_{n}(I) \implies A_{ij},B_{ij} \in I \implies A_{ij} + B_{ij} \in I \implies A + B \in M_{n}(I)$ so we get the identity, inverses, and closure.

Further, we have for $A \in M_{n}(I)$, $B \in M_{n}(R)$, each entry satisfies
\begin{align*}
  (AB)_{ij} = \sum_{k=1}^{n}A_{i,k}B_{k,j}
\end{align*}
but since $A_{i,k} \in I$, $A_{i,k}B_{k,j} \in I$ as well, so we get that $(AB)_{ij} \in I$ and so $AB \in M_{n}(I)$. Similarly,
\begin{align*}
  (BA)_{ij} = \sum_{k=1}^{n}B_{i,k}A_{k,j}
\end{align*}
so $A_{k,j} \in I \implies B_{i,k}A_{k,j} \in I \implies (BA)_{ij} \in M_{n}(I) \implies BA \in M_{n}(I)$.

\subsection*{b}

For this part, denote the matrix
\[
  \begin{bmatrix}
    0 & \cdots & 0 & \cdots & 0 \\
    \vdots & \ddots & \vdots & \ddots & \vdots \\
    0 & \cdots & 1 & \cdots & 0 \\
    \vdots & \ddots & \vdots & \ddots & \vdots \\
    0 & \cdots & 0 & \cdots & 0 \\
  \end{bmatrix}
\]
where the 1 is is in the $i^{th}$ row and $j^{th}$ column as $E_{ij}$.

We can show that $S = \{A_{11} \mid A \in I_{n}\}$ (for some ideal $I_{n}$ of $M_{n}(R)$) is an ideal of $R$, and that $I_{n} = M_{n}(S)$. First, to see that it is an ideal, we can consider for any element $r \in R$ and element $a \in S$ the matrix $rE_{11} \in M_{n}(R)$ and the matrix $sE_{11} \in I_{n}$, such that since $I_{n}$ is an ideal, $(rE_{11})(sE_{11}) = rsE_{11}^{2} = rsE_{11}$ and $(sE_{11})(rE_{11}) = srE_{11}$ are both in $I_{n}$, and thus $rs, sr \in S$. That $S$ is a subgroup under addition is easy, since we get that $0 \in I_{n} \implies 0 \in S$, $a, b \in S \implies aE_{11} + bE_{11} = (a + b)E_{11} \implies a + b \in S$, and $a \in S \implies -aE_{11} \in I_{n} \implies -a \in S$, so we get the identity, closure, and inverses.

To see that $I_{n} = M_{n}(S)$, consider $A \in I_{n}$ and any entry $A_{ij}$. Then, we can consider the product $P_{i}AP_{j}$, where $P_{k}$ is the permutation matrix corresponding to the permutation $(1,k)$ (that is, the identity matrix with the first row and the $k^{th}$ row swapped). Then, we have that this is contained in the ideal, and $(P_{i}AP_{j})_{11} = A_{ij}$, so $A_{ij} \in S$, so $S$ contains all of the entries of $I_{n}$, but by definition $S$ is a subset of the entries of $I_{n}$, so $I_{n}$ has coefficients exactly in $S$, giving $I_{n} \subseteq M_{n}(S)$. Then, consider that for any $A \in M_{n}(S)$, we can write $A = \sum_{i=1}^{n}\sum_{j=1}^{n}A_{ij}E_{ij}$, but each $A_{ij}E_{ij} \in I_{n}$, so we get that $A \in I_{n}$, so $M_{n}(S) \subseteq I_{n}$ as well, so finally $I_{n} = M_{n}(S)$.

\end{document}
% LocalWords:  NetID fancyplain LocalWords colorlinks linkcolor linkbordercolor
