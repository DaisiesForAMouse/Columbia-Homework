\documentclass[12pt,letterpaper]{article}
\usepackage{fullpage}
\usepackage[top=2cm, bottom=4.5cm, left=2.5cm, right=2.5cm]{geometry}
\usepackage{amsmath,amsthm,amsfonts,amssymb,amscd}
\usepackage{lastpage}
\usepackage{enumerate}
\usepackage{fancyhdr}
\usepackage{mathrsfs}
\usepackage{xcolor}
\usepackage{graphicx}
\usepackage{listings}
\usepackage{hyperref}
\usepackage{tikz}
\usepackage{relsize}
\usepackage{fancyvrb}
\usepackage{import}
\usetikzlibrary{shapes.geometric,fit}

\hypersetup{%
  colorlinks=true,
  linkcolor=blue,
  linkbordercolor={0 0 1}
}

\setlength{\parindent}{0.0in}
\setlength{\parskip}{0.05in}

\theoremstyle{definition}
\newtheorem*{statement}{Statement}
\newtheorem*{claim}{Claim}
\newtheorem*{theorem}{Theorem}
\newtheorem*{lemma}{Lemma}

\newcommand{\contra}{\Rightarrow\!\Leftarrow}
\newcommand{\R}{\mathbb{R}}
\newcommand{\F}{\mathbb{F}}
\newcommand{\Z}{\mathbb{Z}}
\newcommand{\Zeq}{\mathbb{Z}_{\geq 0}}
\newcommand{\Zg}{\mathbb{Z}_{>0}}
\newcommand{\Req}{\mathbb{R}_{\geq 0}}
\newcommand{\Rg}{\mathbb{R}_{>0}}
\newcommand{\N}{\mathbb{N}}
\newcommand{\Q}{\mathbb{Q}}
\newcommand{\C}{\mathbb{C}}

\newcommand{\incfig}[1] {%
    % \def\svgwidth{\columnwidth}
    \import{./figures/}{#1.pdf_tex}
}

\title{MATH 4061 HW 1}
\author{David Chen, dc3451}

\begin{document}

\maketitle
\today

\section*{Ch 1, Q2}

Euclid's lemma is that if $p \mid ab$, then $p \mid a$ or $p \mid b$. In
particular, an easy corollary (the contrapositive) is that if $p \nmid a$ and $p \nmid b$, then
we have that $p \nmid ab$.

Let $(p / q)^2 = 12$, where $p,q$ are relatively prime, such that $\nexists n \in \N$ such that $n > 1$
and $n \mid p$, $n \mid q$. 

Then, we have that $p^2 = 12q^2 = 3(4q^2)$. Clearly $3 \mid p^2$, and so $3 \mid
p$ (if $3 \nmid p$, then $3 \nmid p^2$ as well as a consequence of the earlier
stated corollary to Euclid's lemma). Then, if $p = 3k$, we have that
$9k^2 = 3(4q^2) \implies 3k^2 = 4q^2$. Since $3 \mid 4q^2$, and $3 \nmid 4$, by
Euclid's lemma we have that $3 \mid q^2$, and from the same logic as earlier, we
have that $3 \mid q$. 

Since we have that 3 divides both $p$ and $q$, we have $\contra$ as we earlier
assumed that $p, q$ were relatively prime.

\subsection*{Lemma}

Now, we first show that for any set $X \subset \R$ bounded above there is an element
$\epsilon-$close to $\sup X$ that is contained in $X$ (that is, $\forall \epsilon
> 0, \exists x \in X
\mid x > \sup X - \epsilon$). To see this, note that if
we no such element, then we would have $\sup X - \epsilon$ as a lower upper
bound of $X$.

In particular, this shows that if $\sup(X) > y$, then $\exists x \in X \mid y <
x \leq \sup(X)$.

\section*{Ch 1, Q6}

\subsection*{a}

% Note that it is good enough to show this for nonnegative rationals, as we have that 
% $b^{-m / n}$ is the multiplicative inverse of $b^{m / n}$. However, these
% inverses are unique, and as $b^{-m / n}b^{n / m}$

First, for $a,b \geq 0$, we have that since the $n^{th}$ root of $a,b$ is
distinct. This shows the following: if $a^n = b^n$, then the $n^{th}$ root of 
$b^n = a^n$ is $a$, and the $n^{th}$ root of $a^n = b^n$ is also $b$. Since these roots are
unique, we must have that $a = b$.

For $n, m \in \N$, we see that $x^{nm} = x \cdot x \cdot \dots \cdot x$ $nm$
times. However, we also have that $(x^n)^m = (x \cdot x \cdot \dots \cdot x)
\cdot \dots \cdot (x \cdot \dots \cdot x)$, where each group in side the
parentheses contains $n$ $x$'s and there are $m$ groups, for a total of $nm$
$x$'s. However, since multiplication is associative, we have that $(x^n)^m$ is
then the same as $x^{nm}$.

If $n,m \in \Z$, $nm < 0$, we have the same thing as above, but with $x$
replaced by $1 / x$, and we still have $(x^n)^m = x^{nm}$. This allows us to justify the following
manipulations.

Consider now $((b^m)^{1 / n})^{nq} = b^{mq}$ and $((b^p)^{1 / q})^{nq} =
b^{pn}$. However, we have that $m / n = p / q \implies mq = pn$, so we have that
$b^{mq} = b^{pn}$. From the earlier statement about distinct $n^{th}$ roots,
this suggests that $(b^m)^{1 / n} = (b^p)^{1 / q}$.

\subsection*{b}

Let $r = p_r / q_r$ and $s = p_s / q_s$. Then, we have that
\[
  b^{r + s} = b^{\frac{p_rq_s + p_sq_r}{q_sq_r}} = (b^{p_rq_s + p_sq_r})^{1 /
  q_rq_s} = (b^{p_rq_s}b^{p_sq_r})^{1 / q_rq_s}
\]

The last step follows from multiplication commuting, and the next step
distributing the exponent is
justified by Rudin in the book:
\[
  (b^{p_rq_s}b^{p_sq_r})^{1 / q_rq_s} = (b^{p_rq_s})^{1 / q_rq_s}(b^{p_sq_r})^{1
  / q_rq_s} = (b^{\frac{p_rq_s}{q_rq_s}})(b^{\frac{p_sq_r}{q_rq_s}}) =
  b^{\frac{p_r}{q_r}}b^{\frac{p_s}{q_s}} = b^rb^s
\]

\subsection*{c}

First, we will show that for $b > 1, r > 0$ with $b \in \R, r \in \Q$ that $b^r > 1$.
Put that $r = p / q$. Since $b > 1$, we have that $b^p > 1$ (Rudin claims $0 < y_1 <
y_2 \implies y_1^n < y_2^n$ earlier).

Then, since we have that $b^r$ is $q^{th}$ root of $b^p$, and we know that for
$0 < y \leq 1$ that $y^n \leq 1$, we have that $b^r > 1$ (otherwise, we would
have that $(b^r)^q < 1$, but we have that $(b^r)^q = b^p > 1$).

Now, consider $r,s \in \Q$, and $r > s$. Then, we have that $b^r - b^s =
b^s(b^{r-s}-1)$. Since we have that $b^s > 0$, and that $b^{r-s} > 1$, $b^r -
b^s > 0$, and so $b^r > b^s$.

We will show that $b^r$ is an upper bound of $B(r)$ first. For any $b^t \in
B(r)$, we have that $t \leq r \implies b^t \leq b^r$. Then, $b^r$ is an upper
bound of $B(r)$. To see that it is the least upper bound, note that $b^r \in
B(r)$, such that any real number $< b^r$ is not an upper bound for $B(r)$.

\subsection*{d}

We have that $b^{x+y} = \sup B(x + y)$. We will show that $b^xb^y = \sup B(x)
\sup B(y)$ is the least upper bound of $B(x + y)$. 

First, to see that it is in fact an upper bound, consider any $t \leq x + y$. We
want to construct two rationals $r,s$ such that $r \leq x$, $s \leq y$ and $r +
s = t$. 

I realized when I'm about to submit that there is an easier construction than
what I originally did: pick
some rational $r$ in the interval $[t-y,x]$ (since we have $t - y \leq x$ and
one such rational is guaranteed to exist by the density $\Q$ as shown by Rudin), and
take $s = t-r$. Then, we have
that $r < x$ and $s = t - r \leq t - (t-y) = y$ which gives $r,s$ as desired. The boxed
construction \textit{should also} work, but it's irrelevant.

\fbox{\begin{minipage}{\textwidth}
  To do this, consider that the denseness of $\Q$ in $\R$ (as shown earlier by Rudin)
  gives some rational $x'$ in the interval $[\frac{x-y+t}{2},x]$ and some other rational $y'$
  in $[\frac{y-x+t}{2},y]$. Note that $x' + y' \geq \frac{x-y+t}{2} +
  \frac{y-x+t}{2} = t$. Then, consider the quantities
  \[
    r = x' - \frac{x' + y' - t}{2}, s = y' - \frac{x' + y' - t}{2}
  \]

  Note that $r + s = t$, and since $r\leq x' \leq x, s\leq y' \leq y$, we
  have our desired $r,s$.
\end{minipage}}

Now, we have that $b^t = b^rb^s$, with $b^t \in B(x+y), b^r \in B(x), b^s \in
B(y)$. Now, we have that $\sup B(x) \geq b^r, \sup B(y) \geq b^s$, such that
$\sup B(x) \sup B(y) \geq b^rb^s = b^t$. This shows that $\sup B(x) \sup B(y)$
is a upper bound.


Now consider any real $a$ that is less than $\sup B(x) \sup
B(y)$. Then, we have that $a < \sup B(x) \sup B(y)$. Then, by the earlier
lemma, we have some $b^{\beta} \in B(x)$ such that $\alpha / \sup B(y) <
b^{\beta}$, and now some $b^{\gamma} \in B(y)$ such that $\alpha / b^{\beta} <
b^{\gamma}$ with $\beta, \gamma$ rational. Now, we have that $\alpha < b^\beta
b^\gamma = b^{\beta + \gamma}$. However, since we have that $\beta \leq x$ and
$\gamma \leq y$ from the definitions of $B(x), B(y)$, we have that $\beta +
\gamma \leq x + y \implies b^{\beta\gamma} \in B(x + y)$. Then, we have that $a$
cannot be an upper bound for $\sup B(x + y)$, and thus $\sup B(x) \sup B(y) = \sup B(x + y)$.

\section*{Ch1, Q5}

% To show that $\inf A = - \sup (-A)$, we need to show that $\inf A + \sup(-A) =
% 0$. First, we will show that $\inf A + \sup(-A) \leq 0$: consider any $a_p \in A,
% a_n \in -A$. Then, we have that 

We will show first that $- \sup(-A)$ is a lower bound of $A$. In particular,
for any element $a \in A$, we have that $-a \in -A$, and by definition $-a \leq
\sup(-A)$. Then, $-(-a) = a \geq -\sup(-A)$. 

To see that it is the greatest lower bound, consider any $x > -\sup(-A)$. Then,
we would have that $-x < \sup(-A)$. Now, in an
earlier problem we showed that there is some element $a \in -A$ such that
$\sup(-A) - a < \sup(-A) - (-x) \implies -a < x$. However, since we have that
$-A$ has the form of $\{ -\alpha \mid \alpha \in A\}$, we have that any $a \in
-A \implies -a \in A$. However, since $-a < x$, $x$ cannot be a lower bound for
$A$.

Thus, we have that $-\sup(-A)$ is the greatest lower bound of $A$, $\inf A$.

% pick any element of $a_n \in -A$, and note that $a_n = -(a_p)$ for some $a_p \in
% A$. Then, we have that $-a_n = -(-a_p) = a_p$, and since $\sup(-A) \geq a_n
% \implies -a_n \geq -\sup(-A)$, we have that $-\sup (-A) \leq $

\section*{Ch1, Q7}

\subsection*{a}

We have the identity that for any positive integer $n$, $x^n - y^n =
(x-y)(x^{n-1} + x^{n-2}y + \dots + y^{n-1})$ as used by Rudin earlier. Then,
taking $x = b, y = 1$, we have that $b^n - 1 = (b-1)(b^{n-1} + b^{n-2} + \dots +
1)$. In particular, since $b > 1, b^n > 1$ (for $n$ any positive
integer) and taking the convention
that $b^0 = 1$, we have that
\[
  \frac{b^n - 1}{b-1} = \frac{(b-1)\left(\sum_{i=0}^{n-1}b^i\right)}{b-1} =
  \sum_{i=0}^{n-1}b^i \geq \sum_{i=0}^{n-1} 1 = n
\]

Then, this gives us what we wanted, as $\frac{b^n - 1}{b-1} \geq n \implies b^n
- 1 \geq n(b-1)$.

\subsection*{b}

We have that the above holds for any $b > 1$. In particular, if we wish to show
that $b - 1 \geq n(b^{1 / n} - 1)$, then we simply take part a and apply it to
$b^{1 / n}$, which gives us what we wanted.

The only thing to check is that for any $b > 1$, we still have that $b^{1 / n} >
1$. Suppose otherwise; then we what that $b^{1 / n} \leq 1 \implies (b^{1 /
n})^n \leq 1^n = 1$, $\contra$.

\subsection*{c}

Applying the above, we have
\[
  n > \frac{b-1}{t-1} \implies n(t-1) > b - 1 \geq n(b^{1 / n} - 1) \implies t -
  1 > b^{1 / n} - 1 \implies t > b^{1 / n}
\]

\subsection*{d}

As Rudin suggests, applying the above with $t = yb^{-w}$ yields that for any
positive integer $n$, we have that
\[
  n > \frac{b-1}{t-1} \implies yb^{-w} > b^{1 / n} \implies yb^{-w}b^w > b^{1 /
  n}b^w \implies y > b^{w + 1 / n}
\]

In particular, since $b, y, w$ are fixed (meaning that $(b-1) / (t-1)$ is some
fixed real) and that the integers have no greatest
element, there exists some positive integer $n > (b-1) / (t-1)$.

The only thing left to check is that $t = yb^{-w} > 1$, but this is directly
given by $b^w > y \implies b^{w}b^{-w} = 1 > yb^{-w}$.

\subsection*{e}

Using the same trick as before but with $t = b^w / y$ (note that $b^w > y
\implies b^w / y > 1$), we have that for any
positive integer $n$,

\[
  n > \frac{b-1}{t-1} \implies \frac{b^{w}}{y} > b^{1 / n} \implies b^{w}b^{- 1
  / n} > b^{-1 / n}b^{1 / n}y \implies b^{w - 1 / n} > y
\]

Again, we have some fixed bound $(b-1) / (t-1)$ above which $b^{w - 1 / n} > y$.

\subsection*{f,g}

We will first show part g.

If $x > 0$, we have that $b^x > 1$ as there is some rational $r = p / q$ in the
interval $(0, x)$ such that $r \in B(x)$, and we show in Q5, part c that $b^r >
0$. Then, since $b^x = \sup B(x)$, we have that $b^x \geq b^r > 1$.
Correspondingly, $b^x > 1 \implies b^xb^{-x} = 1 < b^{-x}$. Now, if $b^x
= y$ and $x' > x$, $b^{x'} = b^xb^{x'-x} = yb^{x'-x} > y$. Similarly, if $x' <
x$, we have that $b^{x'} = b^xb^{x'-x} = yb^{-(x'-x)} < y$. Then, $b^x = b^{x'}
\iff x = x'$.


First, we have to see that $A$ is nonempty: for any $y > 1$, $0 \in A$.
Otherwise, $\exists n$ such
that $b^n > 1 / y$; in particular, we know that $n$ such that $n(b - 1) > 1 / y$ from
the Archimedean property, which gives some $b^n \geq n(b-1) > 1 / y$. Then, we have
that $b^{-n} < y$.

Now, if $b^x > y$, then we have that $\exists n \in \Z^+$ such that $b^{x - 1 / n}
> y > b^w \implies x - 1 / n > w$ for any $w \in A$ via the proof of part g, and so $x - 1 / n$ is a lower upper bound of $A$.
$\contra$

If $b^x < y$, then we have that $\exists n \in \Z^+$ such that $b^{x + 1 / n} <
y$, which means that $x + 1 / n \in A$, and $x$ cannot be an upper bound of $A$.
$\contra$

This only leaves $b^x = y$.


\end{document}
% LocalWords:  NetID fancyplain LocalWords colorlinks linkcolor linkbordercolor
