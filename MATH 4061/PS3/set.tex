\documentclass[12pt,letterpaper]{article}
\usepackage{fullpage}
\usepackage[top=2cm, bottom=4.5cm, left=2.5cm, right=2.5cm]{geometry}
\usepackage{amsmath,amsthm,amsfonts,amssymb,amscd}
\usepackage{lastpage}
\usepackage{enumerate}
\usepackage{fancyhdr}
\usepackage{mathrsfs}
\usepackage{xcolor}
\usepackage{graphicx}
\usepackage{listings}
\usepackage{hyperref}
\usepackage{tikz}
\usepackage{relsize}
\usepackage{fancyvrb}
\usepackage{import}
\usetikzlibrary{shapes.geometric,fit}

\hypersetup{%
  colorlinks=true,
  linkcolor=blue,
  linkbordercolor={0 0 1}
}

\setlength{\parindent}{0.0in}
\setlength{\parskip}{0.05in}

\theoremstyle{definition}
\newtheorem*{statement}{Statement}
\newtheorem*{claim}{Claim}
\newtheorem*{theorem}{Theorem}
\newtheorem*{lemma}{Lemma}

\newcommand{\contra}{\Rightarrow\!\Leftarrow}
\newcommand{\R}{\mathbb{R}}
\newcommand{\F}{\mathbb{F}}
\newcommand{\Z}{\mathbb{Z}}
\newcommand{\Zeq}{\mathbb{Z}_{\geq 0}}
\newcommand{\Zg}{\mathbb{Z}_{>0}}
\newcommand{\Req}{\mathbb{R}_{\geq 0}}
\newcommand{\Rg}{\mathbb{R}_{>0}}
\newcommand{\N}{\mathbb{N}}
\newcommand{\Q}{\mathbb{Q}}
\newcommand{\C}{\mathbb{C}}

\newcommand{\incfig}[1] {%
    % \def\svgwidth{\columnwidth}
    \import{./figures/}{#1.pdf_tex}
}

\title{MATH 4061 HW 3}
\author{David Chen, dc3451}

\begin{document}

\maketitle

\section*{Ch 2, Q7}

\subsection*{a}

First, we can show that $(E \cup F)' = E' \cup F'$. In the case that $p \in E'$ or $p \in F'$, then we have that for any $r > 0$, $(E \cup F) \cap B_{r}^{\circ}(b) = (E \cap B_{r}^{\circ}(b)) \cup (F \cap B_{r}^{\circ}(b))$, but the right hand side cannot be empty, or else $p \notin E', F'$. Then, we have that $p \in (E \cup F)'$. Thus, we have that $E' \cup F' \subseteq (E \cup F)'$.

To show that $(E \cup F)' \subseteq E' \cup F'$, suppose we have some $p \in (E \cup F)'$, but $p \notin E', p \notin F'$. Then, we have some $r_{E} > 0$ such that $E \cup B_{r_{E}}^{\circ}(p) \setminus \{p\} = \emptyset$ and some $r_{F} > 0$ such that $E \cup B_{r_{F}}^{\circ}(p)  \setminus \{p\} = 0$ (otherwise, $p$ would be a limit point of $E$ or $F$). Then, we have that $r = \min(r_{E},r_{F})$, such that $(E \cup F) \cap B_{r}^{\circ}(p) \setminus \{p\}  = (E \cap B_{r}^{\circ}(p)  \setminus \{p\} ) \cup (F \cap B_{r}^{\circ}(p) \setminus \{p\} ) = \emptyset \cup \emptyset = \emptyset$. Then, clearly $p \notin (E \cup F)'$, $\contra$, and so $(E \cup F)' \subseteq E' \cup F'$.

We can now show that $\overline{E \cup F} = \overline{E} \cup \overline{F}$, as we have that $\overline{E \cup F} = (E \cup F) \cup (E \cup F)' = (E \cup F) \cup (E' \cup F') = (E \cup E') \cup (F \cup F') = \overline{E} \cup \overline{F}$.

Then, we can induct to show that $\overline{B} = \overline{\bigcup_{i=1}^{n}A_{i}} = \bigcup_{i=1}^{n}\overline{A}_{i}$. $n = 1$ is trivial. Then, if it holds for $n$, we have that
\begin{align*}
  \overline{\bigcup_{i=1}^{n+1}A_{i}} &= \overline{\bigcup_{i=1}^{n}{A}_{i} \cup A_{n+1}} \\
  \intertext{By the first statement,}
                                      &= \overline{\bigcup_{i=1}^{n}A_{i}} \cup \overline{A}_{n+1} \\
  \intertext{By inductive hypothesis,}
                                      &= \bigcup_{i=1}^{n}\overline{A}_{i} \cup \overline{A}_{n+1} \\
                                      &= \bigcup_{i=1}^{n+1`}\overline{A}_{i}\\
\end{align*}

so we have that for any $n, \overline{B} = \overline{\bigcup_{i=1}^{n}A_{i}} = \bigcup_{i=1}^{n}\overline{A}_{i}$

\subsection*{b}

If $p \in \bigcup_{i=1}^{\infty}\overline{A}_{i}$, then we have that we have some $A_{j}$ such that $p \in \overline{A}_{j}$. Then, we have two cases: if $p \in A_{j}$, then $p \in \bigcup_{i=1}^{\infty}A_{i} = B$. The second case is that $p \in A_{i}'$; then, for any $r > 0$, we have that $A_{j} \cap B_{r}^{\circ}(p) \setminus \{p\}  \neq \emptyset$, which gives that as $A_{j} \subset B$, $B \cap B_{r}^{\circ}(p) \setminus \{p\}  \neq \emptyset$, so $p \in B'$ as well, so $p \in \overline{B}$. In both cases, $p \in \overline{B}$, so $B \supset \bigcup_{i=1}^{\infty}A_{i}$.

\section*{Ch 2, Q9}
\subsection*{d}

We want to show that $(E^{\circ})^{c} = \overline{E^{c}}$.

First, for any $p$, we have two cases. If $p \notin E$, then we have that since $E^{\circ} \subset E$, $p \notin E^{\circ} \implies p \in (E^{\circ})^{c}$. Similarly, we have that $\overline{E^{c}} = E^{c} \cup (E^{c})'$, so $p \in \overline{E^{c}}$, so we have that for $p \notin E$, $p \in (E^{\circ})^{c} \iff p \in \overline{E^{c}}$. Then, the only points which remain are the ones which are in $E$.

Now, for $p \in E$, if $p \in (E^{\circ})^{c}$, then we have that $p$ is not an interior point of $E$; that is, for every $r>0$, $B_{r}^{\circ}(p) \not\subset E$, so $B_{r}^{\circ}(p) \setminus \{p\}$ contains some point not in $E$, which then is in $E^{c}$. Then, for any $r > 0$, we have that $E^{c} \cap B_{r}^{\circ}(p)  \setminus \{p\} \neq \emptyset$, so $p \in (E^{c})' \implies p \in \overline{E^{c}}$.

Further, for $p \in E$, if $p \in \overline{E^{c}}$, we have that for any $r > 0$, $E^{c} \cap B_{r}^{\circ}(p) \setminus \{p\} \neq 0$. This means that $B_{r}^{\circ}(p) \not\subset E$, as it contains some point in $E^{c}$,  and so $p$ is not an interior point of $E$, so $p \notin E^{\circ} \implies p \in (E^{\circ})^{c}$.

Since in both cases $p \in E$ and $p \notin E$, $p \in (E^{\circ})^{c} \iff p \in \overline{E^{c}}$, the two sets are equal.

\subsection*{e}

No: consider something like $E = \Q \subset \R$. Then, the closure of $E$ is $\R$ as $\Q$ is dense in the reals, but the interior of $\Q$ is empty, and the interior of $\R$ is $\R$ itself, which is definitely \textit{not} empty.

\subsection*{f}

Again no: consider the same example. $E = \Q$ has interior $\emptyset$, the closure of which is $\emptyset$, but the closure of $\Q$ is $\R$.

\section*{Ch 2, Q10}

For any $p, q \in X$, we have that if $p \neq q$, $d(p,q) = 1 > 0$ and if $p = q$, $d(p,q) = 0$ so it satisfies the property of a metric.

Further,
\[
  d(q,p) = \begin{cases}
    1 & q \neq p \\
    0 & q = p
  \end{cases}
  = \begin{cases}
    1 & p \neq q \\
    0 & p = q
  \end{cases}
  = d(p,q)
\]
so $d$ is symmetric.

Lastly, $d(p,r) + d(r,q) \geq d(p,q)$ can be verified in a few cases.
\begin{enumerate}
  \item $d(p,q) = 0$. Since $d(p,r) \geq 0$, $d(r,q) \geq 0$ by the first property, we have that $d(p,r) + d(r,q) \geq 0 = d(p,q)$.
  \item $d(p,q) = 1$. Then, we have that $p \neq q$, so we cannot have that both $r = p$ and $r = q$. In that case, we have at least one of $r \neq p$ or $r \neq q$, so $d(p,r) + d(r,q) \geq 1 = d(p,q)$.
\end{enumerate}

Thus, $d$ is a metric.

Every set is open. First, note that for any $p \in X$, $B_{\frac{1}{2}}^{\circ}(p) = \{q \in X \mid d(p,q) < \frac{1}{2}\}$. However, $d(p,q) < \frac{1}{2} \implies d(p,q) = 0 \implies p = q$, so $B_{\frac{1}{2}}^{\circ}(p) = \{p\}$. Now, consider $E \subseteq X$. Then, for any point $p \in E$, we have that $B_{\frac{1}{2}}^{\circ}(p) = \{p\} \subset E$, so $E$ is open.

Every set is also closed. First, note that since every set is open, any $E \subset X$ has that $E^{c}$ is open, and so $E = (E^{c})^{c}$ is the complement of an open set, and is thus closed. Alternatively, note that there are no limit points, since for any $p \in E$, $B_{\frac{1}{2}}^{\circ}(p) \cap (E \setminus \{p\}) = \{p\} \cap (E \setminus \{p\}) = \emptyset$. Thus, every set contains all of its (none) limit points, and is therefore closed.

Only sets of finite order are compact. To see this, we can clearly give a finite subcover of any finite set. Let $E = \bigcup_{\alpha \in A}G_{\alpha}$ for some indexing set $A$. Then, since $E$ is finite (say with order $n$), we can select for each $p_{i} \in E$ that $p_{i} \in G_{\alpha_{i}}$ for some $\alpha_{i} \in A$ (if no such $\alpha_{i}$ exists, then $\bigcup_{\alpha \in A}G_{\alpha}$ does not contain $p$ and is thus not an open cover of $E$). Then, we have that $\bigcup_{i=1}^{n}G_{\alpha_{i}}$ contains every $p_{i} \in E$, so we have a finite subcover of $E$.

Furthermore, we can give an open cover of any infinite set that contains no finite subcover. In particular, consider for any infinite set $E \subseteq X$ the open cover $\bigcup_{p \in E}\{p\}$. Any finite subcover $\bigcup_{i=1}^{n}\{p_{i}\}$ has finite order, as it is the finite union of finite sets, and so $E$ cannot be a subset of $\bigcup_{i=1}^{n}\{p_{i}\}$, as $E$ is infinite. Thus, this open cover admits no finite subcover so no infinite set can be compact.

\section*{Ch 2, Q12}

Suppose that we have an open cover such that $K \subset \bigcup_{\alpha \in A}G_{\alpha}$ for some indexing set $A$. Then, there is some $G_{\alpha_{0}}$ for $\alpha_{0} \in A$ such that $0 \in G_{\alpha_{0}}$. Then, since $G_{\alpha_{0}}$ is open, we have that there is some $r > 0$ such that $B_{r}^{\circ}(0) \subset G_{\alpha_{0}}$.

Then, for $n \in \N$, $n > 1/r \implies 1/n < r \implies 1/n \in G_{\alpha_{0}}$. Let $N$ be the greatest natural such that $N \leq 1/r$. Then, for $1 \leq n \leq N$, we know that $\exists$ some $G_{\alpha_{n}}$ such that $\alpha_{n} \in A$ and $1/n \in G_{\alpha_{n}}$, since $\bigcup_{\alpha \in A}G_{\alpha}$ is an open cover of a set containing $1/n$. Then, we have that $\bigcup_{i=0}^{N}G_{\alpha_{i}}$ is a finite subcover of $K$, as $0 \in G_{\alpha_{0}}$, and for any $n \in N$, we have that either $n > 1/r \implies 1/n \in G_{\alpha_{0}}$ or $n \leq 1/r \implies 1/n \in G_{\alpha_{n}}$.

\section*{Ch 2, Q25}

For every $n \in \N$, consider the open cover of $K$, $\bigcup_{p \in K}B_{\frac{1}{n}}^{\circ}(p)$. Then, since $K$ is compact, there is some finite collection $\{p_{n_{i}}\}$ such that $\bigcup_{i=1}^{m_{n}}B_{\frac{1}{n}}^{\circ}(p_{n_{i}})$ is an open cover of $K$. Then, we have for each $n$ some finite associated collection of sets $E_{n} = \{B_{\frac{1}{n}}^{\circ}(p_{n_{i}})\}_{i=1}^{m_{n}}$. We know that as the countable union of finite sets, $\bigcup_{n=1}^{\infty}E_{n}$ is countable itself.

We can check that $\bigcup_{n=1}^{\infty}E_{n}$ is a base for $K$: for every $p \in K$ and open $G \subset K$ that contains $p$, we have that since $G$ is open, there is some $r > 0$ such that $B_{r}^{\circ}(p) \subset G$. Then, there is some $n > 2/r \implies 1/n < \frac{r}{2}$; consider now $E_{n}$, which is an open cover of $K$, and thus also an open cover of $B_{r}^{\circ}(p)$. Therefore, we must have some $B_{\frac{1}{n}}^{\circ}(p_{n_{i}}) \in E_{n}$ such that $p \in B_{\frac{1}{n}}^{\circ}(p_{n_{i}})$. However, for any $q \in B_{\frac{1}{n}}^{\circ}(p_{i})$, we have that $d(p,q) \leq d(p,p_{n_{i}}) + d(p_{n_{i}}, q) \leq \frac{1}{n} + \frac{1}{n} < r$, so $p \in B_{\frac{1}{n}}^{\circ}(p_{n_{i}}) \subset B_{r}^{\circ}(p) \subset G$, so we have that $E_{n}$ is a countable base for $K$.

To conclude that $K$ is separable, note that having a countable basis is sufficient to show that a set is separable. To see this, we can construct a countable dense subset of $K$ from a countable basis $\{V_{\alpha}\}_{\alpha \in A}$ where $A$ is some countable index set. In particular, select one point $p_{\alpha}$ from every $V_{\alpha}$. The set of all $\{p_{\alpha}\}_{\alpha \in A}$ is a countable dense subset; to see this, we need to show that every point $p \in K$ is a limit point of $E = \{p_{\alpha}\}_{\alpha \in A}$ or in $E$.

We have two choices for $p \in K$. Either $p \in E$ or $p \notin E$; in the first case, we are immediately done. In the second case, for any $r > 0$, we have that $B_{r}^{\circ}(p)$ contains some $p_{\alpha} \in E$ where $p_{\alpha} \neq p$, as $V_{\alpha} \subset B_{r}$ for some $\alpha$ by the definition of base (as $B_{r}^{\circ}(p)$ is a nonempty open set), and $p_{\alpha} \in V_{\alpha}$ by construction. Then, $B_{r}^{\circ}(p) \cap (E \setminus{p}) \neq \emptyset$ and $p$ is a limit point of $E$, so $\overline{E} = K$, so $E$ is a countable dense subset of $K$.

\end{document}
% LocalWords:  NetID fancyplain LocalWords colorlinks linkcolor linkbordercolor
