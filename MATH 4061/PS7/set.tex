\documentclass[12pt,letterpaper]{article}
\usepackage{fullpage}
\usepackage[top=2cm, bottom=4.5cm, left=2.5cm, right=2.5cm]{geometry}
\usepackage{amsmath,amsthm,amsfonts,amssymb,amscd}
\usepackage{mathtools}
\usepackage{lastpage}
\usepackage{enumerate}
\usepackage{fancyhdr}
\usepackage{mathrsfs}
\usepackage{xcolor}
\usepackage{graphicx}
\usepackage{listings}
\usepackage{hyperref}
\usepackage{tikz}
\usepackage{relsize}
\usepackage{fancyvrb}
\usepackage{import}
\usetikzlibrary{shapes.geometric,fit}

\hypersetup{%
  colorlinks=true,
  linkcolor=blue,
  linkbordercolor={0 0 1}
}

\setlength{\parindent}{0.0in}
\setlength{\parskip}{0.05in}

\theoremstyle{definition}
\newtheorem*{statement}{Statement}
\newtheorem*{claim}{Claim}
\newtheorem*{theorem}{Theorem}
\newtheorem*{lemma}{Lemma}

\newcommand{\contra}{\Rightarrow\!\Leftarrow}
\newcommand{\R}{\mathbb{R}}
\newcommand{\F}{\mathbb{F}}
\newcommand{\Z}{\mathbb{Z}}
\newcommand{\Zeq}{\mathbb{Z}_{\geq 0}}
\newcommand{\Zg}{\mathbb{Z}_{>0}}
\newcommand{\Req}{\mathbb{R}_{\geq 0}}
\newcommand{\Rg}{\mathbb{R}_{>0}}
\newcommand{\N}{\mathbb{N}}
\newcommand{\Q}{\mathbb{Q}}
\newcommand{\C}{\mathbb{C}}

\newcommand{\incfig}[1] {%
    % \def\svgwidth{\columnwidth}
    \import{./figures/}{#1.pdf_tex}
}

\title{MATH 4061 HW 7}
\author{David Chen, dc3451}

\begin{document}

\maketitle

\section*{2}

Let $E'$ be the set of all limit points of $E$, such that $\overline{E} = E \cup E'$. Then, if $x \in E$, then $f(x) \in f(E) \subset \overline{f(E)}$ by definition, so we are only concerned about $x \in E'$.

Left superscripts denotes the metric space we are working in.

We need to show that for any $x \in E'$, $f(x)$ is a limit point of $f(E)$ or is in $f(E)$. Since $x$ is a limit point of $E$, we have that for any $\delta > 0$, ${}^{X}B_{\delta}^{\circ}(x) \cap (E \setminus \{x\}) \neq \emptyset$, and since $f$ is continuous, we have that for any $\epsilon > 0$, we can find $\delta > 0$ such that $y \in {}^{X}B_{\delta}^{\circ}(x) \implies f(y) \in {}^{Y}B_{\epsilon}(f(x))$. Then, the first statement gives exactly some $y$ such that $y \in E$ and $y \in {}^{X}B_{\delta}^{\circ}(x)$, and thus some $f(y)\in {}^{Y}B_{\epsilon}(f(x))$ for $y \in E$.

Now, either $f(y) = f(x)$, for $f(y) \neq f(x)$. In the first case, then $f(x) = f(y) \in f(E)$, and we are done. In the other case, then $f(y) \in f(E) \setminus \{f(x)\} \implies f(y) \in {}^{Y}B_{\epsilon}(f(x)) \cap f(E) \setminus \{f(x)\}$, so any neighborhood of $f(x)$ contains an element of $f(E)$ not itself, and so $f(x)$ is a limit point of $f(E)$. In either case, $x \in E' \implies f(x) \in \overline{f(E)}$, and we have what we want.

To see that it can be a proper subset, consider $f: [1, \infty) \rightarrow \R$ given by $f(x) = 1/x$. This is clearly continuous (since $x$ is continuous, and does not vanish on $[1, \infty)$), but $f([1,\infty)) = (0,1]$. Then, the closure of $[1,\infty)$ is itself, so
\[
  f(\overline{[1,\infty)}) = (0,1] \subset \overline{f([1,\infty))} = \overline{(0, 1]} = [0,1]
\]

\section*{3}

This follows immediately from the fact that $Z(f)$ is defined the same as $f^{-1}(\{0\})$, but $\{0\}$ is a closed set and the preimage of a closed set is closed by a theorem from class (also Rudin corollary to 4.8).

\section*{4}

We want to show that every point of $f(X)$ is either a point of $f(E)$, or a limit point of $f(E)$.

From problem 2, we have that $f(\overline{E}) \subset \overline{f(E)}$, but since $E$ is dense in $X$, the closure of $E$ is $X$, so we have that $f(\overline{E}) = f(X)$, so $f(X) \subset \overline{f(E)}$, so every point in $f(X)$ is either a limit point of or member of $f(E)$, and we are done.

Now consider $h: X \rightarrow Y$, $h(x) = f(x) - g(x)$, a continuous function which vanishes on $E$, a dense subset of $X$. Then, suppose that $h(x) \neq 0$ for some $x \in X$. Then clearly $x \notin E$, since in $E$ by assumption $f(x) = g(x) \implies f(x) - g(x) = 0$, so $x$ must be a limit point of $E$.

Suppose that $d(h(x),0) = \epsilon > 0$; then, we have some $\delta > 0$ such that for $y \in B_{\delta}^{\circ}(x) \implies h(y) \in B_{\epsilon/2}^{\circ}(h(x))$, so by the reverse triangle inequality,
\[
  d(h(x),h(y)) \geq d(h(y),0) - d(h(x), 0) \implies d(h(x),0) \leq d(h(y),0) - d(h(x),h(y)) < \epsilon - \frac{\epsilon}{2} = \frac{\epsilon}{2}
\]
so $h(y) \neq 0$. However, we have that since $E$ is dense in $X$, $B_{\delta}^{\circ}(x) \cap (E \setminus \{x\})$ is nonempty; pick an element $y$ inside of $B_{\delta}^{\circ}(x) \cap (E \setminus \{x\})$. Then, by continuity, $h(x) \neq 0$, but by assumption, since $y \in E$, we have that $h(x) = 0$. $\contra$, so $h$ is identically zero, and so $f - g = 0 \implies f = g$.

\section*{7}

To see that $f$ is bounded, we have
\[
  \frac{1}{2} - f(x) = \frac{x^{2} + y^{4} - 2xy^{2}}{2(x^{2}+y^{4})} = \frac{(x - y^{2})^{2}}{2(x^{2}+y^{4})} \geq 0
\]
since both numerator and denominator are nonnegative, so $f(x,y) \leq 1/2$ for any $x,y$; then, since $f(-x,y) = -f(x,y)$, if $f(x,y) < -1/2$ for any $x$, we have that $f(-x,y) > 1/2$, so $\contra$ and thus $|f(x,y)| < 1/2$. This handles when $x^{2}+y^{4} \neq 0$, but $x^{2}+y^{4} = 0$ only when $x = y = 0$ and $f(0,0) = 0$ by definition, so we're good.

To see that $g$ is always unbounded in a neighborhood of 0, consider $x = y^{3}$. Then, we have that $g(y^{3},y) = \frac{y^{5}}{2y^{6}} = \frac{1}{2y}$. We have that $\lim_{y \rightarrow 0}|g(y^{3},y)| = \lim_{y \rightarrow 0}\left|\frac{1}{2y}\right| = \infty$, so $g$ is always unbounded in a neighborhood of 0.

To see that $f$ is discontinuous at the origin, we have that if we take $x = y^{2}$, $f(y^{2},y) = \frac{y^{4}}{2y^{4}} = \frac{1}{2}$, so $\lim_{y \rightarrow 0}f(y^{2},y) = \frac{1}{2} \neq f(0,0)$.

To see that the restriction of $f,g$ is always continuous, note that for lines which avoid $(0,0)$, $f,g$ are already continuous everywhere but at the origin (since their denominators don't vanish and $f,g$ are the quotient of continuous functions). Then, their restriction must also be continuous (since the limit converges on every path, including the line we are restricting to).

Any line through the origin is either given by $x = 0$, in which case $f,g$ both vanish everywhere and are thus continuous or is given by $y = ax$. Then, away from the origin, $f(x,ax) = \frac{a^{2}x^{3}}{x^{2}+a^{4}x^{4}} = \frac{a^{2}x}{1 + a^{4}x^{2}}$, which is easily seen to be continuous everywhere,  except maybe at $x = 0$ (again as the quotient of two continuous functions), but $\lim_{x \rightarrow 0}\frac{a^{2}x}{1+a^{4}x^{2}} = 0 = f(0,0)$.

For $g$, $g(x,ax) = \frac{a^{2}x^{3}}{x^{2}+a^{6}y^{6}} = \frac{a^{2}x}{1 + a^{6}y^{4}}$, which is continuous everywhere except at the origin, but again, $\lim_{x \rightarrow 0}\frac{a^{2}x}{1+a^{6}x^{6}} = 0 = f(0,0)$.


\end{document}
% LocalWords:  NetID fancyplain LocalWords colorlinks linkcolor linkbordercolor
