\documentclass[12pt,letterpaper]{article}
\usepackage{fullpage}
\usepackage[top=2cm, bottom=4.5cm, left=2.5cm, right=2.5cm]{geometry}
\usepackage{amsmath,amsthm,amsfonts,amssymb,amscd}
\usepackage{lastpage}
\usepackage{enumerate}
\usepackage{fancyhdr}
\usepackage{mathrsfs}
\usepackage{xcolor}
\usepackage{graphicx}
\usepackage{listings}
\usepackage{hyperref}
\usepackage{tikz}
\usepackage{relsize}
\usepackage{fancyvrb}
\usetikzlibrary{shapes.geometric,fit}

\hypersetup{%
  colorlinks=true,
  linkcolor=blue,
  linkbordercolor={0 0 1}
}

\setlength{\parindent}{0.0in}
\setlength{\parskip}{0.05in}

% Edit these as appropriate
\newcommand\course{MATH 1207}
\newcommand\hwnumber{6}                  % <-- homework number
\newcommand\NetIDa{dc3451}           % <-- NetID of person #1
\newcommand\NetIDb{David Chen}           % <-- NetID of person #2 (Comment this line out for problem sets)

\theoremstyle{definition}
\newtheorem*{statement}{Statement}
\newtheorem*{claim}{Claim}
\newtheorem*{theorem}{Theorem}
\newtheorem*{lemma}{Lemma}

\newcommand{\contra}{\Rightarrow\!\Leftarrow}
\newcommand{\R}{\mathbb{R}}
\newcommand{\F}{\mathbb{F}}
\newcommand{\Z}{\mathbb{Z}}
\newcommand{\Ze}{\mathbb{Z}_{\geq 0}}
\newcommand{\Zg}{\mathbb{Z}_{>0}}
\newcommand{\N}{\mathbb{N}}
\newcommand{\Q}{\mathbb{Q}}
\newcommand{\C}{\mathbb{C}}

\pagestyle{fancyplain}
\headheight 35pt
\lhead{\NetIDa}
\lhead{\NetIDa\\\NetIDb}                 % <-- Comment this line out for problem sets (make sure you are person #1)
\chead{\textbf{\Large Homework \hwnumber}}
\rhead{\course \\ \today}
\lfoot{}
\cfoot{}
\rfoot{\small\thepage}
\headsep 1.5em

\begin{document}

\subsection*{Apostol p.125 no.21}

\begin{claim}
  Let $f,g$ be functions that are integrable on every interval and satisfying
  the following: $f$ is odd, $g$ is even, $f(5) = 7, f(0) = 0, g(x) = f(x+5),
  f(x) = \int_0^xg(t)dt$ for all $x$. Then (a) $\forall x, f(x-5) = -g(x)$; (b)
  $\int_0^5f(x)dx = 7$; (c) $\int_0^xf(t)dt = g(0) - g(x)$.
\end{claim}

\begin{proof}
  (a)
  \begin{alignat*}{2}
    && g(x) &= f(x + 5) \\
    &\implies& g(x) &= \int_0^{x+5}g(t)dt \\
    && g(x) &= g(-x) = \int_0^{-x + 5}g(t)dt \\
    &\implies& g(x) &= f(-x + 5) = -f(x - 5) \\
    &\implies& f(x - 5) &= -g(x)
  \end{alignat*}

  (b) Note that since $f,g$ are integrable on every interval, then we have that
  $g(x) = f(x + 5) \implies g(y - 5) = f(y)$ by simply taking $y = x + 5$. Since
  the choice of variables is arbitrary, in general, we have that $g(x - 5) = f(x)$.

  
  \begin{alignat*}{2}
    && \int_0^5f(t)dt &= \int_0^5g(x - 5)dx \\
    && &=\int_{-5}^0g(t)dt \\
    && &=\int_0^5g(-t)dt \\
    && &=\int_0^5g(t)dt \\
    && &=f(5) = 7
  \end{alignat*}

  (c) Similarly to above,

  \begin{alignat*}{2}
    && \int_0^xf(t)dt &= \int_0^xg(t - 5)dt \\
    && &=\int_{-5}^{x-5}g(t)dt \\
    && &=\int_{-x + 5}^5g(-t)dt \\
    && &=\int_{-x+5}^5g(t)dt \\
    && &=\int_{-x+5}^0g(t)dt + \int_0^5g(t)dt \\
    && &=\int_0^{x-5}g(t)dt + f(5) \\
    && &=f(x-5) + f(5) \\
    && &= -g(x) + g(0)
  \end{alignat*}
\end{proof}

\subsection*{Apostol p.138-139 no.5}

\begin{lemma}
  \[
    \lim_{x \rightarrow 0} \frac{x}{x} = 1
  \]
\end{lemma}

\begin{proof}
  For any $\epsilon > 0$, taking $\delta = 1224323121$, we have that $0 < |x - 0| < 1224323121 \implies x \neq 0
  \implies |\frac{x}{x} - 1| = 0 < \epsilon$. Thus, the limit is then just $1$. 
\end{proof}

\begin{alignat*}{3}
  && \lim_{h\rightarrow 0} \frac{(t+h)^2 - t^2}{h} &= \lim_{h\rightarrow 0}
  \frac{t^2 + 2th +h^2 - t^2}{h} & \\
  && &=\lim_{h\rightarrow 0}\frac{2th + h^2}{h} & \\
  && &=\lim_{h\rightarrow 0}\frac{2th}{h} + \frac{h^2}{h} &\\
  && &=\lim_{h\rightarrow 0}\frac{h}{h}(2t + h) &\\
  && &=\lim_{h\rightarrow 0}\frac{h}{h} \cdot \lim_{h\rightarrow 0} 2t + h
  & \text{ (multiplicativity of limits) } \\
  && &=\lim_{h\rightarrow 0} 2t + h & \text{ (lemma) }\\
  && &= 2t & \text{ (see below) }\\
\end{alignat*}

To evaluate this last limit, consider that for any $\epsilon > 0$, taking $\delta =
\frac{\epsilon}{2} \implies \forall x, 0 < |x - 0| < \frac{\epsilon}{2} \implies
|2t + \frac{\epsilon}{2} - 2t| = \frac{\epsilon}{2} < \epsilon$. 

\subsection*{Apostol p.138-139 no.8}

\begin{alignat*}{3}
  && \lim_{x\rightarrow 0} \frac{x^2 -a^2}{x^2 + 2ax + a^2} &=
  \lim_{x\rightarrow 0} \frac{(x-a)(x+a)}{(x+a)^2} &\\
  && &= \lim_{x\rightarrow a}\frac{x-a}{x+a} & (\text{since } x + a \neq 0
  \text{ when } x = a) \ \\
  && &= \frac{\lim_{x\rightarrow a} x-a}{\lim_{x\rightarrow a} x + a} &
  \text{ (multiplicativity of limits) } \\
  && &= \frac{0}{2a} = 0
\end{alignat*}

We can see that $\lim_{x \rightarrow a} x - a = 0$ as we have that for $\epsilon
> 0$, $\delta = \epsilon \implies \forall x, 0 < |x - a| < \epsilon$.

Similarly, we see that $\lim_{x \rightarrow a} x + a = 2a$ as we have that for
$\epsilon > 0$, $\delta = \epsilon \implies \forall x, 0 < |x - a| < \epsilon
\implies |x + a - 2a| = |x - a| < \epsilon$. 

\subsection*{Apostol p.138-139 no.21}

\begin{alignat*}{3}
  && \lim_{x\rightarrow 0} \frac{1 - \sqrt{1-x^2}}{x^2} &= \lim_{x\rightarrow 0}
  \frac{1 - \sqrt{1-x^2}}{x^2}(\frac{1 + \sqrt{1-x^2}}{1 + \sqrt{1-x^2}}) &\\
  && &= \lim_{x\rightarrow 0} \frac{1 - (1 - x^2)}{x^2(1 +\sqrt{1-x^2})} &\\
  && &= \lim_{x\rightarrow 0} \frac{x^2}{x^2}\lim_{x\rightarrow 0}\frac{1}{1
    +\sqrt{1-x^2}} & \text{ (multiplicativity of limits) }\\
  && &= \lim_{x\rightarrow 0}\frac{1}{1 + \sqrt{1 -x^2}} & \text{ (lemma) }\\
  && &= \frac{\lim_{x\rightarrow 0} 1}{\lim_{x\rightarrow 0} 1 +
    \lim_{x\rightarrow 0} \sqrt{1 - x^2}} & \text{ (additivity, multiplicativity
    of limits) } \\
  && &= \frac{1}{1 + \lim_{x\rightarrow 0} \sqrt{1 - x^2}} &\\
  && &= \frac{1}{2}
\end{alignat*}

To compute the last limit, note that we have $(\lim_{x\rightarrow 0}
\sqrt{1-x^2})^2 = \lim_{x\rightarrow 0} 1-x^2 = 1$, as for any $\epsilon > 0$,
take $\delta = \sqrt{\epsilon} \implies 0 < |x| < \sqrt{\epsilon} \implies |1 -
x^2 - 1| = |x^2| = |x| < \epsilon$. Thus, $\lim_{x\rightarrow 0} \sqrt{1-x^2} =
\sqrt{1} = 1$.

\subsection*{Apostol p.138-139 no.31}

Consider

\[
  f(x) = \begin{cases}
    x & x \in \R \setminus \Q \\
    0 & x \in \Q
  \end{cases}
\]

\begin{claim}
  This is continuous at 0, but nowhere else.
\end{claim}

\begin{proof}
  At 0, we observe that for any $\epsilon > 0$, the density of $\R \setminus \Q$
  in $\R$ furnishes $\delta \in \R \setminus \Q$ such that $0 < \delta <
  \epsilon$. Then, $\forall x, 0 < |x| < \delta \implies |f(x)| < \delta$, as
  either $x \in \R \setminus \Q \implies |f(x)| = |x| < \delta$ or $x \in \Q
  \implies |f(x)| = 0 < \delta$. Thus, we have that $\lim_{x\rightarrow 0}f(x) =
  0 = f(0)$.

  However, for any $c \in \R, c \neq 0$, we have that $\lim_{x\rightarrow
    c}f(x)$ does not exist. Suppose that it did, and it had value $K$. Take
  $\epsilon = |\frac{c}{2}| > 0$. For any
  $\delta$, pick $x_1 \in \Q, x_2 \in \R \setminus \Q, 0 < |x_1 - c| <\delta, 0
  < |x_2 - c| < \delta$, and consider that the
  existence of the limit has $|f(x_0) - K| = |K| < |\frac{c}{2}|, |f(x_1) - K| =
  |x_1 - K| = |x_1 - K + c - c| \leq |x_1 - c| + |K - c| < |\frac{c}{2}|$.
  However, $|K - c| > |\frac{c}{2}|$ as $|K| < |\frac{c}{2}|$, and so we have
  that both $|K -c| < |\frac{c}{2}|$ and $|K -c| > |\frac{c}{2}|$. $\contra$, so
  the limit does not exist and $f$ cannot be continuous for $x \neq 0$.
\end{proof}

\section*{Problem 1}

\begin{claim}
  $f$ is integrable $\implies |f|$ is integrable.
\end{claim}

\begin{proof}
  Let $f$ be integrable over $[a,b]$.

  We have from the triangle inequality that $|a - b| \leq |a| + |b| \implies |a - b| -|b| \leq |a|$.
  Replacing $a$ with $x - y$ and $b$ with $-y$, we have that $|x| - |y| \leq |x - y|$.

  % Now consider that we now have that $\forall x,y \in [a,b], |f(x)| - |f(y)| \leq |f(x) - f(y)|$.
  Then, over any partition $P = \{x_0,...,x_n\}$ of $[a,b]$, we have that on a open subinterval of
  the partition $(x_i,x_{i+1})$, the theorem on approximation gives $x_1,x_2 \in
  (x_i,x_{i=1}), \epsilon > 0 \mid \sup(|f|) -
  \inf(|f|) - 2\epsilon < |f(x_1)| - |f(x_2)| \leq |f(x_1 - f(x_2)| < |\sup(f) -
  \inf(f)| = \sup(f) - \inf(f)$.

  Then, for that partition and putting $\inf_I, \sup_I$ for the infimum and
  supremum over $I$, we have that
  \begin{align*}
    \sum_{i=0}^n(\inf_{(x_i,x_{i+1})}(f))(x_{i+1} - x_i) \in \underline{I}(f) \\
    \sum_{i=0}^n(\sup_{(x_i,x_{i+1})}(f))(x_{i+1} - x_i) \in \overline{I}(f) \\
    \sum_{i=0}^n(\inf_{(x_i,x_{i+1})}(|f|))(x_{i+1} - x_i) \in \underline{I}(|f|) \\
    \sum_{i=0}^n(\sup_{(x_i,x_{i+1})}(|f|))(x_{i+1} - x_i) \in \overline{I}(|f|) \\
  \end{align*}

  By properties of sums proved on previous homework, we then have that
  \begin{align*}
    \sum_{i=0}^n(\sup_{(x_i,x_{i+1})}(|f|))(x_{i+1} - x_i) &- \sum_{i=0}^n(\inf_{(x_i,x_{i+1})}(|f|))(x_{i+1} - x_i) \\
    &= \sum_{i=0}^n(\sup_{(x_i,x_{i+1})}(|f|) - \inf_{(x_i,x_{i+1})}(|f|))(x_{i+1} - x_i) \\
    &\leq \sum_{i=0}^n(\sup_{(x_i,x_{i+1})}(f) - \inf_{(x_i,x_{i+1})}(f))(x_{i+1} - x_i) \\
    &= \sum_{i=0}^n(\sup_{(x_i,x_{i+1})}(f))(x_{i+1} - x_i) - \sum_{i=0}^n(\inf_{(x_i,x_{i+1})}(f))(x_{i+1} - x_i) \\
  \end{align*}

  Finally, since $f$ is integrable, we have that we can pick a partition such
  that the $RHS < \epsilon$ for any $\epsilon > 0$. Thus, we can find $x \in
  \overline{I}(|f|), y \in \underline{I}(|f|)$ such that $0 \leq x - y <
  \epsilon$, and so $x - y = 0$ by a previous homework result.
\end{proof}

\section*{Problem 2}

\subsection*{a}

\begin{claim}
  $x^n, n \in \Ze$ is monotonic on both $(-\infty, 0]$ and $[0,\infty)$.
\end{claim}

\begin{proof}
  We induct on $n$. The base case is $n = 0 \implies x^n = 1$, and so for any
  $x,y$ in $(-\infty,0]$ or $x,y$ in $[0,\infty)$, we have that $x > y \implies x^0 = y^0 = 1$.

  For the inductive case, suppose that the claim holds for $n = k$. Then,
  $x^{k+1} = x^k \cdot x$.

  For any $x,y$ in $(-\infty, 0]$, we have that if
  $x^k$ is monotonically increasing, then $x > y \implies x^k \geq y^k \implies
  x^k\cdot x < y^k \cdot y$, as in general we have shown $a \geq b, 0 \geq c > d
  \implies ac < bd$ as a property of the ordering. Similarly, if $x^k$ is monotonically decreasing, then $x >
  y \implies x^k \geq y^k \implies x^k \cdot x > y^k \cdot y$.

  
  For any $x,y$ in $[0,\infty)$, we have that if
  $x^k$ is monotonically increasing, then $x > y \implies x^k \geq y^k \implies
  x^k\cdot x > y^k \cdot y$. Similarly, if $x^k$ is monotonically decreasing, then $x >
  y \implies x^k \geq y^k \implies x^k \cdot x < y^k \cdot y$.

  Thus, $x^n$ is monotonic on both $(-\infty,0]$ and $[0,\infty)$.
\end{proof}

\subsection*{b}

\begin{claim}
  All monomials are integrable on any closed interval.
\end{claim}

\begin{proof}
  In the cases that $[a,b] \subseteq(-\infty, 0]$, or $[a,b] \subseteq[0,\infty)$, we have
  that the function is monotonic and bounded (in general, $x^n$ bounded by
  $\max(a^n,b^n)$ over the interval $[a,b]$). This has been shown to be integrable
  in class.

  In the last case that $[a,b] \nsubseteq (-\infty, 0]$ or $[0,\infty)$, and since
  $ [a,b] \subseteq (-\infty, 0] \cup [0,\infty) = \R$, then $[a,b] = [a,0] \cup
  [0,b]$. This can bee seen by the fact that there must be some element in $[a,b]$
  that is greater than zero, and one that is less than zero. Further, since $a,b$
  are bounds on the interval, they must be less and and greater than zero each.

  Then, $\int_a^0x^ndx + \int_0^bx^ndx = \int_a^bx^ndx$, as the two parts on the
  LHS are integrable as they are bounded and monotonic.
\end{proof}

\subsection*{c}

\begin{claim}
  \[
    g(x) = \sum_{i=0}^nc_ix^i 
  \]
  is integrable.
\end{claim}

\begin{proof}
  Note that we have already proved that $f(x) = c$ is integrable, and that the
  sum and product of integrable functions is itself integrable in class.

  We induct on $n$. The base case, $n = 0$, has $g(x) = c_0$, which is
  integrable. Then, if the claim for $n = k$ holds, then $g_{k+1}(x) =
  \sum_{i=0}^{k+1}c_ix^i = \sum_{i=0}^kc_ix^i + c_{k+1}x^{k+1}$. We have that
  $\sum_{i=0}^kc_ix^i$ is integrable by the inductive premise, and that $c_{k+1}$
  and $x^{k+1}$ are both integrable as well. Thus, $ \sum_{i=0}^kc_ix^i +
  c_{k+1}x^{k+1} = g_{k+1}(x)$ is then integrable.

  Induction then yields that $g(x)$ is integrable for any $n$.
\end{proof}


\section*{Problem 3}

\begin{align*}
  &f:(a,b) \rightarrow \R, x \in (a,b) \\
  a) \ &\lim_{h\rightarrow 0}|f(x + h) - f(x)| = 0 \\
  b) \ &\lim_{h\rightarrow 0}|f(x + h) - f(x - h)| = 0
\end{align*}

\begin{claim}
  $a) \implies b)$.
\end{claim}

\begin{proof}
  We have that for any $\epsilon > 0$, $\exists \delta \mid 0 < |h| < \delta
  \implies |f(x+h) - f(x)| < \epsilon$.

  We want to show that also $\lim_{h\rightarrow 0}|f(x + h) - f(x)| = 0 \implies
  \lim_{h\rightarrow 0}|f(x - h) -f(x)| = 0$. Consider that for $\epsilon > 0$,
  we can take the same $\delta$ as before. For $0 < |h| < \delta$ note that for
  any given $h$, we have that $f(x-h) =
  f(x+(-h))$, but $|-h| < \delta \implies |f(x + (-h)) - f(x)| < \epsilon$.
  
  Further, for any $\epsilon' = 2\epsilon > 0$, we have that $|f(x + h) - f(x - h)| = |f(x +
  h) - f(x - h) + f(x) - f(x)| \leq |f(x + h) - f(x)| + |f(x -h) - f(x)| <
  2\epsilon = \epsilon'$. Thus, $\lim_{h\rightarrow 0}|f(x+h)-f(x-h)| = 0$.
\end{proof}

We do not in fact that $b) \implies a)$. Consider the following function
$f:(-1,1) \rightarrow \R$:

\[
  f(x) = \begin{cases}
    1 & x \neq 0 \\
    0 & x = 0 \\
  \end{cases}
\]

Then, take $x = 0$. $\lim_{h\rightarrow 0}|f(h) - f(-h)| = 0$, as we have that
for $\forall \epsilon > 0, \delta = 1 \implies \forall h, 0 < |h| < 1 \implies |f(h) -
f(-h) - 0| = |1-1 - 0| = 0 < \epsilon$.

However, we have that $\lim_{h\rightarrow 0}|f(h) - f(0)| = \lim_{h\rightarrow
  0}|f(h)| = 1$, as $\forall \epsilon > 0, \delta = 1 \implies \forall h, 0 <
|h| < 1 \implies |f(h) - 1| = 0 < \epsilon$. Thus, $\lim_{h\rightarrow
  0}|f(0+h) - f(0)| \neq 0$.

\section*{Problem 4}

\[
  f(x) = \begin{cases}
    1 & x \in \R \setminus \Q \\
    \frac{1}{n} & x = \frac{m}{n} \mid m, n \in \Zg, (m,n) = 1 
  \end{cases}
\]

\subsection*{a}

\begin{claim}
  $f$ is continuous at $x$ if and only if $x$ is irrational.
\end{claim}

\begin{proof}
  $(\implies)$ Suppose that $x$ is not irrational. Then, since $f$ is continuous
  at $x$, we must have that 
\end{proof}

\subsection*{b}

\end{document}

% LocalWords:  NetID fancyplain LocalWords colorlinks linkcolor linkbordercolor
