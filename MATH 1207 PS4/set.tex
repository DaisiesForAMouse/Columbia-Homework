\documentclass[12pt,letterpaper]{article}
\usepackage{fullpage}
\usepackage[top=2cm, bottom=4.5cm, left=2.5cm, right=2.5cm]{geometry}
\usepackage{amsmath,amsthm,amsfonts,amssymb,amscd}
\usepackage{lastpage}
\usepackage{enumerate}
\usepackage{fancyhdr}
\usepackage{mathrsfs}
\usepackage{xcolor}
\usepackage{graphicx}
\usepackage{listings}
\usepackage{hyperref}
\usepackage{tikz}
\usepackage{relsize}
\usepackage{fancyvrb}
\usetikzlibrary{shapes.geometric,fit}

\hypersetup{%
  colorlinks=true,
  linkcolor=blue,
  linkbordercolor={0 0 1}
}

\setlength{\parindent}{0.0in}
\setlength{\parskip}{0.05in}

% Edit these as appropriate
\newcommand\course{MATH 1207}
\newcommand\hwnumber{3}                  % <-- homework number
\newcommand\NetIDa{dc3451}           % <-- NetID of person #1
\newcommand\NetIDb{David Chen}           % <-- NetID of person #2 (Comment this line out for problem sets)

\theoremstyle{definition}
\newtheorem*{statement}{Statement}
\newtheorem*{claim}{Claim}
\newtheorem*{theorem}{Theorem}

\newcommand{\contra}{\Rightarrow\!\Leftarrow}
\newcommand{\R}{\mathbb{R}}
\newcommand{\F}{\mathbb{F}}
\newcommand{\Z}{\mathbb{Z}}
\newcommand{\Ze}{\mathbb{Z}_{\geq 0}}
\newcommand{\Zg}{\mathbb{Z}_{>0}}
\newcommand{\N}{\mathbb{N}}
\newcommand{\Q}{\mathbb{Q}}
\newcommand{\C}{\mathbb{C}}

\pagestyle{fancyplain}
\headheight 35pt
\lhead{\NetIDa}
\lhead{\NetIDa\\\NetIDb}                 % <-- Comment this line out for problem sets (make sure you are person #1)
\chead{\textbf{\Large Homework \hwnumber}}
\rhead{\course \\ \today}
\lfoot{}
\cfoot{}
\rfoot{\small\thepage}
\headsep 1.5em

\begin{document}



\subsection*{Apostol p.28 no.1}

\begin{claim}
  For $x,y \in \R, x < y \implies \exists z \in \R \mid x < z < y$.
\end{claim}

\begin{proof}
  Consider $z = \frac{x}{2} + \frac{y}{2}$. We have $z \in \R$ as $\R$ is closed under
  addition and multiplication as it is a field under those operations.

  Further, we have that $x \leq y \implies \frac{x}{2} < \frac{y}{2} \implies
  \frac{x}{2} + \frac{x}{2} < \frac{x}{2} + \frac{y}{2}, \frac{x}{2} +
  \frac{y}{2} < \frac{y}{2} + \frac{y}{2}$.

  As $\forall a \in \R, \frac{a}{2} + \frac{a}{2} = \frac{1}{2}(a+a) =
  \frac{1}{2}(2a) = a$, we have that $x < \frac{x}{2} + \frac{y}{2} < y$.
\end{proof}

\subsection*{Apostol p.28 no.3}

\begin{claim}
  For $x \in \R, x>0 \implies \exists n \in \Zg \mid \frac{1}{n} < x$.
\end{claim}

\begin{proof}
  The Archimedian property of the reals furnishes an $n \in \Zg \mid nx > 1$.
  Then, we see that $nx > 1 \implies 1 < nx \implies n^{-1}(1) < n^{-1}(nx)
  \implies \frac{1}{n} < x$. 
\end{proof}

\subsection*{Apostol p.28 no.4}

\begin{claim}
  For $x \in \R, \exists ! n \in \Z \mid n < x < n+1$.
\end{claim}

\begin{proof}
  We will first show existence. Consider $S := \{n \in \Z \mid n \leq x\}$. This
  must be nonempty, or else $x$ would be a lower bound to $\Z$, as $\neg \exists
  n \in \Z \mid n \leq x \implies \forall n \in \Z, \neg(n \leq x) \implies
  \forall n \in \Z, x \leq n$.

  Now, note that if $x$ is a lower bound for $\Z$, then $-x$ is an upper bound
  for $\Z$. This follows as $x \leq n \implies -x \geq -n$, but as $n \in \Z \implies
  -n \in \Z$, we have that $\forall n \in \Z, x \leq n \implies \forall n \in
  \Z, x \leq -n \implies \forall n \in \Z, -x \geq -(-n) = n$.

  However, we proved that $\Ze \subset \Z$ has no upper bound, meaning that $-x$
  cannot be an upper bound of $\Z$. Thus, $S$ must be nonempty.

  Now, the approximation theorem proved in class furnishes $n \in S \mid
  \sup(S) - 1 < n$. Thus, since we have $\sup(S) - 1 < n \implies \sup(S) < n +
  1 \implies n + 1 \notin S$, and by definition of $S$, $n \in S \implies n \leq x$
  and $n+1 \notin S \implies \neg(n + 1 \leq x) \implies x < n + 1 \implies n
  \leq x < n+1$.

  We will now show uniqueness: suppose that $\exists n, n' \in \Z \mid n \neq n', n \leq x
  < n+1, n' \leq x < n'+1$. $n' > n \implies n' \geq n + 1 > x$. However, $n' <
  n$, then we have that $n \geq n' + 1 > x$. Either way, we have $\contra$, so
  $n = n'$.

  The above relies on the fact that $a,b \in Z, a > b \implies a \geq b + 1$.
  This follows from $a > b \implies a - b > 0$, and as $a - b \in \Z$,
  the fact that there is no integer between $0$ and $1$ (proved in an earlier
  homework) allows that $a - b = 1$ or $a - b > 1$ by trichotomy. However, this
  means that $a-b \geq 1 \implies a \geq b + 1$.
\end{proof}

\subsection*{Apostol p.28 no.6}

\begin{claim}
  $\Q$ is dense in $\R$.
\end{claim}

\begin{proof}
  We shall start by proving at for $x,y \in \R, x < y, \exists r \in \Q \mid x <
  r < y$. The Archimedian property furnishes $n \in \Zg \mid n(y-x) > 1 \implies
  ny > nx + 1$. Now consider $[nx]$. We have that $[nx] \leq nx \implies [nx] +
  1 \leq nx + 1 < ny$, and also $nx < [nx] + 1$.

  These together yield that 
  \begin{align*}
    & nx < [nx] + 1 \leq nx + 1 < ny \\
    \implies& n^{-1}(nx) < n^{-1}([nx] +1) \leq n^{-1}(nx + 1) < n^{-1}(ny) \\
    \implies& x < \frac{[nx] + 1}{n} < y \\
  \end{align*}

  Critically, $[nx] \in \Z$, meaning that as $[nx] + 1, n \in \Z$, we have
  $\frac{[nx] + 1}{n} \in \Q$.

  Now that we have one such $r$, we can construct infinitely many: simply use
  the above process to find $r'$ such that $r < r' < y$. This can be repeated ad infinitum.
\end{proof}

\subsection*{Apostol p.64 no.4b}

\begin{claim}
  \[
    [-x] = \begin{cases}
      -[x] & x \in \Z \\
      -[x] - 1 & x \notin \Z
    \end{cases}
  \]
\end{claim}

\begin{proof}
  Note that if we find one such $a$ such that $a \leq x < a + 1$, we have that
  $[x] = a$ as we have shown previously that such an $a$ must be unique.

  Suppose that $x \in \Z$. Then we have that $-x \leq -x < -x + 1$, and so $[-x]
  = -x$.

  Otherwise, we have that $[x] \leq x$. However, we have that as $[x] \in \Z,
  x \notin \Z$, $[x] < x$. This then provides that $-[x] > -x$. Further, $x < [x]
  + 1 \implies -x > -([x] + 1) = -[x] - 1$.

  These together give us $-[x]-1<-x<-[x] \implies -[x] -1 \leq -x < -[x]
  \implies [-x] = -[x]-1$.
\end{proof}

\subsection*{Apostol p.64 no.4d}

\begin{claim}
  $[2x] = [x] + [x + \frac{1}{2}]$
\end{claim}

\begin{proof}
  Consider $[x] + \frac{1}{2}$. We have that by trichotomy, exactly one of
  $x < y, x = y, x > y$ is true.

  If $x < [x]+\frac{1}{2}$, then $[x] \leq x < [x] + \frac{1}{2} \implies [x]
  < x + \frac{1}{2} < [x] + \frac{1}{2} + \frac{1}{2} = [x] + 1 \implies [x +
  \frac{1}{2}] = [x]$. Further, $2[x] \leq 2x < 2([x] + \frac{1}{2}) = 2[x] + 1 \implies [2x]
  = 2[x] = [x] + [x] = [x] + [x + \frac{1}{2}]$.

  If $x = [x]+\frac{1}{2}$, then $x + \frac{1}{2} = [x] + \frac{1}{2} + \frac{1}{2} = [x] +
  1$, and $[x] + 1 \in Z \implies [x] + 1 \leq [x] + 1 < [x] + 2 \implies [x +
  \frac{1}{2}] = [[x] + 1] = [x] + 1$. Further, $2x = 2([x] + \frac{1}{2}) = 2[x]
  + 1 = [x] + [x] + 1 = [x] + [x + \frac{1}{2}]$.

  If $x > [x]+\frac{1}{2}$, then $[x] + \frac{1}{2} \leq x < [x] + 1 \implies [x] +
  \frac{1}{2} + \frac{1}{2} \leq x + \frac{1}{2} < [x] + 1 + \frac{1}{2}
  \implies [x] + 1 \leq x + \frac{1}{2} < [x] + 2 \implies [x + \frac{1}{2}] =
  [x] + 1$. Further, $2x > 2([x] + \frac{1}{2}) = 2[x] + 1$, and $x < [x] + 1
  \implies 2x < 2[x] + 2 \implies 2[x] + 1 < 2x < 2[x] + 2 \implies [2x] = 2[x]
  + 1 = [x] + [x] + 1 = [x] + [x + \frac{1}{2}]$.
\end{proof}

\section*{Problem 1}
\section*{Problem 2}
\section*{Problem 3}
\section*{Problem 4}

\end{document}

% LocalWords:  NetID fancyplain LocalWords colorlinks linkcolor linkbordercolor