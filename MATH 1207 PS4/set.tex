\documentclass[12pt,letterpaper]{article}
\usepackage{fullpage}
\usepackage[top=2cm, bottom=4.5cm, left=2.5cm, right=2.5cm]{geometry}
\usepackage{amsmath,amsthm,amsfonts,amssymb,amscd}
\usepackage{lastpage}
\usepackage{enumerate}
\usepackage{fancyhdr}
\usepackage{mathrsfs}
\usepackage{xcolor}
\usepackage{graphicx}
\usepackage{listings}
\usepackage{hyperref}
\usepackage{tikz}
\usepackage{relsize}
\usepackage{fancyvrb}
\usetikzlibrary{shapes.geometric,fit}

\hypersetup{%
  colorlinks=true,
  linkcolor=blue,
  linkbordercolor={0 0 1}
}

\setlength{\parindent}{0.0in}
\setlength{\parskip}{0.05in}

% Edit these as appropriate
\newcommand\course{MATH 1207}
\newcommand\hwnumber{3}                  % <-- homework number
\newcommand\NetIDa{dc3451}           % <-- NetID of person #1
\newcommand\NetIDb{David Chen}           % <-- NetID of person #2 (Comment this line out for problem sets)

\theoremstyle{definition}
\newtheorem*{statement}{Statement}
\newtheorem*{claim}{Claim}
\newtheorem*{theorem}{Theorem}

\newcommand{\contra}{\Rightarrow\!\Leftarrow}
\newcommand{\R}{\mathbb{R}}
\newcommand{\F}{\mathbb{F}}
\newcommand{\Z}{\mathbb{Z}}
\newcommand{\Ze}{\mathbb{Z}_{\geq 0}}
\newcommand{\Zg}{\mathbb{Z}_{>0}}
\newcommand{\N}{\mathbb{N}}
\newcommand{\Q}{\mathbb{Q}}
\newcommand{\C}{\mathbb{C}}

\pagestyle{fancyplain}
\headheight 35pt
\lhead{\NetIDa}
\lhead{\NetIDa\\\NetIDb}                 % <-- Comment this line out for problem sets (make sure you are person #1)
\chead{\textbf{\Large Homework \hwnumber}}
\rhead{\course \\ \today}
\lfoot{}
\cfoot{}
\rfoot{\small\thepage}
\headsep 1.5em

\begin{document}



\subsection*{Apostol p.28 no.1}

\begin{claim}
  For $x,y \in \R, x < y \implies \exists z \in \R \mid x < z < y$.
\end{claim}

\begin{proof}
  Consider $z = \frac{x}{2} + \frac{y}{2}$. We have $z \in \R$ as $\R$ is closed under
  addition and multiplication as it is a field under those operations.

  Further, we have that $x \leq y \implies \frac{x}{2} < \frac{y}{2} \implies
  \frac{x}{2} + \frac{x}{2} < \frac{x}{2} + \frac{y}{2}, \frac{x}{2} +
  \frac{y}{2} < \frac{y}{2} + \frac{y}{2}$.

  As $\forall a \in \R, \frac{a}{2} + \frac{a}{2} = \frac{1}{2}(a+a) =
  \frac{1}{2}(2a) = a$, we have that $x < \frac{x}{2} + \frac{y}{2} < y$.
\end{proof}

\subsection*{Apostol p.28 no.3}

\begin{claim}
  For $x \in \R, x>0 \implies \exists n \in \Zg \mid \frac{1}{n} < x$.
\end{claim}

\begin{proof}
  The Archimedian property of the reals furnishes an $n \in \Zg \mid nx > 1$.
  Then, we see that $nx > 1 \implies 1 < nx \implies n^{-1}(1) < n^{-1}(nx)
  \implies \frac{1}{n} < x$. 
\end{proof}

\subsection*{Apostol p.28 no.4}

\begin{claim}
  For $x \in \R, \exists ! n \in \Z \mid n < x < n+1$.
\end{claim}

\begin{proof}
  We will first show existence. Consider $S = \{n \in \Z \mid n \leq x\}$. This
  must be nonempty, or else $x$ would be a lower bound to $\Z$, as $\neg \exists
  n \in \Z \mid n \leq x \implies \forall n \in \Z, \neg(n \leq x) \implies
  \forall n \in \Z, x \leq n$.

  Now, note that if $x$ is a lower bound for $\Z$, then $-x$ is an upper bound
  for $\Z$. This follows as $x \leq n \implies -x \geq -n$, but as $n \in \Z \implies
  -n \in \Z$, we have that $\forall n \in \Z, x \leq n \implies \forall n \in
  \Z, x \leq -n \implies \forall n \in \Z, -x \geq -(-n) = n$.

  However, we proved that $\Ze \subset \Z$ has no upper bound, meaning that $-x$
  cannot be an upper bound of $\Z$. Thus, $S$ must be nonempty.

  Now, the approximation theorem proved in class furnishes $n \in S \mid
  \sup(S) - 1 < n$. Thus, since we have $\sup(S) - 1 < n \implies \sup(S) < n +
  1 \implies n + 1 \notin S$, and by definition of $S$, $n \in S \implies n \leq x$
  and $n+1 \notin S \implies \neg(n + 1 \leq x) \implies x < n + 1 \implies n
  \leq x < n+1$.

  We will now show uniqueness: suppose that $\exists n, n' \in \Z \mid n \neq n', n \leq x
  < n+1, n' \leq x < n'+1$. $n' > n \implies n' \geq n + 1 > x$. However, $n' <
  n$, then we have that $n \geq n' + 1 > x$. Either way, we have $\contra$, so
  $n = n'$.

  The above relies on the fact that $a,b \in \Z, a > b \implies a \geq b + 1$.
  This follows from $a > b \implies a - b > 0$, and as $a - b \in \Z$,
  the fact that there is no integer between $0$ and $1$ (proved in an earlier
  homework) allows that $a - b = 1$ or $a - b > 1$ by trichotomy. However, this
  means that $a-b \geq 1 \implies a \geq b + 1$.
\end{proof}

\subsection*{Apostol p.28 no.6}

\begin{claim}
  $\Q$ is dense in $\R$.
\end{claim}

\begin{proof}
  We shall start by proving at for $x,y \in \R, x < y, \exists r \in \Q \mid x <
  r < y$. The Archimedian property furnishes $n \in \Zg \mid n(y-x) > 1 \implies
  ny > nx + 1$. Now consider $[nx]$. We have that $[nx] \leq nx \implies [nx] +
  1 \leq nx + 1 < ny$, and also $nx < [nx] + 1$.

  These together yield that 
  \begin{align*}
    & nx < [nx] + 1 \leq nx + 1 < ny \\
    \implies& n^{-1}(nx) < n^{-1}([nx] +1) \leq n^{-1}(nx + 1) < n^{-1}(ny) \\
    \implies& x < \frac{[nx] + 1}{n} < y \\
  \end{align*}

  Critically, $[nx] \in \Z$, meaning that as $[nx] + 1, n \in \Z$, we have
  $\frac{[nx] + 1}{n} \in \Q$.

  Now that we have one such $r$, we can construct infinitely many: simply use
  the above process to find $r'$ such that $r < r' < y$. This can be repeated ad infinitum.
\end{proof}

\subsection*{Apostol p.64 no.4b}

\begin{claim}
  \[
    [-x] = \begin{cases}
      -[x] & x \in \Z \\
      -[x] - 1 & x \notin \Z
    \end{cases}
  \]
\end{claim}

\begin{proof}
  Note that if we find one such $a$ such that $a \leq x < a + 1$, we have that
  $[x] = a$ as we have shown previously that such an $a$ must be unique.

  Suppose that $x \in \Z$. Then we have that $-x \leq -x < -x + 1$, and so $[-x]
  = -x$.

  Otherwise, we have that $[x] \leq x$. However, we have that as $[x] \in \Z,
  x \notin \Z$, $[x] < x$. This then provides that $-[x] > -x$. Further, $x < [x]
  + 1 \implies -x > -([x] + 1) = -[x] - 1$.

  These together give us $-[x]-1<-x<-[x] \implies -[x] -1 \leq -x < -[x]
  \implies [-x] = -[x]-1$.
\end{proof}

\subsection*{Apostol p.64 no.4d}

\begin{claim}
  $[2x] = [x] + [x + \frac{1}{2}]$
\end{claim}

\begin{proof}
  Consider $[x] + \frac{1}{2}$. We have that by trichotomy, exactly one of
  $x < y, x = y, x > y$ is true.

  If $x < [x]+\frac{1}{2}$, then $[x] \leq x < [x] + \frac{1}{2} \implies [x]
  < x + \frac{1}{2} < [x] + \frac{1}{2} + \frac{1}{2} = [x] + 1 \implies [x +
  \frac{1}{2}] = [x]$. Further, $2[x] \leq 2x < 2([x] + \frac{1}{2}) = 2[x] + 1 \implies [2x]
  = 2[x] = [x] + [x] = [x] + [x + \frac{1}{2}]$.

  If $x = [x]+\frac{1}{2}$, then $x + \frac{1}{2} = [x] + \frac{1}{2} + \frac{1}{2} = [x] +
  1$, and $[x] + 1 \in Z \implies [x] + 1 \leq [x] + 1 < [x] + 2 \implies [x +
  \frac{1}{2}] = [[x] + 1] = [x] + 1$. Further, $2x = 2([x] + \frac{1}{2}) = 2[x]
  + 1 = [x] + [x] + 1 = [x] + [x + \frac{1}{2}]$.

  If $x > [x]+\frac{1}{2}$, then $[x] + \frac{1}{2} \leq x < [x] + 1 \implies [x] +
  \frac{1}{2} + \frac{1}{2} \leq x + \frac{1}{2} < [x] + 1 + \frac{1}{2}
  \implies [x] + 1 \leq x + \frac{1}{2} < [x] + 2 \implies [x + \frac{1}{2}] =
  [x] + 1$. Further, $2x > 2([x] + \frac{1}{2}) = 2[x] + 1$, and $x < [x] + 1
  \implies 2x < 2[x] + 2 \implies 2[x] + 1 < 2x < 2[x] + 2 \implies [2x] = 2[x]
  + 1 = [x] + [x] + 1 = [x] + [x + \frac{1}{2}]$.

  (What an awful proof)
\end{proof}

\section*{Problem 1}

Suppose $S \subseteq \R, c \in \R$. Let $cS = \{cx \mid x \in S\}$.

\subsubsection*{a)}

\begin{claim}
  If $c > 0$ and $S$ is bounded above, then $cS$ is also bounded above.
\end{claim}

\begin{proof}
  Let $r \in \R$ be an upper bound of $S$. Then $\forall s \in S, s \leq r \implies
  \forall s \in S, cs \leq cr$. However, for any element $t \in cS$, we have that
  $\exists s \in S \mid t = cs$. This means that for any element $t \in cS$, we
  have that $t = cs \leq cr$, so $cr$ is an upper bound on $cS$.
\end{proof}

\subsubsection*{b)}

\begin{claim}
  If $c > 0$, then $\sup(cS) = c\sup(S)$.
\end{claim}

\begin{proof}
  (I use problem 2 in this proof freely, as that proof does not rely on this one.)

  We will first show that if one exists only if the other exists. Suppose that
  $\sup(cS)$ exists. Then, we can see that for any element
  $t \in cS$ we have $t = cs$ for some $s \in S$, meaning that $\forall t \in
  cS, \sup(cS) \geq t \implies \forall s \in S, \sup(cS) \geq cs \implies \forall s
  \in S, c^{-1}\sup(cS) \geq s$, so $c^{-1}\sup(cS)$ in particular is an upper
  bound for $S$.

  Suppose now $\sup(S)$ exists. Then we can see for any $t \in cS$, we have $t =
  cs$ for some $s \in S$, such that $\forall s \in S, \sup(S) \geq s \implies
  \forall s \in S, c\sup(S) \geq cs \implies \forall t \in cS, c\sup(S) \geq t$, so
  $c\sup(S)$ in particular is an upper bound for $cS$.

  We will now show that $\sup(cS)$ is exactly $c\sup(S)$.  Now suppose
  that $c\sup(S)$ is not the least upper bound of $cS$, meaning that
  $\exists \epsilon > 0 \mid \sup(cS) + \epsilon < c\sup(S)$. By
  approximation theorem, we have another $\epsilon' > 0 \mid s + \epsilon >
  \sup(S)$ for some $s \in S$. Then, we have that $c(s + \epsilon') = cs +
  c\epsilon' > c\sup(S)$. However, $\sup(cS) > cs,$ as $cs \in cS$, so we have
  that $\sup(cS) + c\epsilon' > cs + c\epsilon > c\sup(S)$. However, since we
  can take $\epsilon, \epsilon'$ to be any two positive reals, we can take $\epsilon'
  = c^{-1}\epsilon$, such that we have $\sup(cS) + \epsilon > c\sup(S)$ as well
  as $\sup(cS) + \epsilon < c\sup(S)$. This violates trichotomy, so $\contra$
  and thus $\sup(cS) = c\sup(S)$.
\end{proof}

\subsubsection*{c)}

Take $c = -1, S = (0,1)$. Clearly $cS = (-1,0), \sup(S) = 1, \sup(cS) = 0$, and
so $c\sup(S) = -1(1) = -1 \neq \sup(cS)$. In fact, if $c<0$, then $\sup(cS) =
c\inf(S)$. This is most clearly seen by noticing that multiplying by $c < 0$
swaps the order, so the infimum gets mapped to the supremum of the new set and
vice versa.

\section*{Problem 2}

\begin{claim}
  Suppose $S \subseteq \R, t \in \R$. $t = \sup(S) \iff \forall s \in S, t \geq
  s$, and $\forall \epsilon > 0, \exists x \in S \mid x > t - \epsilon$.
\end{claim}

\begin{proof}
  ($\implies$) $t = \sup(S)$ implies that $t$ is an upper bound of $S$, as
  $\sup(S)$ is an upper bound of $S$ by definition. The rest follows from the
  approximation theorem exactly, which can be proved as follows: 

  We proceed via contradiction. Suppose that $\exists \epsilon \mid \forall x \sup(S) - \epsilon \geq x$.
  Then $\sup(S) - \epsilon$ is an upper bound for $S$.
  By definition of $\sup$, we have the statement $\sup(S) < \sup (S) - \epsilon$
  but as $\epsilon > 0$, $\contra$

  ($\impliedby$) The first half establishes $t$ as an upper bound. Further,
  suppose that $t$ is not the least upper bound, such that $\exists t'  \in \R \mid
  \forall s \in S, t' \geq s, t' < t$. However, $t' < t \implies t - t' > 0$,
  meaning that we have for $\epsilon = \frac{t - t'}{2}$, we have that $\forall x \in
  S, t - \frac{t - t'}{2} = \frac{t}{2} - \frac{t'}{2} > \frac{t'}{2} +
  \frac{t'}{2} = t' > x$. $\contra$, as thus there is no $x \in S$
  that can satisfy the premise without violating trichotomy, so $t$ is the least upper bound.
\end{proof}

\section*{Problem 3}

\begin{claim}
  Suppose $S,T \subseteq R$, both nonempty and bounded above, with a bijective
  function $f: S \rightarrow T$ such that $\forall x \in S, x \geq f(x)$. Then
  $\sup(S) \geq \sup(T)$.
\end{claim}

  % First, we must note a property of bijections where we can create another
  % bijection by swapping the outputs of two arguments. More specifically, if $f:
  % S \rightarrow T$ is a bijection, then for any $a,b \in S$,

  % \[
  %   g(x) = \begin{cases}
  %     f(x) & x \neq a,b \\
  %     f(a) & x = b\\
  %     f(b) & x = a\\
  %   \end{cases}
  % \]

  % is also bijective. This is clearly injective as all three cases are injective
  % and have disjoint domains, and also clearly surjective as the first case
  % covers $T \setminus \{a,b\}$, and the second case covers $\{f(a)\}, \{f(b)\}$
  % respectively. Note that this does not require $a$ distinct from $b$, as taking
  % them equal yields $f = g$, still a bijection.

\begin{proof}
  Now, suppose that $\sup(S) < \sup(T)$. Approximation furnishes $t \in T$ such
  that for arbitrary $\epsilon > 0, \sup(T) - \epsilon < t$. Further, we have
  that $\sup(S) < \sup(T) \implies \sup(T) - \sup(S) > 0$. Taking $\epsilon
  = \sup(T) - \sup(S)$, we see that $\exists t \in T \mid t > \sup(T) - (\sup(T)
  - \sup(S)) = \sup(S)$. Thus, we have, as $f$ is surjective, that $\exists s \in S \mid t = f(s) > \sup(S)
  \geq s$. However, we have that $\forall s \in S, s \geq f(s)$. $\contra$, so
  $\sup(S) > \sup(T)$.
\end{proof}

We can't conclude that $x > f(x) \implies \sup(S) > \sup(T)$, as the above
reasoning fails in that we can only say, after assuming the opposite of
$\sup(S) \geq \sup(T)$, that $\sup(T) - \sup(S) \geq 0$. Then, we cannot
reason with $\epsilon = \sup(T) - \sup(S) > 0$, as it  is possible $\sup(T) -
\sup(S) = 0$.

An actual example is that $f: (\frac{1}{2},1) \rightarrow (0,1)$, where $f(x)
= 2x - 1$ is a bijection such that $\forall x \in (\frac{1}{2}, 1), x > f(x)$,
as $x > 2x - 1  \iff x < 1$. However, $\sup((\frac{1}{2}, 1)) = \sup((0,1)) = 1$.
  
\begin{claim}
  However, we can claim that $\sup(T) \notin T$.
\end{claim}
  
\begin{proof}
  Note that otherwise we could take the above $\epsilon = 0$, as $\sup(T)$ is
  exactly  the element in $T$ such that $\sup(T) - 0 = \sup(T) \in T$. Then,
  moving in the same line of reasoning as the original proof, we have that
  $\forall s \in S, s < f(s)$ and $\exists s \in S \mid f(s) \geq s$. $\contra$.
\end{proof}

\section*{Problem 4}

\subsubsection*{a)}

\begin{claim}
  For $S = \{\frac{n-1}{n} \mid n \in \Zg\}$, $\sup(S) = 1$.
\end{claim}

First, we have that $\frac{n-1}{n} < \frac{n-1}{n} + \frac{1}{n} =
\frac{n-1+1}{n} = 1$. This holds as $\frac{1}{n} > 0 \impliedby n \in \Zg$.
Thus, $1$ is an upper bound on $S$.

Now, for any $\epsilon > 0$, we have that the Archimedian property furnishes an
$n \in \Zg$ such that $n\epsilon > 1$. Then, $\exists n \in \Zg \mid \epsilon >
\frac{1}{n} \implies -\epsilon < -\frac{1}{n} \implies 1 - \epsilon < 1 -
\frac{1}{n} = \frac{n-1}{n} \in S$. 

We use problem 2 here to show that since $1$ is an upper bound for $S =
\{\frac{n-1}{n} \mid n \in \Zg\}$ and $\forall \epsilon > 0, \exists x \in
S \mid x > 1 - \epsilon$, we can conclude $\sup(S) = 1$.

\subsubsection*{b)}

\begin{claim}
  For $S = \{\frac{n+1}{n} \mid n \in \Zg\}$, $\inf(S) = 1$.
\end{claim}

\begin{proof}
  We first must show that for $S \subseteq \R, t \in \R$, $\forall s \in S, t
  \leq s$, and $\forall \epsilon > 0, \exists x \in S \mid x  > t + \epsilon \implies t = \inf(S)$.

  The first half establishes $t$ as an lower bound. Further,
  suppose that $t$ is not the greatest lower bound, such that $\exists t' \in \R \mid
  \forall s \in S, t' \geq s, t' > t$. However, $t' > t \implies t' - t > 0$,
  meaning that we have for $\epsilon = \frac{t' - t}{2}$, we have that $\forall x \in
  S, t + \frac{t' - t}{2} = \frac{t}{2} + \frac{t'}{2} < \frac{t'}{2} +
  \frac{t'}{2} = t' < x$. $\contra$, thus there is no $x \in S$ that can satisfy
  the premise without violating trichotomy, so $t$ is the greatest lower bound.

  First, we have that $\frac{n+1}{n} > \frac{n+1}{n} - \frac{1}{n} =
  \frac{n-1+1}{n} = 1$. This holds as $-\frac{1}{n} < 0 \impliedby -n < 0
  \impliedby n \in \Zg$. Thus, $1$ is an lower bound on $S$.

  Now, for any $\epsilon > 0$, we have that the Archimedian property furnishes an
  $n \in \Zg$ such that $n\epsilon > 1$. Then, $\exists n \in \Zg \mid \epsilon >
  \frac{1}{n} \implies 1 + \epsilon < 1 + \frac{1}{n} = \frac{n+1}{n} \in S$. 

  Thus, $1$ must be the greatest lower bound of $S$.
\end{proof}

\end{document}

% LocalWords:  NetID fancyplain LocalWords colorlinks linkcolor linkbordercolor