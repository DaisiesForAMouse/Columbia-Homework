
\documentclass[12pt,letterpaper]{article}
\usepackage{fullpage}
\usepackage[top=2cm, bottom=4.5cm, left=2.5cm, right=2.5cm]{geometry}
\usepackage{amsmath,amsthm,amsfonts,amssymb,amscd}
\usepackage{lastpage}
\usepackage{enumerate}
\usepackage{fancyhdr}
\usepackage{mathrsfs}
\usepackage{xcolor}
\usepackage{graphicx}
\usepackage{listings}
\usepackage{hyperref}
\usepackage{tikz}
\usepackage{xfrac}
\usepackage{nicefrac}
\usepackage{xcolor}
\allowdisplaybreaks


\usetikzlibrary{shapes.geometric,fit}
\usetikzlibrary{patterns}

\hypersetup{
  colorlinks=true,
  linkcolor=blue,
  linkbordercolor={0 0 1}
}

\setlength{\parindent}{0.0in}
\setlength{\parskip}{0.05in}

\newcommand\course{ECON 3213}
\newcommand\hwnumber{1}
\newcommand\NetIDa{dc3451}
\newcommand\NetIDb{David Chen}

\newcommand\R{\mathbb{R}}

\theoremstyle{definition}
\newtheorem*{statement}{Statement}
\newtheorem*{claim}{Claim}
\newtheorem*{theorem}{Theorem}

\newcommand{\contra}{\Rightarrow\!\Leftarrow}
\newcommand{\Lag}{\mathcal{L}}

\pagestyle{fancyplain}
\headheight 35pt
\lhead{\NetIDa}
\lhead{\NetIDa\\\NetIDb}
\chead{\textbf{\Large Problem Set \hwnumber}}
\rhead{\course \\ \today}
\lfoot{}
\cfoot{}
\rfoot{\small\thepage}
\headsep 1.5em

\begin{document}

The following \verb!R! code was used to compute the solutions. 

\begin{scriptsize}
\begin{verbatim}
growth <- function(start, end, first, last) {
  return ((end / start) ^ (1 / (last - first)))
}

present_value <- function(start, end, first, last, years_later, discount) {
  g <- growth(start, end, first, last)
  years <- last - first
  value <- 0
  for (i in 0:years) {
    value <- (value + (start * (g ^ i) / ((1 + discount) ^ (i + years_later))))
  }
  return (value)
}

project <- function(start, years, growth) {
  return ((start) * (growth ^ years))
}

"Problem 2"
cat("Irish growth is ", (growth(1775, 2736, 1870, 1913) - 1) * 100, "%.\n")
cat("American growth is ", (growth(2445, 5301, 1870, 1913) - 1) * 100, "%.\n")
cat("Argentenian growth is ", (growth(1311, 3797, 1870, 1913) - 1) * 100, "%.\n")

"Problem 3"
cat("Irish present value is ", present_value(1775, 2736, 1870, 1913, 0, 0.03), ".\n")
cat("American present value is ", present_value(2445, 5301, 1870, 1913, 0, 0.03) , ".\n")
cat("Argentenian present value is ", present_value(1311, 3797, 1870, 1913, 0, 0.03), ".\n")

"Problem 4"
ire_proj <- project(2736, 1929 - 1913, growth(1775, 2736, 1870, 1913))
ame_proj <- project(5301, 1929 - 1913, growth(2445, 5301, 1870, 1913))
arg_proj <- project(3797, 1929 - 1913, growth(1311, 3797, 1870, 1913))
cat("Projected rGDP for Ireland is ", ire_proj, ".\n")
cat("Projected rGDP for America is ", ame_proj, ".\n")
cat("Projected rGDP for Argentina is ", arg_proj, ".\n")

"Problem 5"
cat("Irish present value is ", present_value(2736, ire_proj, 1913, 1929, 0, 0.03), ".\n")
cat("American present value is ", present_value(5301, ame_proj, 1913, 1929, 0, 0.03) , ".\n")
cat("Argentenian present value is ", present_value(3797, arg_proj, 1913, 1929, 0, 0.03), ".\n")

"Problem 6"
cat("Irish present value is ", present_value(1775, 2736, 1870, 1913, 0, 0.03) +
                               present_value(2736, 2824, 1913, 1929, 1913 - 1870, 0.03) -
                               (2736 / ((1 + 0.03) ^ (1913 - 1870))), ".\n")
cat("American present value is ", present_value(2445, 5301, 1870, 1913, 0, 0.03) +
                                  present_value(5301, 6899, 1913, 1929, 1913 - 1870, 0.03) -
                                  (5301 / ((1 + 0.03) ^ (1913 - 1870))), ".\n")
cat("Argentenian present value is ", present_value(1311, 3797, 1870, 1913, 0, 0.03) +
                                     present_value(3797, 4367, 1913, 1929, 1913 - 1870, 0.03) -
                                     (3797 / ((1 + 0.03) ^ (1913 - 1870))), ".\n")

\end{verbatim}
\end{scriptsize}

\subsection*{Problem 1}

The Irish were on average poorer by $\frac{2445-1775}{1775} = 37.7\%$ relative to the
average American.

The Irish were on average richer by $\frac{1775-1311}{1775} = 26.1\%$ relative to the
average Argentine.

\subsection*{Problem 2}

Irish growth is  1.011351\%.

American growth is  1.815943\%.

Argentine growth is  2.503906\%.

\subsection*{Problem 3}

Irish present value is 52946.59.

American present value is 84808.94.

Argentine present value is 52094.26.

Thus, we see that the American made the best choice with the benefit of
hindsight since we see that she has the highest computed value, and the
Argentine sister made the worst choice due to having the lowest such value.

The reason that these values show the best/worst decisions is because it is a
measure of the value placed on total earnings in the timespan with future
earnings devalued properly.

\subsection*{Problem 4}

Projected rGDP for Ireland is 3213.949.

Projected rGDP for America is 7069.854.

Projected rGDP for Argentina is 5640.106.

\subsection*{Problem 5}

Irish projected present value is 39976.77.

American projected present value is 82287.09.

Argentine projected present value is 62120.73.

The best choice was America, then Argentina, then Ireland, for the same reason
as above (they are ranked descending with respect to present value).

\subsection*{Problem 6}

Irish present value is 62739.85 .

American present value is 106136.5 .

Argentenian present value is 66439.6 .

The sister who when to Argentina made the second best choice because she has
higher present value that her Ireland-remaining sister.

\subsection*{Problem 7}

What is anachronistic is that we making evaluations from future
standpoints with econometric data that doesn't yet exist. The stability and
value of growth rates from 1870 - 1913 cannot be ascertained from 1870, and thus
doesn't really well reflect the best decision with available information at that time.


\end{document}
% LocalWords:  nodecirc