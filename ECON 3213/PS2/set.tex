
\documentclass[12pt,letterpaper]{article}
\usepackage{fullpage}
\usepackage[top=2cm, bottom=4.5cm, left=2.5cm, right=2.5cm]{geometry}
\usepackage{amsmath,amsthm,amsfonts,amssymb,amscd}
\usepackage{lastpage}
\usepackage{enumerate}
\usepackage{fancyhdr}
\usepackage{mathrsfs}
\usepackage{xcolor}
\usepackage{graphicx}
\usepackage{listings}
\usepackage{hyperref}
\usepackage{tikz}
\usepackage{xfrac}
\usepackage{nicefrac}
\usepackage{xcolor}
\allowdisplaybreaks


\usetikzlibrary{shapes.geometric,fit}
\usetikzlibrary{patterns}

\hypersetup{
  colorlinks=true,
  linkcolor=blue,
  linkbordercolor={0 0 1}
}

\setlength{\parindent}{0.0in}
\setlength{\parskip}{0.05in}

\newcommand\course{ECON 3213}
\newcommand\hwnumber{2}
\newcommand\NetIDa{dc3451}
\newcommand\NetIDb{David Chen}

\newcommand\R{\mathbb{R}}

\theoremstyle{definition}
\newtheorem*{statement}{Statement}
\newtheorem*{claim}{Claim}
\newtheorem*{theorem}{Theorem}

\newcommand{\contra}{\Rightarrow\!\Leftarrow}
\newcommand{\Lag}{\mathcal{L}}

\pagestyle{fancyplain}
\headheight 35pt
\lhead{\NetIDa}
\lhead{\NetIDa\\\NetIDb}
\chead{\textbf{\Large Problem Set \hwnumber}}
\rhead{\course \\ \today}
\lfoot{}
\cfoot{}
\rfoot{\small\thepage}
\headsep 1.5em

\begin{document}

% The following \verb!R! code was used to compute the solutions. 

% \begin{scriptsize}
% \begin{verbatim}
% \end{verbatim}
% \end{scriptsize}

\subsection*{Part a}

If we have any positive income to the bottom $90\%$, then $Y_t^{90} > 0$; and
since we know that $Y_t > 0$, we have that $\alpha_t > 0$.

Since we have that $i < j \implies y_t^i \leq y_t^j$, we have that
$\sum_{i=k}^{k+9}y_t^i \leq \sum_{i=91}^{100}y_t^i$ for any $k < 91$. Taking $k
= 1, 11, ..., 81$ and summing, we have that $\sum_{i=1}^{90}y_t^i \leq
9\sum_{i=91}^{100}$.

\begin{align*}
  \alpha_t &= \frac{Y_t^{90}}{Y_t} \\ 
           &= \frac{Y_t^{90}}{Y_t^{90} + Y_t^{10}} \\
           &= 1 - \frac{Y_t^{10}}{Y_t^{90} + Y_t^{10}} \\
           &= 1 - \frac{\sum_{i=91}^{100}y_t^i}{\sum_{i=1}^{90}y_t^i + \sum_{i=91}^{100}y_t^i} \\
           &\leq 1 - \frac{\sum_{i=91}^{100}y_t^i}{10\sum_{i=91}^{100}y_t^i} \\
           &= 1 - \frac{1}{10} = 0.9
\end{align*}

\subsection*{Part b}

If there is no income inequailty, for any $i,j$ we have that $y_t^i = y_t^j$.

\begin{align*}
  \alpha_t &= \frac{Y_t^{90}}{Y_t} \\ 
           &= \frac{\sum_{i=91}^{100}y_t^i}{\sum_{i=1}^{100}y_t^i} \\
           &= \frac{90y_t^1}{100y_t^1} \\
           &= 0.9
\end{align*}

\subsection*{Part c}

\[
  \frac{Y_{2018}^{90}}{Y_{1980}^{90}} = \frac{3Y_{1947}^{90}}{2Y_{1947}^{90}}
  = \frac{3}{2} = 1.5
\]

This implies that the total amount of income accrued to the bottom $90\%$
increased by a magnitude of $50\%$ over the $38$ years inbetween 1980 and 2018.

\subsection*{Part d}

\[
  \frac{Y_{2018}}{Y_{1980}} = \frac{4Y_{1947}}{2Y_{1947}} = 2
\]


This implies that the total amount of income to the entire population doubled
over the $38$ years inbetween 1980 and 2018; since this is significantly higher
than the previous $Y_{2018}^{90} / Y_{1980}^{90}$, it implies that income grew
much faster for the top decile than the rest.

\subsection*{Part e}

\begin{align*}
  Y_{1980}^{10} &= \alpha_{1980}Y_{1980} \\
  \frac{Y_{2018}^{10}}{Y_{1980}^{10}} &= \frac{Y_{2018} -
    Y_{2018}^{10}}{Y_{1980} - Y_{1980}^{90}} \\
                &= \frac{2Y_{1980} - \frac{3}{2}Y_{1980}^{90}}{Y_{1980} - Y_{1980}^{90}} \\
                &= \frac{2Y_{1980} - \frac{3\alpha_{1980}}{2}Y_{1980}}{Y_{1980} - \alpha_{1980}Y_{1980}} \\
                &= \frac{4 - 3\alpha_{1980}}{2 - 2\alpha_{1980}}
\end{align*}

\subsection*{Part f}

\[
  \frac{Y_{2018}^{10}}{Y_{1980}^{10}} = \frac{4 - 3\alpha_{1980}}{2 -
    2\alpha_{1980}} = \frac{4 - 2}{2 - \frac{4}{3}} = 3
\]

This means that the amount of income accrued to the top decile in 2018 is triple of what
it was in 1980.

\subsection*{Part g}

\[
  \alpha_{2018} = \frac{Y_{2018}^{90}}{Y_{2018}} =
  \frac{\frac{3}{2}Y_{1980}^{90}}{2Y_{1980}} = \frac{3}{4}\alpha_{1980}
\]

\subsection*{Part h}

We have that

\[
  Y_t^{90} = \alpha_tY_t = \alpha_t(Y_t^{90} + Y_t^{10}) \implies
  \frac{Y_t^{10}}{Y_t^{90}} = \frac{1 - \alpha_t}{\alpha_t}
\]

\begin{align*}
  \frac{x_t^{10}}{x_t^{90}} &= \frac{9Y_t^{10}}{Y_t^{90}} \\
                            &= \frac{9 - 9\alpha_t}{\alpha_t}
\end{align*}

For the respective years, we have that

\[
  \frac{x_{1980}^{10}}{x_{1980}^{90}} = \frac{3}{\frac{2}{3}} = \frac{9}{2},
\]

\[
  \frac{x_t^{10}}{x_t^{90}} = \frac{\frac{9}{2}}{\frac{1}{2}} = 9, 
\]

We have that inequailty explodes from $1980 - 2018$, as we see that the average
real income per capita doubles over this period.

\end{document}
% LocalWords:  nodecirc