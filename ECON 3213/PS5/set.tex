
\documentclass[12pt,letterpaper]{article}
\usepackage{fullpage}
\usepackage[top=2cm, bottom=4.5cm, left=2.5cm, right=2.5cm]{geometry}
\usepackage{amsmath,amsthm,amsfonts,amssymb,amscd}
\usepackage{lastpage}
\usepackage{enumerate}
\usepackage{fancyhdr}
\usepackage{mathrsfs}
\usepackage{xcolor}
\usepackage{graphicx}
\usepackage{listings}
\usepackage{hyperref}
\usepackage{tikz}
\usepackage{xfrac}
\usepackage{nicefrac}
\usepackage{xcolor}

\allowdisplaybreaks


\usetikzlibrary{shapes.geometric,fit}
\usetikzlibrary{patterns}

\hypersetup{
  colorlinks=true,
  linkcolor=blue,
  linkbordercolor={0 0 1}
}

\setlength{\parindent}{0.0in}
\setlength{\parskip}{0.05in}

\newcommand\course{ECON 3213}
\newcommand\hwnumber{5}
\newcommand\NetIDa{dc3451}
\newcommand\NetIDb{David Chen}

\newcommand\R{\mathbb{R}}

\theoremstyle{definition}
\newtheorem*{statement}{Statement}
\newtheorem*{claim}{Claim}
\newtheorem*{theorem}{Theorem}

\newcommand{\contra}{\Rightarrow\!\Leftarrow}
\newcommand{\Lag}{\mathcal{L}}

\pagestyle{fancyplain}
\headheight 35pt
\lhead{\NetIDa}
\lhead{\NetIDa\\\NetIDb}
\chead{\textbf{\Large Problem Set \hwnumber}}
\rhead{\course \\ \today}
\lfoot{}
\cfoot{}
\rfoot{\small\thepage}
\headsep 1.5em

\begin{document}

\subsection*{Part 1}

We see that as they have the same growth / depreciation / savings rates and production
function, that they converge to the same steady state behaviour. However, we
have that in country $B$, GDP per capita is much lower, and so we expect them to
experience faster initial growth as the further from steady state, the faster
the initial growth.

\subsection*{Part 2}

In the long run, we see that as measured by GDP per capita, that both grow at
rate $g = 0.02$, as predicted by the Solow model in SGU. In the long run, at
steady state, we have that they grow at the same rate.

\subsection*{Part 4}

We see that country A is already in the steady state and thus all growth comes
from technological growth. However, in country B, which starts off far poorer,
has most of its growth coming in terms of transitional dynamics rather than
technological growth because it is far below the steady state.

\break

\subsection*{Part 3}

\begin{footnotesize}
\begin{verbatim}
library(xtable)

n = 0.01
g = 0.02
delta = 0.0698
sigma = 0.25
alpha = 0.5

solow <- function(gdp_cap, e, years) {
  gdp_caps <- double(years)
  es <- double(years)
  capital_caps <- double(years)
  growths <- double(years)

  gdp_caps[1] <- gdp_cap
  es[1] <- e
  capital_caps[1] <- ((gdp_cap) / (e ^ (1 - alpha))) ^ (1 / alpha)

  for (year in 2:years) {
    es[year] <- es[year - 1] * (1 + g)
    capital_caps[year] <- (capital_caps[year - 1] * (1 - delta) +
                           gdp_caps[year - 1] * sigma) / (1 + n)
    gdp_caps[year] <- (capital_caps[year] ^ alpha) * (es[year] ^ (1 - alpha))
    growths[year] <- (gdp_caps[year] / gdp_cap) ^ (1/(year - 1))
  }

  tech <- (log(es) - log(e)) / (log(gdp_caps) - log(gdp_cap))

  return (list(gdp=gdp_caps, tech=tech, growth=growths))
}

A <- solow(2.5, 1, 26)
B <- solow(0.25, 1, 26)

part_three <- data.frame(A$gdp, A$growth, A$tech, B$gdp, B$growth, B$tech)

colnames(part_three) <- c("A GDP/Capita", "A % growth", "A % from tech",
                          "B GDP/Capita", "B % growth", "B % from tech")

xtable(part_three, type="latex")
\end{verbatim}
\end{footnotesize}


\begin{table}[p]
\centering
\begin{tabular}{rrrrrrr}
  \hline
 & A $Y_t/L_t$ & A growth & A \% from tech & B $Y_t/L_t$ & B growth & B \% from tech \\ 
  \hline
1 & 2.50 & 0.00 &  & 0.25 & 0.00 &  \\ 
  2 & 2.55 & 1.02 & 100 & 0.35 & 1.40 & 6 \\ 
  3 & 2.60 & 1.02 & 100 & 0.45 & 1.34 & 7 \\ 
  4 & 2.65 & 1.02 & 100 & 0.56 & 1.31 & 7 \\ 
  5 & 2.71 & 1.02 & 100 & 0.66 & 1.28 & 8 \\ 
  6 & 2.76 & 1.02 & 100 & 0.77 & 1.25 & 9 \\ 
  7 & 2.82 & 1.02 & 100 & 0.88 & 1.23 & 9 \\ 
  8 & 2.87 & 1.02 & 100 & 0.99 & 1.22 & 10 \\ 
  9 & 2.93 & 1.02 & 100 & 1.10 & 1.20 & 11 \\ 
  10 & 2.99 & 1.02 & 100 & 1.21 & 1.19 & 11 \\ 
  11 & 3.05 & 1.02 & 100 & 1.32 & 1.18 & 12 \\ 
  12 & 3.11 & 1.02 & 100 & 1.43 & 1.17 & 13 \\ 
  13 & 3.17 & 1.02 & 100 & 1.54 & 1.16 & 13 \\ 
  14 & 3.23 & 1.02 & 100 & 1.65 & 1.16 & 14 \\ 
  15 & 3.30 & 1.02 & 100 & 1.76 & 1.15 & 14 \\ 
  16 & 3.36 & 1.02 & 100 & 1.87 & 1.14 & 15 \\ 
  17 & 3.43 & 1.02 & 100 & 1.98 & 1.14 & 15 \\ 
  18 & 3.50 & 1.02 & 100 & 2.09 & 1.13 & 16 \\ 
  19 & 3.57 & 1.02 & 100 & 2.20 & 1.13 & 16 \\ 
  20 & 3.64 & 1.02 & 100 & 2.31 & 1.12 & 17 \\ 
  21 & 3.71 & 1.02 & 100 & 2.42 & 1.12 & 17 \\ 
  22 & 3.79 & 1.02 & 100 & 2.53 & 1.12 & 18 \\ 
  23 & 3.86 & 1.02 & 100 & 2.64 & 1.11 & 18 \\ 
  24 & 3.94 & 1.02 & 100 & 2.76 & 1.11 & 19 \\ 
  25 & 4.02 & 1.02 & 100 & 2.87 & 1.11 & 19 \\ 
  26 & 4.10 & 1.02 & 100 & 2.98 & 1.10 & 20 \\ 
   \hline
\end{tabular}
\end{table}

Growth is given in the form: if each entry is $g$, then $Y_1/L_1 \cdot g^{T - 1}
= Y_T/L_T$

\end{document}
% LocalWords:  nodecirc