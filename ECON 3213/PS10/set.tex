
\documentclass[12pt,letterpaper]{article}
\usepackage{fullpage}
\usepackage[top=2cm, bottom=4.5cm, left=2.5cm, right=2.5cm]{geometry}
\usepackage{amsmath,amsthm,amsfonts,amssymb,amscd}
\usepackage{lastpage}
\usepackage{enumerate}
\usepackage{fancyhdr}
\usepackage{mathrsfs}
\usepackage{xcolor}
\usepackage{graphicx}
\usepackage{listings}
\usepackage{hyperref}
\usepackage{tikz}
\usepackage{xfrac}
\usepackage{nicefrac}
\usepackage{xcolor}

\allowdisplaybreaks


\usetikzlibrary{shapes.geometric,fit}
\usetikzlibrary{patterns}

\hypersetup{
  colorlinks=true,
  linkcolor=blue,
  linkbordercolor={0 0 1}
}

\setlength{\parindent}{0.0in}
\setlength{\parskip}{0.05in}

\newcommand\course{ECON 3213}
\newcommand\hwnumber{10}
\newcommand\NetIDa{dc3451}
\newcommand\NetIDb{David Chen}

\newcommand\R{\mathbb{R}}

\theoremstyle{definition}
\newtheorem*{statement}{Statement}
\newtheorem*{claim}{Claim}
\newtheorem*{theorem}{Theorem}

\newcommand{\contra}{\Rightarrow\!\Leftarrow}
\newcommand{\Lag}{\mathcal{L}}

\pagestyle{fancyplain}
\headheight 35pt
\lhead{\NetIDa}
\lhead{\NetIDa\\\NetIDb}
\chead{\textbf{\Large Problem Set \hwnumber}}
\rhead{\course \\ \today}
\lfoot{}
\cfoot{}
\rfoot{\small\thepage}
\headsep 1.5em
\allowdisplaybreaks

\begin{document}

\section*{Problem 1}

\subsection*{a}

In 2007,
\[
  NIIP = A - L = 20705 - 21984 = -1279 \text{ billions of \$}
\]

In 2018,
\[
  NIIP = A - L = 25241 - 34796 = -9555 \text{ billions of \$}
\]

\subsection*{b}

In 2007,
\[
  NII = r^AA - r^LL \implies r^A = 2.85 \%
\]

In 2018,
\[
  NII = r^AA - r^LL \implies r^A = 4.12 \%
\]

\subsection*{c}

The exorbitant privilege has increased in the US; $NII$ has increased while
$NIIP$ has declined, which is reflected in the fact that the difference between
asset and liability returns has increased from $2.85\%$ to $4.12\%$.

\subsection*{d}

If $r^A = 2.85, r^L = 4.12$,
\[
  NII = r^AA - r^LL = -62.9 \text{ billions of \$}
\]

We see that if the return differential had not increased, then the $NII$ of the
US would've been negative in 2018.

\subsection*{e}

\[
  NII = r^AA - r^LL \implies A = 36497
\]

Then, we can give a quantitative value to exorbitant privilege, which is the
counterfactual difference of observed $A$ and the needed $A$ to generate the
same $NII$ holding $r^A, r^L,$ and $L$ constant. In this case, it is $36497 -
25241 = 11256$ in billions of USD.

\section*{Problem 2}

\subsection*{1}

We have that $CA_1 = -0.5(120) = -6$.

Since we have that $CA_t = TB_t + rB_{t-1}^*$,
\[
  TB_1 = CA_1 - rB_{0}^* = -6 + 10 = 4
\]

Then, $B_1^* = CA_1 + B_0^* = -6 - 100 = -106$

\subsection*{2}

We compute $B_2^* = CA_2 + B_1^* = 0 \implies CA_2 = 106$. Then, we have that
\[
  TB_2 = CA_2 - rB_1^* = 106 + 10.6 = 116.6
\]

They are barely living within their means; they would have to export practically
all of their production in period 2.

\subsection*{3}

We have that $CA_1 = -0.1(120) = -12$.

Then, $B_1^* = CA_1 + B_0^* = -12  - 100= -112$, and $B_2^* = 0\implies CA_2 =
112$. Then, the needed trade balance would be
\[
  TB_2 = CA_2 - rB_1^* = 112 + 11.2 = 123.2 > GDP
\]

so they are living beyond their means.

\section*{Problem 3}

\subsection*{1}

We care about the domain $C_1, C_2 > 0$. Then, both partial derivatives exist
and are positive, with $\frac{\partial U}{\partial C_i} = C_i^{-2}$ and
both partial second derivatives are also extant and negative, with
$\frac{\partial^2 U}{\partial C_i^2} = -C_i^{-3}$ ($i = 0,1$).

\subsection*{2}

We have that the optimality condition is
\[
  U'(C_1) + \beta(1+r_1) U'((1+r_1)(\overline{Y} - C_1)) = C_1^{-2} -
  (1+r_1)^{-1}(\overline{Y} - C_1)^{-2} = 0
\]

Solving, we get that the optimal consumption is
\[
  C_1 = \overline{Y}\left(\frac{1+r_1 - \sqrt{1+r_1}}{r_1}\right), C_2 = \overline{Y}(1+r_1)\left(\frac{\sqrt{1+r_1}-1}{r_1}\right)
\]

Abbreviating $\sqrt{1+r_1}$ to $\rho$, we get that
\[
  C_1 = \overline{Y}\left( \frac{\rho}{\rho+1} \right), C_2 = \overline{Y}\left( \frac{\rho^2}{\rho+1} \right)
\]


\subsection*{3}

\begin{gather*}
  \Delta C_1 = \frac{\rho}{\rho+1} \Delta \overline{Y} = 
    \frac{\rho}{\rho+1} \Delta Q_1 \\
  \Delta TB_1 = \Delta Q_1 - \Delta C_1 = \Delta Q_1 - \left(
    \frac{\rho}{\rho+1} \right)\Delta Q_1 = \frac{1}{\rho + 1}\Delta Q_1 \\
  \Delta CA_1 = \Delta TB_1 + \Delta(r_0B_0) = \Delta TB_1 = \frac{1}{\rho +
    1}\Delta Q_1
\end{gather*}

\subsection*{4}

Putting $\Delta Q_1 = \Delta Q_2 = \Delta Q$,

\begin{gather*}
  \Delta C_1 =  \frac{\rho}{\rho+1} \Delta \overline{Y} = \left(
    \frac{\rho}{\rho+1} \right)\left(\frac{2 + r_1}{1+r_1}\right)\Delta Q =
   \frac{\rho^2+1}{\rho^2 + \rho} \Delta Q \\
  \Delta TB_1 = \Delta Q_1 - \Delta C_1 = \Delta Q_1 - \left(
    \frac{\rho^2+1}{\rho^2+\rho} \right)\Delta Q_i = \frac{\rho-1}{\rho^2 + \rho}\Delta Q_1 \\
  \Delta CA_1 = \Delta TB_1 + \Delta(r_0B_0) = \Delta TB_1 = \frac{\rho-1}{\rho^2 + \rho}\Delta Q_1
\end{gather*}

\subsection*{5}

Note that for realistic values of $r_1$, $\rho = \sqrt{1 + r_1} \approx 1$, such that
changes look a lot like they do with logarithmic preferences. We get the same
conclusions that temporary shocks to endowment in period one yields nontrivial changes in
consumption funded by either deterioration or improvement in the trade
balance; in either set of preferences, this fall in consumption and
corresponding increase in the trade balance is about half of the shock in
magnitude.

For permanent shocks, we get that consumption eats almost all of the shock, just
as with logarithmic preferences.


\end{document}