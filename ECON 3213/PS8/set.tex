
\documentclass[12pt,letterpaper]{article}
\usepackage{fullpage}
\usepackage[top=2cm, bottom=4.5cm, left=2.5cm, right=2.5cm]{geometry}
\usepackage{amsmath,amsthm,amsfonts,amssymb,amscd}
\usepackage{lastpage}
\usepackage{enumerate}
\usepackage{fancyhdr}
\usepackage{mathrsfs}
\usepackage{xcolor}
\usepackage{graphicx}
\usepackage{listings}
\usepackage{hyperref}
\usepackage{tikz}
\usepackage{xfrac}
\usepackage{nicefrac}
\usepackage{xcolor}

\allowdisplaybreaks


\usetikzlibrary{shapes.geometric,fit}
\usetikzlibrary{patterns}

\hypersetup{
  colorlinks=true,
  linkcolor=blue,
  linkbordercolor={0 0 1}
}

\setlength{\parindent}{0.0in}
\setlength{\parskip}{0.05in}

\newcommand\course{ECON 3213}
\newcommand\hwnumber{8}
\newcommand\NetIDa{dc3451}
\newcommand\NetIDb{David Chen}

\newcommand\R{\mathbb{R}}

\theoremstyle{definition}
\newtheorem*{statement}{Statement}
\newtheorem*{claim}{Claim}
\newtheorem*{theorem}{Theorem}

\newcommand{\contra}{\Rightarrow\!\Leftarrow}
\newcommand{\Lag}{\mathcal{L}}

\pagestyle{fancyplain}
\headheight 35pt
\lhead{\NetIDa}
\lhead{\NetIDa\\\NetIDb}
\chead{\textbf{\Large Problem Set \hwnumber}}
\rhead{\course \\ \today}
\lfoot{}
\cfoot{}
\rfoot{\small\thepage}
\headsep 1.5em
\allowdisplaybreaks

\begin{document}

\section*{Problem 1}

\subsection*{1}

We have the following maximization problem on households:
\[
  \max_{C_1, C_2}\{\log(C_1) + \log(C_2)\} \mid C_1 + \frac{C_2}{1 + r} = Y_1 -
  T_1 + \frac{Y_2 - T_2}{1 + r}
\]

In particular, we have that $G_1 = 2, G_2 = 0, Y_1 = 1, T_1 + \frac{T_2}{1 + r}
= G_1 = 2$. Further, for the given technology we have the following optimality
condition for investment:
\[
  \frac{5}{\sqrt{I_1}} = 1 + r \implies I_1 = \left(\frac{5}{1+r}\right)^2
\]

Then,
\begin{align*}
  \mathcal{L}(C_1, C_2, \lambda) &= \log(C_1) + \log(C_2) - \lambda(C_1 +
  \frac{C_2}{1+r} - 8 - \frac{\Pi_2}{1 + r}) \\
  \frac{\partial \mathcal{L}}{\partial C_1} &= \frac{1}{C_1} - \lambda = 0 \\
  \frac{\partial \mathcal{L}}{\partial C_1} &= \frac{1}{C_2} -
  \frac{\lambda}{1+r} = 0 \\ 
  \intertext{The above two conditions yield that $(1+r)C_1 = C_2$. Then, }
  \frac{\partial \mathcal{L}}{\partial \lambda} &= C_1 + \frac{C_2}{1 + r} - 8 -
  \frac{\Pi_2}{1+r} =  0 \\
  \intertext{Here, we have that $C_1 = 4 + \frac{\Pi_2}{2(1 + r)}, C_2 = C_1(1 + r)$. Furthermore, we have that private savings is}
  S^p &= Y_1 - T_1 - C_1 \\
                                 &= 5 - \frac{\Pi_2}{2(1 + r)} \\
  \intertext{With government savings $T_1 - G_1 = -1,$ we have total savings}
  S &= 4 - \frac{\Pi_2}{2(1+r)} \\
  \intertext{Since in equilibrium we have that savings is identical to investment,}
  \left(\frac{5}{1+r}\right)^2 &= 4 - \frac{25}{2(1+r)^2} \\
  \intertext{The equilibrium interest rate is then}
  r &= 2.06 = 206\% \\
  \intertext{With corresponding investment and consumption}
  I_1 &= 2.67, C_1 = 5.33
\end{align*}

\subsection*{2}

We take the same process as before, and note that we now have $T_1 +
\frac{T_2}{1+r} = G_1 = 2.2$, and that
\begin{align*}
  S &= 3.9 - \frac{\Pi_2}{2(1 + r)} \\
  \left(\frac{5}{1+r}\right)^2 &= 3.9 - \frac{25}{2(1+r)^2} \\
\end{align*}
and arrive at equilibrium values
\[
  r = 2.10 = 210\%, I_1 = 2.60, C = 5.2
\]

We have essentially lower investment (as there are less savings due to government
deficit spending), which creates a situation of higher equilibrium interest
rates and also lower consumption in period one (in line with higher interest rates).

\section*{Problem 2}

\subsection*{1}

In period one, households face the following
\[
  Y_1 - T_1 = C_1 + S 
\]

In period two, they face the following
\[
  Y_2 - T_2 + S(1 + r) = C_2 
\]

The intertemporal budget constraint is then
\[
  C_1 + \frac{C_2}{1 + r} = Y_1 - T_1 + \frac{Y_2 - T_2}{1 + r}
\]

\subsection*{2}

If firms borrow $I_1$ in period one, then
\[
  \frac{T_2}{1 + r} = \tau^I(1 + r)I_1 \implies T_2 = \tau^I(1 + r)^2I_1 
\]

\subsection*{3}

\[
  \max_{I_1}\{2\sqrt{I_1} - (1 - \tau^I)(1 + r)I_1\}
\]

\subsection*{4}

The firm must have that
\[
  f'(I_1) = \frac{1}{\sqrt{I_1}} = (1 - \tau^I)(1 + r) \implies I_1 =
  \frac{1}{(1 - \tau^I)^2(1 + r)^2}
\]

Note that this means that the optimal level of investment is a strictly
decreasing function of $r$, but a strictly increasing function of $\tau^I$ for
$0 < \tau^I < 1$, as
\[
  \frac{\partial I_1}{\partial \tau^I} = \frac{2}{(1 + r)^2(1 - \tau^I)^3} > 0
\]

This makes sense: borrowing is more expensive at higher interest rates, but
cheaper the larger the subsidy.

For $\tau \geq 1$, we have negative costs to borrowing: they do it
infinitely in this case.

\subsection*{5}

If the subsidy is zero, then all taxes and spending vanish, such that
equilibrium investment follows
\[
  I_1 = \frac{1}{(1+r)^2}, \Pi = \frac{1}{1 + r}
\]
and has savings
\[
  S = Y_1 - C_1 = Y_1 - \frac{1}{2}(Y_1 + \frac{\Pi}{1 + r}) = \frac{1}{2} -
  \frac{1}{2(1 + r)^2} 
\]

In equilibrium, we have that the above quantities are equal, such that
\[
  r = 0.732 = 73.2\%, I_1 = 0.33, C_1 = 0.67, C_2 = 1.15
\]

\subsection*{6}

In the case of $\tau^I = 0.5$, we have that
\[
  I_1 = \frac{1}{(1 - \tau^I)^2(1+r)^2}, \Pi = \frac{1}{(1-\tau^I)(1 + r)}
\]

For government spending, we have, when firms profit maximize
\[
  T_2 = \tau^I(1 + r)^2I_1 = \frac{\tau^I}{(1 - \tau^I)^2}
\]

Then, we have savings
\[
  S = Y_1 - G_1 - C_1 = 1 - \frac{\tau^I}{(1+r)(1-\tau^I)^2} - \frac{1}{2}(1 + \frac{\Pi -\frac{\tau^I}{(1 -
      \tau^I)^2}}{1 + r}) = \frac{1}{2} - \frac{1 + \tau^I r}{2(1-\tau^I)(1+r)^2}
  %     \tau^I)^2}}{1 + r}) = \frac{1}{2} -
  % \frac{1-2\tau^I-\tau^Ir}{2(1-\tau^I)^2(1+r)^2}
\]

% Equating, we have
% \[
%   \frac{1}{(1 - \tau^I)^2(1+r)^2} = \frac{1}{2} -
%   \frac{1}{2(1-\tau^I)(1+r)^2} \implies 1 + r = \frac{\sqrt{3-\tau}}{1-\tau^I}
% \]

% With $\tau^I = 0.5$, we have that
% \[
%   r = 3.32 = 332\%, I_1 = 0.215, C_1 = 0.255, C_2 = 0.591
% \]

% \subsection*{7}

% We do have higher investment, moving from $I_1 = 0.33$ to $I_1 = 0.743$. In terms
% of welfare, we have in the first case utility $\log(0.67) + \log(1.15) = -0.26$, and the
% second $\log(0.257) + \log(1.72) = -1.896$, which is a decrease!

% This decrease stem from the fact that the amount of investment incurs a heavy
% tax burden in period two, turning $\Pi - T_2$ negative, and thus causing a loss
% of consumption in both periods.

% \subsection*{8}

% There is probably a clever way to do this; I don't see it. I will however hazard
% a guess that it is $\tau^I = 0$ in lieu of actually doing the math.

% % Note that we have $C_1(1 + r) = C_2 \implies
% % C_1\frac{\sqrt{3-\tau^I}}{1-\tau^I}= C_2$. Then.
% % \begin{align*}
% %   U = \log(C_1) + \log(C_2) &= \log(\frac{\sqrt{3-\tau^I}}{1-\tau^I}C_1^2) \\
% %                         &= \log(\frac{\sqrt{3-\tau^I}}{1-\tau^I}(1 - I_1)^2) \\
% %                         &= \log(\frac{\sqrt{3-\tau^I}}{1-\tau^I}(1 - \frac{1}{3-\tau^I})^2) \\
% %                             &= \log((1-\tau^I)^{-1}(2-\tau^I)^2(3-\tau^I)^{-1.5}) \\
% %                             &= -\log(1-\tau^I) + 2\log(2-\tau^I) - \frac{3}{2}\log(3-\tau^I) \\
% %   \frac{dU}{d\tau^I} &= \frac{1}{1-\tau^I} - \frac{2}{2 - \tau^I} + \frac{3}{2(3-\tau^I)} > 0 \forall x \in (0, 1)
% % \end{align*}

% % Thus, the utility maximizing subsidy is 

\end{document}