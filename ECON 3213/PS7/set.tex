
\documentclass[12pt,letterpaper]{article}
\usepackage{fullpage}
\usepackage[top=2cm, bottom=4.5cm, left=2.5cm, right=2.5cm]{geometry}
\usepackage{amsmath,amsthm,amsfonts,amssymb,amscd}
\usepackage{lastpage}
\usepackage{enumerate}
\usepackage{fancyhdr}
\usepackage{mathrsfs}
\usepackage{xcolor}
\usepackage{graphicx}
\usepackage{listings}
\usepackage{hyperref}
\usepackage{tikz}
\usepackage{xfrac}
\usepackage{nicefrac}
\usepackage{xcolor}

\allowdisplaybreaks


\usetikzlibrary{shapes.geometric,fit}
\usetikzlibrary{patterns}

\hypersetup{
  colorlinks=true,
  linkcolor=blue,
  linkbordercolor={0 0 1}
}

\setlength{\parindent}{0.0in}
\setlength{\parskip}{0.05in}

\newcommand\course{ECON 3213}
\newcommand\hwnumber{7}
\newcommand\NetIDa{dc3451}
\newcommand\NetIDb{David Chen}

\newcommand\R{\mathbb{R}}

\theoremstyle{definition}
\newtheorem*{statement}{Statement}
\newtheorem*{claim}{Claim}
\newtheorem*{theorem}{Theorem}

\newcommand{\contra}{\Rightarrow\!\Leftarrow}
\newcommand{\Lag}{\mathcal{L}}

\pagestyle{fancyplain}
\headheight 35pt
\lhead{\NetIDa}
\lhead{\NetIDa\\\NetIDb}
\chead{\textbf{\Large Problem Set \hwnumber}}
\rhead{\course \\ \today}
\lfoot{}
\cfoot{}
\rfoot{\small\thepage}
\headsep 1.5em

\begin{document}

\section*{Problem 1}

\subsection*{a}

The budget constraint in period one and two satisfy that
\begin{gather*}
  P_1C_1 + S = P_1Y_1 \\
  P_2C_2 = P_2Y_2 + (1 + i)S
\end{gather*}

The intertemporal budget constraint is the same as the model with sticky prices with $P_1 =
P_2 = 1$:
\[
  C_1 + \frac{C_2}{1 + i} = Y_1 + \frac{Y_2}{1 + i} = y
\]

\subsection*{b}

\begin{align*}
  \mathcal{L}(C_1, C_2, \lambda) &= \log(C_1) + \beta \log(C_2) - \lambda(C_1 + \frac{C_2}{1+i} - y)\\
  \frac{\partial \mathcal{L}}{\partial C_1} &= \frac{1}{C_1} - \lambda = 0\\
  \frac{\partial \mathcal{L}}{\partial C_2} &= \frac{\beta}{C_2} - \frac{\lambda}{1+i} = 0 \\
  \frac{\partial \mathcal{L}}{\partial \lambda} &= C_1 + \frac{C_2}{1+i} - y = 0\\
  \intertext{Taking that $\lambda = C_1$, }
  \frac{C_2}{C_1} &= \beta(i + i) \\
\end{align*}

Then, we have that if the central bank wishes to ensure $C_1 = Y_1, C_2 = Y_2$,
then $\beta(1 + i) = \implies i = 0.1$. In equilibrium, we have that $C_1 = Y_1
= 10,$ and that savings is $0$.

\subsection*{c}

\[
  E_1(\overline{Y_2}) = \frac{1}{2}(6) + \frac{1}{2}(12) = 9
\]

\subsection*{d}

We are subject to the following constraints, assuming that nominal prices are
fixed at $1$:

\begin{gather*}
  C_1 + S = Y_1 \implies C_1 = Y_1 - S\\
  C_2^g = Y_2^g + (1 + i)S \\
  C_2^b = Y_2^b + (1 + i)S \\
\end{gather*}

We now have expected utility
\[
  V(S) = \log(Y_1 - S) + \frac{\beta}{2}\log(Y_2^g + (1 + i)S) + \frac{\beta}{2}\log(Y_2^b + (1 + i)S)
\]

Maximizing with respect to savings,
\begin{gather*}
  \frac{\partial V}{\partial S} = 0 = \frac{1}{Y_1 - S} + \frac{\beta(1+i)}{2(Y_2^g
    + (1+i)S)} + \frac{\beta(1+i)}{2(Y_2^b + (1+i)S)} \\
  \frac{1}{Y_1 - S} = \frac{\beta(1+i)}{2(Y_2^g
    + (1+i)S)} + \frac{\beta(1+i)}{2(Y_2^b + (1+i)S)}
\end{gather*}

If the central bank does not adjust monetary policy (i.e. keeps rates at $i =
0.1$), then we have that
\[
  \frac{1}{10 - S} = \frac{1}{2(6 + 1.1S)} + \frac{1}{2(12 + 1.1S)}
\]

This results in $S = 0.905$, and so $C_1 = 9.095$.

\subsection*{e}

In this class, we have worked with the assumption that the market clears, such
that $S = 0$. Then, we have that
\[
  \frac{1}{1+i} = \frac{\beta}{Y_1}(\frac{1}{2Y_2^g} + \frac{1}{2Y_2^b}) = 0.011
\]

Thus, we have that we are in a liquidity trap, and the optimal policy is $i = 0$.

At the zero lower bound, we have that
\[
  \frac{1}{10 - S} = \frac{\beta}{2(6 + 1.1S)} + \frac{\beta}{2(12 + 1.1S)}
\]
such that $S = 0.514$, and thus $C_1 = 9.486$.
\section*{Problem 2}

\subsection*{a}

This is identical to before; $i^* = 0.1$.

\subsection*{b}

We still have that, since the market clears and is in full employment in the
long run, $C_1 = Y_1, C_2 = \overline{Y}$
\[
  \frac{C_2}{C_1} = \beta(1 + i) \implies Y_1 = \frac{\overline{Y}}{\beta(1 + i)}
\]

Thus, we have that $Y_1 = 8.18$, and that the output gap is
\[
  \left(  \frac{\overline{Y}}{Y_1} - 1\right)100 = 22.2
\]

\subsection*{c}

If the bank acts quickly, then we have that $\beta(1 + i) = 1 \implies i < 0$,
such that we are in a liquidity trap. At $i = 0$, we have that $Y_1 =
0.9\overline{Y}$, and so the output gap is $11.1$.

\subsection*{d}

We wish to spend $G^*$ such that at $i = 0$, $Y_1 = \overline{Y}$. In particular,
since we still have that
\[
  \frac{C_2}{C_1} = \beta(1 + i)
\]
we now arrive at
\[
  Y_1 = \frac{\overline{Y}}{\beta(1 + i)} + G^*
\]

Thus, if we want that $Y_1 = \overline{Y}$, it must be that $G^* = 0.1\overline{Y}
= 1$.

\subsection*{e}

If instead $\tilde{G} = 0.11\overline{Y}$ is spent, then the central bank will raise interest
rates to compensate, such that
\[
  1 = \frac{1}{\beta(1 + i)} + 0.11 \implies i = 0.123
\]

Private consumption is now $\overline{Y} - G = 0.89\overline{Y} = 8.9$.

We see that government fiscal spending has crowded out private consumption due
to its miscalculation, such that the government spending multiplier is now $< 1$.

\end{document}