
\documentclass[12pt,letterpaper]{article}
\usepackage{fullpage}
\usepackage[top=2cm, bottom=4.5cm, left=2.5cm, right=2.5cm]{geometry}
\usepackage{amsmath,amsthm,amsfonts,amssymb,amscd}
\usepackage{lastpage}
\usepackage{enumerate}
\usepackage{fancyhdr}
\usepackage{mathrsfs}
\usepackage{xcolor}
\usepackage{graphicx}
\usepackage{listings}
\usepackage{hyperref}
\usepackage{tikz}
\usepackage{xfrac}
\usepackage{nicefrac}
\usepackage{xcolor}

\allowdisplaybreaks


\usetikzlibrary{shapes.geometric,fit}
\usetikzlibrary{patterns}

\hypersetup{
  colorlinks=true,
  linkcolor=blue,
  linkbordercolor={0 0 1}
}

\setlength{\parindent}{0.0in}
\setlength{\parskip}{0.05in}

\newcommand\course{ECON 3213}
\newcommand\hwnumber{3}
\newcommand\NetIDa{dc3451}
\newcommand\NetIDb{David Chen}

\newcommand\R{\mathbb{R}}

\theoremstyle{definition}
\newtheorem*{statement}{Statement}
\newtheorem*{claim}{Claim}
\newtheorem*{theorem}{Theorem}

\newcommand{\contra}{\Rightarrow\!\Leftarrow}
\newcommand{\Lag}{\mathcal{L}}

\pagestyle{fancyplain}
\headheight 35pt
\lhead{\NetIDa}
\lhead{\NetIDa\\\NetIDb}
\chead{\textbf{\Large Problem Set \hwnumber}}
\rhead{\course \\ \today}
\lfoot{}
\cfoot{}
\rfoot{\small\thepage}
\headsep 1.5em

\begin{document}

\subsection*{Problem 1}

\subsubsection*{a}

We have that the final overall steady state criterion is that

\begin{alignat*}{2}
  && K^* &= (1 - \delta)K^* + \sigma I^* \\
  &\implies& k^* &= \frac{1 - \delta}{(1 + g)(1 + n)}k^* + \frac{\sigma
    \sqrt{k^*}}{(1 + g)(1 + n)} \\
  &\implies& \sigma \sqrt{k^*} &= ((1 + g)(1 + n) - (1 - \delta))k^*  \\
  &\implies& k^* &= \left(\frac{\sigma}{(1 + n)(1 + g) - (1 - \delta)}\right)^2
\end{alignat*}

\subsubsection*{b}


\begin{align*}
  \intertext{From above, we have that}
  k^* &= \left(\frac{\sigma}{(1 + n)(1 + g) - (1 - \delta)}\right)^2 \\
      &= \left( \frac{0.25}{(1 + 0.015)(1 + 0.02) - (1 - 0.10)} \right)^2 \\
      &= 3.414 \\
  \intertext{Since we have that in steady state $y^* = \sqrt{k^*}$,} 
  y^* &= \sqrt{3.414} = 1.847 \\
\end{align*}

\subsubsection*{c}

\begin{alignat*}{2}
  && \frac{Y_{t+1}}{L_{t+1}E_{t+1}} &= \frac{Y_t}{L_tE_t} \\
  &\implies& Y_{t+1} &= (1 + g)(1 + n)Y_t \\
  && &= 1.0353Y_t
\end{alignat*}

Thus, it takes $\frac{\log(2)}{\log(1.0353)} = 20$ years for overall output to double.

\subsubsection*{d}

\begin{alignat*}{2}
  && \frac{Y_{t+1}}{L_{t+1}E_{t+1}} &=  \frac{Y_{t}}{L_tE_t} \\
  &\implies& \frac{Y_{t+1}}{L_{t+1}} &= (1 + n)\frac{Y_t}{L_t} \\
  &\implies& \frac{Y_{t+1}}{L_{t+1}} &= 1.015\frac{Y_t}{L_t} \\
\end{alignat*}

Thus, it takes $\frac{\log(2)}{\log(1.015)} = 46.5$ years for output per capita to double.

\subsubsection*{e}

\begin{alignat*}{2}
  && \frac{K_{t+1}}{L_{t+1}E_{t+1}} &= \frac{K_t}{L_tE_T} \\
  &\implies& K_{t+1} &= (1 + g)(1 + n)K_t \\
\end{alignat*}

From above, it takes also 20 years for the capital stock to double.

\subsection*{Problem 2}


\subsubsection*{a}

We have from SGU that as growth in GDP per capita is computed to be near
constant over 1870 to 2016, and is equal to $1.97 \%$. Thus, GDP per capital
grew about 18 fold over the mentioned 148 years.

\subsubsection*{b}

No, there is not; in fact, the plot has the lowest values for GDP per capital
are about $500$ in 1990 \$, and the highest at about $3000$. This is then a
maximum of a 6-fold increase in any time period shown, so no country sees growth
like the US from 1870 to 2018.

\subsubsection*{c}

Keynes is mostly correct: there are varations, but all countries on the plotted
timespan have maximum value about $200\%$ of their minimum value, and so there
is no such explosive, continual growth.

\subsubsection*{d}

Long flat growth followed by nonzero growth looks like a hockey stick on its
side when output is plotted logarithmically.

\end{document}
% LocalWords:  nodecirc