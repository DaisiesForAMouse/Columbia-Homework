\documentclass[12pt,letterpaper]{article}
\usepackage{fullpage}
\usepackage[top=2cm, bottom=4.5cm, left=2.5cm, right=2.5cm]{geometry}
\usepackage{amsmath,amsthm,amsfonts,amssymb,amscd}
\usepackage{lastpage}
\usepackage{enumerate}
\usepackage{fancyhdr}
\usepackage{mathrsfs}
\usepackage{xcolor}
\usepackage{graphicx}
\usepackage{mathdots}
\usepackage{listings}
\usepackage{hyperref}
\usepackage{tikz}
\usepackage{relsize}
\usepackage{fancyvrb}
\usepackage{import}
\usetikzlibrary{shapes.geometric,fit}

\hypersetup{%
  colorlinks=true,
  linkcolor=blue,
  linkbordercolor={0 0 1}
}

\setlength{\parindent}{0.0in}
\setlength{\parskip}{0.05in}

\theoremstyle{definition}
\newtheorem*{statement}{Statement}
\newtheorem*{claim}{Claim}
\newtheorem*{theorem}{Theorem}
\newtheorem*{lemma}{Lemma}

\newcommand{\contra}{\Rightarrow\!\Leftarrow}
\newcommand{\R}{\mathbb{R}}
\newcommand{\F}{\mathbb{F}}
\newcommand{\Z}{\mathbb{Z}}
\newcommand{\Zeq}{\mathbb{Z}_{\geq 0}}
\newcommand{\Zg}{\mathbb{Z}_{>0}}
\newcommand{\Req}{\mathbb{R}_{\geq 0}}
\newcommand{\Rg}{\mathbb{R}_{>0}}
\newcommand{\N}{\mathbb{N}}
\newcommand{\Q}{\mathbb{Q}}
\newcommand{\C}{\mathbb{C}}
\DeclareMathOperator{\Id}{id}
\DeclareMathOperator{\Aut}{Aut}
\DeclareMathOperator{\Stab}{Stab}
\DeclareMathOperator{\lcm}{lcm}
\DeclareMathOperator{\Ker}{Ker}
\DeclareMathOperator{\Image}{Im}

\newcommand{\incfig}[1] {%
    % \def\svgwidth{\columnwidth}
    \import{./figures/}{#1.pdf_tex}
}

\title{MATH 4041 Final}
\author{David Chen, dc3451}
\date{\today}

\begin{document}

\maketitle

I had some extra time, so I decided to type up some of this to make the graders' lives easier. The handwritten copies are at the end.

\section*{Q1}
\subsection*{i}

Note first that under the construction of $\N$ given in this class, we define addition as $\text{succ}(n)$ for some successor function; in particular, $f$ here is exactly the successor function by definition, and via the construction, we get that there is no preimage of $1$ (or $0$, I can't remember the convention used in this class for the least element of $\N$, I just use $1$ as the least element, none of the proofs change at all).

For injectivity, $f(n) = f(m) \implies n + 1 = m + 1 \implies n = m$. To see $f$ is not surjective (alternatively, see note above) we have that $f(n) = 1 \implies n + 1 = 1 \implies n < 1$, but $1$ is the least element of $\N$. $\contra$, so we get that $1$ has no preimage under $f$ and thus $f$ is not surjective. However, for $n > 1$, $n - 1 \in \N$, so $f(n - 1) = n$ and thus the image of $f$ is all of $\N \setminus \{1\}$.

\subsection*{ii}

Pick any $k \in \N$, and define
\[
  g_{k}(n) = \begin{cases}
    k & n = 1 \\
    n - 1 & \text{ otherwise }
  \end{cases}
\]
such that for any $n \in \N$, $g_{k}(f(n)) = g(n + 1)$ and since $n + 1 \neq 1$, $g_{k}(n + 1) = (n + 1) - 1 = n$. Thus, $g_{k}$ is a left inverse for $f$.

In particular, for $n > 1$ and any left inverse $g$, $n - 1 \in \N$, so $g(f(n - 1)) = g(n) = n - 1$ so for $n > 1$, $g$ must map $n \mapsto n - 1$; however, we can pick any $k \in \N$ for $g(1)$, so we have found all such left inverses. This gives an infinite amount of left inverses for $f$.

\subsection*{iii}

Since we have that for any $n \in \N$, $n + 1 \in \N$, we have $g_{k}(n + 1) = n$, so every element of $\N$ has a preimage, so $g_{k}$ is surjective, but $g(1) = g(k + 1) = k$, and since $k + 1 \neq 1$ (otherwise $f(k) = 1$, $\contra$), we have that $g_{k}$ is not injective for any $k$, and thus no left inverse is ever injective.

\subsection*{iv}

We have that $\Image(f \circ h) \subset \Image(f)$, since $(f \circ h)(n) = m \implies f(h(n)) = m$, $h(n) \in \N$. Then, $\not\exists n \in \N \mid (f \circ h)(n) = 1$, since we would then get $h(n)$ as a preimage under $f$ for 1; $\contra$, so no right inverse exists.

\section*{Q2}

\subsection*{(a)}

The order of $(\Z/25\Z)^{*}$ is $\varphi(25) = 25 \cdot (1 - 1/5) = 20$.

\subsection*{(b)}

\begin{align*}
  2^{1} &= 2  \mod 25\\
  2^{2} &= 2 \cdot 2 = 4  \mod 25\\
  2^{4} &= (2^{2})^{2} = 16  \mod 25\\
  2^{5} &= 2^{4} \cdot 2 = 32 = 7 \mod 25 \\
  2^{10} &= (2^{5})^{2} = 49 = -1 = 24 \mod 25 \\
  2^{20} &= (2^{10})^{2} = -1^{2} = 1 \mod 25 \\
\end{align*}

The order of $2$ in $(\Z/25\Z)^{*}$ must divide $|(\Z/25\Z)^{*}| = 20$, so it must be one of $1,2,4,5,10,20$. Since we checked manually that it is not $1,2,4,5,10$, it must be of order 20 and thus a generator of $(\Z/25\Z)^{*}$, since $(\Z/25\Z)^{*}$ is finite, $|\langle 2 \rangle| = |(\Z/25\Z)^{*}|$ and $\langle 2 \rangle \leq (\Z/25\Z)^{*}$.

\subsection*{(c)}

This happens when $a$ is coprime to $20$, as seen in class, so $a = 1, 3, 7, 8, 11, 13, 17, 19$. There $\varphi(20) = 20 \cdot (1 - 1/2)(1-1/5) = 8$ of these.

\subsection*{(d)}

An element $2^{a}$ is of order $n$ when $20/\gcd(20,a) = n$. In particular, for order $4$, we have $2^{5}, 2^{15}$, for order $5$, we have $2^{4}, 2^{8}, 2^{12}, 2^{16}$ and there are no elements of order $3$.

\section*{Q3}

\subsection*{i}

We have
\[
  \sigma = (1, 6, 7, 2)(4,8,5)
\]

Note that disjoint cycles commute, so if $\sigma = \rho_{1}\rho_{2}\cdots\rho_{n}$,
\[
  \sigma^{2} = (\rho_{1}\rho_{2}\cdots\rho_{n})(\rho_{1}\rho_{2}\cdots\rho_{n}) = (\rho_{1}\rho_{1}\rho_{2}\rho_{2}\cdots\rho_{n}\rho_{n}) = \rho_{1}^{2}\rho_{2}^{2}\cdots\rho_{n}^{2}
\]
and via induction,
\[
  \sigma^{k+1} = (\rho_{1}^{k}\rho_{2}^{k}\cdots\rho_{n}^{k})(\rho_{1}\rho_{2}\cdots\rho_{n}) = (\rho_{1}^{k}\rho_{1}\rho_{2}^{k}\rho_{2}\cdots\rho_{n}^{k}\rho_{n}) = \rho_{1}^{k+1}\rho_{2}^{k+1}\cdots\rho_{n}^{k+1}
\]
so
\[
  \sigma^{4} = (1,6,7,2)^{4}(4,8,5)^{3}(4,8,5) = (4,8,5)
\]
and
\[
  \sigma^{6} = (1,6,7,2)^{4}(1,6,7,2)^{2}(4,8,5)^{3}(4,8,5)^{3} = (1,7)(6,2)
\]
and
\[
  \sigma^{12} = ((1,7)(6,2))^{2} = (1,7)^{2}(6,2)^{2} = 1
\]
and in particular, this shows that $\sigma^{2} \neq 1$ and $\sigma^{3} \neq 1$, since otherwise $\sigma^{4} = (\sigma^{2})^{2} = 1$ and same for $\sigma^{6} = (\sigma^{3})^{2}$. However, the order of $\sigma$ must divide 12, so it is one of $1, 2, 3, 4, 6, 12$; but we checked that it is not 1,2,3,4,6, so it must be 12.

\subsection*{ii}

We have seen in class that an $n$-cycle can be written as the product of $n - 1$ transpositions, so
\[
  \varepsilon(\sigma) = \varepsilon((1,6,7,2))\varepsilon((4,8,5)) = (-1)^{3}(-1)^{2} = -1
\]
so $\sigma$ is odd.

\subsection*{iii}
\subsubsection*{a}

There are 2 non-trivial orbits, and 1 orbit of size 1. Take $\sigma^{n} \in \langle \sigma \rangle$, and note that
\[
  \sigma^{n} = (1,6,7,2)^{n}(4,8,5)^{n}
\]
so if $k \in \{1,6,7,2\}$, $\sigma^{n}(k) \in \{1,6,7,2\}$ (the first orbit) and if $k \in \{4,8,5\}$, $\sigma^{n}(k) \in \{4,8,5\}$ (the second orbit), since only one of the cycles in $\sigma$ moves these $k$. Lastly, the orbit of order 1 is $G \cdot 3 = \{3\}$.

\subsubsection*{b}

As above, this is $\{4,8,5\}$, which is of order 3.

\subsubsection*{c}

Take any element $\sigma^{n} \in G$. Then, $\sigma^{n} = \sigma^{3q + r}$ for $r = 0, 1, 2$, so $\sigma^{n} = (1,6,7,2)^{n}(4,8,5)^{3q}(4,8,5)^{r}$, but clearly the part $(1,6,7,2)^{n}$ fixes $\{4,8,5\}$, so we only care about $(4,8,5)^{3q}(4,8,5)^{r} = (4,8,5)^{r}$. Then, we have that $r = 1$ takes $5 \mapsto 4$ and $r = 2$ takes $5 \mapsto 8$, so we get that $r = 0$, so the stabilizer is $\{1, \sigma^{3}, \sigma^{6}, \sigma^{9}\}$, of order 4 (also computable via $|G|/|G_{5}| = 12/3$).

\section*{Q4}

See the scans at the end of the pdf.

\section*{Q5}

We have that Fermat gives $7^{22} \equiv 1 \mod 23$, so $7^{68} = (7^{22})^{3} \cdot 7^{2} \equiv 49 \equiv 3 \mod 23$, so the remainder will be $a = 3$.

\section*{Q6}
\subsection*{i}

We have directly from class that
\[
  |G| = |G_{x}| \cdot |G \cdot x|
\]
or as we will use later $G_{x} = \frac{|G|}{|G \cdot x|}$.

\subsection*{ii}

We have that conjugation of a product of disjoint cycles of lengths $r_{1}, r_{2}, \dots, r_{n}$ results in a product of disjoint cycles of lengths $r_{1}, r_{2}, \dots, r_{n}$ from one of the homeworks. Then, we have that the conjugacy class is some subset of $\{(1,2)(3,4), (1,3)(2,4), (1,4)(2,3)\}$ which on inspection are the only products of disjoint transpositions in $S_{4}$. Then, checking that all 3 are achieved,
\begin{align*}
  1 \cdot (1,2)(3,4) \cdot 1 = (1,2)(3,4) \\
  (1,3) \cdot (1,2)(3,4) \cdot (1,3) = (1,4)(2,3) \\
  (1,4) \cdot (1,2)(3,4) \cdot (1,4) = (1,3)(2,4) \\
\end{align*}
so the orbit is exactly $\{(1,2)(3,4), (1,3)(2,4), (1,4)(2,3)\}$ which is of order $3$.

\subsection*{iii}

By definition, the centralizer of $(1,2)(3,4)$ is
\[
  \{g \in S_{4} \mid g(1,2)(3,4)g^{-1} = (1,2)(3,4)\}
\]
but this is also exactly the stablizer of $(1,2)(3,4)$ w.r.t the earlier group action. Then, from the first part, we get that the stablizer is going to have order $24/3 = 8$; since $8 = 2^{3}$ and $24 = 3 \cdot 2^{3}$, it must also be a 2-Sylow subgroup of $S_{4}$, which has order 24.

\section*{Q7}

See the scans below.

\section*{Q8}

Note $44 = 2^{2} \cdot 11$, and has divisors $1,2,4,11,44$.

\subsection*{i}

The order of a 2-Sylow subgroup will be $2^{2} = 4$. The odd (that is, $\equiv 1 \mod 2$) divisors of 44 are $1$ and $11$, so there is either 1 or 11 2-Sylow subgroups.

\subsection*{ii}

The order of an 11-Sylow subgroup will be $11$. The $\equiv 1 \mod 11$ divisor of 44 is exactly 1, so there is a unique 11-Sylow subgroup.

\subsection*{iii}

Since conjugating an 11-Sylow subgroup gives rise to another 11-Sylow subgroup, if $H$ is that unique 11-Sylow subgroup, $gHg^{-1} = H$ for all $g \in G$, so $H$ is normal and a nontrivial subgroup of $G$, so $G$ is not simple.

\end{document}
% LocalWords:  NetID fancyplain LocalWords colorlinks linkcolor linkbordercolor