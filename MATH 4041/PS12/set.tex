\documentclass[12pt,letterpaper]{article}
\usepackage{fullpage}
\usepackage[top=2cm, bottom=4.5cm, left=2.5cm, right=2.5cm]{geometry}
\usepackage{amsmath,amsthm,amsfonts,amssymb,amscd}
\usepackage{lastpage}
\usepackage{enumerate}
\usepackage{fancyhdr}
\usepackage{mathrsfs}
\usepackage{xcolor}
\usepackage{graphicx}
\usepackage{listings}
\usepackage{hyperref}
\usepackage{tikz}
\usepackage{relsize}
\usepackage{fancyvrb}
\usepackage{import}
\usetikzlibrary{shapes.geometric,fit}

\hypersetup{%
  colorlinks=true,
  linkcolor=blue,
  linkbordercolor={0 0 1}
}

\setlength{\parindent}{0.0in}
\setlength{\parskip}{0.05in}

\theoremstyle{definition}
\newtheorem*{statement}{Statement}
\newtheorem*{claim}{Claim}
\newtheorem*{theorem}{Theorem}
\newtheorem*{lemma}{Lemma}

\newcommand{\contra}{\Rightarrow\!\Leftarrow}
\newcommand{\R}{\mathbb{R}}
\newcommand{\F}{\mathbb{F}}
\newcommand{\Z}{\mathbb{Z}}
\newcommand{\Zeq}{\mathbb{Z}_{\geq 0}}
\newcommand{\Zg}{\mathbb{Z}_{>0}}
\newcommand{\Req}{\mathbb{R}_{\geq 0}}
\newcommand{\Rg}{\mathbb{R}_{>0}}
\newcommand{\N}{\mathbb{N}}
\newcommand{\Q}{\mathbb{Q}}
\newcommand{\C}{\mathbb{C}}
\DeclareMathOperator{\Id}{id}
\DeclareMathOperator{\Aut}{Aut}
\DeclareMathOperator{\lcm}{lcm}

\newcommand{\incfig}[1] {%
    % \def\svgwidth{\columnwidth}
    \import{./figures/}{#1.pdf_tex}
}

\title{MATH 4041 HW 12}
\author{David Chen, dc3451}
\date{\today}

\begin{document}

\maketitle

\section*{Problem 1}

\subsection*{i}

We have that
\[
  f_{a,b} \circ f_{r,0} \circ f^{-1}_{a,b} = f_{ra,b} \circ f_{a^{-1},-a^{-1}b} = f_{r, b-rb}
\]
so for example,
\[
  f_{2,2} \circ f_{2,0} \circ f^{-1}_{2,2} = f_{4,2} \circ f_{1/2,-1} = f_{2,-2}
\]
but $f_{2,-2} \notin K$, while $f_{2,0} \in K$, so $K$ is not normal.

\subsection*{ii}

We have that
\[
  f_{a,b} \circ f_{1,s} \circ f^{-1}_{a,b} = f_{a,as + b} \circ f_{a^{-1},-a^{-1}b} = f_{1, as}
\]
and $as \in \R$ since $a, s \in \R$, so $hNh \subseteq N$ for any $h \in H$, so $N$ is normal.

Next, $H/N \cong \R^{*}$. To see this, we use Noether's first isomorphism theorem: consider the homomorphism $F: H \rightarrow \R^{*}$ which takes $f_{a,b} \mapsto a$ and $F(f_{a,b} \circ f_{c,d}) = F(f_{ac,ad + b}) = ac = F(f_{a,b})F(f_{c,d})$. Then, $F(f_{a,b}) = 1 \iff a = 1$, so the kernel of $F$ is $N$. Then, the theorem gives that the image of $F$, which is all of $\R^{*}$ (since every $r \in \R^{*}$ has preimage $f_{r,0}$, for example), is isomorphic to $H/N$.

\section*{Problem 2}

\subsection*{i}

First, note that $gH = H \implies g \cdot 1 = g \in H$ since $1 \in H$, and if $g \in H$, then $gH = H$ since $H$ is a subgroup and closed under the operation. Then, $gH = H \iff g \in H$.

From definition of coset multiplication, we have that $(aH)^{n} = a^{n}H$; then, $a^{n}H = H$ (here, $H$ is the identity for coset multiplication) if and only if $a^{n} \in H$, so the integers $n$ which satisfy $a^{n} \in H$ and which satisfy $(aH)^{n} = H$ are the same, and by definition of order, the order of $aH$ is the least positive integer $n$ such that $(aH)^{n} = H$ and thus the least positive integer that $a^{n} \in H$. Further, if $a^{n} \notin H$ for all positive integers $n$, then we have that $(aH)^{n} \neq H$, so $aH$ has infinite order in this case.

\subsection*{ii}

Consider $G = \Z$, $H = \langle 2 \rangle$. Clearly $a = 1 \in \Z$ has infinite order, but $2 \cdot 1 = 2 \in \langle 2 \rangle$, so $aH$ has finite order 2 (the subgroup is normal since $\Z$ is abelian).

\subsection*{iii}

We have that if $a^{n} = 1$, then $(aH)^{n} = a^{n}H = 1H = H$, so $aH$ has order at most $n$ in $G/H$. To see that $aH$ must have order dividing $n$, suppose that $aH$ has order $m$ and $n = qm + r$ where $0 \leq r < m$. Then,
\[
  H = (aH)^{n} = (aH)^{qm + r} = (aH)^{qm}(aH)^{r} = ((aH)^{m})^{q}(aH)^{r} = H^{q}(aH)^{r} = aH^{r}
\]
but then $aH^{r} = H$, so since $r < m$, if $r > 0$, then $aH$ has order $r < m$; $\contra$, so $r = 0$ and $n = qm$, so $m \mid n$.

The order of $aH$ is not always $n$: consider the cyclic group $G = \Z/4\Z$, and let $H = \{0, 2\}$, $a = 1$. $a$ has order $4$ in $G$, but $aH = \{1, 3\}$, $a^{2}H = \{2,4\} = \{0, 2\} = H$, so $aH$ only has order $2$ in $G/H$.

\section*{Problem 3}

\subsection*{i}

$f^{-1}(H_{2}) = \{h \in G_{1} \mid f(h) \in H_{2}\}$. Suppose that $f^{-1}(H_{2})$ was not normal, such that $\exists h \in f^{-1}(H_{2})$ and $g \in G_{1}$ such that $ghg^{-1} \notin f^{-1}(H_{2}) \implies f(ghg^{-1}) \notin H_{2}$. But then, $f(g)^{-1} = f(g^{-1})$, so $f(ghg^{-1}) = f(g)f(h)f(g)^{-1} \notin H_{2}$, and $f(h) \in H_{2}$, so $H_{2}$ is not normal. $\contra$, so the preimage of a normal group under a homormorphism is itself normal.

\subsection*{ii}

Consider the mapping $f: S_{3} \rightarrow S_{4}$ where $f(\sigma) = \tau$ where $\sigma(n) = \tau(n)$ for $1 \leq n \leq 3$ and $\tau(4) = 4$. This is a homomorphism: $(f(\sigma_{1}\sigma_{2}))(n) = (\sigma_{1}\sigma_{2})(n) = (f(\sigma_{1})f(\sigma_{2}))(n)$ for $1 \leq n \leq 3$, since $\sigma_{2}(n) \in \{1,2,3\}$ for $1 \leq n \leq 3$. Lastly, if $n = 4$, $(f(\sigma_{1}\sigma_{2}))(4) = 4 = (\sigma_{1}\sigma_{2})(4) = (f(\sigma_{1})f(\sigma_{2}))(4)$ Then, $f(A_{3})$ is not normal even though $A_{3}$ is normal; consider
\[
  (3,4)f((1,2,3))(4,3) = (3,4)(1,2,3)(4,3) = (3,4)(1,2,3,4) = (1, 2, 4)
\]
but $(1,2,4)$ moves $4$, so $(1,2,4) \notin f(A_{3})$.

\subsection*{iii}

Pick any $g \in G_{2}$ and any $h \in f(H_{1})$. We wish to show that $ghg^{-1} \in f(H_{1})$. In particular, let $h = f(h')$ for some $h' \in H_{1}$, and $g = f(g')$ for some $g' \in G$ since $f$ is surjective. Then, $f(g'^{-1}) = f(g')^{-1} = g^{-1}$, and so
\[
  f(g'h'g'^{-1}) = f(g')f(h')f(g'^{-1}) = ghg^{-1}
\]
but since $H_{1}$ is normal, $g'h'g'^{-1} \in H_{1}$, so $f(g'h'g'^{-1}) = ghg^{-1} \in f(H_{1})$, and so $f(H_{1})$ is normal.

\section*{Problem 4}

The first isomorphism theorem gives that $G/K \cong H$ where $K$ is the kernel of $f$ ($H$ is the image of $f$ since $f$ is surjective). Then, since $|G/K| = |G|/|K|$, and there is a bijection between $G/K$ and $H$, $|G|/|K| = |H| \implies |K| = |G|/|H|$.

\section*{Problem 5}

If $g = 1$, then $H = \{1\}$ and is trivially both normal and contained in the center. Assume otherwise for the rest of the problem:

$(\implies)$ If $H$ is normal, then $hgh^{-1} \in H$ for any $h \in G$. Then, either $hgh^{-1} = 1$ or $hgh^{-1} = g$; in the first case, we have that $hgh^{-1} = 1 \implies hgh^{-1}h = h \implies hg = h \implies g = 1$, which contradicts the earlier assumption that $g \neq 1$. Then, $hgh^{-1} = g \implies hg = gh \implies g \in Z(G)$.

$(\impliedby)$ If $H \leq Z(G)$, then $1, g$ must both commute with every element of $G$. In particular, this is always true for the identity, and for any $h \in G$, $h \cdot 1 \cdot h^{-1} = hh^{-1} = 1 \in H$, and since $g \in Z(G)$, $hgh^{-1} = hh^{-1}g = g \in H$ as well, so $H$ is normal.

\section*{Problem 6}

$\Z \times \Z$ is abelian, so $\langle (a,b) \rangle$ is normal.

We have that all homomorphisms $\Z \times \Z \rightarrow \Z$ take the form $f(n,m) = cn + dm$ for integers $c,d$. Then, if we want that the kernel of $f$ is $\langle (a,b) \rangle$, consider $f(n,m) = bn - am$, such that $f(ka, kb) = bka - akb = 0$, so every element $(ka, kb) \in \langle (a,b) \rangle$ is killed by $f$. Further, if $bn - am = 0$, then $bn = am$, and since $gcd(a,b) = 1$, we have that $b \mid m$ and $a \mid n$, so $b(k_{n}a) = a(k_{m}b) \implies k_{n} = k_{m}$ so the only $(n,m)$ which are killed are $(n,m) = (ka, kb)$, and so the kernel of $f$ is $\langle (a,b) \rangle$. Then, $f$ is still surjective since $\gcd(a,b) = 1$, as was proved in the earlier homework 10.

The first isomorphism theorem then gives that $(\Z \times \Z) / \langle (a,b) \rangle \cong \Z$, since the image of $f$ is $\Z$ since it is surjective and $\langle (a,b) \rangle$ is the kernel of $f$.

\section*{Problem 7}

Consider any element $n(a,b) + m(c,d) = (na + mc, nb + md) \in K$. Then, $f(na + mc, nb + md) = -b(na + mc) + a(nb + md) = -bmc + amd = m(ad - bc) = mN \in \langle N \rangle$, so the image of $f \subseteq \langle N \rangle$. Then, we have that any element $kN \in \langle N \rangle$ has preimage $0(a,b) + k(c,d)$, so the image of $f \supseteq \langle N \rangle$, so the image of $f$ is exactly $\langle N \rangle$. This gives by the first isomorphism theorem that (since everything is abelian, $K$ is a normal subgroup of $G$ and $H$ is a normal subgroup of $K$) $K/H \cong \langle N \rangle$. Then, by the third isomorphism theorem, $(\Z \times \Z) / K \cong ((\Z \times \Z)/H) / (K/H)$.

Now, since $(\Z \times \Z)/H \cong \Z$ and $K/H \cong \langle N \rangle$, let $g$ be the isomorphism given in the first isomorphism theorem taking $(\Z \times \Z)/ H \rightarrow \Z$ and $\pi: \Z \rightarrow \Z / \langle N \rangle$ the quotient homomorphism. Then, the function $\pi \circ g: (\Z \times \Z) / H \rightarrow \Z / \langle N \rangle$ is a surjective homomorphism. Further, the kernel of $\pi \circ g$ is the preimage of the kernel of $\pi$ under $g$, namely $g^{-1}(\langle N \rangle)$. However, the way that the first isomorphism theorem constructs the isomorphism $g$ gives that $f = g \circ \pi'$ where $\pi': (\Z \times \Z) \rightarrow (\Z \times \Z)/H$ is the quotient homomorphism, but $f(K) = \langle N \rangle$, so $g(\pi'(K)) = g(K/H) = \langle N \rangle$. In particular, since $g$ is a bijection, we can invert this to $K/H = g^{-1}(\langle N \rangle)$. Finally, we have that by the first isomorphism theorem, $((\Z \times \Z)/H) / (K/H) \cong \Z / \langle N \rangle = \Z/N\Z$, as desired.


\section*{Problem 8}

$(\implies)$ Let $f$ by a homomorphism $G \rightarrow H$; then, since $G$ is simple and $\ker f$ is a normal subgroup of $G$, either $\ker f = \{1\}$ or $\ker f = G$. In the first case, $f$ is injective, and in the second, $f$ is trivial since every element is mapped to the identity.

$(\impliedby)$ If every homomorphism $f: G \rightarrow H$ is either trivial or injective, the quotient homomorphism $\pi: G \rightarrow G/K$ for some normal subgroup $K$ of $G$ is either trivial or injective. In particular, if it is trivial, then every $g \in G$ satisfies $\pi(g) = gK = K$ for every $g \in G$, so $g \in K$ for every $g \in G$, so $K = G$. If it is injective, then $K = \{1\}$; to see this suppose otherwise, such that $K$ contains both $1$ and some other distinct element $g \in G$ (we can do this since $G \neq \{1\}$). Then, $\pi(1) = K$ and $\pi(g) = gK = K$ since $K$ is closed under the operation, so $\pi$ is not injective; $\contra$, so $K = \{1\}$. These two are clearly mutually exclusive, so we have that the only possibilities for normal subgroups of $G$ are $\{1\}$ and $G$ itself, so it is simple.

\section*{Problem 9}

Consider that for any normal subgroup $H$ of $S_{n}$, $H \cap A_{n} \lhd A_{n}$ by past homework and also the notes. However, since the only normal subgroups of $A_{n}$ are itself and the trivial group, either $H \cap A_{n} = \{1\}$ or $H \cap A_{n} = A_{n}$. In the latter case, we have that $A_{n} \leq H \leq S_{n} \implies |A_{n}| \leq |H| \leq |S_{n}|$ since all of these groups are finite. Then, $|S_{n}|/|H| \leq |S_{n}|/|A_{n}| = 2$, but by Lagrange $|S_{n}|/|H|$ is integral, so $|S_{n}|/|H| = 1 \implies |H| = |S_{n}|$, and since these groups are finite and $H$ is contained in $S_{n}$, $|H| = |S_{n}| \implies H = S_{n}$. The second case is that $|S_{n}|/|H| = 2 \implies |H| = |A_{n}| = |H \cap A_{n}|$, but then $|H \cap A_{n}| = |H| \iff H \cap A_{n} = H$ for finite sets. In either case, either $H = S_{n}$ or $H = A_{n}$.

Then, if $H \cap A_{n} = \{1\}$, we have that $H$ must be composed of odd permutations only (and the identity). Then, consider that $\varepsilon: H \rightarrow \{\pm 1\}$ where $\varepsilon$ returns the sign of the permutation, is a homomorphism since it is a homomorphism on $S_{n}$. Clearly only one element of $H$ can be even (that is, the identity), since otherwise $H \cap A_{n}$ would contain multiple elements. Then, the kernel of $\varepsilon$ is exactly the identity, so $\varepsilon$ is injective, so $|H| \leq 2$ (otherwise, there are $> 2$ distinct nonidentity elements, all of which have sign $-1$, so $\varepsilon$ would not be injective). Then, by problem 5, $H \leq Z(G)$, but by an earlier homework (problem 8, HW 8), the only element that commutes with everything in $S_{n}$, $n \geq 3$ is the identity, so $H = \{1\}$. Thus, any normal subgroup of $S_{n}$ is either $1, A_{n}, S_{n}$ (for $n \geq 5$).

\end{document}
% LocalWords:  NetID fancyplain LocalWords colorlinks linkcolor linkbordercolor