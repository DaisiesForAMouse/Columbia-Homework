\documentclass[12pt,letterpaper]{article}
\usepackage{fullpage}
\usepackage[top=2cm, bottom=4.5cm, left=2.5cm, right=2.5cm]{geometry}
\usepackage{amsmath,amsthm,amsfonts,amssymb,amscd}
\usepackage{lastpage}
\usepackage{enumerate}
\usepackage{fancyhdr}
\usepackage{mathrsfs}
\usepackage{xcolor}
\usepackage{graphicx}
\usepackage{listings}
\usepackage{hyperref}
\usepackage{tikz}
\usepackage{relsize}
\usepackage{fancyvrb}
\usepackage{import}
\usetikzlibrary{shapes.geometric,fit}

\hypersetup{%
  colorlinks=true,
  linkcolor=blue,
  linkbordercolor={0 0 1}
}

\setlength{\parindent}{0.0in}
\setlength{\parskip}{0.05in}

\theoremstyle{definition}
\newtheorem*{statement}{Statement}
\newtheorem*{claim}{Claim}
\newtheorem*{theorem}{Theorem}
\newtheorem*{lemma}{Lemma}

\newcommand{\contra}{\Rightarrow\!\Leftarrow}
\newcommand{\R}{\mathbb{R}}
\newcommand{\F}{\mathbb{F}}
\newcommand{\Z}{\mathbb{Z}}
\newcommand{\Zeq}{\mathbb{Z}_{\geq 0}}
\newcommand{\Zg}{\mathbb{Z}_{>0}}
\newcommand{\Req}{\mathbb{R}_{\geq 0}}
\newcommand{\Rg}{\mathbb{R}_{>0}}
\newcommand{\N}{\mathbb{N}}
\newcommand{\Q}{\mathbb{Q}}
\newcommand{\C}{\mathbb{C}}

\newcommand{\incfig}[1] {%
    % \def\svgwidth{\columnwidth}
    \import{./figures/}{#1.pdf_tex}
}

\title{MATH 4041 HW 1}
\author{David Chen, dc3451}
\date{September 10, 2020}

\begin{document}

\maketitle

\section*{Problem 1}

In class it was stated that $X \subseteq Y \iff \forall x \in X, x \in Y$.

\subsection*{i}

Consider any $x_1 \in X_1$. Then, since $X_1 \cup X_2$ is defined to be $\{x \mid
x\in X_1 \lor x \in X_2\}$, we have that $x_1 \in X_1 \cup X_2$. From the
definition of $\subseteq$, we have that $X_1 \subseteq X_1 \cup X_2$.

Consider any $x_2 \in X_2$. Then, since $X_1 \cup X_2$ is defined to be $\{x \mid
x\in X_1 \lor x \in X_2\}$, we have that $x_2 \in X_1 \cup X_2$. From the
definition of $\subseteq$, we have that $X_2 \subseteq X_1 \cup X_2$.

To show that $X_1 \cup X_2$ is the smallest set containing $X_1, X_2$, consider
any $x \in X_1 \cup X_2$. In that case, via the definition of the union of two sets,
we have that $x \in X_1$ or $x \in X_2$ (or both). In either case, we have that
since $X_1 \subseteq Y \implies \forall x_1 \in X_1, x_1 \in Y$ and $X_2 \subseteq Y
\implies \forall x_2 \in X_2, x_2 \in Y$, that $x \in Y$. From the definition of
$\subseteq$, we have that $(X_1 \cup X_2) \subseteq Y$.

\subsection*{ii}

Consider any $x \in X_1 \cap X_2$. Then, since $X_1 \cap X_2$ is defined to be $\{x \mid
x\in X_1 \land x \in X_2\}$, we have that $x \in X_1 \cap X_2 \implies x \in
X_1$ and $x \in X_2$. From the definition of $\subseteq$, we have that $X_1 \cap X_2
\subseteq X_1, X_1 \cap X_2 \subseteq X_2$.

To show that $X_1 \cap X_2$ is the largest set contained in $X_1, X_2$, consider
any $y \in Y$. Then, we have that $Y \subseteq X_1 \implies y \in X_1$ and $Y
\subseteq X_2 \implies y \in X_2$. Since $X_1 \cap X_2$ is defined to be $\{x
\mid x \in X_1 \land x \in X_2\}$, we have that $y \in X_1 \cap X_2$. From the
definition of $\subseteq$, we have that $Y \subseteq (X_1 \cap X_2)$.

\section*{Problem 2}

Let $x_1, x_2$ be any two distinct elements of $X$. Then, $(x_1, x_2) \notin
\Delta_X$, (if it were in fact in the diagonal, we would have that $x_1 = x_2$, which contradicts the
earlier assumption).

However, we have that $(x_1, x_1), (x_2, x_2) \in \Delta_X$. Now, suppose that
$\Delta_X = A \times B$ for some $A,B \subseteq X$. Then, since $A \times B$ is
defined to be $\{(a,b) \mid a \in A, b \in B\}$, and $(x_1, x_1) \in A \times
B$, we must have that $x_1 \in A, x_1 \in B$. Similarly, since $(x_2, x_2) \in A \times
B$, we must have that $x_2 \in A, x_2 \in B$. 

Then, again recalling that $A \times B = \{(a,b) \mid a \in A, b \in B\}$, we
have that $(x_1, x_2) \in A \times B$. However, we showed earlier that $(x_1,
x_2) \notin \Delta_X$, so $\contra$, and $\Delta_X \neq A \times B$ for any $A,B
\subseteq X$.

\section*{Problem 3}

\subsection*{i}

$g(x) = e^x$ is injective, as the inverse function $g^{-1}(x) = \ln(x)$ shows.
It is not surjective, as $e^x > 0$ on the real line. Since it is not surjective,
it is not a bijection.

The image of $g$ is the positive reals $\R^+$.

\subsection*{ii}

$g(x) = 5x-12$ is a bijection: for injectivity, we see that if we have $x,y$
such that $g(x) = g(y)$, $5x - 12 = 5y - 12 \implies 5x = 5y \implies x = y$.

For surjectivity, we have that $f^{-1}(x) = \frac{x + 12}{5}$ gives a preimage
to any given $x \in \R$.

The image of $g$ is the entire real line $\R$.

\subsection*{iii}

$g(x) = x^3$ is also a bijection: for injectivity, we see that if we have $x,y$
such that $g(x) = g(y)$, $x^3 = y^3 \implies x = y$. For surjectivity, we have
that $g^{-1}(x) = x^{\frac{1}{3}}$ furnishes a preimage for any given input $x$.
{
The image of $g$ is the entire real line $\R$.

\subsection*{iv}

$g(x) = x^3 - 3x$ is not a bijection: it is surjective, but not injective. To
see that it is surjective, consider for any $y \in \R$, $g(|y| + 2) =
|y|^3 + 2|y|^2 + |y| + 8 \geq |y|$ and $g(-|y|
- 2) = -|y|^3 - 2|y|^2 - |y| - 8 \leq -|y|$. Intermediate value theorem gives
some preimage of $y$ between those $|y| + 2$ and $-|y| - 2$, since $-|y| \leq y \leq |y|$.

To see that it is not injective, note that $g(-\sqrt{3}) = g(0) = g(\sqrt{3}) =
0$.

The image of $g$ is the entire real line $\R$.

\begin{figure}[ht]
  \centering
  \incfig{3g}
  \caption{$g(x) = x^3 - 3x$}
\end{figure}

\section*{Problem 4}

$f$ is surjective, but not injective. Note that for any $n$, $n + 1$ is a
preimage of $n$, but $f(1) = f(31) = 30$.

\subsection*{a}

\[
  f(\{1,2,3,4,5\}) = \{30,1,2,3,4\}
\]

\subsection*{b}

\[
  f(\{1,31\}) = \{30\}
\]

\subsection*{c}

\[
  f^{-1}(1) = \{2\}
\]

\subsection*{d}

\[
  f^{-1}(\{1,2,3\}) = \{2,3,4\}
\]

\subsection*{e}

\[
  f^{-1}(30) = \{1,31\}
\]

\subsection*{f}

\[
  f^{-1}(\{1,30\}) = \{1,2,31\}
\]

\section*{Problem 5}

If $X$ is nonempty:

A constant function $f: X \rightarrow Y$ is surjective only if $Y$ is a
singleton set $\{c\}$ (otherwise, if $Y$ has more than one element, then those
other elements have no preimage).

$f$ is injective only if $X$ is a singleton set as well (otherwise, if $X$ has
more than one element, then for two distinct elements $x_1, x_2$, we have that
$f(x_1) = f(x_2)$).
\
Combining the two above conditions, we see that $f:X \rightarrow Y$ is bijective between two
nonempty sets if $X,Y$ both consist of a single element.

If $X = \emptyset$, then the empty function $f = \emptyset$ is constant
irregardless of $Y$ so long as $Y$ is nonempty. Picking any arbitrary element of
$Y$, we would have that $\forall x \in X, f(x) = c$.

Then, the empty function is always injective, as $\nexists x_1, x_2 \in
\emptyset$ such that $f(x_1) = f(x_2)$, but never surjective (as long as $Y$ is
nonempty) as no element in $Y$ can have a preimage in the empty set. Consequentially, the empty
function is never a bijection from the empty set to any nonempty range.

\section*{Problem 6}

Recall the definition of a function $f: X \rightarrow Y$ as a subset $G$ of $X \times Y$ such that
$\forall x \in X, \exists ! y \in Y$ such that $(x,y) \in G$.

Now consider the empty function with graph $\emptyset$. Then, the above
condition with quantifier $\forall x \in X$ holds vacuously (note that
$\emptyset \times Z = Z \times \emptyset = \emptyset$ for any set $Z$), and as
such the empty function is a function.

As mentioned earlier, the empty function is always injective, but only
surjective if $Y$ is also the empty set. Then, the empty function $f: \emptyset
\rightarrow Y$ is bijective only if $Y = \emptyset$.

For a function $f: X \rightarrow \emptyset$, note that if $X = \emptyset$, we
showed earlier that the empty function is such a function $\emptyset \rightarrow
\emptyset$.

Suppose that we have some function $f: X \rightarrow \emptyset$ and that $X$ is
nonempty. Then we have that for any given $x \in X$, there must be
some $(x,y) \in G_f \subset X \times \emptyset$. However, since $X \times \emptyset = \emptyset$, we have
that $\contra$, so no such function can exist.

\section*{Problem 7}

If either $X$ or $Y$ is the empty set, we have that $X \times Y = \emptyset$.
Then, from earlier, we showed that $\pi_1: X \times Y \rightarrow X$ is
well-defined only if $X = \emptyset$ (in which case it is a bijection), and that similarly, $\pi_2$ is
well-defined only if $Y = \emptyset$ (in which case it is a bijection).

Now, if neither is empty:

$\pi_1$ is injective when $Y$ is a singleton set; otherwise, let $y_1, y_2 \in
Y$ be distinct. Then, $\pi_1(x,y_1) = \pi_1(x,y_2) = x$.

Similarly, $\pi_2$ is injective when $X$ is a singleton set.

$\pi_1$ is always surjective, as $x$ has the preimage $\{(x,y) \mid y \in Y\}$
which is nonempty for nonempty $Y$.

Similarly, $\pi_2$ is also always surjective.


\section*{Problem 8}

Remember that $(x,y) \in X \times Y$ for \textit{any} $x \in X, y \in Y$. In
particular, note that this suggests that $(x,y) \in X \times Y \implies x \in X,
y \in Y \implies (y,x) \in Y \times X$.

To show surjectivity, note that for any element in the range $(y,x) \in Y \times
X$, we have that $(x,y) \in X \times Y$ satisfies $t(x,y) =
(y,x)$.

For injectivity, we will need to show that if $t(x_1,y_1) = t(x_2,y_2)$ then
$(x_1,y_1) = (x_2,y_2)$. Suppose that $t(x_1,y_1) = t(x_2,y_2)$. Then, we have
that $(y_1, x_1) = (y_2, x_2)$. By the definition of ordered pairs we have that
$y_1 = y_2$ and $x_1 = x_2$, which also gives that $(x_1,y_1) = (x_2,y_2)$,
which is what we wanted. Hence, $t$ is a bijection.

\section*{Problem 9}
\subsection*{i}

We have that $\chi_A^{-1}(0) = \{x \in X \mid \chi_A(x) = 0\}$. However, from
the definition of $\chi_A$, we have that $\chi_A(x) = 0 \iff x \notin A$. Then,
we have that $\chi_A^{-1}(0) = \{x \in X \mid x \notin A\}$, which is exactly $X
\setminus A$.

We have that $\chi_A^{-1}(1) = \{x \in X \mid \chi_A(x) = 1\}$. However, from
the definition of $\chi_A$, we have that $\chi_A(x) = 1 \iff x \in A$. Then,
we have that $\chi_A^{-1}(1) = \{x \in X \mid x \in A\} = \{x \in A\}$, which is
exactly $A$.

$\chi_A$ is a constant function when $X$ is empty, or when $A = X$ or $A =
\emptyset$. We showed
earlier that $\chi_A: \emptyset \rightarrow \{0,1\}$ is a constant function.

If $X$ is nonempty and $A$ is a proper nonempty subset of $X$, then let $a \in
A$ and $x \in X \setminus A$. Then, $\chi_A(a) = 1, \chi_A(x) = 0$, and so
$\chi_A$ is nonconstant.

Now we also verify that $\chi_A$ is constant for $A = X$ and $A = \emptyset$. In the first
case, $a \in X \implies a\in A$, so $\chi_A(x) = 1$ for every $x \in X$ if $A =
X$. Similarly, no $x \in X$ has $x \in \emptyset$, so $\chi_A(x) = 0$ for every $x
\in X$ if $A = \emptyset$.

Again, if $X = \emptyset$ then the function is injective but not surjective, as
shown earlier. Otherwise, $\chi_A$ is injective in the following cases:
\begin{enumerate}
  \item $X$ is a singleton. In this case, there is only one element of the domain, so
  the preimage of both 0 and 1 has at most one element, irregardless of $A$.
  \item $X$ has two elements, and $A$ has one element. Then, for the unique $a
    \in A$, we have that $\chi_A(a) = \{1\}$ and the preimage of the other
    element in $X$ is $\{0\}$.
\end{enumerate}

 If $X$ has more than 2 elements, then one of the two sets $A$ and $X
 \setminus A$ has at least 2 elements. Then, letting $b_1, b_2$ be two
 distinct elements of the larger of $A$ and $X \setminus A$, we have that
 $\chi_A(b_1) = \chi_A(b_2)$.

 $X$ is surjective only if $A \neq \emptyset$ and $A \neq X$. To show
 this, we have that $A$ is nonempty, and that there exists some $x \in X$
 that also has $x \notin A$. Then, we have that $\chi_A(a) = 1$ for some $a \in
 A$, and that $\chi_A(x) = 0$, which covers the entire range.

 Now, if $X$ is a singleton, then the only subsets of $X$ are $\emptyset$ and
 $X$ itself, so $\chi_A$ cannot be surjective. Then, for $\chi_A$ to be
 surjective, we must have that $X$ contains two elements, and that $A$ must
 contain exactly one element. This satisfies the conditions for being both
 injective and surjective, such that $\chi_A$ is a bijection.

\subsection*{ii}

Take any $x \in X$. Then, we have two cases: $x \in S_f$ or $x \notin S_f$. In
the first case, we have that $x \in S_f \implies x \in f^{-1}(1) \implies f(x) =
1 = \chi_{S_f}(x)$. In the second case, we have that $x \notin S_f \implies x \notin f^{-1}(1)$;
however, since every element of the domain gets mapped to some element of the
range, we have that $f(x) = 0 = \chi_{S_f}(x)$.

Thus, for every $x \in X$, we have that $f(x) = \chi_{S_f}(x)$.

\subsection*{iii}

We have already shown that $X_{S_f} = f$ in the last part.

% To show that $S_{\chi_A} = A$, we have that $x \in S_{\chi_A} \implies \chi_A(x)
% = 1$ from the earlier shown relationship with $f = \chi_A$. Then, $\chi_A(x) = 1
% \implies x \in A$ from definition of the characteristic function, and so
% $S_{\chi_A} \subseteq A$.

% Similarly, if $x \in A$, then we have that $\chi_A(x) = 1 \implies x \in
% \chi_A^{-1}(1) = f^{-1}(1) = S_{\chi_A}$, and so $A \subseteq S_{\chi_A}$. However, since
% $S_{\chi_A} \subseteq A$, we have that $A = S_{\chi_A}$.

The first
part (i) showed that $\chi_A^{-1}(1) = A$, and the second part (ii) has by definition of
$S_{\chi_A}$ that $\chi_A^{-1}(1) = S_{\chi_A}$. Then, we have that
\[
  A = \chi_A^{-1}(1) = S_{\chi_A}.
\]

\end{document}
% LocalWords:  NetID fancyplain LocalWords colorlinks linkcolor linkbordercolor
