\documentclass[12pt,letterpaper]{article}
\usepackage{fullpage}
\usepackage[top=2cm, bottom=4.5cm, left=2.5cm, right=2.5cm]{geometry}
\usepackage{amsmath,amsthm,amsfonts,amssymb,amscd}
\usepackage{lastpage}
\usepackage{enumerate}
\usepackage{fancyhdr}
\usepackage{mathrsfs}
\usepackage{xcolor}
\usepackage{graphicx}
\usepackage{mathdots}
\usepackage{listings}
\usepackage{hyperref}
\usepackage{tikz}
\usepackage{relsize}
\usepackage{fancyvrb}
\usepackage{import}
\usetikzlibrary{shapes.geometric,fit}

\hypersetup{%
  colorlinks=true,
  linkcolor=blue,
  linkbordercolor={0 0 1}
}

\setlength{\parindent}{0.0in}
\setlength{\parskip}{0.05in}

\theoremstyle{definition}
\newtheorem*{statement}{Statement}
\newtheorem*{claim}{Claim}
\newtheorem*{theorem}{Theorem}
\newtheorem*{lemma}{Lemma}

\newcommand{\contra}{\Rightarrow\!\Leftarrow}
\newcommand{\R}{\mathbb{R}}
\newcommand{\F}{\mathbb{F}}
\newcommand{\Z}{\mathbb{Z}}
\newcommand{\Zeq}{\mathbb{Z}_{\geq 0}}
\newcommand{\Zg}{\mathbb{Z}_{>0}}
\newcommand{\Req}{\mathbb{R}_{\geq 0}}
\newcommand{\Rg}{\mathbb{R}_{>0}}
\newcommand{\N}{\mathbb{N}}
\newcommand{\Q}{\mathbb{Q}}
\newcommand{\C}{\mathbb{C}}
\DeclareMathOperator{\Id}{id}
\DeclareMathOperator{\Aut}{Aut}
\DeclareMathOperator{\Stab}{Stab}
\DeclareMathOperator{\lcm}{lcm}

\newcommand{\incfig}[1] {%
    % \def\svgwidth{\columnwidth}
    \import{./figures/}{#1.pdf_tex}
}

\title{MATH 4041 HW 13}
\author{David Chen, dc3451}
\date{\today}

\begin{document}

\maketitle

\section*{Problem 1}

We have that
\[
  g_{1} \cdot (g_{2} \cdot (x_{1}, x_{2})) = g_{1} \cdot (g_{2} \cdot x_{1}, g_{2} \cdot x_{2}) = (g_{1} \cdot (g_{2} \cdot x_{1}), g_{1} \cdot (g_{1} \cdot x_{2}))
\]
and since $X_{1}, X_{2}$ are $G-$sets, we have that this becomes
\[
  (g_{1} \cdot (g_{2} \cdot x_{1}), g_{1} \cdot (g_{1} \cdot x_{2})) = ((g_{1}g_{2}) \cdot x_{1}, (g_{1}g_{2}) \cdot x_{2}) = (g_{1}g_{2}) \cdot (x_{1}, x_{2})
\]
which is what we want.

Checking the identity,
\[
  1 \cdot (x_{1}, x_{2}) = (1 \cdot x_{1}, 1 \cdot x_{2}) = (x_{1}, x_{2})
\]

Any element that fixes $(x_{1}, x_{2})$ under the group action must fix both $x_{1}$ and $x_{2}$, since
\[
  g \cdot (x_{1}, x_{2}) = (g \cdot x_{1}, g \cdot x_{2}) = (x_{1}, x_{2}) \implies g \cdot x_{1} = x_{1}, g\cdot x_{2} = x_{2}
\]
so $g \in G_{x_{1}}, G_{x_{2}}$. Clearly, if $g \in G_{x_{1}}, G_{x_{2}}$ then we get that
\[
  g \cdot (x_{1}, x_{2}) = (g \cdot x_{1}, g \cdot x_{2}) = (x_{1}, x_{2})
\]
so the stabilizer of $(x_{1}, x_{2})$ under $G$ is exactly $G_{x_{1}} \cap G_{x_{2}}$.

\section*{Problem 2}

\begin{enumerate}
  % \item Any element in $G \cdot (1,2,3)$ must be some permutation of $(1,2,3)$, in that if $\sigma \cdot (1,2,3) = (a,b,c)$, $a \neq b \neq c$: if they were, then $\sigma$ would take two distinct elements of $\{1,2,3\}$ to the same output, so it is not bijective. On the other hand, if we have some ordering of $(1,2,3)$, say $(a,b,c)$, $a \neq b \neq c$, then $\sigma$ which takes $1 \mapsto a$, $2 \mapsto b$, $3 \mapsto c$ gives an element of $S_{3}$ that $\sigma \cdot (1,2,3) = (a,b,c)$. Then,
  \item We can just compute these directly; after all, there are only 6 elements in $|S_{3}|$.
        \[
          S_{3} \cdot (1,2,3) = \{(1,2,3), (1,3,2), (2,1,3), (3,1,2), (2,3,1), (3,2,1)\}
        \]
  which has order $6$, which divides $|S_{3}| = 6$ as well. Any element in the stabilizer of $(1,2,3)$ needs $(\sigma(1), \sigma(2), \sigma(3)) = (1,2,3)$, so $\sigma = 1$ and the stabilizer is trivial.
  \item
        \[
          S_{3} \cdot (1,1,2) = \{(1,1,2), (1,1,3), (2,2,1), (2,2,3), (3,3,1), (3,3,2)\}
        \]
  which has order $6$ dividing $|S_{3}|$, and any element in the stabilizer has $\sigma(1) = 1, \sigma(2) = 2 \implies \sigma(3) = 3$ and so must be the identity, so the stabilizer is trivial.
  \item
        \[
          S_{3} \cdot (1,1,1) = \{(1,1,1), (2,2,2), (3,3,3)\}
        \]
  which has order $3$ dividing $|S_{3}| = 6$, and the stabilizer must satisfy that $\sigma(1) = 1$, so
        \[
          (S_{3})_{(1,1,1)} = \left\{1, (2,3)\right\}
        \]
\end{enumerate}

\section*{Problem 3}

\subsection*{i}

We need to check that
\[
  \tau \cdot (\sigma \cdot (x_{1}, x_{2}, x_{3})) = P(\tau) (P(\sigma)(x_{1}e_{1} + x_{2}e_{2} + x_{3}e_{3})) = P(\tau)P(\sigma)(x_{1}e_{1} + x_{2}e_{2} + x_{3}e_{3}) = (\tau \sigma) \cdot (x_{1}, x_{2}, x_{3})
\]
but that $P(\tau)P(\sigma) = P(\tau\sigma)$ is a basic fact of permutation matrices, and associativity of matrix multiplication gives the rest of what we want. More explicitly,
\begin{align*}
  P(\tau)(P(\sigma)(x_{1}e_{1} + x_{2}e_{2} + x_{3}e_{3})) &= P(\tau)(x_{1}e_{\sigma(1)} + x_{2}e_{\sigma(2)} + x_{3}e_{\sigma(3)})) \\
                                                           &= x_{1}e_{\tau(\sigma(1))} + x_{2}e_{\tau(\sigma(2))} + x_{3}e_{\tau(\sigma(3))} \\
                                                           &= P(\tau\sigma)(x_{1}e_{1} + x_{2}e_{2} + x_{3}e_{3})
\end{align*}
which was what we wanted. Then, the identity permutation clearly maps $(x_{1}, x_{2}, x_{3}) \mapsto (x_{1}, x_{2}, x_{3})$ since the preimage of any element is the element itself.

\begin{enumerate}
  \item We can again just compute these directly.
        \[
          S_{3} \cdot (1,2,3) = \{(1,2,3), (1,3,2), (2,1,3), (3,1,2), (2,3,1), (3,2,1)\}
        \]
        which has order $6$ dividing $|S_{3}| = 6$. The stabilizer is then of order 1, and thus only the identity.
  \item
        \[
          S_{3} \cdot (1,1,2) = \{(1,1,2), (1,2,1), (2,1,1)\}
        \]
        which has order $3$ dividing $|S_{3}| = 6$. The stabilizer then satisfies that $\sigma^{-1}(1), \sigma^{-1}(2) \in \{1,2\}$ and $\sigma^{-1}(3) = 3$. This is then two elements of $S_{3}$, and
        \[
          (S_{3})_{(1,1,2)} = \{1, (1,2)\}
        \]
  \item
        \[
          S_{3} \cdot (1,1,1) = \{1,1,1\}
        \]
        which has order $1$ dividing $|S_{3}| = 6$. The stabilizer must be of order $6$ and since it is a subgroup of $S_{3}$, a finite group, must be all of $S_{3}$ (which is also apparent since every permutation in $S_{3}$ fixes $(1,1,1)$, since the orbit is extactly itself).
        \[
          (S_{3})_{(1,1,2)} = S_{3}
        \]
\end{enumerate}

\subsection*{ii}

\begin{enumerate}
  \item The orbit of $(1,2,3,4)$ is any $(a,b,c,d) \in \{1,2,3,4\}^{4} \subset \R^{4}$, where $a \neq b \neq c \neq d$. In particular, the element of $S_{4}$ which moves $(1,2,3,4)$ to $(a,b,c,d)$ is the inverse to the one taking $1 \mapsto a, 2 \mapsto b, 3 \mapsto c, 4 \mapsto d$, so the orbit is of order 16. Furthermore, any element $\sigma \in S_{4}$ fixing $(1,2,3,4)$ must satisfy that $\sigma^{-1}(1) = 1, \sigma^{-1}(2) = 2,\sigma^{-1}(3) = 3,\sigma^{-1}(4) = 4$. This clearly shows that $\sigma = 1$, so the stabilizer is exactly the identity element (which gives us another way to compute the orbit's size as $|S_{4}|/1 = 16$).
  \item The stabilizer of $(1,1,2,2)$ is any $\sigma$ that is the composition of $(1,2)$ and $(3,4)$, since $\sigma$ can swap the first two elements and the second two elements, but nothing else. This gives a total of $4$ elements in the stabilizer and thus $|S_{4}|/4 = 4$ elements in the orbit (also since the orbit is just all permutations, it is also $\frac{4!}{2!2!} = 4$).
  \item The stabilizer of $(1,1,1,1)$ is any $\sigma \in S_{4}$, since any permutation of $(1,1,1,1)$ is itself $(1,1,1,1)$. This gives a total of $16$ elements in the stabilizer and thus $|S_{4}|/16 = 1$ elements in the orbit, which is just $\{(1,1,1,1)\}$.
\end{enumerate}

\section*{Problem 4}

Let $A$ be an element of $O_{n}$ which fixes $e_{3}$, and let the element in the $i^{th}$ row and $j^{th}$ column be $a_{ij}$. Then, basic matrix multiplication gives
\[
  Ae_{n} = (a_{1n},a_{2n}, \dots, a_{nn}) = e_{n} \implies a_{1n} = a_{2n} = \dots = a_{(n-1)n} = 0, a_{nn} = 1
\]
but since $A$ is orthonormal, $\sum_{i=1}^{n}a_{ni} = a_{nn} + \sum_{i=1}^{n-1}a_{ni} = 1 \implies a_{n1} = a_{n2} = \dots = a_{n(n-1)} = 1$, so
\[
  A = \begin{bmatrix} a_{11} & a_{12} & \cdots & 0 \\  a_{21} & a_{22} & \cdots & 0 \\ \vdots & \vdots & \ddots & 0 \\ 0 & 0 & 0 & 1 \end{bmatrix} = \begin{bmatrix} & & & 0 \\ & B & & 0 \\ & & & 0 \\ 0 & 0 & 0 & 1 \end{bmatrix}
\]
where $B$ is some $n - 1 \times n - 1$ matrix. Then, we have that $B$ must be orthonormal, since the length of the first $n-1$ row vectors of $A$ are the same as the lengths of the row vectors of $B$, so $B$ must have row vectors of length 1. Furthermore, the dot product between any of the first $n - 1$ row vectors of $A$ is the same as the dot product between the respective row vectors of $B$, so the row vectors of $B$ must be orthgonal (and from earlier, orthonormal). Then, $B \in O_{n-1}$ gives us our $A$. Further, any $B \in O_{n - 1}$ embeded as above into an $n \times n$ matrix gives the rows as an orthonormal basis since the lengths are all 1 and the dot product of any two distinct rows must be 0.

In particular, $A$ fixes the subspace spanned by $e_{n}$, but moves the subspace spanned by $e_{1}, \dots, e_{n-1}$ by $B$, so in some sense it is easily identifiable with an element of $O_{n-1}$.

Now, the mapping $f: (O_{n})_{e_{n}} \rightarrow O_{n-1}$ given by taking $A$ to the $n-1 \times n-1$ submatrix as above is well-defined and bijective; we can also check easily that
\[
  \begin{bmatrix} & & & 0 \\ & B_{1} & & 0 \\ & & & 0 \\ 0 & 0 & 0 & 1 \end{bmatrix} \begin{bmatrix} & & & 0 \\ & B_{2} & & 0 \\ & & & 0 \\ 0 & 0 & 0 & 1 \end{bmatrix} = \begin{bmatrix} & & & 0 \\ & B_{1}B_{2} & & 0 \\ & & & 0 \\ 0 & 0 & 0 & 1 \end{bmatrix}
\]
so $f$ is an isomorphism between the stabilizer of $e_{n}$ and $O_{n-1}$.

\section*{Problem 5}

Pick some generator $xH \in G/H$, $x \in G$. Then, we have that every element $g \in G$ is contained in some coset $H' \in G/H$, and if $H' = (xH)^{n} = x^{n}H$, then $g = x^{n}h$ for some $h \in H$ and some positive integer $n$. Now consider any two elements of $G$, represented as $x^{n_{1}}h_{1}$, $x^{n_{1}}h_{2}$. Then, since $H \leq Z(G)$, $h_{1}$ and $h_{2}$ commute with every element of $G$, so
\[
  x^{n_{1}}h_{1}x^{n_{2}}h_{2} = x^{n_{1}}x^{n_{2}}h_{1}h_{2} = x^{n_{1} + n_{2}}h_{1}h_{2} = x^{n_{2}}x^{n_{1}}h_{2}h_{1} = x^{n_{1}}h_{2}x^{n_{1}}h_{1}
\]
so $G$ is abelian.

\section*{Problem 6}

We have that $kh^{-1}k^{-1} \in H$ since $H$ is normal, and similarly $hkh^{-1} \in K$. Then, $hkh^{-1}k^{-1} = hh' = k'k$ for some $h' \in H, k' \in K$. Then, $hkh^{-1}k^{-1} \in H \cap K \implies hkh^{-1}k^{-1} = 1 \implies (hkh^{-1}k^{-1})kh = hk = kh$.

\section*{Problem 7}

This is just the right coset $Hg$. In particular, the it is the set (by definition) $\{h \cdot g \mid h \in H\}$, but the group action is just left multiplication, so it is $\{hg \mid h \in H\}$, which is $Hg$. The action is transitive when $Hg = G$ for some $g$, but this means that $1 \in Hg \implies Hg = H = G$, so the action is transitive only when $H$ is the entire group $G$. The stabilizer of $g$ is trivial; $hg = g \implies h = 1$. This can also be noted for finite groups since $|Hg| = |H|$ since all cosets are the same size, but then $|H_{g}| = |Hg|/|H| =1$.

\end{document}
% LocalWords:  NetID fancyplain LocalWords colorlinks linkcolor linkbordercolor