\documentclass[12pt,letterpaper]{article}
\usepackage{fullpage}
\usepackage[top=2cm, bottom=4.5cm, left=2.5cm, right=2.5cm]{geometry}
\usepackage{amsmath,amsthm,amsfonts,amssymb,amscd}
\usepackage{lastpage}
\usepackage{enumerate}
\usepackage{fancyhdr}
\usepackage{mathrsfs}
\usepackage{xcolor}
\usepackage{graphicx}
\usepackage{listings}
\usepackage{hyperref}
\usepackage{tikz}
\usepackage{relsize}
\usepackage{fancyvrb}
\usepackage{import}
\usetikzlibrary{shapes.geometric,fit}

\hypersetup{%
  colorlinks=true,
  linkcolor=blue,
  linkbordercolor={0 0 1}
}

\setlength{\parindent}{0.0in}
\setlength{\parskip}{0.05in}

\theoremstyle{definition}
\newtheorem*{statement}{Statement}
\newtheorem*{claim}{Claim}
\newtheorem*{theorem}{Theorem}
\newtheorem*{lemma}{Lemma}

\newcommand{\contra}{\Rightarrow\!\Leftarrow}
\newcommand{\R}{\mathbb{R}}
\newcommand{\F}{\mathbb{F}}
\newcommand{\Z}{\mathbb{Z}}
\newcommand{\Zeq}{\mathbb{Z}_{\geq 0}}
\newcommand{\Zg}{\mathbb{Z}_{>0}}
\newcommand{\Req}{\mathbb{R}_{\geq 0}}
\newcommand{\Rg}{\mathbb{R}_{>0}}
\newcommand{\N}{\mathbb{N}}
\newcommand{\Q}{\mathbb{Q}}
\newcommand{\C}{\mathbb{C}}
\DeclareMathOperator{\Id}{id}
\DeclareMathOperator{\lcm}{lcm}

\newcommand{\incfig}[1] {%
    % \def\svgwidth{\columnwidth}
    \import{./figures/}{#1.pdf_tex}
}

\title{MATH 4041 HW 6}
\author{David Chen, dc3451}
\date{\today}

\begin{document}

\maketitle

\section*{Problem 1}

\begin{align*}
  \begin{bmatrix}
    0 & -1 \\
    1 & 0
  \end{bmatrix}
  \begin{bmatrix}
    0 & -1 \\
    1 & 0
  \end{bmatrix} &=
  \begin{bmatrix}
    0 - 1 & 0 + 0 \\
    0 + 0 & -1 + 0
  \end{bmatrix} \\
  &=
  \begin{bmatrix}
    -1 & 0 \\
    0 & -1
  \end{bmatrix} \\
      &= -I
\end{align*}

Then, we have that $A^{2} = -I \implies A^{3} = -(IA) = -A \neq I$ and then $A^{4} = A^{2}A^{2} = (-I)(-I)=  I$, so the order of $A$ is 4.

\begin{align*}
  \begin{bmatrix}
    0 & 1 \\
    -1 & -1
  \end{bmatrix}
  \begin{bmatrix}
    0 & 1 \\
    -1 & -1
  \end{bmatrix} &=
  \begin{bmatrix}
    0 - 1 & 0 -1 \\
    0 + 1 & -1 + 1
  \end{bmatrix} \\
  &=
  \begin{bmatrix}
    - 1 & -1 \\
    1 & 0
  \end{bmatrix} \\
  \begin{bmatrix}
    - 1 & -1 \\
    1 & 0
  \end{bmatrix}
  \begin{bmatrix}
    0 & 1 \\
    -1 & -1
  \end{bmatrix} &=
  \begin{bmatrix}
    0 + 1 & -1 + 1 \\
    0 + 0 & 1 + 0
  \end{bmatrix} \\
      &=
  \begin{bmatrix}
    1 & 0 \\
    0 & 1
  \end{bmatrix} \\
  &= I
\end{align*}
so we have that $B^{3} = I$.

\begin{align*}
  \begin{bmatrix}
    0 & -1 \\
    1 & 0
  \end{bmatrix}
  \begin{bmatrix}
    0 & 1 \\
    -1 & -1
  \end{bmatrix} &=
  \begin{bmatrix}
    0 + 1 & 0 + 1 \\
    0 + 0 & 1 + 0
  \end{bmatrix} \\
      &=
  \begin{bmatrix}
    1 & 1 \\
    0 & 1
  \end{bmatrix} \\
\end{align*}
as desired.

Then,
\begin{align*}
  \begin{bmatrix}
    1 & 1 \\
    0 & 1
  \end{bmatrix}
  \begin{bmatrix}
    1 & 1 \\
    0 & 1
  \end{bmatrix} &=
  \begin{bmatrix}
    1 + 0 & 1 + 1 \\
    0 + 0 & 0 + 1
  \end{bmatrix} \\
  &=
  \begin{bmatrix}
    1 & 2 \\
    0 & 1
  \end{bmatrix}
\end{align*}

We can show inductively that \((AB)^{n} = \begin{bmatrix} 1 & n \\  0 & 1 \end{bmatrix}\). Clearly this holds for the $n = 1$ case, since we already showed that \((AB)^{1} = AB = \begin{bmatrix} 1 & 1 \\  0 & 1 \end{bmatrix}\). Then, if \((AB)^{n} = \begin{bmatrix} 1 & n \\  0 & 1 \end{bmatrix}\), then
\begin{align*}
  \begin{bmatrix}
    1 & n \\
    0 & 1
  \end{bmatrix}
  \begin{bmatrix}
    1 & 1 \\
    0 & 1
  \end{bmatrix} &=
  \begin{bmatrix}
    1 + 0 & 1 + n \\
    0 + 0 & 0 + 1
  \end{bmatrix} \\
  &=
  \begin{bmatrix}
    1 & n+1 \\
    0 & 1
  \end{bmatrix}
\end{align*}
so this holds for \(n+1\), and by induction for all \(n \in \N\). Therefore, \(AB\) cannot be of finite order, as we have that if \(AB\) has order \(n\), \((AB)^{n} = I \implies \begin{bmatrix} 1 & n \\  0 & 1 \end{bmatrix} = \begin{bmatrix} 1 & 0 \\  0 & 1 \end{bmatrix} \implies n = 0\), and since order is defined to be a positive integer, \(n > 0\). \(\contra\), so \(AB\) does not have finite order.

\section*{Problem 2}

\subsection*{i}

We can compute all of these directly:
\begin{align*}
  ([1],[1]) + ([1],[1]) &= ([2],[2]) \\
                        &= ([0],[0]) \\
  \intertext{So \(([1],[1])\) has order \(2\) in \((\Z/2\Z) \times (\Z/2\Z)\).}
  ([1],[1]) + ([1],[1]) &= ([2],[2]) \\
                        &= ([0],[2]) \\
  ([0],[2]) + ([1],[1]) &= ([1],[3]) \\
                        &= ([1],[0]) \\
  ([1],[0]) + ([1],[1]) &= ([2],[1]) \\
                        &= ([0],[1]) \\
  ([0],[1]) + ([1],[1]) &= ([1],[2]) \\
  ([1],[2]) + ([1],[1]) &= ([2],[3]) \\
                        &= ([0],[0]) \\
  \intertext{So \(([1],[1])\) has order \(6\) in \((\Z/2\Z) \times (\Z/3\Z)\).}
  ([1],[1]) + ([1],[1]) &= ([2],[2]) \\
  ([2],[2]) + ([1],[1]) &= ([3],[3]) \\
  ([3],[3]) + ([1],[1]) &= ([4],[4]) \\
                        &= ([0],[4]) \\
  ([0],[4]) + ([1],[1]) &= ([1],[5]) \\
  ([1],[5]) + ([1],[1]) &= ([2],[6]) \\
  ([2],[6]) + ([1],[1]) &= ([3],[7]) \\
  ([3],[7]) + ([1],[1]) &= ([4],[8]) \\
                        &= ([0],[0]) \\
  \intertext{So \(([1],[1])\) has order \(8\) in \((\Z/4\Z) \times (\Z/8\Z)\).}
  ([2],[4]) + ([2],[4]) &= ([4],[8]) \\
                        &= ([0],[0]) \\
  \intertext{So \(([2],[4])\) has order \(2\) in \((\Z/4\Z) \times (\Z/8\Z)\).}
\end{align*}

\subsection*{ii}

First, we show that in any group \(G\), if \(g\) has order \(n\), \(g^{k} = 1 \iff n \mid k\).

\((\implies)\) Suppose that $n \nmid k$, such that $k = nq + r$ for some $0 \leq r \leq n - 1$. Then, $1 = g^{k} = g^{nq + r} = g^{nq}g^{r} = (g^{n})^{q}g^{r} = g^{r} = 1$. Then, we have that $g^{r} = 1$ for $r < n$, and so $g$ cannot be of order $n$. $\contra$, so $n \mid k$.

\((\impliedby)\) We have that $n \mid k \implies k = nq$ for some $q \in \Z$. Then, $g^{k} = g^{nq} = (g^{n})^{q} = 1^{q} = 1$.

The order of \(g,h\) will be \(\lcm(n,m)\), or the least common multiple of \(n\) and \(m\). To see this, note that if we have \((g,h)^{k} = (e_{G}, e_{H})\) where \(e_{G}, e_{H}\) are the identities of \(G\) and \(H\) respectively, we have that
\[
  (g,h)^{k} = (g^{k},h^{k}) = (e_{G}, e_{H}) \implies g^{k} = e_{G}, h^{k} = e_{H}
\]
However, from above, we have that this holds if and only if $n \mid k$ and $m \mid k$. Since the lcm of $n, m$ is exactly the least positive integer $k$ which satisfies $n \mid k$ and $m \mid k$ and the order is the least positive integer which satisfies the above relation, the order must be the lcm of $n, m$.

\section*{Problem 3}

We can see that the torsion subgroup of $\Z/n\Z$ is exactly $\Z/n\Z$ itself: note that $[m] \in \Z/\Z$ satisfies that
\[
  n[m] = \left[nm\right] = [0]
\]
so each element has order at most $n$, and thus has finite order.

The torsion subgroup of $\Z$ is exactly $\{0\}$. No other element has finite order (but $0$ as the identity has order $1$). To see this, $n \in \Z$, $n \neq 0$ alongside some $m \in \Z$, $m > 0$ satisfies that $n < 0 \implies nm < 0$ and $n > 0 \implies nm > 0$, so $nm \neq 0$ and $n$ does not have finite order in $\Z$.

The torsion subgroup of $\Z \times (\Z/n\Z)$ is exactly all elements of the form $(0,[m])$ for $[m] \in \Z/n\Z$. To see this, we have that this has order at most $n$ by the observation that $0$ has order $1$ and from above that $[m]$ has order at most $n$, so we have from problem 1 that $(0,m)$ has order at most $\lcm(1,n) = n$. Further, we can see that for $a \neq 0$, $(a,[m])^{k} = (ka, k[m]) = (0, [0])$ yields that $a$ has order $k$, which is impossible, so we have that $(a,[m])$ cannot have finite order for $a \neq 0$.

\section*{Problem 4}

$\mu_{\infty}$ cannot be cyclic. Suppose that $\mu_{\infty}$ was in fact generated by some element $\mu'$. In particular, since $\mu' \in \mu_{\infty}$, $\mu'$ must have finite order, say $n$. Then, we have that for any $k$, $(\mu'^{k})^{n} = (\mu'^{n})^{k} = 1^{k} = 1$, so every element in $\langle \mu' \rangle$ must have order at most $n$. However, we can very easily see that $\mu_{\infty}$ contains an element of order $2n$, say $e^{\pi i / n} \in \mu_{2n}$. Thus, $\mu'$ cannot generate the entirety of $\mu_{\infty}$.

\section*{Problem 5}

\subsection*{i}

Since we have for any $g \in G$ that $f(g) = f(g \cdot 1) = f(g) \cdot f(1)$ and $f(g) = f(1 \cdot g) = f(1) \cdot f(g)$, we have that $f(1) \cdot f(g) = f(g) = f(g) \cdot f(1)$, so $f(1)$ is an identity in $G_{2}$, as any element in $G_{2}$ can be represented as $f(g)$ for $g \in G_{1}$ since $f$ is surjective. Then, since identities are unique in groups, we have that $f(1) = 1$ as \textit{the} identity.

\subsection*{ii}

\[
  f(1) = 1 \implies 1 = f(g^{-1}g) = f(g^{-1})f(g), 1 = f(gg^{-1}) = f(g)f(g^{-1}) \implies f(g^{-1}) = (f(g))^{-1}
\]
by definition and uniqueness of inverses.

\subsection*{iii}

$\implies$ That $f(H)$ is a subgroup of $G_{2}$ was also the last problem on the last problem set.

We need to show that $f(H)$ contains the identity, inverses, and is closed. Since $H$ is a subgroup, it satisfies all three of those things. Then, $1 \in H \implies f(1) = 1 \in f(H)$, $g \in H \implies g^{-1} \in H$, so $f(g) \in f(H) \implies f(g^{-1}) = (f(g))^{-1} \in H$, and lastly, $g,h \in H \implies gh \in H \implies f(gh) = f(g)f(h) \in H$. Thus, $f(H)$ is a subgroup.

$\impliedby$ Applying the first part with $f^{-1}:G_{2} \rightarrow G_{1}$, we have that $f^{-1}(f(H))$ is a subgroup of $G_{1}$; all we need to show is that $f^{-1}(f(H)) = H$ as sets to show that they are the same subgroup. To see this, note that $h \in H \implies f(h) \in f(H) \implies f^{-1}(f(h)) = h \in f^{-1}(f(H))$, and $h \in f^{-1}(f(H)) \implies \exists h = f^{-1}(h')$ for some $h' \in f(H)$, and $h' \in f(H) \implies h' = f(h'')$ for some $h'' \in H$. Then, $h = f^{-1}(f(h'')) = h'' \in H$, so we have that $H \subseteq f^{-1}(f(H))$ and $f^{-1}(f(H)) \subseteq H$, so the two sets are equal and we are done.

\subsection*{6}

Note that a permutation of a set $X$ is defined exactly to be a bijection from $X$ to $X$.

We need to show that $H_{n+1}$ contains the identity, inverses, and is closed. In particular, we have that $\Id \in H_{n+1}$, as we have that $\Id(n+1) = n+1$ as desired (in particular, $\Id: S_{n+1} \rightarrow S_{n+1}$ takes $x \mapsto x$ for any $x$ so it is clearly a bijection, and $f(\Id(x)) = f(x) = \Id(f(x))$). Further, we have that for $f,g \in S_{n+1}$, $(f \circ g)(n+1) = f(g(n+1)) = f(n+1) = n + 1$, so we have that $f \circ g \in H_{n+1}$ as well (we know that $f\circ g$ as the composition of two bijections is itself a bijection).

The last thing to handle is inverses. Since $f \in H_{n+1}$ is a bijection, there is clearly some inverse $f^{-1} \in S_{n+1}$ that is bijective as well; further, since $f(n+1) = n + 1$, we get that $f^{-1}(f(n+1)) = f^{-1}(n+1) \implies n + 1 = f^{-1}(n+1)$, so $f^{-1} \in H_{n+1}$ as well.

Then, an isomorphism $\phi$ from $S_{n}$ to $H_{n+1}$ can be given as
\[
  \phi(f) = g, \text{ where } g(m) = \begin{cases}
    f(m) & 1 \leq m \leq n \\
    n + 1 & m = n + 1
  \end{cases}
\]

We need to first show that $g$ is in fact a bijection: we have that the unique preimage of $n+1$ is $n+1$, as $1 \leq f(m) \leq n$. Then, each $k$ where $1 \leq k \leq n$ also has a unique preimage, given by $f^{-1}(k)$, which we know exists since $f \in S_{n} \implies f$ is a bijection. Then, $g$ has an inverse given by
\[
  g^{-1}(m) = \begin{cases}
    f^{-1}(m) & 1 \leq m \leq n \\
    n + 1 & m = n + 1
  \end{cases}
\]
and is then a bijection, so $\phi$ is well-defined.

Now, we need to show that $\phi$ is a bijection. For $f, f' \in S_{n}$, $\phi(f) = \phi(f')$ implies that for every $m$, where $1 \leq m \leq n$, $(\phi(f))(m) = (\phi(f'))(m) \implies f(m) = f'(m)$, and since $f, f'$ have domain $\{1, 2, \dots, n\}$, $f = f'$ and $\phi$ is injective.

Similarly, every $f \in H_{n+1}$ must have a preimage: since $f$ is a bijection with $f(n+1) = n+1$, for any $m$, $1 \leq m \leq n \iff 1 \leq f(m) \leq n$ (to see $\implies$, supposing otherwise gives $f(m) = n + 1$, contradicting that $f$ is injective, and to see $\impliedby$, supposing otherwise gives $m = n + 1 \implies f(m) = n + 1$, contradicting that $1 \leq f(m) \leq n$). Then, any $m'$, $1 \leq m' \leq n$ must have a preimage $f^{-1}(m)$ (since $f$ is surjective) which must satisfy $1 \leq f^{-1}(m) \leq n$ as before, and must be unique since $f$ is surjective. Then, the restriction $f|_{\{1,2,\dots,n\}}$ is a bijection from $\{1,2,\dots,n\}$ to itself, and $\phi(f|_{\{1,2,\dots,n\}}) = f$, so $\phi$ is surjective.

Finally, given $f,f' \in S_{n}$, we have that
\[
  (\phi(f \circ f'))(m) = \begin{cases}
    f(f'(m)) & 1 \leq m \leq n \\
    n + 1 & m = n + 1
  \end{cases}
\]
and
\[
  (\phi(f) \circ \phi(f'))(m) = \begin{cases}
    f((\phi(f'))(m)) & 1 \leq (\phi(f'))(m) \leq n \\
    n + 1 & (\phi(f'))(m) = n + 1
  \end{cases}
\]
however, as shown above, $1 \leq (\phi(f'))(m) \leq n \iff 1 \leq m \leq n$, and $(\phi(f'))(m) = n + 1 \iff m = n + 1$ as $\phi(f') \in H_{n+1}$, so the conditions simplify to
\[
  (\phi(f) \circ \phi(f'))(m) = \begin{cases}
    f((\phi(f'))(m)) & 1 \leq m \leq n \\
    n + 1 & m = n + 1
  \end{cases} = \begin{cases}
    f(f'(m))) & 1 \leq m \leq n \\
    n + 1 & m = n + 1
  \end{cases} = (\phi(f \circ f'))(m)
\]

Since $\phi(f \circ f') = \phi(f) \circ \phi(f')$, $\phi$ is an isomorphism and we are done.

\subsection*{7}

Note that we have from class that there are integer solutions $x,y$ to $ax + by = d$ if and only if $\gcd(a,b) \mid d$. Then, $n \in \langle a, b\rangle \implies n = ax + by$ for integers $x, y \implies \gcd(a,b) \mid n$, so every $n \in \langle a, b \rangle$ can be written as $k\gcd(a,b)$ for some integer $k$. Further, since we have that $ax + by = k\gcd(a,b)$ has solutions for every integer $k$, we know that $\langle a, b \rangle = \{k\gcd(a,b) \mid k \in \Z\}$.

Then, this is exactly $\langle \gcd(a,b) \rangle$ for positive $(a,b)$. Then, $\langle 2,3\rangle = \langle 1 \rangle = \Z$, $\langle 3,5\rangle = \langle 1 \rangle = \Z,$ and $\langle 4,6 \rangle = \langle 2 \rangle$, or the even integers.

\subsection*{8}

Notice that $9 \cdot 9 = 81$ and $16 \cdot 5 = 80 = 81 - 1$, so we have that $n = 9$, $m = -5$ satisfies $9n + 16m = 1$.

\subsection*{9}

No: we saw in class that there are integer solutions $x,y$ to $ax + by = d$ if and only if $\gcd(a,b) \mid d$. and since we have that $3 \nmid 2$, there are no solutions to $57x + 93y = 2$ (also note that $57x + 93y = 3(19x + 21y)$, so any solution would give that $19x + 21y = \frac{2}{3}$, where the LHS is an integer). However, since $3 \mid -6$, there are solutions to $57x + 93y = -6$. In particular, if $x',y'$ satisfy $57x' + 93y' = 3$, then $x = -2x', y = -2y'$ satisfy $57x + 93y = -6$.

\end{document}
% LocalWords:  NetID fancyplain LocalWords colorlinks linkcolor linkbordercolor