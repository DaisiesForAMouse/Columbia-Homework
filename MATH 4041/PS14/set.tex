\documentclass[12pt,letterpaper]{article}
\usepackage{fullpage}
\usepackage[top=2cm, bottom=4.5cm, left=2.5cm, right=2.5cm]{geometry}
\usepackage{amsmath,amsthm,amsfonts,amssymb,amscd}
\usepackage{lastpage}
\usepackage{enumerate}
\usepackage{fancyhdr}
\usepackage{mathrsfs}
\usepackage{xcolor}
\usepackage{graphicx}
\usepackage{mathdots}
\usepackage{listings}
\usepackage{hyperref}
\usepackage{tikz}
\usepackage{relsize}
\usepackage{fancyvrb}
\usepackage{import}
\usetikzlibrary{shapes.geometric,fit}

\hypersetup{%
  colorlinks=true,
  linkcolor=blue,
  linkbordercolor={0 0 1}
}

\setlength{\parindent}{0.0in}
\setlength{\parskip}{0.05in}

\theoremstyle{definition}
\newtheorem*{statement}{Statement}
\newtheorem*{claim}{Claim}
\newtheorem*{theorem}{Theorem}
\newtheorem*{lemma}{Lemma}

\newcommand{\contra}{\Rightarrow\!\Leftarrow}
\newcommand{\R}{\mathbb{R}}
\newcommand{\F}{\mathbb{F}}
\newcommand{\Z}{\mathbb{Z}}
\newcommand{\Zeq}{\mathbb{Z}_{\geq 0}}
\newcommand{\Zg}{\mathbb{Z}_{>0}}
\newcommand{\Req}{\mathbb{R}_{\geq 0}}
\newcommand{\Rg}{\mathbb{R}_{>0}}
\newcommand{\N}{\mathbb{N}}
\newcommand{\Q}{\mathbb{Q}}
\newcommand{\C}{\mathbb{C}}
\DeclareMathOperator{\Id}{id}
\DeclareMathOperator{\Aut}{Aut}
\DeclareMathOperator{\Stab}{Stab}
\DeclareMathOperator{\lcm}{lcm}

\newcommand{\incfig}[1] {%
    % \def\svgwidth{\columnwidth}
    \import{./figures/}{#1.pdf_tex}
}

\title{MATH 4041 HW 13}
\author{David Chen, dc3451}
\date{\today}

\begin{document}

\maketitle

\section*{Problem 1}

First, $(1 \cdot f)(x) = f(1 \cdot x) = f(x)$. Secondly,
\[
  (h \cdot (g \cdot f))(x) = (g \cdot f)(h\cdot x) = f(g \cdot h\cdot x) = f((gh) \cdot x) = (gh \cdot f)(x)
\]
so it does not define a group action, but if we take
\[
  (g \cdot f)(x) = f(g^{-1} \cdot x)
\]
instead, we get
\[
  (h \cdot (g \cdot f))(x) = (g \cdot f)(h^{-1} \cdot x) = f(g^{-1} \cdot h^{-1} \cdot x) = f((hg)^{-1} \cdot x) = (hg \cdot f)(x)
\]
which gives us a group action.

\section*{Problem 2}

Since we have that $20 = 2^{2} \cdot 5$, the order of a 2-Sylow subgroup is $4$, and a 5-Sylow subgroup has order $5$.

The divisors of 20 are 1,2,4,5,10,20.

The possible amount of 2-Sylow subgroups is 1 or 5. The possible amount of 5-Sylow subgroups is 1.

\section*{Problem 3}

Since we have that $36 = 2^{2} \cdot 3^{2}$, the order of a 2-Sylow subgroup is $4$, and a 3-Sylow subgroup has order $9$.

The divisors of 36 are 1,2,3,4,6,9,12,18,36.

The possible amount of 2-Sylow subgroups is 1, 3, 9. The possible amount of 3-Sylow subgroups is 1 or 4.

\section*{Problem 4}

\subsection*{i}

Taking the hint in the problem set, a 2-Sylow subgroup $H$ of $G$ has order 2, and thus if $H = \{1, h\}$, since it is unique, it is normal, so $g^{-1}hg = 1$ or $g^{-1}hg = h$ for $g \in G$; then, if $g^{-1}hg = 1$, then $hg = g \implies h = 1$, but by assumption $h \neq 1$, so we instead have that $g^{-1}hgh = h \implies h \in Z(G)$, so $H \leq Z(G)$.

Then, by an earlier homework, we get that $G/H$ is cyclic since it is of prime order (I can't actually remember if this was on an earlier homework now that I think about it, so suppose $k \neq 1$ is an element of some group $K$ with prime order, and thus $\langle k \rangle \leq K$ but by Lagrange, $|\langle k \rangle| = |K|$ since $K$ has prime order, and thus $\langle k \rangle = K$ since $K$ is finite), and from another (different) earlier homework, we have that this implies $G$ to be abelian.

Now, we have a subgroup of order 2 $H$ and a subgroup of order 3 $H'$, both necessarily cyclic (since they are of prime order). Let $g_{1}$ be a generator of $H$, and $g_{2}$ a generator of $H'$. Then, we have that since $G$ is abelian, and $2, 3$ are coprime, that the order of $g_{1}g_{2}$ is $6$, so $G$ is generated by $g_{1}g_{2}$.

\subsection*{ii}

Taking the group action $g \cdot H_{i} = g^{-1}H_{i}g$ for $g \in G, H_{i} \in X$, we have that $G$ is transitive on $X$ by the Sylow theorem, since each 2-Sylow subgroup is conjugate to each other. Then, we get that for any $H_{i} \in X$, $X \cong G/G_{H_{i}}$, so $|G|/|G_{H_{i}}| = 3$, so the stabilizers are of order 2. But then, we have that clearly for any $h \in H_{i}$, $h \cdot H_{i} = H_{i}$, so the stabilizer of $H_{i}$ is exactly $H_{i}$ itself. Then, if we define $f: G \rightarrow S_{X}$ to be the homomorphism induced by the group action, we have that $g \in \ker(f)$ satisfies that $g \cdot H_{i} = H_{i}$ for any $H_{i} \in X$; from earlier, this means that $g \in H_{1} \cap H_{2} \cap H_{3}$, so $g = 1$ and the kernel is therefore trivial, so $f$ is an isomorphism giving $G \cong S_{X}$, and taking $H_{i} \mapsto i$ gives $S_{X} \cong S_{3}$.

\section*{Problem 5}
\subsection*{i}

Since we have that $|G:H| = 2$, $H$ must be normal, and by Sylow has a subgroup of order 3 as well. Further, since the divisors of 6 are 1,2,3,6, there can only be one subgroup of order 3 of $H$, so this subgroup must be normal in $H$. Call this subgroup $K$. Now, we have that since $H$ is normal, $g^{-1}Kg$ is contained in $H$, but in particular, this is still a subgroup: $g^{-1}1g = 1$, $g^{-1}h_{1}gg^{-1}h_{2}h = g^{-1}h_{1}h_{2}g$, and $g^{-1}hgg^{-1}h^{-1}g = 1$, so we get the indentity, closure, and inverses, and the mapping $h \mapsto g^{-1}hg$ is injective. However, since $K$ is the unique subgroup of order 3, $g^{-1}Kg = K$.

\subsection*{ii}

Suppose that $A_{6}$ had such a subgroup. Then, by the last part, $A_{6}$ has a normal subgroup of order 3, call it $H$; since it is of order 3, it is necessarily cyclic. Then, consider the possible even permutations of $\{1,2,3,4\}$. In particular, any such (non-identity) permutation must either move 3 or 4 elements, since if it moved only two, it would be a transposition. If it moves 4 elemements, then writing it as a product of disjoint cycles, it must either be of the form $(a,b,c,d)$ or $(a,b)(c,d)$ (to see this, note that the size of the support is the sum of the length of the cycles \textit{except} the 1-cycles, and we can only write $4 = 1 + 3 = 2 + 2$, so either it is the composition of 2 2-cycles or just one 1 4-cycle), but $(a,b,c,d) = (a,b)(b,c)(c,d)$, so it is not even and thus must be of the form $(a,b)(c,d)$. However, every non-identity element in $H$ must be a generator since it must be of order $3$, but $(a,b)(c,d)(a,b)(c,d) = 1$, so every element in $H$ must be a 3-cycle (or the identity). Then, if $(a,b,c) \in H$, then $H = \langle (a,b,c) \rangle = \{1, (a,b,c), (a,b,c)^{2}\}$, but $(a,b,c)^{2} = (c,b,a)$. Then, if you just pick something like $(a,b)(c,d)$ which is its own inverse, we get that
\[
  (a,b)(c,d)(a,b,c)(a,b)(c,d)
\]
sends $a \mapsto d$, so it is not in $H$, and thus $H$ cannot be normal.

\section*{Problem 6}

There is exactly one $p$-Sylow subgroup of $G$. In particular, the possible amounts of $p$-Sylow subgroups must all be 1 modulo $p$. However, this amount must also divide $p^{r}m$, which has divisors $1, p, p^{2}, \dots, p^{r}, m, pm, \dots, p^{r}m$. Clearly no $p^{i}$ for $1 < i \leq r$ is 1 mod $p$, and neither are $p^{i}m$, since these all vanish modulo $p$. Then, the only choices are $1$ and $m$, but by assumption, $1 < m < p$, so $m \neq 1$ mod $p$. Then, the only choice left is 1, so the unique $p$-Sylow subgroup of $G$ is normal and nontrivial since it is of order $p^{r}$ for $r > 0$.

\section*{Problem 7}

\subsection*{i}

We have that $\ker(f)$ is a normal subgroup of $G$. Since $G$ is simple and by assumption $\ker(f) \neq G$, we have that the kernel is trivial and thus $f$ is injective.

Now consider $g \in f^{-1}(A_{n})$. If we have any $h \in G$, $f(h^{-1}gh) = f(h^{-1})f(g)f(h)$, but $f(n) \in A_{n}$, a normal subgroup of $S_{n}$ since $\ker(\varepsilon) = A_{n}$, so we get that $f(h^{-1}gh) \in A_{n}$, so $h^{-1}gh \in f^{-1}(A_{n})$. Then, the preimage of $A_{n}$ is normal and thus either exactly $\{1\}$ or all of $G$. However, if it were exactly $\{1\}$, then $G$ has at most 3 elements, since if there are 3 distinct non-identity elements, there must be a pair $g,h$ such that $gh \neq 1$ (otherwise, if the three elements are $g,h,k$, we get that $gh = kh = 1\implies g = k$). Then, $G$ has order exactly $3$, since by assumption $|G| > 2$; however, if $G = \{1, g, g^{2}\}$, we have that $f(g^{2}) = (f(g))^{2}$ which is even, but $g^{2} \notin f^{-1}(A_{n})$ so $\contra$ and thus the preimage of $A_{n}$ must be a nontrivial normal subgroup of $G$, namely $G$ itself.

\subsection*{ii}

The divisors of 60 are 1, 2, 3, 4, 5, 6, 10, 12, 15, 20, 30, 60. The only ones that are 1 modulo 2 (that is, odd) are 1, 3, 5, 15. Thus, these are the only possibilities.

Now if there is only 1 2-Sylow subgroup, we get that we have a normal subgroup of order 2, $\contra$ since by assumption $G$ is simple.

If there are 3 2-Sylow subgroups $H_{1}, H_{2}, H_{3}$, we get that the group action of $G$ on $X = \{H_{1}, H_{2}, H_{3}\}$ given by $g \cdot H_{i} = g^{-1}H_{i}g$ induces a homomorphism $f: G \rightarrow S_{X}$. In particular, this group action is transitive, such that there is some $g \in G$ such that $g^{-1}H_{1}g = g^{-1}H_{2}g$, so $f(g) \neq \Id$, and this homormophism is not trivial and therefore injective. However, $|G| = 60, |S_{X}| = 3! = 6$, so $f$ cannot be injective. $\contra$

If there are 5 2-Sylow subgroups, we consider the same group action as before, which now induces a homomorphism $f: G \rightarrow S_{X}$ for $X$ the set of 2-Sylow subgroups. Then, we have that $f(G) \subseteq A_{5}$, and $f$ is injective, such that $|G| = |f(G)| = 60$. Then, a subgroup of order $60$ in $A_{5}$ which itself has order $6!/2 = 60$ must be all of $A_{5}$, so $f(G) = A_{5}$, and thus $f$ is an isomorphism, and we have what we wanted.

\end{document}
% LocalWords:  NetID fancyplain LocalWords colorlinks linkcolor linkbordercolor