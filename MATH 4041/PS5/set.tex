\documentclass[12pt,letterpaper]{article}
\usepackage{fullpage}
\usepackage[top=2cm, bottom=4.5cm, left=2.5cm, right=2.5cm]{geometry}
\usepackage{amsmath,amsthm,amsfonts,amssymb,amscd}
\usepackage{lastpage}
\usepackage{enumerate}
\usepackage{fancyhdr}
\usepackage{mathrsfs}
\usepackage{xcolor}
\usepackage{graphicx}
\usepackage{listings}
\usepackage{hyperref}
\usepackage{tikz}
\usepackage{relsize}
\usepackage{fancyvrb}
\usepackage{import}
\usetikzlibrary{shapes.geometric,fit}

\hypersetup{%
  colorlinks=true,
  linkcolor=blue,
  linkbordercolor={0 0 1}
}

\setlength{\parindent}{0.0in}
\setlength{\parskip}{0.05in}

\theoremstyle{definition}
\newtheorem*{statement}{Statement}
\newtheorem*{claim}{Claim}
\newtheorem*{theorem}{Theorem}
\newtheorem*{lemma}{Lemma}

\newcommand{\contra}{\Rightarrow\!\Leftarrow}
\newcommand{\R}{\mathbb{R}}
\newcommand{\F}{\mathbb{F}}
\newcommand{\Z}{\mathbb{Z}}
\newcommand{\Zeq}{\mathbb{Z}_{\geq 0}}
\newcommand{\Zg}{\mathbb{Z}_{>0}}
\newcommand{\Req}{\mathbb{R}_{\geq 0}}
\newcommand{\Rg}{\mathbb{R}_{>0}}
\newcommand{\N}{\mathbb{N}}
\newcommand{\Q}{\mathbb{Q}}
\newcommand{\C}{\mathbb{C}}
\DeclareMathOperator{\Id}{id}

\newcommand{\incfig}[1] {%
    % \def\svgwidth{\columnwidth}
    \import{./figures/}{#1.pdf_tex}
}

\title{MATH 4041 HW 5}
\author{David Chen, dc3451}
\date{\today}

\begin{document}

\maketitle

\section*{Problem 1}

We can induct on \(n\) for nonnegative \(n\): we have that for \(n = 0\), \(f(g^{0}) = f(e_{1})\), where \(e_{1}\) is the identity element of \(G_{1}\). Further, since we have that for every \(g \in G_{1}\), \(f(g) = f(e_{1}  g) = f(e_{1})  f(g)\), we have that \(f(e_{1})\) must be the identity of \(G_{2}\), as we know that identities are unique in groups. Then, \(f(g^{0}) = f(e_{1}) = (f(g))^{0}\).

Now, if \(f(g^{n}) = (f(g))^{n}\), then
\[
  f(g^{n+1}) = f(g^{n}  g) = f(g^{n})  f(g) = (f(g))^{n}  f(g) = (f(g))^{n+1}
\]

So we have that \(f(g^{n}) = (f(g))^{n}\) holds for \(n \geq 0\). Now for \(n < 0\), we have by the result for \(n > 0\) that \(f(g^{n}) = f((g^{-1})^{-n}) = (f(g^{-1}))^{-n}\). Further, since we have that \(f(e_{1}) = f(g^{-1}  g) = f(g^{-1})  f(g)\), and since the identity element in \(G_{2}\) is exactly \(f(e_{1})\), as shown earlier, by the definition of inverses, \(f(g^{-1}) = f(g)^{-1}\). Then, we have that \(f(g^{n}) = (f(g^{-1}))^{-n} = ((f(g))^{-1})^{-n} = (f(g))^{n}\).

Thus, if \(g\) has finite order in \(G_{1}\), then there is some \(n \in \Z\) such that \(g^{n} = e_{1}\). Then \(f(e_{1}) = f(g^{n}) = (f(g))^{n}\), and since \(f(e_{1})\) is the identity in \(G_{2}\), we have that \(f(g)\) has also order at most \(n\), and thus has finite order. The same result holds, since \(f\) admits another isomorphism \(f^{-1}: G_{2} \rightarrow G_{1}\), so if \(f(g)\) has finite order, then there is some \(m \in \Z\) such that \((f(g))^{m} = f(e_{1})\), so \(f^{-1}((f(g))^{m}) = f^{-1}(f(e_{1})) = e_{1}\). Then, \(f^{-1}((f(g))^{m}) = f^{-1}(f(g^{m})) = g^{m}\), so \(g^{m} = e_{1}\) so we have that \(g\) has finite order of at most \(m\).

From above, we also can see that the order of \(f(g)\) is at most \(n\), the order of \(g\), and the order of \(g\) is at most \(m\), the order of \(f(g)\). Thus, \(n \leq m\) and \(m \leq n \implies n = m\).

\section*{Problem 2}

Write \(1\) as the identity element in $G$.

The least common multiple of \(d_{1}\) and \(d_{2}\) is the least \(n \in \N\) such that \(n = d_{1}k_{1}\) and \(n = d_{2}k_{2}\) for integers \(k_{1}, k_{2}\). Then, we have that \((gh)^{n} = g^{n}h^{n} = g^{d_{1}k_{1}}h^{d_{2}k_{2}} = (g^{d_{1}})^{k_{1}}(h^{d_{2}})^{k_{2}} = 1^{k_{1}}1^{k_{2}} = 1\) (commutativity is used in the first equality), so \(gh\) has order at most \(n\). Since we know that the least common multiple of \(d_{1}\) and \(d_{2}\) for finite \(d_{1}, d_{2}\) is at most \(d_{1}d_{2}\), \(gh\) must have finite order when \(g\), \(h\) have finite order.

This is not always true: consider in \(\Z_{2}\) that \([1]\) has order \(2\), but \([1] + [1] = [0]\) has order 1, whereas the least common multiple of \(2\) and \(2 > 1\).

If $g$ has finite order and $h$ has finite order, then $gh$ cannot have finite order. To see this, note that $h^{n} = 1 \implies (h^{-1})^{n} = (h^{n})^{-1} = 1^{-1} = 1$ (the first equality follows from $h^{n}h^{-n} = (hh^{-1})^{n} = 1 \implies (h^{-1})^{n} =  h^{-n} = (h^{n})^{-1}$), so $h^{-1}$ is of order $n$ as well. Then, if $gh$ has finite order, we have that $g = (gh)h^{-1}$ as the product of two elements of finite order must have finite order, so $\contra$ and $gh$ cannot have finite order.

If $g, h$ both have infinite order, $gh$ may have finite order: consider that $-1$ and $1$ both have infinite order in $\Z$ under addition, but $-1 + 1 = 0$ has order $1$ in $\Z$.

\section*{Problem 3}

We need to show that $H = \{g \in G \mid g^{n} = 1\}$ contains the identity, inverses, and is closed. The identity satisfies that $1^{n} = 1$, so $1 \in H$. Then, we have that $g^{n} = 1 \implies (g^{-1})^{n} = (g^{n})^{-1} = 1^{-1} = 1$ (first equality was shown in one of the proofs for a question in problem 2) so every element in $H$ has an inverse in $H$. Lastly, if $g^{n} = 1, h^{n} = 1$, we have that since $G$ is abelian $1 = g^{n}h^{n} = (gh)^{n}$, so $gh \in H$ as well.

\section*{Problem 4}

Again, we need to check that $H = \{g \in G \mid \exists N \in \N \text{ s.t. } g^{N} = 1\}$ contains the identity, inverses, and is closed. First, we have that $1^{1} = 1$, so $1 \in H$. Again, if $g^{N} = 1$, we have from the last problem that $(g^{-1})^{N} = 1$ and so $g^{-1} \in H$. Lastly, we have from problem 2 that since $g,h \in H$ have finite order, $gh$ must also have finite order at most the lcm of the orders of $g$ and $h$, and thus $gh \in H$.

\section*{Problem 5}

\subsection*{i}

We have that $A_{\theta}A_{\theta} = A_{2\theta}$ from earlier homework, so $(A_{\theta})^{n} = A_{n\theta}$. Then, we have that $(A_{2\pi / n})^{n} = A_{2\pi} = I$. If $0 < m < n$, then we have that $(A_{2\pi/n})^{m} = A_{2m\pi/n}$, where $0 < m/n < 1$. However, as shown in earlier homework, $A_{\theta} = I$ if and only if $\theta = 2\pi k$ for some integer $k$. Since there is no such integer $m/n$ where $0 < m/n < 1$, we have that $(A_{2\pi /n})^{m} \neq I$, so we have that $n$ is the least positive integer such that $(A_{2\pi / n})^{n} = I$, so $A_{2\pi / n}$ has order $n$.

The elements of finite order are exactly $A_{\theta}$ where $\theta = 2\pi r$ for some $r \in \Q$. To see that this is sufficient, we have that if $r = p / q$, then $(A_{2\pi r})^{q} = A_{2 p \pi} = I$, so $A_{2\pi r}$ has order at most $q$. Further, if $\theta \neq 2\pi r$ for any $r \in \Q$, we have that $\theta = 2\pi x$ for some irrational $x \in \R$. Then, we have that if there is some $n \in \N$ such that $(A_{2\pi x})^{n} = I$, then $2\pi nx = 2\pi k$ for some $k \in \Z$, so $x = n / k \implies x \in \Q$, so $\contra$ and $A_{\theta}$ must be of the form $\theta = 2\pi r$ for $r \in \Q$.

In the second homework, we showed that $B_{\theta}B_{\theta} = I$, so $(B_{\theta})^{2} = I$ and thus $B_{\theta}$ always has order 2 (note that the only element in a group with order 1 is the identity, and $B_{\theta} \neq I$ for any $\theta$, as we would need that $\cos(\theta) = -\cos(\theta) = 1$, which is clearly impossible).

\subsection*{ii}

We have from the second homework that $B_{\theta_{1}}B_{\theta_{2}} = A_{\theta_{1}-\theta_{2}}$. Then, from earlier, we know that this has finite order if and only if $\theta_{1} - \theta_{2} = 2\pi r$ for some $r \in \Q$.

\section*{Problem 6}
\subsection*{i}

Every element of $\langle (3,-5) \rangle$ is given by $n(3, -5)$ for some $n$. Then, since the operation is defined componentwise, we have that $n(3,-5) = (n3, n(-5)) = (3n, -5n)$, so we have that
\[
   \langle (3,-5) \rangle = \{(3n, -5n) \mid n \in \Z\} = \{\dots,(-6, 10), (-3,5), (0, 0), (3, -5), (6, -10)\dots\}
\]

\subsection*{ii}

To show that it is a proper subgroup, we only need to show that there is some element in $\Z \times \Z$ that is not contained in $\langle (a,b) \rangle$ for any given $(a,b)$ (in particular, we already showed that this would be a subgroup in class).

If $(a,b) = (0,0)$, then $\langle (a,b) \rangle = \{(0, 0)\}$, as we have that $n(0,0) = (0n, 0n) = (0, 0)$ for any $n$. This is clearly a proper subgroup of $\Z \times \Z$.

If $(a,b) \neq (0, 0)$, then we can show that $(-b,a) \notin (a,b)$. Suppose that $n(a,b) = (-b, a)$ for some $n \in \Z$. Then, we have that since $0(a,b) = (0, 0)$, we have that $n \neq 0$ as well. Since we have that for any $n$, $n(a,b) = (na, nb)$, if we have $n(a,b) = (-b,a)$, then $na = -b$ and $nb = a$. Then, $na(a) = -b(nb) \implies na^{2} = -nb^{2} \implies a^{2} = -b^{2}$ (we can cancel the $n$ since we already know $n \neq 0$). However, $b \in \Z \implies b^{2} \geq 0 \implies a^{2} = -b^{2} \leq 0$, but $a \in \Z \implies a^{2} \geq 0$. Thus, the only option is that $a^{2} = -b^{2} = 0 \implies a = b =0$, but is contrary to the initial assumption that $(a,b) \neq (0,0)$. $\contra$, so $\langle(a,b)\rangle$ does not contain $(-b, a)$, and is thus a proper subgroup of $\Z \times \Z$.

\section*{Problem 7}

Note that repeated addition in $\Q$ is just multiplication by an integer, so $\sum_{i=1}^{n}r = nr$ in the sense of multiplication in $\Q$, so being able to cancel something like $r/2 = nr$ to $1/2 = n$ is not just a coincidence of notation.

Suppose that $r/2 \in \langle r \rangle$, such that $r/2 = nr$ for some $n \in \Z$. Then, $1/2 = n$, which clearly is not an integer, so $r/2$ cannot be generated by $r$.

\section*{Problem 8}

Write $1$ as the identity of $G$.

We need to show it contains the identity, inverses, and is closed. First, since $H_{1}, H_{2}$ are subgroups, $1 \in H_{1}$ and $1 \in H_{2}$, so $1 \in H_{1} \cap H_{2}$. Next, $g \in H_{1} \cap H_{2} \implies g \in H_{1}$ and $g \in H_{2}$, and since $H_{1}, H_{2}$ are subgroups, each much contain inverses for their elements, so $g \in H_{1} \implies g^{-1} \in H_{1}$ and $g \in H_{2} \implies g^{-1} \in H_{2}$, so $g^{-1} \in H_{1} \cap H_{2}$. Lastly, if $g, h \in H_{1} \cap H_{2}$, we have that since subgroups are closed, $g,h \in H_{1} \implies gh \in H_{1}$ and $g,h \in H_{2} \implies gh \in H_{2}$, so $gh \in H_{1} \cap H_{2}$.

\section*{Problem 9}

We have that $H$ is a subgroup, so it contains the identity, inverses, and is closed. We want to show the same things for $f(H)$. Earlier in the homework (problem 1) we showed that if $f$ is an isomorphism and $e_{1}$ is the identity of $G_{1}$, then $f(e_{1})$ is the identity of $G_{2}$, so $e_{1} \in H \implies f(e_{1}) \in f(H)$, so $f(H)$ contains the identity of $G_{2}$, which will be written $e_{2}$.

Since $e_{2} = f(e_{1}) = f(g  g^{-1}) = f(g)  f(g^{-1})$, we have that $(f(g))^{-1} = f(g^{-1})$. Thus, given any $f(g) \in f(H)$, we have that since $H$ is a subgroup $g^{-1} \in H \implies (f(g))^{-1} = f(g^{-1}) \in f(H)$. Lastly, given any two $f(g), f(h) \in f(H)$, we have that $f(g) f(h) = f(g  h)$. However, $g, h \in H \implies gh \in H$ since $H$ as a subgroup is closed, so $f(g) f(h) = f(g  h) \in f(H)$.

\end{document}
% LocalWords:  NetID fancyplain LocalWords colorlinks linkcolor linkbordercolor