\documentclass[12pt,letterpaper]{article}
\usepackage{fullpage}
\usepackage[top=2cm, bottom=4.5cm, left=2.5cm, right=2.5cm]{geometry}
\usepackage{amsmath,amsthm,amsfonts,amssymb,amscd}
\usepackage{lastpage}
\usepackage{enumerate}
\usepackage{fancyhdr}
\usepackage{mathrsfs}
\usepackage{xcolor}
\usepackage{graphicx}
\usepackage{listings}
\usepackage{hyperref}
\usepackage{tikz}
\usepackage{relsize}
\usepackage{fancyvrb}
\usepackage{import}
\usetikzlibrary{shapes.geometric,fit}

\hypersetup{%
  colorlinks=true,
  linkcolor=blue,
  linkbordercolor={0 0 1}
}

\setlength{\parindent}{0.0in}
\setlength{\parskip}{0.05in}

\theoremstyle{definition}
\newtheorem*{statement}{Statement}
\newtheorem*{claim}{Claim}
\newtheorem*{theorem}{Theorem}
\newtheorem*{lemma}{Lemma}

\newcommand{\contra}{\Rightarrow\!\Leftarrow}
\newcommand{\R}{\mathbb{R}}
\newcommand{\F}{\mathbb{F}}
\newcommand{\Z}{\mathbb{Z}}
\newcommand{\Zeq}{\mathbb{Z}_{\geq 0}}
\newcommand{\Zg}{\mathbb{Z}_{>0}}
\newcommand{\Req}{\mathbb{R}_{\geq 0}}
\newcommand{\Rg}{\mathbb{R}_{>0}}
\newcommand{\N}{\mathbb{N}}
\newcommand{\Q}{\mathbb{Q}}
\newcommand{\C}{\mathbb{C}}
\DeclareMathOperator{\Id}{id}
\DeclareMathOperator{\lcm}{lcm}

\newcommand{\incfig}[1] {%
    % \def\svgwidth{\columnwidth}
    \import{./figures/}{#1.pdf_tex}
}

\title{MATH 4041 HW 7}
\author{David Chen, dc3451}
\date{\today}

\begin{document}

\maketitle

I use that $a \mid b \implies |a| \leq |b|$ for nonzero $b$ a few times in this homework, and I can't recall if it was shown in class. I'll just show it here at the top of the problem set: $a \mid b \implies b = ak$ for some $k \in \Z$; then, since $b$ is nonzero, $k$ cannot be $0$, so $k \neq 0$, then $k \geq 1$ or $k \leq -1$, so we have that $|ak| - |a| = (|k| - 1)|a|$, which is either $0$ if $k = \pm 1$ or positive otherwise, so $|b| = |ak| \geq |a|$.

\section*{Problem 1}

\begin{enumerate}
  \item \begin{align*}
          11 &= 5 \cdot 2 + 1 \\
          1 &= 11 - 5 \cdot 2
        \end{align*}
        so we can take $a^{-1} = -2, 9$, or more generally any $x \equiv -2 \equiv 9 \mod 11$.

  \item $21^{-1} \mod 28$ does not exist, since $\gcd(21, 28) = 7$, and so there is no integer solution to $21x = 1 + 28y$.

  \item \begin{align*}
          101 &= 2 \cdot 50 + 1 \\
          1 &= 101 - 2 \cdot 50
        \end{align*}
        so we can take $a^{-1} = -50, 51$, or more generally any $x \equiv -50 \equiv 51 \mod 101$.

  \item \begin{align*}
          101 &= 4 \cdot 25 + 1 \\
          1 &= 101 - 4 \cdot 25
        \end{align*}
        so we can take $a^{-1} = -25, 75$, or more generally any $x \equiv -25 \equiv 76 \mod 101$.
\end{enumerate}

\section*{Problem 2}

When $n = 2k$ is even, then $\gcd(2, n) \geq 2$ since $2 \mid 2$ and $2 \mid 2k$, so we have that $2x = 1 + ny$ has no integer solutions, and so no multiplicative inverse exists for $2$ modulo $n$.

If $n = 2k + 1$ instead, note that $2k + 1 = n \implies 2k + 2 = n + 1 \implies 2(k + 1) \equiv 1 \mod n$, so $k + 1$ is a suitable inverse; in general, any $x \equiv k  +1 \mod n$ is a suitable inverse.

\section*{Problem 3}

We can show that the least positive integer $a$ such that when viewed as an element of $\Z/n\Z$, $\langle m \rangle = \langle a \rangle$ is $\gcd(n, m)$. In particular, we already have from class that $\langle m \rangle = \langle \gcd(n,m) \rangle$; note that if $\gcd(n,m) = 1$, we are done as 1 is the least positive integer. If there is some $1 \leq a < \gcd(n,m)$ such that $\langle \gcd(n,m) \rangle = \langle a \rangle$, then we have that $a \in \langle \gcd(n,m) \rangle \implies a \equiv \gcd(n,m)x \mod n$ for some $x$, so $a = \gcd(n,m)x + ny$ for some $x, y \in \Z$; however, since $a < \gcd(n,m) \implies \gcd(\gcd(n,m), n) = \gcd(n,m) \nmid a$, this has no solutions, so $\contra$ and $\gcd(n,m)$ is the least positive integer that generates $\langle m \rangle$ as an element of $\Z/n\Z$.

\subsection*{i}

The order is $12$, as we saw in class that the order is $36 / \gcd(21, 36)$, and similarly we have $a = \gcd(21,36) = 3$.

\subsection*{ii}

The order is $3$, as we saw in class that the order is $45 / \gcd(30, 45)$, and similarly we have $a = \gcd(30,45) = 15$.

\subsection*{iii}

By earlier homeworks, the order is $\lcm(12,3) = 12$.

\section*{Problem 4}

For $0 \leq a < 11$, $[a]_{11} \in (\Z/11\Z)^{*}$ only if $\gcd(a, 11) = 1$. Since $11$ is prime, this is everything $1 \leq a \leq 10$, so the order is 10, as $[1]_{11}, [2]_{11}, \dots, [10]_{11} \in (\Z/11\Z)^{*}$.

We can find an explicit generator, so $(\Z/11\Z)^{*}$ is cyclic and thus isomorphic to $\Z/10\Z$ (brackets dropped in the table):

\begin{center}
  \begin{tabular}{c|c|c|c|c|c|c|c|c|c|c|c}
    $n$ & 0 & 1 & 2 & 3 & 4 & 5 & 6 & 7 & 8 & 9 & 10 \\ \hline
    $2^{n}$ & 1 & 2 & 4 & 8 & 5 & 10 & 9 & 7 & 3 & 6 & 1 \\
  \end{tabular}
\end{center}

The order of any subgroup of $(\Z/11\Z)^{*} \cong \Z/10\Z$ has order dividing $10$, and this subgroup is the unique subgroup with that order.

Then, we have that the subgroups of $(\Z/11\Z)^{*}$ are $\langle 1 \rangle, \langle 2 \rangle, \langle 4 \rangle,$ and $\langle 10 \rangle$, which have orders $1, 10, 5, 2$ respectively, as can be checked in the table; this is given since the subgroup of order $d$ is generated by $2^{n/d}$, as seen in class for any divisor $d$ of 10.

\section*{Problem 5}

From class, every subgroup can be given by the form $\langle d \rangle$ for some divisor $d$ of $n$, as for any $a$, $\langle a \rangle = \langle \gcd(a,n) \rangle$. Then, since there is at most one subgroup of any given order in $\Z/n\Z$, the subgroups are (generators are computed by taking all $1 \leq g \leq 18$ with the same order $\gcd(g,n)$):
\begin{enumerate}
  \item $\Z/18\Z = \langle 1 \rangle = \langle 5 \rangle = \langle 7 \rangle = \langle 11 \rangle = \langle 13 \rangle = \langle 17 \rangle$ which has order $18$. Also, $\varphi(18) = 6$
  \item $\langle 2 \rangle = \langle 4 \rangle = \langle 8 \rangle = \langle 10 \rangle = \langle 14 \rangle = \langle 16 \rangle $ which has order $9$. Also, $\varphi(9) = 6$.
  \item $\langle 3 \rangle = \langle 15 \rangle$ which has order $6$. Also, $\varphi(6) = 2$.
  \item $\langle 6 \rangle = \langle 12 \rangle$ which has order $3$. Also, $\varphi(3) = 2$.
  \item $\langle 9 \rangle$ which has order $2$. Also, $\varphi(2) = 1$.
  \item $\langle 18 \rangle = \langle 0 \rangle$ which has order $1$. Also, $\varphi(1) = 1$.
\end{enumerate}
the totient of $n$ is calculated by counting the amount of generators of order $n$, which is an equivalence shown in class. Adding, we have that $\sum_{d \mid 18}\varphi(d) = 1 + 1 + 2 + 2 + 6 + 6 = 18$.

\section*{Problem 6}

\subsection*{a}

$(\implies)$ We have that $d \mid a \implies a = dk$ for some $k \in \Z$. Then, $[d]_{n}^{k} = k \cdot [d]_{n} = [kd]_{n} = [a]_{n} \implies [a]_{n} \in \langle [d]_{n} \rangle$, which was what we wanted.

$(\impliedby)$ We have that $[a]_{n} \in \langle [d]_{n} \rangle \implies [a]_{n} = t \cdot [d]_{n}$ for $t \in \Z \implies [a]_{n} = [td]_{n}$. Then, by the construction of these equivalence classes, $a = td + nu$ for $u \in \Z$. However, we have that $d\mid n \implies n =dv$ for $v \in \Z$. Finally, we arrive at $a = td + uvd = d(t + uv)$, so $d \mid a$.

\subsection*{b}

$(\implies)$ $a \equiv a' \mod n \implies a = a' + nt$ for some $t \in \Z \implies a = a' + tud$ as $d \mid n \implies n = du$ for some $u \in \Z$; then, $d \mid a \implies a = vd$ for $v \in \Z$, so $a' = vd - tud = d(v - tu)$ so $d \mid a'$.

$(\impliedby)$ The above proof is symmetric; replace $a$ with $a'$ and vice versa.

$a' \equiv a \mod n \implies a' = a + nt$ for some $t \in \Z \implies a' = a + tud$ as $d \mid n \implies n = du$ for some $u \in \Z$; then, $d \mid a' \implies a' = vd$ for $v \in \Z$, so $a = vd - tud = d(v - tu)$ so $d \mid a$.

\subsection*{c}

Let $a' = a + nk$ for $k \in \Z$. Then, if $d \mid a$ and $d \mid n$, we have that $d \mid a + nk = a'$; similarly, if $d \mid a'$ and $d \mid n$, $d \mid a' - nk = a$, so we have that for any integer $d$, that $d \mid a$ and $d \mid n \implies d \mid a'$, as well as $d \min a'$ and $d \mid n \implies d \mid a$. Then take $d = \gcd(a,n)$ so by definition of the gcd, we have that $\gcd(a,n) \mid a$, $\gcd(a,n) \mid n$, so $\gcd(a,n) \mid a'$. However, this then gives that $\gcd(a,n) \mid \gcd(a', n)$, and since they are both positive, $\gcd(a,n) \leq \gcd(a',n)$. Taking $d = \gcd(a',n)$, we see that $\gcd(a',n) \mid a$ as well, so $\gcd(a',n) \mid \gcd(a, n)$, and $\gcd(a',n) \leq \gcd(a,n)$, so combining with before, $\gcd(a,n) = \gcd(a',n)$.

\section*{Problem 7}

Note that the existence of integers $x,y$ such that $1 = ax + by$ gives that $gcd(a,b) \mid 1$, but the only positive divisor of $1$ is $1$, so $\gcd(a,b) = 1$.

\subsection*{i}

Since $a,b$ are relatively prime,  $1 = ax + by$ for some $x, y$; then, for any divisor $d$ of $a$, $a = dk$ for some $k \in \Z$, so $1 = dkx + by$, so there are integers $kx, y$ satisfying $1 = d(kx) + by$, so from class $\gcd(d,b) = 1$.

\subsection*{ii}

Since $a$ is relatively prime to $n,m$, we can write $1 = ax + ny = aw + mz$; then, we have that $1 = (ax + ny)(aw + mz) = a^{2}xw + awny + axmz + nmyz = a(axw + wny + xmz) + nm(yz)$, so by class, $1 = \gcd(a, nm)$.

If $a$ is relatively prime to $mn$, then $1 = ax + nmy$, so there are integers $x, my$ such that $1 = ax + n(my)$, so $\gcd(a,n) = 1$; similarly, there are integers $x, ny$ such that $1 = ax + m(xy)$, so $\gcd(a,m) = 1$.

\section*{Problem 8}
\subsection*{i}

We can define $\lcm(a,b)$ to be a positive integer $m$ such that $a \mid m$ and $b \mid m$; further, if $a \mid n$ and $b \mid n$ for some integer $n$, then $m \mid n$ as well.

To see that this is unique, suppose that $m, m'$ have the above property. Then, $m \mid m' \implies |m| \leq |m'|$ and $m' \mid m \implies |m'| \leq |m|$ Since both $|m| \leq |m'|$ and $|m'| \leq |m|$, and both are positive, $m = m'$.

\subsection*{ii}
We have that any element $mk \in \langle m \rangle$ satisfies that $mk \in \langle a \rangle \implies mk = ak'$ and $n \in \langle b \rangle \implies mk = bk''$ for $k', k'' \in \Z$, so all elements of $\langle m \rangle$ are common multiples of $a,b$. In particular, if $k = 1$, then $m = ak' = bk''$, so $a \mid m$ and $b \mid m$. Further, any common multiple of $a, b$ is an element of $\langle a \rangle \cap \langle b \rangle$: a common multiple is some number $l$ such that $l = ak'= bk''$, but this is exactly the condition to be in $\langle a \rangle \cap \langle b \rangle$, since $l = ak' \implies l \in \langle a \rangle$, and $l = bk'' \implies l \in \langle b \rangle$. Further, there is no element $n$ in $\langle m \rangle$ such that $1 \leq n < m$ as then $m \nmid n$ (since $m \mid n \implies m \geq n$), so $m$ is the least positive integer in the list of common multiples of $a,b$, which we just saw to be $\langle a \rangle \cap \langle b \rangle$, and in that sense is the least common multiple.

Then, clearly $m \mid mk$ for $k \in \Z$, so this also satisfies the definition of part i, as $m \mid mk = ak' = bk''$ (since for any $n$, $a \mid n, b \mid n \implies n = ak', n = bk'', k',k'' \in \Z \implies n = mk$ for some integer $k$, as shown earlier).

\subsection*{iii}

If $a \mid bk$ for some $k \in \Z$, then $a \mid k$ by a lemma from class since $a,b$ are relatively prime. Now, for any common multiple $n$, if $a \mid n$ and $b \mid n$, we have that $n = bk$ for some $k$, so $n = b(ak') = (ab)k'$ for some $k' \in \Z$ since $a \mid n = bk$ and thus $a \mid k$. Then, $ab \mid n \implies |ab| \mid n$. Furthermore, clearly $a \mid |ab|$ and $b \mid |ab|$. This gives that $ab$ satisfies all the conditions in part i of the lcm.

\subsection*{iv}

Suppose that $e = \gcd\left(\frac{a}{d},\frac{b}{d}\right) > 1$. Then, $e \mid a/d \implies e = (a/d)k \implies ed \mid a$; similarly, $e \mid b/d \implies ed \mid b$. However, we now have that $ed$ is a common factor of $a, b$, but since $e > 1 \implies ed > d$, $ed \nmid d$ (as $ed \mid d \implies ed \leq d$, since both are positive), so $d$ cannot be the gcd of $a, b$. $\contra$, so $e = 1$, and $a,b$ are relatively prime.

\subsection*{v}

Put $e = \lcm\left(\frac{a}{k}, \frac{b}{k}\right)$. Then, $a/k \mid e \implies a/k = ek' \implies a = ekk' \implies a \mid ek$ and $b/k \mid e \implies b \mid ek$ similarly, so $ek$ is a common multiple.

Now let $n$ be any common multiple, so $a \mid n$ and $b \mid n$. Note that $a \mid n$, $k \mid a \implies k \mid n$. Let $n = kx, a = ky$, so $n/k = x$, $a/k = y$. Further, $a \mid n \implies n = ak', k' \in \Z \implies kx = kyk' \implies x =  yk' \implies n/k = (a/k)k' \implies a/k \mid n/k$. Similarly, $b \mid n \implies b/k \mid n/k$, so $n/k$ is a common multiple of $a/k$ and $b/k$. Then, since $e = \lcm\left(\frac{a}{k},\frac{b}{k}\right)$, we have that $e \mid n/k \implies e = (n/k)k', k' \in \Z, \implies ek = nk' \implies ek \mid n$, which was what we wanted.

\subsection*{vi}

From above, we have that $\lcm(a,b) = \gcd(a,b) \lcm\left(\frac{a}{\gcd(a,b)}, \frac{b}{\gcd(a,b)}\right)$. From iv, we have that $\frac{a}{\gcd(a,b)}, \frac{b}{\gcd(a,b)}$ are relatively prime, and so from iii, $\lcm\left(\frac{a}{\gcd(a,b)}, \frac{b}{\gcd(a,b)}\right) = \left|\frac{a}{\gcd(a,b)}\frac{b}{\gcd(a,b)}\right| = \frac{|ab|}{\gcd(a,b)^{2}}$. Then, finally, we get that $\lcm(a,b) = \gcd(a,b) \lcm\left(\frac{a}{\gcd(a,b)}, \frac{b}{\gcd(a,b)}\right) = \frac{|ab|}{\gcd(a,b)}$.

\subsection*{vii}

If $a = \prod_{i=1}^{n}p_{i}^{r_{i}}, b = \prod_{i=1}^{m}q_{i}^{s_{i}}$. Consider $\{u \mid u = p_{i}, 1 \leq i \leq n, \text{ or } u = q_{i}, 1 \leq i \leq m\}$. Then, let
\[
  t_{u} = \begin{cases}
    \max(r_{i},s_{j}) & u = p_{i}, u = q_{j} \\
    r_{i} & u = p_{i}, u \neq q_{j}, 1 \leq j \leq m \\
    s_{i} & u = q_{i}, u \neq p_{j}, 1 \leq j \leq n
  \end{cases}
\]

We then have the following, if $\{u_{i}\}_{i=1}^{k}$ is some ordering of the earlier set:
\[
  \lcm(a,b) = \prod_{i=1}^{k}u_{i}^{t_{u_{i}}}
\]

Morally, this is just saying that the lcm is the product of all the primes in factorizations of $a,b$ with the exponent chosen to be the greater of the two exponents in the factorizations of $a, b$ (if it only shows up in one factorization, pick the exponent in the one it shows up in).

\end{document}
% LocalWords:  NetID fancyplain LocalWords colorlinks linkcolor linkbordercolor