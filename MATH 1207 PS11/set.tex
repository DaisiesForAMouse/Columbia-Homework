\documentclass[12pt,letterpaper]{article}
\usepackage{fullpage}
\usepackage[top=2cm, bottom=4.5cm, left=2.5cm, right=2.5cm]{geometry}
\usepackage{amsmath,amsthm,amsfonts,amssymb,amscd}
\usepackage{lastpage}
\usepackage{enumerate}
\usepackage{fancyhdr}
\usepackage{mathrsfs}
\usepackage{xcolor}
\usepackage{graphicx}
\usepackage{listings}
\usepackage{hyperref}
\usepackage{tikz}
\usepackage{relsize}
\usepackage{fancyvrb}
\usetikzlibrary{shapes.geometric,fit}

\hypersetup{%
  colorlinks=true,
  linkcolor=blue,
  linkbordercolor={0 0 1}
}

\setlength{\parindent}{0.0in}
\setlength{\parskip}{0.05in}

\newcommand\course{MATH 1207}
\newcommand\hwnumber{11}
\newcommand\NetIDa{dc3451}
\newcommand\NetIDb{David Chen}

\theoremstyle{definition}
\newtheorem*{statement}{Statement}
\newtheorem*{claim}{Claim}
\newtheorem*{theorem}{Theorem}

\newcommand{\contra}{\Rightarrow\!\Leftarrow}
\newcommand{\R}{\mathbb{R}}
\newcommand{\F}{\mathbb{F}}
\newcommand{\Z}{\mathbb{Z}}
\newcommand{\Ze}{\mathbb{Z}_{\geq 0}}
\newcommand{\Zg}{\mathbb{Z}_{>0}}
\newcommand{\N}{\mathbb{N}}
\newcommand{\Q}{\mathbb{Q}}
\newcommand{\C}{\mathbb{C}}

\pagestyle{fancyplain}
\headheight 35pt
\lhead{\NetIDa}
\lhead{\NetIDa\\\NetIDb}
\chead{\textbf{\Large Homework \hwnumber}}
\rhead{\course \\ \today}
\lfoot{}
\cfoot{}
\rfoot{\small\thepage}
\headsep 1.5em

\begin{document}

\subsection*{Apostol p.391 no.3}

\begin{align*}
  \sum_{n = 2}^\infty a_n = \sum_{n = 2}^\infty \frac{1}{n^2 - 1} &= \sum_{n = 2}^\infty \frac{1}{(n-1)(n+1)} \\
                                                                  &= \sum_{n = 2}^\infty \frac{1}{2}(\frac{1}{n-1} - \frac{1}{n+1}) \\
                                                                  &= \sum_{n=2}^\infty\frac{1}{2} ((\frac{1}{n-1} + \frac{1}{n}) - (\frac{1}{n} + \frac{1}{n+1})) \\
\end{align*}

Note that we now have that if we take $b_n = -\frac{1}{2}(\frac{1}{n} + \frac{1}{n+1})$, we
have then $a_n = b_n - b_{n-1}$. Then, $\sum_{n=2}^m a_n = b_m - b_1$. Then,
$\sum_{n=2}^\infty \frac{1}{n^2 - 1} = \lim_{m\rightarrow \infty} (b_m - b_1) = \lim_{m\rightarrow
  \infty} (-\frac{1}{2}(\frac{1}{m} + \frac{1}{m+1}) + (\frac{1}{1} + \frac{1}{2})) = \frac{3}{4}$.

\subsection*{Apostol p.391 no.4}

\begin{align*}
  \sum_{n=1}^\infty \frac{2^n + 3^n}{6^n} &= \sum_{n=1}^\infty \frac{2^n}{6^n} + \frac{3^n}{6^n} \\
                                          &= \sum_{n=1}^\infty (\frac{1}{3})^n + (\frac{1}{2})^n \\
                                          &= \sum_{n=1}^\infty (\frac{1}{3})^n + \sum_{n=1}^\infty(\frac{1}{2})^n \\
                                          &= \frac{\frac{1}{3}}{1 - \frac{1}{3}} + \frac{\frac{1}{2}}{1 - \frac{1}{2}} \\
                                          &= \frac{1}{2} + 1 = \frac{3}{2}
\end{align*}

Note that the formula for geometric series was derived in class and that the
splitting of the sum is also justified in class as well in the proof of the
propositions about absolute convergence.

\subsection*{Apostol p.411 no.50}

\begin{claim}
  If $\sum |a_n|$ converges, then $\sum a_n^2$ also converges.
\end{claim}

\begin{proof}
  This proof will use freely the result from part C that $\sum_{i=0}^\infty a_i$
  converges if and only if $\sum_{i=l}^\infty a_i$ converges, as that does not
  depend on this result, as well as the result, also from part C, that
  $\sum_{i=1}^\infty a_i$ converging implies that $\lim_{i\rightarrow \infty}a_i
  = 0$.

  Since we have that $\sum |a_n|$ converges, we know that $\{|a_n|\}$ converges
  to 0. Then $\exists N \in \Zg \mid n \geq N \implies ||a_n| - 0| = |a_n| < 1
  \implies |a_n|^2 = a_n^2 < 1$. Consider now the sums $\sum_{i=N}^\infty|a_n|,
  \sum_{i=N}^\infty a_n^2$. We have by the comparison test, since $|a_n| >
  a_n^2$, that $\sum_{i=N}^\infty a_n^2$ must also converge, and so $\sum a_n^2$
  in general also converges.
\end{proof}

The converse must be false. Consider the harmonic series; $\sum_{i=1}^\infty |a_n| = \sum_{i=1}^\infty
\frac{1}{n}$ diverges, but $\sum_{i=1}^\infty a_n^2 = \sum_{i=1}^\infty \frac{1}{n^2}$ converges.

\section*{Problem 1}

\begin{claim}
  If $\{a_n\}$ is indexed from 0, then for any $k \in \Ze$,
  $\sum_{i=0}^\infty a_n$ converges if and only if $\sum_{i=k}^\infty a_n :=
  \sum_{n=0}^\infty a_{i+k}$ also converges.
\end{claim}

\begin{proof}
  ($\implies$) Let $b_n = \sum_{i=0}^na_i$, and let $c_n = \sum_{i=0}^na_{i+k} =
  \sum_{i=k}^{k+n}a_i$.
  Then, we have that for $n \geq k$, $b_n = \sum_{i=0}^na_i = \sum_{i=0}^{k-1}
  a_i + \sum_{i=k}^n a_i = b_{k-1} + c_{n-k}$. Reindexing, we have that $c_{n} =
  b_{n+k} - b_{k-1}$. Then, since we have that
  $\lim_{n\rightarrow \infty} b_n$ exists, for any $\epsilon > 0 \exists N \in
  \Ze \mid n > N \implies |b_n - L| < \epsilon$. Then, we must also have that
  $\lim_{n\rightarrow \infty} b_{k-1} + c_n$ exists, as for $\epsilon$, consider 
  take $N' = \max(k, N)$. Then, $n > N' \implies |c_n - (L - b_{k-1})| =
  |b_{n+k} - b_{k-1} - (L - b_{k-1})| = |b_{n+k} - L| < \epsilon$.

  ($\impliedby$) Let $b_n, c_n$ be as before, such that $b_n = b_{k-1} +
  c_{n-k}$.  Then, we have that $\lim_{n\rightarrow \infty} c_n = L$, so for any
  $\epsilon > 0, \exists N \mid n > N \implies |c_n - L| < \epsilon$. Then,
  $\lim_{n\rightarrow \infty}b_n$ must also converge, as for $\epsilon$ take $N'
  = \max(k, N)$, so $n > N' \implies |b_n - (L + b_{k-1})| = |b_{k-1} + c_n - (L
  + b_{k-1})| = |c_n - L| < \epsilon$.
\end{proof}

Note that this statement immediately generalizes to sums not indexed from 0;
for any sequence $\{a_n\}$ indexed from $m$, let $a_0, a_1,...,a_m = 0$. Then,
we have that $\sum_{i=m}^k a_i = \sum_{i=0}^k$, and the result shows that
$\sum_{i=m}^k a_i$ converges if and only if $\sum_{i=m'}^k a_i$ converges for
$m' \geq m$.

\section*{Problem 2}

\begin{claim}
  For sequences $\{ a_n \}, \{ b_n \}$ and a number $M \in \Zg$ such that $n
  \geq M \implies a_n = b_n$, prove that the series $\sum_{i=1}^\infty a_n$
  converges if and only if $\sum_{i=1}^\infty b_n$ does.
\end{claim}

\begin{proof}
  $\sum_{i=1}^\infty a_n$ converges if and only if, as above, $\sum_{i=M}^\infty
  a_n = \sum_{i=M}^\infty b_n$ also converges, which converges if and only if
  $\sum_{i=1}^\infty b_n$ converge.
\end{proof}

\section*{Problem 3}

\begin{claim}
  If $\{ a_n \}$ is a sequence and the sum $\sum_{i=1}^\infty a_i$ converges,
  then the sequence converges to 0.
\end{claim}

\begin{proof}
  Let $b_n = \sum_{i=1}^n a_i$,
  and $\lim_{n\rightarrow \infty}b_n = L$. Then, for any $\frac{\epsilon}{2} > 0, \exists
  N \in \Zg \mid n > N \implies |b_n - L| < \frac{\epsilon}{2} \implies |b_{n+1} - L| +
  |b_{n} - L| = |b_{n+1} - L| + |L - b_n|< \epsilon \implies |b_{n+1} - L + (L - b_n)| = |b_{n+1} - b_n| =
  |a_{n+1} - 0| < \epsilon$. Then, we have that $\forall \epsilon > 0$, $\exists
  N \in \Zg \mid n > N \implies |a_n - 0| < \epsilon$, so $\{a_n\}$ must
  converge to 0.
\end{proof}

\section*{Problem 4}

\begin{claim}
  A function $f: \R \rightarrow \R$ is continuous if and only if whenever $\{
  x_n \}$ converges with $\lim_{n\rightarrow \infty} x_n = x$, then $\{f(x_n)\}$
  converges with $\lim_{n\rightarrow \infty} f(x_n) = f(x)$.
\end{claim}

\begin{proof}
  ($\implies$) $f$ is continuous means that for any $\epsilon > 0$, $\exists
  \delta \mid |x_n - x| < \delta \implies |f(x_n) - f(x)| < \epsilon$. Then,
  since $x_n \rightarrow x$, for $\delta \exists N \in \Zg \mid n > N \implies
  |x_n - x| < \delta \implies |f(x_n) - f(x)| < \epsilon$.

  ($\impliedby$) Suppose that $f$ is not continuous. Then $\exists \epsilon > 0
  \mid \forall \delta > 0, \exists x' \mid |x' - x| < \delta$ and $|f(x') -
  f(x)| \geq \epsilon$. We define $x_n$ as follows: let $\delta = \frac{1}{n}$.
  Then, $\exists x' \mid |x' - x| < \frac{1}{n}$ and $|f(x') - f(x)| \geq
  \epsilon$ since $f$ is discontinuous at $x$. Let $x_n = x'$.

  Now, $\forall n, |f(x_n) - f(x)| \geq \epsilon$, so $\lim_{n\rightarrow
    \infty}f(x_n) \neq f(x)$. Further, for any $\epsilon > 0$, let $N = \lceil
  \frac{1}{\epsilon} \rceil$. Then, we have that $|x_n - x| < \frac{1}{n}$, so
  $n > N \implies |x_n - x| < \frac{1}{N} = \frac{1}{\lceil
    \frac{1}{\epsilon} \rceil} < \frac{1}{\frac{1}{\epsilon}} = \epsilon$, so
  $\lim_{n\rightarrow \infty} x_n = x$. $\contra$, so $f$ must be continuous.
\end{proof}

\section*{Problem 5}
\subsection*{a}

\begin{claim}
  Let $\{ a_n \}$ be a sequence. Suppose that for all $c \in \R$, $\exists  N
  \in \Zg \mid \forall n \in \Zg, n \geq N \implies |a_n| > c$. $\{a_n\}$ is divergent.
\end{claim}

\begin{proof}
  Any convergent series $\{ a_n \}$ must be bounded above. Let $\lim_{n\rightarrow
    \infty}a_n = a$. To see why, pick $\epsilon = 1$, such that $\exists N \mid
  n \geq N \implies |a_n - a| < 1$. Then, let $M_1$ be the maximum of
  $\{a_1,a_2,...a_N\}$. Then, we see that $\forall n$, $a_n \leq \max(M_1, a + 1)$.

  Now suppose that the initial hypothesis holds, and $\{ |a_n| \}$ is bounded by
  $M$. However, for $M \in \R$, $\exists N \in \Zg \mid n \geq N \implies
  |a_n| > M$. $\contra$, so $\{ |a_n| \}$ is not bounded, and thus diverges.
  Further, since  $\{|a_n|\}$ diverges then $\{a_n\}$ must also diverge as shown
  on an earlier homework.
\end{proof}

\subsection*{b}

\begin{claim}
  $|x| > 1 \implies \{x^n\}$ is divergent.
\end{claim}

\begin{proof}
  We will show that for all $c \in \R$, $\exists N \in \Zg \mid \forall n \in
  \Zg, n \geq N \implies |a_n| > c$.
  
  First, we have that $|x| > 1 \implies |x^{n+1}| = |x^n \cdot x| = |x^n||x| >
  |x^n|$, so $a > b \implies |x^a| > |x^b|$.

  If $c \leq 1$, we have that $|x^1| = |x| > c$, so $n > 1 \implies |x^n| > c$.
 
  For $c \geq 1$, consider that $x^y = e^{y\log(x)}$. We know that that $y\log(x): \R
  \rightarrow \R$, where $\log(x)$ is a constant, is surjective and
  monotonically increasing, and also that $e^y: \R \rightarrow \Rg$ is also
  surjective and monotonically increasing, so $x^y: \R \rightarrow \Rg$ must be
  surjective and monotonically increasing, and so $\exists y' \mid x^{y'} = c$.
  Then, let $N = \lceil y \rceil$. We have that $n > N \implies x^n > x^N \geq
  x^{y'} = c$, as $x^{y'}$ is monotonically increasing. Thus, $\{x^n\}$ must diverge.
\end{proof}

\subsection*{c}

\begin{claim}
  $|x| > 1 \implies \sum_{n=0}^\infty x^n$ is divergent.
\end{claim}

\begin{proof}
  Suppose $\sum_{n=0}^\infty x^n$ converges. Then, we must have that
  $\lim_{n\rightarrow \infty} \{x^n\}$ converges to 0. However,
  $\lim_{n\rightarrow \infty} \{x^n\}$ diverges, so the sum must diverge.
\end{proof}

\end{document}

% LocalWords:  NetID fancyplain LocalWords colorlinks linkcolor linkbordercolor
