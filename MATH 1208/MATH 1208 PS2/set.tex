\documentclass[12pt,letterpaper]{article}
\usepackage{fullpage}
\usepackage[top=2cm, bottom=4.5cm, left=2.5cm, right=2.5cm]{geometry}
\usepackage{amsmath,amsthm,amsfonts,amssymb,amscd}
\usepackage{lastpage}
\usepackage{enumerate}
\usepackage{fancyhdr}
\usepackage{mathrsfs}
\usepackage{xcolor}
\usepackage{graphicx}
\usepackage{listings}
\usepackage{hyperref}
\usepackage{tikz}
\usepackage{relsize}
\usepackage{fancyvrb}
\usetikzlibrary{shapes.geometric,fit}

\hypersetup{%
  colorlinks=true,
  linkcolor=blue,
  linkbordercolor={0 0 1}
}

\setlength{\parindent}{0.0in}
\setlength{\parskip}{0.05in}

\newcommand\course{MATH 1208}
\newcommand\hwnumber{1}
\newcommand\NetIDa{dc3451}
\newcommand\NetIDb{David Chen}

\theoremstyle{definition}
\newtheorem*{statement}{Statement}
\newtheorem*{claim}{Claim}
\newtheorem*{theorem}{Theorem}
\newtheorem*{lemma}{Lemma}

\newcommand{\contra}{\Rightarrow\!\Leftarrow}
\newcommand{\R}{\mathbb{R}}
\newcommand{\F}{\mathbb{F}}
\newcommand{\Z}{\mathbb{Z}}
\newcommand{\Zeq}{\mathbb{Z}_{\geq 0}}
\newcommand{\Zg}{\mathbb{Z}_{>0}}
\newcommand{\Req}{\mathbb{R}_{\geq 0}}
\newcommand{\Rg}{\mathbb{R}_{>0}}
\newcommand{\N}{\mathbb{N}}
\newcommand{\Q}{\mathbb{Q}}
\newcommand{\C}{\mathbb{C}}

\pagestyle{fancyplain}
\headheight 35pt
\lhead{\NetIDa}
\lhead{\NetIDa\\\NetIDb}
\chead{\textbf{\Large Homework \hwnumber}}
\rhead{\course \\ \today}
\lfoot{}
\cfoot{}
\rfoot{\small\thepage}
\headsep 1.5em

\begin{document}

\section*{Problem 1}



\section*{Problem 2}

\subsection*{a}

We will first show that $Id_V - T$ is linear.
\begin{align*}
  (Id_v-T)(cx) &= Id_V(cx) - T(cx) \\
               &= cx - cT(x) \\
               &= c(Id_V(x)) - cT(x) \\
               &= c(Id_V(x) - T(x)) \\
               &= c((Id_v - T)(x)) \\
  (Id_v - T)(x + y) &= Id_V(x + y) - T(x + y) \\
               &= x + y - T(x) - T(y) \\
               &= Id_V(x) - T(x) + Id_V(y) - T(y) \\
               &= (Id_V - T)(x) + (Id_V - T)(y)
\end{align*}

It has an inverse, namely $Id_v + T + T^2$:

\begin{align*}
  (Id_v + T + T^2)((Id_v - T)(x)) &= Id_v(x - T(x)) + T(x - T(x)) + T(T(x - T(x))) \\
                                  &= x - T(x) + T(x) - T(T(x)) + T(T(x)) - T(T(T(x))) \\
                                  &= x
\end{align*}

Via the theorem proved in class, we have that $Id_V - T$ is an isomorphism.

\section*{Problem 3}

\begin{claim}
  $\{\sin(x), \sin(2x),..., \sin(2^nx),...\}$ is linearly independent.
\end{claim}

\begin{proof}
  Suppose that we have some linear combination $\sum_{i=0}^na_isin(2^ix) = 0$.
  Consider $x = \frac{\pi}{2^{k+1}}$, where $k$ is the least integer such that
  $a_k \neq 0$.

  Then, we have that $\sin(\frac{2^i\pi}{2^{k+1}}) = \sin(2^{i - k - 1}\pi) = 0$ for
  any $i > k$; for any $i < k$, we have that $a_i = 0$; for $i = k$, we have
  that $\sin(\frac{2^k\pi}{2^{k+1}}) = \sin(\frac{\pi}{2}) = 1$.
\end{proof}

\section*{Problem 4}

\begin{claim}
  $\{1, 1 + x, 1 + x + x^2, ..., 1 + x + x^2 + ... + x^n, ...\}$ is linearly independent.
\end{claim}

\begin{proof}
  % We will show that $\sum_{i=0}^na_i\sum_{j=0}^ix^j = 0 \iff a_i = 0$ for $0
  % \leq i \leq n$ through induction on $n$. The base case is $n = 0$, where
  % $\sum_{i=0}^0a_i\sum_{j=0}^ix^j = a_0$. It follows immediately that
  % $\sum_{i=0}^na_i\sum_{j=0}^ix^j = 0 \iff a_0 = 0$.

  % Suppose that the above hypothesis holds for $n = k$. Then,
  % $\sum_{i=0}^{k+1}a_i\sum_{j=0}^ix^j = $

  We will show that $\sum_{i=0}^na_i\sum_{j=0}^ix^j =
  \sum_{i=0}^{n}(x^i\sum_{j=i}^{n}a_j)$ through induction on $n$. The base case,
  which has $n = 0$, follows immediately as $\sum_{i=0}^0(a_i\sum_{j=0}^ix^j) =
  a_0$.

  Now assume the above hypothesis for $n = k$. Then,
  \begin{align*}
    \sum_{i=0}^{k+1}(a_i\sum_{j=0}^ix^j) &= \sum_{i=0}^k(a_i\sum_{j=0}^ix^j) + a_{k+1}\sum_{j=0}^{k+1}x^j \\
                                       &= \sum_{i=0}^{k}(x^{i}\sum_{j=i}^{k}a_j) + a_{k+1}\sum_{j=0}^{k+1}x^j \\
                                         &= \sum_{i=0}^k(x^i\sum_{j=i}^{k+1}a_j) + a_{k+1}  \\
                                         &= \sum_{i=0}^{k+1}(x^i\sum_{j=i}^{k+1}a_j)  \\
  \end{align*}

  Since we have from earlier that a polynomial
  $\sum_{i=0}^{n}(x^i\sum_{j=i}^{n}a_j)$ is zero everywhere if and only if
  all of its coefficients are zero, we have that all of $\sum_{j=i}^{k+1}a_j$
  must be zero. Since $i$ ranges from 0 to $n$ inclusive; we can show that these
  are all $0$ if and only if all $a_j = 0$.

  The base case of $$
\end{proof}

\end{document}

% LocalWords:  NetID fancyplain LocalWords colorlinks linkcolor linkbordercolor
