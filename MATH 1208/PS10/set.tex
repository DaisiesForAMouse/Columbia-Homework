\documentclass[12pt,letterpaper]{article}
\usepackage{fullpage}
\usepackage[top=2cm, bottom=4.5cm, left=2.5cm, right=2.5cm]{geometry}
\usepackage{amsmath,amsthm,amsfonts,amssymb,amscd}
\usepackage{lastpage}
\usepackage{enumerate}
\usepackage{fancyhdr}
\usepackage{mathrsfs}
\usepackage{xcolor}
\usepackage{graphicx}
\usepackage{listings}
\usepackage{hyperref}
\usepackage{tikz}
\usepackage{relsize}
\usepackage{fancyvrb}
\usetikzlibrary{shapes.geometric,fit}

\hypersetup{%
  colorlinks=true,
  linkcolor=blue,
  linkbordercolor={0 0 1}
}

\setlength{\parindent}{0.0in}
\setlength{\parskip}{0.05in}

\newcommand\course{MATH 1208}
\newcommand\hwnumber{10}
\newcommand\NetIDa{dc3451}
\newcommand\NetIDb{David Chen}

\theoremstyle{definition}
\newtheorem*{statement}{Statement}
\newtheorem*{claim}{Claim}
\newtheorem*{theorem}{Theorem}
\newtheorem*{lemma}{Lemma}

\newcommand{\contra}{\Rightarrow\!\Leftarrow}
\newcommand{\R}{\mathbb{R}}
\newcommand{\F}{\mathbb{F}}
\newcommand{\Z}{\mathbb{Z}}
\newcommand{\Zeq}{\mathbb{Z}_{\geq 0}}
\newcommand{\Zg}{\mathbb{Z}_{>0}}
\newcommand{\Req}{\mathbb{R}_{\geq 0}}
\newcommand{\Rg}{\mathbb{R}_{>0}}
\newcommand{\N}{\mathbb{N}}
\newcommand{\Q}{\mathbb{Q}}
\newcommand{\C}{\mathbb{C}}
\newcommand{\id}{\mathrm{Id}}
\newcommand{\im}{\mathrm{im}}
\newcommand{\tr}{\mathrm{tr}}
\newcommand{\diag}{\mathrm{diag}}
\newcommand{\rank}{\mathrm{rank}}
\newcommand{\spn}{\mathrm{span}}

\pagestyle{fancyplain}
\headheight 35pt
\lhead{\NetIDa}
\lhead{\NetIDa\\\NetIDb}
\chead{\textbf{\Large Homework \hwnumber}}
\rhead{\course \\ \today}
\lfoot{}
\cfoot{}
\rfoot{\small\thepage}
\headsep 1.5em

\begin{document}

\subsection*{Apostol p.328 no.7}

Put $\gamma(t) = (t, 2t, 4t)$, for $0 \leq t \leq 1$. Then,

\begin{align*}
  \int f \cdot d\gamma &= \int_0^1 (t, 2t, 4t^2 - 2t) \cdot (1,2,4)dt \\
                       &= \int_0^1 (16t^2 - 3t)dt \\
                       &= \frac{16}{3} - \frac{3}{2} = \frac{23}{6}
\end{align*}

\subsection*{Apostol p.337 no.5a}

\begin{claim}
  $f(x,y,z) = (P(x,y,z), Q(x,y,z), R(x,y,z)) = (y, x, x)$ is not conservative.
\end{claim}

\begin{proof}
  Using the result of the previous problem, if $f$ were conservative, then
  \[
    \frac{\partial P}{\partial z} = \frac{\partial R}{\partial x}
  \]
  However, we easily see that the left hand side is 0, and the right hand side
  is 1. Thus, $f$ is not conservative.

  Further, consider the closed path $\gamma(t) = (\cos(t), 0, \sin(t))$, for $0
  \leq t \leq 2\pi$.
  \begin{align*}
    \oint f \cdot \gamma &= \int_0^{2\pi} (0, \cos(t), \cos(t)) \cdot (-\sin(t), 0, \cos(t)) dt \\
                         &= \int_0^{2\pi} \cos^2(t)dt > 0
  \end{align*}
\end{proof}


\subsection*{Apostol p.345 no.3}

\begin{align*}
  \varphi(x,y) &= \int (2xe^y + y)dx + A(y) = x^2e^y + xy + A(y) \\
  \varphi(x,y) &= \int (x^2e^y + x - 2y)dy + B(x) = x^2e^y + xy - y^2 + A(y) \\
\end{align*}

Then, we can take $A(y) = -y^2, B(x) = 0$, and arrive at $\varphi =
x^2e^y+xy-y^2 + C$, where $C \in \R$ is a constant.

\subsection*{Apostol p.345 no.9}

% \begin{align}
%   \varphi(x,y,z) &= \int 3y^4z^2dx + A(y,z) = 3y^4z^2x + A(y,z)\\
%   \varphi(x,y,z) &= \int 4x^3z^2dy + B(x,z) = 4x^3z^2y + B(x,z)\\
%   \varphi(x,y,z) &= \int -3x^2^2dz + C(x,y) = -3x^2y^2z + C(x,y)\\
% \end{align}

This is not a gradient; from the earlier results of an Apostol problem,
\[
  D_1f_2 = 12x^2z^2 \neq D_2f_1 = 12y^3z^2
\]
shows that this is not a conservative vector field.

\subsection*{Apostol p.345 no.14}

\begin{claim}
  If $\varphi, \psi$ are potential functions for a continuous vector field $f$
  on an open connected set $S \subset \R^n$, then $\varphi - \psi$ is constant
  on $S$.
\end{claim}

\begin{proof}
  Note that $\nabla(\varphi - \psi) = \nabla \varphi - \nabla \psi = f - f = 0$.

  Then, this implies that all directional derivatives $(\varphi - \psi)'(x;y)$
  are zero (they also exist, as they are assumed differentiable), and by an
  earlier homework problem $\varphi - \psi$ must be constant on $S$.
\end{proof}


\subsection*{Apostol p.346 no.18}

\subsubsection*{a}

\begin{claim}
  Put $T = \R^2 \setminus \{(x,y) \mid y = 0, x \leq 0\}$, and
  \[
    x = r\cos(\theta), y = r\sin(\theta)
  \]
  where $-\pi < \theta < \pi$, and $r > 0$. Then,
  \[
    \theta =
    \begin{cases}
      \arctan(\frac{y}{x}) & x > 0\\
      \frac{\pi}{2} & x = 0 \\
      \arctan(\frac{y}{x}) + \pi & x < 0\\
    \end{cases}
  \]
\end{claim}

\begin{proof}
  If $x > 0, \frac{y}{x} = \tan(\theta) \implies \theta = \arctan(\frac{y}{x})$, and we
  know that $\cos(\theta) > 0 \implies -\frac{\pi}{2} < \theta < \frac{\pi}{2}$,
  which is what we want.

  If $x = 0$, then $\cos(\theta) = 0$, so $\theta = \frac{\pi}{2}$.

  If $x < 0$, we must have that $\cos(\theta) < 0 \implies \frac{\pi}{2} <
  \theta < \pi$ or $-\pi < \theta < -\frac{\pi}{2}$. Apostol seems here to
  allow, in the third quadrant, that $\pi < \theta < \frac{3\pi}{2}$, but these
  are the same angles as $-\pi < \theta < -\frac{\pi}{2}$, as shifting by $2\pi$
  doesn't change the angle.
\end{proof}

\subsubsection*{b}

\begin{claim}
  $\theta$ is a potential function for $f$ on $T$.
\end{claim}

\begin{proof}
  From a formula for the derivative of $\arctan$ in Apostol, we compute the following:
  \begin{align*}
    \frac{\partial \theta}{\partial x} &= -y\frac{1}{x^2}\frac{1}{1 + (\frac{y}{x})^2} =- \frac{y}{x^2 + y^2} \\
    \frac{\partial \theta}{\partial y} &= \frac{1}{x}\frac{1}{1 + (\frac{y}{x})^2} = \frac{x}{x^2 + y^2} \\
  \end{align*}

  Note that even with the piecewise definition of $\theta$, we know that
  $\lim_{x\rightarrow 0}\theta = \frac{\pi}{2}$, and thus $\theta$ is
  continuous; however, for both $x < 0, x >0$, we have that $D_1(\theta) =
  \frac{\partial}{\partial x}\arctan(\frac{y}{x})$ and $D_2(\theta) =
  \frac{\partial}{\partial y}\arctan(\frac{y}{x})$, as desired.

  This gives $\theta$ as a potential for $f$.
\end{proof}


\section*{Problem 1}

\begin{claim}
  Suppose that $F: \R^2 \setminus \{0\} \rightarrow \R^2$ is the vector field
  \[
    F(x,y) = \left( \frac{x+y}{x^2+y^2}, \frac{y-x}{x^2+y^2} \right)
  \]

  Then, $D_1F_2(x,y) = D_2F_1(x,y)$ on the entire domain, but $F$ is not conservative.
\end{claim}

\begin{proof}
  \begin{align*}
    D_1F_2(x,y) &= \frac{-(x^2 + y^2) - 2x(y-x)}{(x^2 + y^2)^2} = \frac{x^2 -2xy - y^2}{(x^2 + y^2)^2} \\
    D_2F_1(x,y) &= \frac{(x^2 + y^2) - 2y(x+y)}{(x^2 + y^2)^2} = \frac{x^2 - 2xy - y^2}{(x^2 + y^2)^2} \\
  \end{align*}

  Thus, we have that $D_1F_2(x,y) = D_2F_1(x,y)$. To show that $F$ is not
  conservative, consider the integral counterclockwise on the unit circle, where
  $\gamma(t) = (\cos(t), \sin(t))$. Then, $\gamma'(t) = (-\sin(t), \cos(t))$.
  \begin{align*}
    \oint F \cdot d\gamma &= \int_0^{2\pi} (\cos(t) + \sin(t), \sin(t) - \cos(t)) \cdot (-\sin(t), \cos(t)) dt \\
                          &= \int_0^{2\pi} (-\sin(t)\cos(t) - \sin^2(t) + \sin(t)\cos(t) - \cos^2(t))dt \\
                          &= \int_0^{2\pi} -1 dt \\
                          &= -2\pi \neq 0
  \end{align*}

  Thus, we have by the theorem proved in class, that $F$ is not conservative.
\end{proof}

\section*{Problem 2}

\begin{claim}
  Suppose that $F: \R^n \setminus \{0\} \rightarrow \R^n$ is a vector field that
  can be expressed as $F(x) = f(||x||)\frac{x}{||x||}$, where $f:(0,\infty)
  \rightarrow \R$ is a continuously differentiable function. $F$ is conservative. 
\end{claim}

\begin{proof}
  First we will show that for some $U \subset \R^n$, if $g: U \rightarrow \R, h:
  \R \rightarrow \R$,
  \[
    \nabla (h \circ g) = (h'\circ g)\nabla g
  \]

  Consider the $i^{th}$ component; the chain rule yields:
  \[
    D_i(h \circ g) = (D_ig)(h' \circ g)
  \]

  Then,
  \[
    \nabla (h \circ g) = ((D_1g)(h' \circ g), (D_2g)(h' \circ g), \dots,
    (D_ng)(h'\circ g)) = (h'\circ g)(D_1g, \dots, D_ng) = (h'\circ g)\nabla g
  \]

  Next, we compute the $j^{th}$ component of $\nabla r = ||x|| = \left( \sum_{i=1}^nx_i^2 \right)^{1/2}$ to be
  \[
    x_j\left( \sum_{i=1}^nx_i^2 \right)^{-1/2}
  \]
  such that
  \[
    \nabla r = \frac{x}{||x||}
  \]
  
  Now consider $\varphi = \int_0^xf(t)dt$ and $r = ||x||$, such that 
  \[
    \nabla (\varphi \circ r) = (\varphi' \circ r) \nabla r = (f \circ r)\nabla r
    = f(||x||)\frac{x}{||x||}
  \]

  We now have a potential $\varphi \circ r$, and so $F$ is conservative.
\end{proof}

\section*{Problem 3}

Let $F:(0,\infty)\times(0,\infty) \rightarrow \R^2$ be defined by
\[
  F(x,y) = \left( \frac{y+1}{x^2y}, \frac{x+1}{xy^2} \right)
\]

\begin{claim}
  $F$ is conservative, and the potential $\varphi$ satisfies
  \[
    \varphi(x,y) = -\frac{x + y + 1}{xy} + C
  \]
  where $C$ is some constant.
\end{claim}

\begin{proof}
  We have that $(0,\infty)\times(0,\infty)$ is an open set (in particular for
  any point $a = (x,y)$, take $\epsilon = \min\{x, y\}$, and $B_\epsilon(a)
  \subset (0,\infty)\times(0,\infty)$), as well as star-shaped, as any point can
  be the center point, as for two points $(x_1, y_1), (x_2, y_2)$, any point on
  the connecting line $(x_1, y_1) + t((x_2-x_1, y_2 - x_1)) = (x_2 + (1-t)x_1,
  y_2 + (1-t)y_1)$ for $0 < t < 1$ has that both coordinates are $> 0$, and thus
  in the domain of $F$.

  Computing,
  \begin{align*}
    D_1F_2(x,y) &= \frac{xy^2 - y^2(x+1)}{x^2y^4} = \frac{-y^2}{x^2y^4} = -\frac{1}{x^2y^2}\\
    D_2F_1(x,y) &= \frac{x^2y - x^2(y+1)}{x^4y^2} = \frac{-x^2}{x^4y^2} = -\frac{1}{x^2y^2}
  \end{align*}
  and we have from class that $F$ is closed on a star-shaped and open domain
  $\implies F$ is conservative.

  The potential $\varphi$ can be found with indefinite integration as specified in
  Apostol:
  \begin{align*}
    \varphi(x, y) &= \int \frac{y+1}{x^2y}dx + A(y) = -\frac{y+1}{xy} + A(y) \\
    \varphi(x, y) &= \int \frac{x+1}{xy^2}dy + A(y) = -\frac{x+1}{xy} + B(x) \\
  \end{align*}
  Setting the two equal, we have $A(y) - B(x) = -\frac{x-y}{xy} = \frac{1}{x} -
  \frac{1}{y}$, yielding $A(y) = -\frac{1}{y}, B(x) = -\frac{1}{x} \implies
  \varphi(x,y) = -\frac{x+y+1}{xy}$. We can arbitraily add some constant $C \in \R$
  that vanishes upon differentiation.
\end{proof}

\end{document}

% LocalWords:  NetID fancyplain LocalWords colorlinks linkcolor linkbordercolor
% LocalWords:  Apostol
 