\documentclass[12pt,letterpaper]{article}
\usepackage{fullpage}
\usepackage[top=2cm, bottom=4.5cm, left=2.5cm, right=2.5cm]{geometry}
\usepackage{amsmath,amsthm,amsfonts,amssymb,amscd}
\usepackage{lastpage}
\usepackage{enumerate}
\usepackage{fancyhdr}
\usepackage{mathrsfs}
\usepackage{xcolor}
\usepackage{graphicx}
\usepackage{listings}
\usepackage{hyperref}
\usepackage{tikz}
\usepackage{relsize}
\usepackage{fancyvrb}
\usetikzlibrary{shapes.geometric,fit}

\hypersetup{%
  colorlinks=true,
  linkcolor=blue,
  linkbordercolor={0 0 1}
}

\setlength{\parindent}{0.0in}
\setlength{\parskip}{0.05in}

\newcommand\course{MATH 1208}
\newcommand\hwnumber{9}
\newcommand\NetIDa{dc3451}
\newcommand\NetIDb{David Chen}

\theoremstyle{definition}
\newtheorem*{statement}{Statement}
\newtheorem*{claim}{Claim}
\newtheorem*{theorem}{Theorem}
\newtheorem*{lemma}{Lemma}

\newcommand{\contra}{\Rightarrow\!\Leftarrow}
\newcommand{\R}{\mathbb{R}}
\newcommand{\F}{\mathbb{F}}
\newcommand{\Z}{\mathbb{Z}}
\newcommand{\Zeq}{\mathbb{Z}_{\geq 0}}
\newcommand{\Zg}{\mathbb{Z}_{>0}}
\newcommand{\Req}{\mathbb{R}_{\geq 0}}
\newcommand{\Rg}{\mathbb{R}_{>0}}
\newcommand{\N}{\mathbb{N}}
\newcommand{\Q}{\mathbb{Q}}
\newcommand{\C}{\mathbb{C}}
\newcommand{\id}{\mathrm{Id}}
\newcommand{\im}{\mathrm{im}}
\newcommand{\tr}{\mathrm{tr}}
\newcommand{\diag}{\mathrm{diag}}
\newcommand{\rank}{\mathrm{rank}}
\newcommand{\spn}{\mathrm{span}}

\pagestyle{fancyplain}
\headheight 35pt
\lhead{\NetIDa}
\lhead{\NetIDa\\\NetIDb}
\chead{\textbf{\Large Homework \hwnumber}}
\rhead{\course \\ \today}
\lfoot{}
\cfoot{}
\rfoot{\small\thepage}
\headsep 1.5em

\begin{document}

\subsection*{Apostol p.263 no.10}

\subsubsection*{a}

\begin{claim}
  If $\nabla f(x) = 0$ for every $x$ in $B(a)$, then $f(x)$ is constant on $B(a)$.
\end{claim}

\begin{proof}
  We have that $f'(x;y) = \nabla f(x) \cdot y = 0$, so $f'(x;y) = 0$ on $B(a)$
  for any $x \in B(a)$.
  
  The mean value theorem then yields for some $0 < \theta < 1$ that on any point
  $x \in B(a)$,
  \[
    f(a) - f(x) = f'(x+\theta(a-x);a-x) = 0 \implies f(x) = f(a)
  \]

  Thus, $f(x)$ is constant everywhere in $B(a)$.
\end{proof}

\subsubsection*{b}

\begin{claim}
  If $f(x) \leq f(a)$ for all $x \in B(a)$, then $\nabla f(a) = 0$.
\end{claim}

\begin{proof}
  Suppose that $\nabla f(a) \neq 0$. Then, we have that
  \[
    f'(a; \nabla f(a)) = \nabla f(a) \cdot \nabla f(a) > 0
  \]

  Then, let 
  \[
    f'(a; \nabla f(a)) = \lim_{t\rightarrow 0}\frac{f(a + t\nabla f(a)) -
      f(a)}{t} = c > 0 
  \]

  In particular, this limit implies that $\exists \delta > 0$ such that $0 < |t|
  < \delta \implies |\frac{f(a + t\nabla f(a)) -
    f(a)}{t} - c| < \frac{c}{2} \implies \frac{f(a + t\nabla f(a)) -
    f(a)}{t} > 0$. Then, $f(a + t\nabla f(a)) > f(a)$, which $\contra$, as we
  can always pick this $t$ such that $0 < |t| < \delta$ and $a + t \nabla f(a) \in
  B(a)$ (if $B(a)$ has radius $\delta'$, then pick $t \mid |t| < \min(\delta,
  \frac{\delta'}{\nabla f(a)})$).
  

  Thus, we have that $\nabla f(x) = 0$.
\end{proof}

\subsection*{Apostol p.269 no.12}

\begin{claim}
  If $\nabla f(x,y,z)$ is always parallel to $(x, y, z)$, then $f$ must assume
  equal values at the points $(0,0,a)$ and $(0,0,-a)$.
\end{claim}

\begin{proof}
  Consider
  \[
    g(x) = f((0, a\cos(x), a\sin(x)))
  \]

  Then, from the chain rule in Apostol,
  \[
    g'(x) = \nabla f(0,a\cos(x),a\sin(x)) \cdot (0,-a\sin(x),a\cos(x)) =- \alpha
    a \cos(x)\sin(x) + \alpha a \sin(x)\cos(x) = 0
  \]

  where $\nabla f(0,a\cos(x),a\sin(x)) = \alpha(0,a\cos(x),a\sin(x))$. Then, we
  have that $g$ is constant and that $f(0,0,a) = g(\frac{\pi}{2}) =
  g(\frac{-\pi}{2}) = f(0,0,-a)$.

\end{proof}

\subsection*{Apostol p.276 no.3ab}

\subsubsection*{a}

From the chain rule given in Apostol that applies to scalar fields,

\begin{gather*}
  \frac{\partial F}{\partial s} = \frac{\partial f}{\partial x}\frac{\partial
    X}{\partial s} + \frac{\partial f}{\partial s} \frac{\partial Y}{\partial s} \\
  \frac{\partial F}{\partial t} = \frac{\partial f}{\partial x}\frac{\partial
    X}{\partial t} + \frac{\partial f}{\partial s} \frac{\partial Y}{\partial t}
\end{gather*}

\subsubsection*{b}

Applying the chain rule to the above,

\begin{align*}
  \frac{\partial^2F}{\partial s^2} &= \frac{\partial}{\partial s}\left(\frac{\partial f}{\partial x}\right)\frac{\partial X}{\partial s} + \frac{\partial f}{\partial x}\frac{\partial}{\partial s}\left({\frac{\partial X}{\partial s}}\right) + \frac{\partial}{\partial s}\left(\frac{\partial f}{\partial y}\right)\frac{\partial Y}{\partial s} + \frac{\partial f}{\partial y}\frac{\partial}{\partial s}\left({\frac{\partial Y}{\partial s}}\right) \\
                                   &= \frac{\partial}{\partial s}\left(\frac{\partial f}{\partial x}\right)\frac{\partial X}{\partial s} + \frac{\partial f}{\partial x}{\frac{\partial^2 X}{\partial s^2}} + \frac{\partial}{\partial s}\left(\frac{\partial f}{\partial y}\right)\frac{\partial Y}{\partial s} + \frac{\partial f}{\partial y}\frac{\partial^2 Y}{\partial s^2} \\
  \intertext{Computing $\frac{\partial}{\partial s}\left( \frac{\partial f}{\partial x} \right)$ with the chain rule,}
  \frac{\partial}{\partial s}\left( \frac{\partial f}{\partial x} \right) &= \frac{\partial D_1f}{\partial x}\frac{\partial X}{\partial s} + \frac{\partial D_1f}{\partial y}\frac{\partial Y}{\partial s} \\
                                   &= \frac{\partial^2 f}{\partial x^2}\frac{\partial X}{\partial s} + \frac{\partial^2 f}{\partial x \partial y}\frac{\partial Y}{\partial s} \\
  \intertext{Similarly,}
  \frac{\partial}{\partial s}\left( \frac{\partial f}{\partial y} \right) &= \frac{\partial D_2f}{\partial x}\frac{\partial X}{\partial s} + \frac{\partial D_2f}{\partial y}\frac{\partial Y}{\partial s} \\
                                   &=  \frac{\partial^2 f}{\partial x \partial y}\frac{\partial X}{\partial s} + \frac{\partial^2 f}{\partial y^2}\frac{\partial Y}{\partial s}\\
  \intertext{Substituting, we arrive at}
  \frac{\partial^2F}{\partial s^2} &= \frac{\partial^2 f}{\partial x^2}\left( \frac{\partial X}{\partial s} \right)^2 + \frac{\partial^2f}{\partial x \partial y}\frac{\partial Y}{\partial s}\frac{\partial X}{\partial s} + \frac{\partial f}{\partial x}\frac{\partial^2 X}{\partial s^2} + \frac{\partial^2 f}{\partial x \partial y}\frac{\partial X}{\partial s}\frac{\partial Y}{\partial s} + \frac{\partial^2 f}{\partial y^2}\left(\frac{\partial Y}{\partial s}\right)^2 + \frac{\partial f}{\partial y}\frac{\partial^2 Y}{\partial s^2} \\
                                   &= \frac{\partial f}{\partial x}\frac{\partial^2 X}{\partial s^2} + \frac{\partial^2 f}{\partial x^2}\left( \frac{\partial X}{\partial s} \right)^2 + 2\frac{\partial^2f}{\partial x \partial y}\frac{\partial Y}{\partial s}\frac{\partial X}{\partial s} +  \frac{\partial f}{\partial y}\frac{\partial^2 Y}{\partial s^2} + \frac{\partial^2 f}{\partial y^2}\left(\frac{\partial Y}{\partial s}\right)^2
\end{align*}


\subsection*{Apostol p.276 no.14}

\subsubsection*{a}

\begin{align*}
  Df &=
       \begin{bmatrix}
         D_1f_1 & D_2f_1 \\
         D_1f_2 & D_2f_2
       \end{bmatrix} \\
     &=
       \begin{bmatrix}
         e^{x + 2y} &  2e^{x + 2y}\\
         2\cos(y + 2x) & \cos(y + 2x)
       \end{bmatrix} \\
  Dg &=
       \begin{bmatrix}
         D_1g_1 & D_2g_1 & D_3g_1 \\
         D_1g_2 & D_2g_2 & D_3g_2\\
       \end{bmatrix} \\
     &=
       \begin{bmatrix}
         1 & 4v & 9w^2 \\
         -2u & 2 & 0 \\
       \end{bmatrix}
\end{align*}

\subsubsection*{b}

\[
  h(u,w,v) = e^{u + v^2 + 3w^3 + 4v - 2u^2}i + \sin(2v - u^2 + 2u + 4v^2 + 6w^3)j
\]

\subsubsection*{c}

\[
  Dh(1, -1, 1) =
  \begin{bmatrix}
    1 & 2 \\
    2\cos(9) & \cos(9)
  \end{bmatrix}
  \begin{bmatrix}
    1 & -4 & 9 \\
    -2 & 2 & 0
  \end{bmatrix} =
  \begin{bmatrix}
    -3 & 0 & 9 \\
    0 & -6\cos(9) & 9\cos(9)
  \end{bmatrix}
\]

\subsection*{Apostol p.281 no.1}

Consider
\[
  f(x,y) =
  \begin{cases}
    0 & xy = 0 \\
    \frac{3}{2}(x + y) &\text{ otherwise} \\
  \end{cases}
\]

Then,
\begin{align*}
  D_1f(0,0) &= \lim_{t\rightarrow 0}\frac{f(t, 0) - f(0,0)}{t} = 0 \\
  D_2f(0,0) &= \lim_{t\rightarrow 0}\frac{f(0, t) - f(0,0)}{t} = 0 \\
  f((0,0); (1,1)) &= \lim_{t\rightarrow 0}\frac{f(t,t) - f(0,0)}{t} = \lim_{t\rightarrow 0}\frac{3t}{t} = 3
\end{align*}

It cannot be differentiable as if it were, we would know that $f'((0,0); (1,1)) =
D_1f(0,0) + D_2f(0,0) = 0$.

\subsection*{Apostol p.281-282 no.12a}

We have that $||\mathbf{r}|| = r(x,y,z) = \sqrt{x^2 + y^2 + z^2}$. Then, for $r \neq 0$,
\begin{align*}
  A \cdot \nabla\left( \frac{1}{r} \right) &= A \cdot \nabla \left( (x^2 + y^2 + z^2)^{-\frac{1}{2}} \right) \\
                                           &= A \cdot \left( -x(x^2 + y^2 + z^2)^{-\frac{3}{2}}, -y(x^2 + y^2 + z^2)^{-\frac{3}{2}}, -z(x^2 + y^2 + z^2)^{-\frac{3}{2}} \right) \\
                                           &= A \cdot \left( - \frac{1}{r^3} \left( x,y,z \right)\right) \\
                                           &= -\frac{A \cdot\mathbf{r}}{r^3}
\end{align*}

\section*{Problem 1}

\begin{claim}
  Let $F: \R^n \rightarrow \R^m$ be linear. The derivative of $F$ at any $x \in \R^n$ is just $F$ itself.
\end{claim}

\begin{proof}
  We want for a total derivative $T$ that
  \begin{align*}
    \lim_{H\rightarrow 0}\frac{||F(x + H) - F(x) - T(H)||}{||H||} &= \lim_{H\rightarrow 0}\frac{||F(x) + F(H) - F(x) - T(H)||}{||H||}\\
                                                              &= \lim_{H\rightarrow 0} \frac{||F(H) - T(H)||}{||H||} = 0
  \end{align*}

  Then, we see that taking $T = F$ clearly works, and since we proved the
  uniqueness of total derivatives in class, we are done.
\end{proof}

\section*{Problem 2}

\begin{claim}
  Let $G: \R^n \rightarrow \R$ be homogeneous and continuous. $G$ is
  differentiable $\iff G$ is linear.
\end{claim}

\begin{proof}
  $(\implies)$ Now, consider that since $G$ is differentiable, we have that for
  any fixed $y$, putting $T$ as the derivative,
  \[
    \lim_{t \rightarrow 0}\frac{G(x + ty) - G(x) - T(ty)}{||ty||} = 0
  \]

  In particular, if we select $y = x$,

  \begin{align*}
    \lim_{t \rightarrow 0}\frac{G(x + tx) - G(x) - T(tx)}{||tx||} &=  \lim_{t \rightarrow 0}\frac{(t+1)G(x) - G(x) - tT(x)}{t||x||} \\
                                                                  &= \lim_{t \rightarrow 0}\frac{t(G(x) -T(x))}{t||x||}\\
                                                                  &= \frac{G(x)-T(x)}{||x||} = 0
  \end{align*}
  Then we have that $G = T$ for any $x$.

  The mean value theorem yields that for any $a,b \in \R^n$ and some $0 < \theta < 1$,
  \[
    G(a) - G(b) = G'(b+\theta(a-b);a-b) 
  \]

  However, we have that $G'(b+\theta(a-b); a-b) = (G'(b+\theta(a-b)))(a-b)$, but
  the total derivative anywhere is just $G$ itself, and thus
  \[
    G(a) - G(b) = G(a-b)
  \]
  
  So we finally have that $G(a) - G(b) = G(a - b)$; replacing $b$ with $-b$,
  since this holds for any $b$, we have that $G(a + b) = G(a) + G(b)$, and that
  $G(tx) = tG(x)$, and so $G$ is linear.

  $(\impliedby)$ As in the first problem, we have that the derivative of $G$ at
  any point is $G$ itself, and is thus differentiable.
\end{proof}

\section*{Problem 3}

\begin{claim}
  Suppose that $G: \R^n \rightarrow \R^m$, with $U$ open in $\R^n$, is written
  with coordinates as
  \[
    F = (F_1, F_2, \dots, F_m)
  \]
  where $F_i:U \rightarrow \R$ for $1 \leq i \leq m$. Then, for any point $x \in
  U$, $F$ is differentiable at $x \iff F_i$ is differentiable at $x$ for all $1
  \leq i\leq m$. 
\end{claim}

\begin{proof}
  $(\implies)$
  The total derivative $T$ satisfies that
  \[
    \lim_{H\rightarrow 0}\frac{||F(x + H) - F(x) - T(x)||}{||H||} = 0 \iff \lim_{H\rightarrow 0}\frac{F(x + H) - F(x) - T(x)}{||H||} = 0
  \]

  In particular, we have that this means that each individual component must
  approach $0$, i.e. for some $X = (x_1, x_2, \dots, x_n)$, $\lim_{H\rightarrow
    0} X = 0 \iff \lim_{H\rightarrow 0}x_i = 0$. This was shown in class.

  Then, putting $T(x) = (T_1(x), T_2(x), \dots, T_m(x))$ in the same way as $F$,
  \begin{align*}
    &\lim_{H\rightarrow 0}\frac{F(x + H) - F(x) - T(H)}{||H||}\\
    = &\lim_{H\rightarrow 0}\frac{(F_1(x + H),\dots,F_m(x+H)) - (F_1(x),\dots,F_m(x)) - (T_1(H),\dots,T_m(H))}{||H||}  \\
    =&\lim_{H\rightarrow 0}\left( \frac{F_1(x+H)-F_1(x)-T_1(H)}{||H||}, \dots, \frac{F_m(x+H)-F_m(x)-T_m(H)}{||H||} \right) \\
    =& 0
  \end{align*}

  From above, we have that for $1 \leq i \leq m$,
  \[
    \lim_{H\rightarrow 0}\frac{F_i(x+H)-F_i(x)-T_i(H)}{||H||} = 0
  \]

  Then, we must have that $F_i$ is differentiable with derivative $T_i$.
  
  $(\impliedby)$ Consider $T = (T_1, T_2, \dots, T_n)$, where $F_i$ has
  derivative $T_i$. Then, the above relation
  \[
    \lim_{H\rightarrow 0}\frac{F(x + H) - F(x) - T(H)}{||H||} =
    \lim_{H\rightarrow 0}\left( \frac{F_1(x+H)-F_1(x)-T_1(H)}{||H||}, \dots, \frac{F_m(x+H)-F_m(x)-T_m(H)}{||H||} \right) \\
  \]
  still holds, and we have that $T$
  is the total derivative of $F$ as each component of $\frac{F(x + H) - F(x) -
    T(x)}{||H||}$ is 0 in the limit, and thus $\frac{F(x + H) - F(x) -
    T(x)}{||H||}$ must then be zero itself as $H \rightarrow 0$.
\end{proof}

\section*{Problem 4}

\begin{claim}
  If $F: U \rightarrow \R^m$ is a function such that all of the entries of its
  Jacobian matrix are well-defined and continuous, then $F$ is differentiable.
\end{claim}

\begin{proof}
  We have that each $F_i$ is differentiable, as all of its partial derivatives,
  which is the $i^{th}$ row of the Jacobian, are continuous, and thus $F_i$ is
  $C^1$ on $U$ by assumption. Since we showed in class that $F_i$ being $C^1$ on
  $U \implies F$ is differentiable on $U$, from the above problem, $F$ itself is differentiable.
\end{proof}

\end{document}

% LocalWords:  NetID fancyplain LocalWords colorlinks linkcolor linkbordercolor
% LocalWords:  Apostol
