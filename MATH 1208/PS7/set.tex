\documentclass[12pt,letterpaper]{article}
\usepackage{fullpage}
\usepackage[top=2cm, bottom=4.5cm, left=2.5cm, right=2.5cm]{geometry}
\usepackage{amsmath,amsthm,amsfonts,amssymb,amscd}
\usepackage{lastpage}
\usepackage{enumerate}
\usepackage{fancyhdr}
\usepackage{mathrsfs}
\usepackage{xcolor}
\usepackage{graphicx}
\usepackage{listings}
\usepackage{hyperref}
\usepackage{tikz}
\usepackage{relsize}
\usepackage{fancyvrb}
\usetikzlibrary{shapes.geometric,fit}

\hypersetup{%
  colorlinks=true,
  linkcolor=blue,
  linkbordercolor={0 0 1}
}

\setlength{\parindent}{0.0in}
\setlength{\parskip}{0.05in}

\newcommand\course{MATH 1208}
\newcommand\hwnumber{6}
\newcommand\NetIDa{dc3451}
\newcommand\NetIDb{David Chen}

\theoremstyle{definition}
\newtheorem*{statement}{Statement}
\newtheorem*{claim}{Claim}
\newtheorem*{theorem}{Theorem}
\newtheorem*{lemma}{Lemma}

\newcommand{\contra}{\Rightarrow\!\Leftarrow}
\newcommand{\R}{\mathbb{R}}
\newcommand{\F}{\mathbb{F}}
\newcommand{\Z}{\mathbb{Z}}
\newcommand{\Zeq}{\mathbb{Z}_{\geq 0}}
\newcommand{\Zg}{\mathbb{Z}_{>0}}
\newcommand{\Req}{\mathbb{R}_{\geq 0}}
\newcommand{\Rg}{\mathbb{R}_{>0}}
\newcommand{\N}{\mathbb{N}}
\newcommand{\Q}{\mathbb{Q}}
\newcommand{\C}{\mathbb{C}}
\newcommand{\id}{\mathrm{Id}}
\newcommand{\im}{\mathrm{im}}
\newcommand{\tr}{\mathrm{tr}}
\newcommand{\diag}{\mathrm{diag}}
\newcommand{\rank}{\mathrm{rank}}
\newcommand{\spn}{\mathrm{span}}

\pagestyle{fancyplain}
\headheight 35pt
\lhead{\NetIDa}
\lhead{\NetIDa\\\NetIDb}
\chead{\textbf{\Large Homework \hwnumber}}
\rhead{\course \\ \today}
\lfoot{}
\cfoot{}
\rfoot{\small\thepage}
\headsep 1.5em

\begin{document}

\subsection*{Apostol p.30 no.2}

\subsubsection*{a}

Proceed with Gram-Schmidt:

\begin{align*}
  u_1 &= (1, 1, 0, 0) \\
  e_1 &= \frac{1}{\sqrt{2}}(1, 1, 0, 0) = (\frac{1}{\sqrt{2}}, \frac{1}{\sqrt{2}}, 0, 0)\\
  u_2 &= (0, 1, 1, 0) - \frac{1}{\sqrt{2}}(\frac{1}{\sqrt{2}}, \frac{1}{\sqrt{2}}, 0, 0) = (-\frac{1}{2}, \frac{1}{2}, 1, 0)\\
  e_2 &= \sqrt{\frac{2}{3}}(-\frac{1}{2}, \frac{1}{2}, 1, 0) = (-\frac{1}{\sqrt{6}}, \frac{1}{\sqrt{6}}, \sqrt{\frac{2}{3}}, 0) \\
  u_3 &= (0,0,1,1) - \sqrt{\frac{2}{3}}(-\frac{1}{\sqrt{6}}, \frac{1}{\sqrt{6}}, \frac{2}{3}, 0) = (\frac{1}{3}, -\frac{1}{3}, \frac{1}{3}, 1)\\
  e_3 &= \frac{\sqrt{3}}{2}(\frac{1}{3}, -\frac{1}{3}, \frac{1}{3}, 1) = (\frac{1}{2\sqrt{3}}, -\frac{1}{2\sqrt{3}}, \frac{1}{2\sqrt{3}}, \frac{\sqrt{3}}{2}) \\
\end{align*}

Note that $(1, 0,0,1) = (1,1,0,0) + (0,0,1,1) - (0,1,1,0)$, so it is linearly
dependent. Thus, $e_1, e_2, e_3$ form an orthonormal basis.

\subsubsection*{b}

Proceed with Gram-Schmidt:

\begin{align*}
  u_1 &= (1, 1, 0, 1) \\
  e_1 &= \frac{1}{\sqrt{3}}(1,1,0,1) = (\frac{1}{\sqrt{3}}, \frac{1}{\sqrt{3}}, 0, \frac{1}{\sqrt{3}}) \\
  u_2 &= (1,0,2,1) - \frac{2}{\sqrt{3}}(\frac{1}{\sqrt{3}}, \frac{1}{\sqrt{3}}, 0, \frac{1}{\sqrt{3}}) = (\frac{1}{3}, -\frac{2}{3}, 2, \frac{1}{3})\\
  e_2 &= \sqrt{\frac{3}{14}}(\frac{1}{3}, -\frac{2}{3}, 2, \frac{1}{3}) = (\frac{1}{\sqrt{42}}, -\frac{2}{\sqrt{42}}, \sqrt{\frac{6}{7}}, \frac{1}{\sqrt{42}})
\end{align*}

Note that $(1, 2, -2 ,1) = 2(1, 1, 0, 1) - (1, 0, 2, 1)$, so $e_1, e_2$ form an
orthonormal basis.

\subsection*{Apostol p.30 no.5}

First we will show that $\int_0^\infty e^{-t}t^ndt = n!$ for $n \in \Zg$.

With integration by parts, $\int_0^\infty e^{-t}t^{n+1}dt =
[-e^{-t}t^{n+1}]\Big|_0^\infty - n\int_0^\infty e^{-t}t^ndt = n\int_0^\infty
-e^{-t}t^ndt$.

Inducting on $n$, for $n = 0$ we have that $\int_0^\infty = e^{-t}dt =
[-e^{-t}]\Big|_0^\infty = 1$. Then, assuming the above for $n = k$,
$\int_0^\infty e^{-t}t^{k+1}dt = (k+1)\int_0^\infty e^{-t}t^{k}dt = (k+1)(k!) = (k+1)!$.

\begin{align*}
  y_0 &= 1 \\
  y_1 &= t - \int_0^\infty e^{-t}tdt \\
      &= t - 1 \\
  y_2 &= t^2 - \frac{\langle t^2, 1 \rangle}{\langle 1,1 \rangle} - \frac{\langle t^2, t-1 \rangle}{\langle t-1,t-1 \rangle}(t-1) \\
      &= t^2 - 2! - \frac{3! - 2!}{2! - 2! + 1!}(t-1) = t^2 -4t +  2 \\
  y_3 &= t^3 - \frac{\langle t^3, 1 \rangle}{\langle 1,1 \rangle} - \frac{\langle t^3, t-1 \rangle}{\langle t-1,t-1 \rangle}(t-1) - \frac{\langle t^3, t^2 - 4t + 2 \rangle}{\langle t^2 - 4t + 2 \rangle}(t^2 - 4t + 2) \\
      &= t^3 - 3! - (4! - 3!)(t-1) - \frac{5! - 4(4!)+2(3!)}{4! - 8(3!) - 20(2!) - 16 + 4}(t^2 - 4t + 2) \\
      &= t^3 - 6 - 18(t-1)  - 9(t^2 - 4t + 2) \\
      &= t^3 - 9t^2 + 18t - 6
  % z_2 &= x_2 - \int_0^\infty e^{-t}x_2dt - \int_0^\infty e^{-t} x_2(t-1)dt \\
  %     &= t^2 - \int_0^\infty e^{-t}t^2 dt - \int_0^\infty e^{-t}t^2(t-1)dt (t - 1) \\
  %     &= t^2 - (-e^{-t}(2+2t+t^2)\Big|_0^\infty) - (-e^{-t}(4+4t+2t^2+t^3)\Big|_0^\infty)(t-1)\\
  %     &= t^2 - 2 - 4(t - 1) = t^2 - 4t + 2 \\
  % z_3 &= x_3 - \int_0^\infty e^{-t}x_3dt - \int_0^\infty e^{-t}(t-1)x_3dt(t-1) - \int_0^\infty e^{-t}(t^2 - 4t+2)x_3dt(t^2-4t+2) \\
  %     &= t^3 - \int_0^\infty e^{-t}t^3dt - \int_0^\infty e^{-t}(t-1)t^3dt(t-1) - \int_0^\infty e^{-t}(t^2 - 4t+2)t^3dt(t^2-4t+2) \\
  %     &= t^3 - (-e^{-t}(6+6+3t^2+t^3))\Big|_0^\infty - (-e^{-t}(18+18t+9t^2+3t^3+t^4))\Big|_0^\infty(t-1) - (-e^{-t}(36+36t+18t^2+6t^3+t^4+t^5))\Big_0^\infty(t^2-4t+2) \\
  %     &= t^3 - 6 - 18(t-1) - 36(t^2 - 4t+2) \\
\end{align*}

\section*{Problem 1}

\begin{claim}
  Let $V$ be a finite dimensional inner product space, and $U \subseteq V$ any
  subspace. Then, $\dim(U) + \dim(U^\perp) = V$.
\end{claim}

\begin{proof}
  We will show sometihng stronger:
  \[
    U + U^\perp = V
  \]
  where $U + U^\perp = \{u+v \mid u \in U, v \in U^\perp \}$.

  We have from class that for any vector $v \in V$, $\exists x, x^\perp \mid x +
  x^\perp = v$ where $ x \in U, x^\perp \in U^\perp$, so $V \subseteq U + U^\perp$. Further, since $U, U^\perp$
  are subspaces of $V$, if $u \in U, v \in V^\perp, u, v \in V \implies u + v 
  \in V$, so $U + U^\perp \subseteq V$.

  Then, $U + U^\perp = V$, and from an ealier homework, $\dim(U) + \dim(U^\perp)
  - \dim(U \cap U^\perp) = V$. However, if $x \in U, U^\perp, \langle x, x
  \rangle = 0 \implies x = 0$. Thus, $\dim(U \cap U^\perp) = 0,$ and the initial
  claim follows.
\end{proof}

\section*{Problem 3}

Let $S: U \rightarrow V$ be a linear map with adjoint $S^*$.

\subsubsection*{a}

\begin{claim}
  $S^*$ has an adjoint and $(S^*)^* = S$. 
\end{claim}

\begin{proof}
  \[
    \langle S(u), v \rangle = \langle u, S^*(v) \rangle \implies
    \overline{\langle v, S(u) \rangle} = \overline{\langle S^*(v), u \rangle}
    \implies  \langle v, S(u) \rangle = \langle S^*(v), u \rangle
  \]

  The last part holds by taking the complex conjugate of both sides, which is a
  bijective operation as shown on an earlier homework. Further, since adjoints are unique, $(S^*)^* = S$.
\end{proof}

\subsubsection*{b}

\begin{claim}
  \[
    \ker(S^*) = (\im S)^\perp
  \]
\end{claim}

\begin{proof}
  If $x \in \ker(S^*)$, then we have that
  \[
    \langle S(u), x \rangle = \langle u, S(x) \rangle = \langle u, 0 \rangle = 0
  \]

  Then $x \in (\im S)^\perp$.

  Similarly, if $x \in (\im S)^\perp$ then
  \[
    0 = \langle S(u), x \rangle = \langle u, S^*(x) \rangle 
  \]

  Then, one of $u, S^*(x) = 0$. Since this holds for any $u$, we have that
  $S^*(x) = 0$.

  Thus, $x \in \ker(S^*) \iff x \in (\im S)^\perp$.
\end{proof}

\subsubsection*{c}

\begin{claim}
  \[
    \ker(S) = (\im S^*)^\perp
  \]
\end{claim}

\begin{proof}
  This follows from part a and b immediately, as we simply take part b and apply
  it to $S^*$. Then, since we have that $(S^*)^* = S$, part b gives that
  \[
    \ker((S^*)^*) = \ker(S) = (\im S^*)^\perp
  \]
\end{proof}

\subsubsection*{d}

\begin{claim}
  If $S$ is invertible, then $S^*$ is invertible, $S^{-1}$ has an adjoint, and
  $(S^*)^{-1} = (S^{-1})^*$.
\end{claim}

\begin{proof}

  We know that the adjoint exists, as it is equivalent to the conjugate
  transpose of the original matrix representing the linear transformation.

  Further, since $I^T = I$, and $I$ is a real matrix, then the adjoint of $I$ is
  still $I$.

  In general, we have that
  \[
    \langle u, v  \rangle = \langle u, \id^*(v) \rangle
  \]

  we easily see that $\id^* = \id$ works, and so $\id^* = \id$ as adjoints are unique.


  Now, from problem 2, we have that
  \[
    (T \circ S)^*= S^* \circ T^*
  \]

  Taking $T = S^{-1}$, we have that $\id^* = S^* \circ (S^{-1})^*$. By the
  definition of inverse, we have that $(S^{-1})^* = (S^*)^{-1}$.
\end{proof}

\section*{Problem 4}

\begin{claim}
  Let $S: U \rightarrow V$ be a linear map between finite inner product spaces.
  Then,
  \[
    \dim(\im S^*) = \dim(\im S).
  \]
\end{claim}

\begin{proof}
  We have that as $\ker(S^*) = (\im(S))^\perp$, $\dim(\ker(S^*)) =
  \dim((\im(S))^\perp)$. From problem 1, we have that $\dim((\im(S))^\perp) =
  \dim(V) - \dim(\im(S))$, and so \[
    \dim(\ker(S^*)) = \dim(V) - \dim(\im(S))
  \]  From rank-nullity, we have that \[
    \dim(V) = \dim(\im(S^*)) + \dim(\ker(S^*))
  \]

  Combining the two, we get that $\dim(\im(S^*)) = \dim(\im(S))$.

  Now, we have from the email that $S^*$ is the complex conjugate of the
  transpose of $A$, where $A$ is the matrix representing $S$ for a choice of
  basis. Further, put $\overline{A}$ for the matrix with entries the complex
  conjugate of $A$.

  Then, $\dim(\im(S^*)) = \dim(\im(S))$, where the image of $S$ is simply the
  column space of $A$, and the image of $S^*$ is then the column space of
  $\overline{A^T}$ (as this is a property of column spaces shown in class).

  Now, we have that the dimension of the span of a set of vectors $v_1, \dots
  v_n$ is the same as the dimension of the space of $\overline{v_1}, \dots,
  \overline{v_n}$.

  To see this, note that for $x = \sum_{i=1}^nc_iv_i$, $\overline{x} =
  \sum_{i=1}^nc_i\overline{v_i}$, by a property proved on a past homework.

  Then, any basis of $v_1, \dots
  v_n$, say $x_1, \dots, x_k$, has a corresponding basis $\overline{x_1}, \dots
  \overline{x_n}$ for $\overline{v_1}, \dots
  \overline{v_n}$, and so they must be of the same dimension.

  Then, the column space of $A^T$ has the same dimension as the column space of
  $\overline{A^T}$; since the column rank of $\overline{A^T} = \dim(\im(S^*))$,
  we have that $\dim(\im(S^*)) = $ the column rank of $A^T$.

  The column rank of $A^T$, however, is also the row rank of $A$, by the
  definition of transposes, and
  so the row rank of $A = \dim(\im(S^*)) = \dim(\im(S)) = $ the column rank of
  $A$.

  Thus, we have that row and column rank are the same.
\end{proof}

\end{document}

% LocalWords:  NetID fancyplain LocalWords colorlinks linkcolor linkbordercolor
% LocalWords:  Apostol
