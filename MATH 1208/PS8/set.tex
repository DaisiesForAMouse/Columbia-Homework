\documentclass[12pt,letterpaper]{article}
\usepackage{fullpage}
\usepackage[top=2cm, bottom=4.5cm, left=2.5cm, right=2.5cm]{geometry}
\usepackage{amsmath,amsthm,amsfonts,amssymb,amscd}
\usepackage{lastpage}
\usepackage{enumerate}
\usepackage{fancyhdr}
\usepackage{mathrsfs}
\usepackage{xcolor}
\usepackage{graphicx}
\usepackage{listings}
\usepackage{hyperref}
\usepackage{tikz}
\usepackage{relsize}
\usepackage{fancyvrb}
\usetikzlibrary{shapes.geometric,fit}

\hypersetup{%
  colorlinks=true,
  linkcolor=blue,
  linkbordercolor={0 0 1}
}

\setlength{\parindent}{0.0in}
\setlength{\parskip}{0.05in}

\newcommand\course{MATH 1208}
\newcommand\hwnumber{8}
\newcommand\NetIDa{dc3451}
\newcommand\NetIDb{David Chen}

\theoremstyle{definition}
\newtheorem*{statement}{Statement}
\newtheorem*{claim}{Claim}
\newtheorem*{theorem}{Theorem}
\newtheorem*{lemma}{Lemma}

\newcommand{\contra}{\Rightarrow\!\Leftarrow}
\newcommand{\R}{\mathbb{R}}
\newcommand{\F}{\mathbb{F}}
\newcommand{\Z}{\mathbb{Z}}
\newcommand{\Zeq}{\mathbb{Z}_{\geq 0}}
\newcommand{\Zg}{\mathbb{Z}_{>0}}
\newcommand{\Req}{\mathbb{R}_{\geq 0}}
\newcommand{\Rg}{\mathbb{R}_{>0}}
\newcommand{\N}{\mathbb{N}}
\newcommand{\Q}{\mathbb{Q}}
\newcommand{\C}{\mathbb{C}}
\newcommand{\id}{\mathrm{Id}}
\newcommand{\im}{\mathrm{im}}
\newcommand{\tr}{\mathrm{tr}}
\newcommand{\diag}{\mathrm{diag}}
\newcommand{\rank}{\mathrm{rank}}
\newcommand{\spn}{\mathrm{span}}

\pagestyle{fancyplain}
\headheight 35pt
\lhead{\NetIDa}
\lhead{\NetIDa\\\NetIDb}
\chead{\textbf{\Large Homework \hwnumber}}
\rhead{\course \\ \today}
\lfoot{}
\cfoot{}
\rfoot{\small\thepage}
\headsep 1.5em

\begin{document}

\subsection*{Apostol p.246 no.5}

\subsection*{a}

\begin{claim}
  $\emptyset$ is open.  
\end{claim}

\begin{proof}
  The empty set is open is a vacuously true statement; that is, the statement
  \[
    \forall x \in \emptyset, \exists \epsilon > 0 \mid B_{\epsilon} \subseteq \emptyset
  \]

  is vacuously true.
\end{proof}


\subsection*{b}
\begin{claim}
  $\R^n$ is open.
\end{claim}

\begin{proof}
  For any $x \in \R^n$, take $\epsilon = 1$. Then, $B_{\epsilon}(x) \subseteq \R^n$,
  as the ball is defined to consist of elements of $\R^n$ that satisfy a certain
  condition.
\end{proof}

\subsection*{c}

\begin{claim}
  The union of any collection of open sets is open.
\end{claim}

\begin{proof}
  If an element $x$ is in the union $\cup_{i=1} A_i$, then $\exists i$ such that $x
  \in A_j$. Then, since $A_j$ is assumed to be open, $\exists \epsilon \mid
  B_\epsilon(x) \subseteq A_j$. Then, by the definition of union, $y \in
  B_{\epsilon}(x) \implies y \in \cup A_i \implies B_{\epsilon}(x) \subseteq \cup A_i$.
\end{proof}

\subsection*{d}

\begin{claim}
  The intersection of a finite collection of open sets is open.
\end{claim}

\begin{proof}
  
  If an element $x$ is in the intersection $\cap_{i=1}^n A_i$, then $x \in A_{1} \land x \in A_{2} \land \dots \land
  x \in A_{n}$.

  Now, since each $A_{j}$ is assumed to be open, we have that $\exists
  \epsilon_j$ such that $B_{\epsilon_j}(x) \in A_{j}$. 

  
  Further, for two open balls $B_{\epsilon_1}(x)$,
  $B_{\epsilon_2}(x)$,
  \[
    \epsilon_2 < \epsilon_1 \implies B_{\epsilon_2}(x) \subset B_{\epsilon_1}(x)
  \]

  as if $y \in B_{\epsilon_2}(x)$, we have that $||x - y|| < \epsilon_2 <
  \epsilon_1 \implies y \in B_{\epsilon_1}(x)$.

  Then, take $\epsilon = \min(\epsilon_j)$. We have that $B_{\epsilon}(x)
  \subseteq B_{\epsilon_j}(x) \implies B_{\epsilon}(x) \subseteq A_{j}$ for any
  $j$ in the range $1$ to $n$. Thus,
  \[
    B_{\epsilon}(x) \subseteq \cap_{i=1}^nA_{i}
  \]
\end{proof}

\subsection*{e}

Consider
\[
  S = \bigcap_{n=1}^\infty(-\frac{1}{n}, \frac{1}{n})
\]

% We have that if $x \in (-\frac{1}{n}, \frac{1}{n})$ for every such $n$, we have
% that

Suppose that $x < 0 \in S$. Then, the archimedian property furnishes some $n$
such that $-x > \frac{1}{n}$. Then, $x \notin (-\frac{1}{n}, \frac{1}{n})
\implies x \notin S$.

Similarly, for $x > 0 \in S$, we have some $n \mid x > \frac{1}{n}$, and so $x
\notin S$.

Thus, $S = \{0\} = [0, 0]$, which is closed as seen by the fact that $[a,b]$ in
general is closed, as stated in class.


\subsection*{Apostol p.252 no.6}

Along $y = mx$, we have that $f(x, y)$ becomes
\[
  f(x) = \frac{x^2(1 - m^2)}{x^2(1 + m^2)}
\]

Then,
\[
  \lim_{x \rightarrow 0}f(x) = \frac{1-m^2}{1+m^2} \lim_{x\rightarrow
    0}\frac{x^2}{x^2} = \frac{1-m^2}{1+m^2}
\]

There is then no so such definition of $f(0, 0)$ such that $f(x ,y)$ is continuous
at the origin; in order for it to be continuous at the origin, the limits along
$y = mx$ must all coincide, but they clearly don't (take for example $m = 0,1$,
which see the limit being $1, 0$ respectively).

\subsection*{Apostol p.256 no.20}

\subsubsection*{a}

\begin{claim}
  If $f'(x; y) = 0$ for any $x \in B_\epsilon(a)$ and any vector $y$, then $f$
  is constant on $B(a)$.
\end{claim}

\begin{proof}
  The mean value theorem yields that for any $y \in B(a)$ and for one $0 < \theta
  < 1$,  $f'(a + \theta (y -a); y-a) = f(y) - f(a) = 0 \implies f(y) = f(a)$. However, since
  this holds for any $y$, $f$ is constant on $B(a)$.
\end{proof}

\subsubsection*{b}

It need not be constant: consider
\[
  F(x_1, x_2) = x_1^2
\]

Then, we have that
\[
  D_2F = 0
\]

everywhere, but obviously $F(x_1, x_2)$ is nonconstant.

We can conclude at the very least that $F$ is periodic, as the mean value
theorem yields that for $x, x + ty \in B(n)$ and some $0 < \theta < 1$, $f'(x + ty\theta; ty) = 0 =
f(x + ty) - f(x) \implies
f(x + ty) = f(x)$, where $t$ is a scalar; we know that $f(x; ty) = tf(x; y) =
0$. Then, $f$ is constant in the direction of $y$, or that $f$ takes the same
value on the line $x + ty$.

\subsection*{Apostol p.256 no.22}

\subsubsection*{a}

\begin{claim}
  There is no $f: \R^n \rightarrow \R$ such that $f(a; y) > 0$ for fixed $a$ and
  any nonzero $y$.
\end{claim}


\begin{proof}
  Suppose that $f'(a; y) > 0$. Now consider $f'(a; -y)$. Putting $t_2 = -t_1$ we have that
  \[
    f'(a; -y) = \lim_{t_1 \rightarrow 0}\frac{F(x - t_1y) - F(x)}{t_1} = \lim_{t_2 \rightarrow
      0}\frac{F(x  + t_2y) - F(x)}{-t_2} = -f'(a; y)
  \]

  Thus we have that $f'(a; -y) < 0$, but by assumption, we have that $f'(a; -y)
  > 0$. $\contra$.
\end{proof}

\subsubsection*{b}

Consider
\[
  F(x_1, x_2) = x_1 
\]

Then,
\[
  D_1F = 1
\]
and therefore $D_1F = F'(x, e_1) = 1 > 0$ for any $x \in \R^2$.

\subsection*{Apostol p.281 no.2}

\begin{align*}
  D_1f(0, 0) &= \lim_{t\rightarrow 0}\frac{f((0,0) + (t,0)) - f(0,0)}{t} \\
             &= \lim_{t\rightarrow 0}\frac{0}{y} = 0 \\
  D_2f(0, 0) &= \lim_{t\rightarrow 0}\frac{f((0,0) + (0,t)) - f(0,0)}{t} \\
             &= \lim_{t\rightarrow 0} \frac{-t}{t} = -1 \\
  D_2f(t_1, 0) &= \lim_{t_2\rightarrow 0} \frac{f(t_1, t_2) - f(t_1, 0)}{t_2} \\
             &= \lim_{t_2\rightarrow 0} \frac{t_2\frac{t_1^2-t_2^2}{t_1^2+t_2^2}}{t_2} \\
             &= 1 \\
  D_{12}f(0,0) &= \lim_{t_1\rightarrow 0} \frac{D_2f(t_1,0) - D_2f(0,0)}{t_1} \\
             &= \lim_{t_1\rightarrow 0} \frac{D_2f(t_1,0) + 1}{t_1} \\
             &= \lim_{t_1\rightarrow 0} \frac{1 + 1}{t_1} \text{ which does not exist.} \\
  D_1f(0, t_2) &= \lim_{t_1\rightarrow 0} \frac{f(t_1, t_2) - f(0, t_2)}{t_1} \\
             &= \lim_{t_1\rightarrow 0} \frac{t_2\frac{t_1^2-t_2^2}{t_1^2+t_2^2} - t_2\frac{-t_2^2}{t_2^2}}{t_1} \\
             &= \lim_{t_1\rightarrow 0}t_2\frac{\frac{t_1^2-t_2^2}{t_1^2+t_2^2} + 1}{t_1} \\
             &= \lim_{t_1\rightarrow 0}t_2\frac{2t_1^2}{t_1} \\
             &= 0 \\
  D_{21}f(0,0) &= \lim_{t_2\rightarrow 0} \frac{D_1f(0,t_2) - D_1f(0,0)}{t_2} \\
             &= \lim_{t_2\rightarrow 0} \frac{D_1f(0,t_2)}{t_2} \\
             &= \lim_{t_2\rightarrow 0} \frac{0}{t_2} \\
             &= 0
\end{align*}


\subsection*{Apostol p.281 no.3}

\subsubsection*{a}

\begin{align*}
  f'(0; (x, y)) &= \lim_{t \rightarrow 0}\frac{F(tx, ty) - F(0,0)}{t} \\
                &= \lim_{t \rightarrow 0}\frac{F(tx, ty)}{t} \\
                &= \lim_{t \rightarrow 0}\frac{xy^3}{x^3+t^3y^6} \\
                &= \frac{y^3}{x^2}
\end{align*}

However, if $x = 0$, then we have that
\begin{align*}
  f'(0; (0, y)) &= \lim_{t \rightarrow 0}\frac{F(0, ty)}{t} \\
                &= \lim_{t \rightarrow 0}\frac{0}{t} = 0
\end{align*}

\subsubsection*{b}

Consider the limit along $x = -y^2$. Then,
\[
  \lim_{y \rightarrow 0}\frac{-y^5}{-y^6+y^6} = \lim_{y \rightarrow 0}\frac{-y^5}{0} 
\]

This limit doesn't exist, and this $f(x,y)$ is not continuous at the origin.
  
\section*{Problem 1}

\begin{claim}
  The product $S = (a_1, b_1) \times \dots \times (a_n, b_n)$ is an open set in
  $\R^n$.
\end{claim}

\begin{proof}
  For any element $x = (x_1, x_2, \dots, x_n) \in S$, take
  \[
    \epsilon = \min(x_1 - a_1, b_1 - x_1, x_2 - a_2, b_2 - x_2, \dots, x_n -
    a_n, b_n - x_n)
  \]
  That is, $\epsilon = \min(x_i - a_i, b_i - x_i)$ for $1 \leq i \leq n$.

  Now, for any $y = (y_1, \dots, y_n) \in B_\epsilon(x)$, we have that
  \[
    ||y - x|| < \epsilon \implies \sum_{i=1}^n(y_i-x_i)^2 < \epsilon^2 \implies
    (y_i - x_i)^2 < \epsilon^2 \implies |y_i-x_i| < \epsilon 
  \]

  Then, if $y_i - x_i > 0$, then
  \[
    y_i < x_i + \epsilon \leq x_i + (b_i - x_i) = b_i \\
  \]
  and if $y_i - x_i< 0$, then
  \[
    y_i > x_i - \epsilon \geq x_i - (x_i - a_i) = a_i \\
  \]

  Thus, we have that for every $1 \leq i \leq n$, $y_i \in (a_i, b_i) \implies y
  \in S \implies B_\epsilon(x) \in S$.
\end{proof}

\section*{Problem 2}

\subsection*{a}

\begin{claim}
  Homogeneous functions of degree 1 have that $F(0) = 0$.
\end{claim}

\begin{proof}
  Since it is continuous, we have that
  \[
    \lim_{x\rightarrow 0}F(x) = F(0)
  \]

  Now, suppose that $F(0) = c \neq 0$. Then, since $\lim_{x\rightarrow 0}F(x) =
  c$, we have that $|F(x) - c| < \frac{|c|}{2}$ on some $B_{\delta}(0)$. Now,
  consider the quantity $\frac{c}{2F(e_1)}$. We have by the archimedian property
  some $n \geq 1 \mid n\delta > |\frac{c}{2F(e_1)}| \implies \delta >
  |\frac{c}{2nF(e_1)}| \implies \frac{c}{2nF(e_1)}e_1 \in B_\delta(0)$, as
  $||\frac{c}{2nF(e_1)}e_1|| = |\frac{c}{2nF(e_1)}| < \delta$.

  However, $F(\frac{c}{2nF(e_1)}e_1) = \frac{c}{2nF(e_1)}F(e_1) = \frac{c}{2n}$.
  Then,
  \[
    |F(\frac{c}{2nF(e_1)}e_1) - c| = |\frac{c}{2n} - c| = |c(1 - \frac{1}{2n})|
  \]

  Further, since $n \geq 1$, $1 - \frac{1}{2n} > 0 \implies |c(1 -
  \frac{1}{2n})| = |c|(1 - \frac{1}{2n})$.

  By assumption, we have that $|F(\frac{c}{2nF(e_1)}e_1) - c| = |c|(1 -
  \frac{1}{2n}) < |\frac{c}{2}| \implies 1 - \frac{1}{2n} < \frac{1}{2} \implies
  1 < \frac{1}{2}(1 + \frac{1}{n}) \leq \frac{1}{2}(1 + 1) = 1$. Thus, we have
  that $1 < 1$. $\contra$, so $F(0) = 0$.
\end{proof}

The above ought to work, but there is a much simpler argument that I realized
only after doing it: by continuity, we
have that

\begin{align*}
  F(0) &= \lim_{x \rightarrow 0}F(x) \\
  F(0) &= \lim_{x \rightarrow 0}F(2x) \\
       &= 2\lim_{x \rightarrow 0}F(x)
\end{align*}

Thus, $F(0) = 2F(0) \implies F(0) = 0$.


\subsection*{b}

\begin{align*}
  F'(0; y) &= \lim_{t\rightarrow 0}\frac{F(0 + ty) - F(0)}{t} \\
           &= \lim_{t\rightarrow 0}\frac{F(ty)}{t} \\
           &= \lim_{t\rightarrow 0}\frac{tF(y)}{t} \\
           &= F(y)
\end{align*}

\section*{Problem 3}

\begin{claim}
  If $F: \R^n \rightarrow \R$ satisfies $|F(x)| \leq c||x||^2$ for some $c \in
  \R$ and all $x \in \R^n$, then for any $y \in \R^n$,
  \[
    F'(0; y) = 0
  \]
\end{claim}

\begin{proof}
  Note that by taking $x = 0$, we have that $|F(0)| \leq c||0||^2 = 0 \implies
  |F(0)| =  0 \implies F(0) = 0$.
  \[
    F'(0; y) = \lim_{t\rightarrow 0}\frac{F(0 + yt) - F(0)}{t} =
    \lim_{t\rightarrow 0}\frac{F(yt)}{t}
  \]

  Further, for any $\epsilon > 0$ take $\delta = \frac{\epsilon}{|c|||y||^2}$.
  Then, for $0 < |t| < \delta$
  \[
    \left|\frac{F(ty)}{t}\right| \leq \frac{|c|||yt||^2}{|t|} = \frac{|c||t|^2||y||^2}{|t|}
    = |c||t|||y||^2 < \epsilon
  \]

  Thus, we have that $F'(0; y) = 0$.

\end{proof}

\end{document}

% LocalWords:  NetID fancyplain LocalWords colorlinks linkcolor linkbordercolor
% LocalWords:  Apostol
 