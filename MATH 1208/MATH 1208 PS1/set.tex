\documentclass[12pt,letterpaper]{article}
\usepackage{fullpage}
\usepackage[top=2cm, bottom=4.5cm, left=2.5cm, right=2.5cm]{geometry}
\usepackage{amsmath,amsthm,amsfonts,amssymb,amscd}
\usepackage{lastpage}
\usepackage{enumerate}
\usepackage{fancyhdr}
\usepackage{mathrsfs}
\usepackage{xcolor}
\usepackage{graphicx}
\usepackage{listings}
\usepackage{hyperref}
\usepackage{tikz}
\usepackage{relsize}
\usepackage{fancyvrb}
\usetikzlibrary{shapes.geometric,fit}

\hypersetup{%
  colorlinks=true,
  linkcolor=blue,
  linkbordercolor={0 0 1}
}

\setlength{\parindent}{0.0in}
\setlength{\parskip}{0.05in}

\newcommand\course{MATH 1208}
\newcommand\hwnumber{1}
\newcommand\NetIDa{dc3451}
\newcommand\NetIDb{David Chen}

\theoremstyle{definition}
\newtheorem*{statement}{Statement}
\newtheorem*{claim}{Claim}
\newtheorem*{theorem}{Theorem}
\newtheorem*{lemma}{Lemma}

\newcommand{\contra}{\Rightarrow\!\Leftarrow}
\newcommand{\R}{\mathbb{R}}
\newcommand{\F}{\mathbb{F}}
\newcommand{\Z}{\mathbb{Z}}
\newcommand{\Zeq}{\mathbb{Z}_{\geq 0}}
\newcommand{\Zg}{\mathbb{Z}_{>0}}
\newcommand{\Req}{\mathbb{R}_{\geq 0}}
\newcommand{\Rg}{\mathbb{R}_{>0}}
\newcommand{\N}{\mathbb{N}}
\newcommand{\Q}{\mathbb{Q}}
\newcommand{\C}{\mathbb{C}}

\pagestyle{fancyplain}
\headheight 35pt
\lhead{\NetIDa}
\lhead{\NetIDa\\\NetIDb}
\chead{\textbf{\Large Homework \hwnumber}}
\rhead{\course \\ \today}
\lfoot{}
\cfoot{}
\rfoot{\small\thepage}
\headsep 1.5em

\begin{document}

\subsection*{Apostol p.451 no.6}

\subsubsection*{a}

The components of $D$ are $(x + z, x + y + z, y + z)$

\subsubsection*{b}

If $D = 0$, then $x + z = 0$, and $ x + y + z = 0$. Then, $y = 0$, and since $y
+ z = 0$, we have that $z = 0$ and finally $x = 0$ as $x + z = 0$. 

\subsubsection*{c}

Take $x = -1, y = 1, z = 2$.

\subsection*{Apostol p.451 no.7}

\subsubsection*{a}

The components of $D$ are $(x + 2z, x + y + z, x + y + z)$.

\subsubsection*{b}

Take $x = -2, y = 1, z = 1$.

\subsubsection*{c}

$D = (1, 2, 3) \implies x + y + z = 2$ and $x + y + z = 3$. $\contra$, so no
such picks of $x, B, z$ have $D = (1, 2, 3)$.

\subsection*{Apostol p.456 no.4}

\begin{claim}
  $\forall B \in \R^N, A \cdot B = 0 \implies A = 0$. 
\end{claim}

\begin{proof}
  Let the $i^{th}$ component of $A$ be $a_i$.
  Taking $B = A$, we have that $A \cdot A = 0 \implies \sum_{i=1}^na_i^2 = 0
  \implies \forall i, a_i = 0 \implies A = 0$.
\end{proof}

\subsection*{Apostol p.456 no.19}

\begin{claim}
  $||A + B||^2 - ||A - B||^2 = 4A \cdot B$
\end{claim}

\begin{proof}
  Let the $i^{th}$ component of $A$ be $a_i$, and the corresponding component of
  $B$ be $b_i$.
  
  \begin{align*}
    ||A + B||^2 - ||A - B||^2 &= \sum_{i=1}^n(a_i+b_i)^2 - \sum_{i=1}^n(a_i - b_i)^2 \\
                              &= \sum_{i=1}^n(a_i + b_i) ^2 - (a_i - b_i)^2 \\
                              &= \sum_{i=1}^n a_i^2 + 2a_ib_i + b_i^2 - a_i^2 +2a_ib_i - b_i^2 \\
                              &= \sum_{i=1}^n 4a_ib_i \\
                              &= 4\sum_{i=1}^n a_ib_i \\
                              &= 4A \cdot B
  \end{align*}
\end{proof}

\subsection*{Apostol p.555 no.9}

Odd functions are indeed a vector space; since they are a subset of
$\mathcal{F}(\R, \R)$, we have that we only need to show closure under scalar
multiplication and vector addition.

$f: \R \rightarrow \R$ is odd $\iff$ $f(x) = -f(-x)$, and so $2f(x) = -2f(-x)$,
and so it is closed under vector multiplication. Similarly, we have that if $g,
f: \R \rightarrow \R$ are odd, then $(f + g)(x) = f(x) + g(x) = -f(-x) - g(-x) =
-(f + g)(-x)$, and it is closed under vector addition. 

\subsection*{Apostol p.555 no.10}

This is also a vector space; we prove the same thing as above; for any function $f$ bounded
by $M$, and scalar multiple $cf$ is bounded by $cM$. Similarly, two functions
bounded by $M_1, M_2$ has their sum bounded by $M_1 + M_2$.

\subsection*{Apostol p.555 no.11}

This is not a vector space, being not closed under scalar multiplication; any
function $f$ that is increasing has additive inverse $-f$ deceasing instead.

\subsection*{Apostol p.555 no.21}

We know that the space of all series is a vector space over $\R$.

This set in particular is also a vector space. It is closed as we have that if $\{a_n\}$ converges to
$L$, then $\{ca_n\}$ converges to $\{cL\}$, and if $\{a_n\}, \{b_n\}$ converge
to $K, L$, then $\{a_n + b_n\}$ converges to $K + L$.

\subsection*{Apostol p.556 no.29}

We denote $\cdot$ and exponentiation in the usual way, and $xy$ will denote the
product of $x, y$ in $V$.

Commutativity:

\[
  x + y = x \cdot y = y \cdot x = y + x
\]

Associativity:

\[
  (x + y) + z = (x \cdot y) \cdot z = x \cdot (y \cdot z) = x + (y + z)
\]

\[
  (cd)x = x^{cd} = (x^d)^c = c(dx)
\]

Distributivity:

\[
  c(x + y) = (x \cdot y)^c = (x^c)\cdot(y^c) = cx + cy
\]

\[
  (c + d)x = x^{c + d} = x^c\cdot x^d = cx + cd
\]

Identity:

\[
  x + 0 = x \cdot 1 = x
\]

\[
  1x = x^1 = x
\]

Inverse:

\[
  cx + c^{-1}x = x^c\cdot x^{-c} = 0
\]

Closure:

\[
  x, y \in \Rg \implies x + y = (x \cdot y) > 0 \implies x + y \in \Rg
\]

\[
  x \in \Rg \implies cx = (x^c) > 0 \implies xc \in \Rg
\]

\section*{Problem 1}

\begin{claim}
  For vector spaces $U, V$ over $F$, a function $f: U \rightarrow V$ is linear
  iff
  \[
    f(\sum_{i=1}^nc_iX_i) = \sum_{i=1}^nc_if(X_i).
  \]
\end{claim}

\begin{proof}
  $(\implies)$ Proceed with induction on $n$. The base case $n = 1$ follows immediately
  from the linearity of $f$: $f(c_iX_i) = c_if(x_i)$.

  Suppose that the above holds for $n = k$. Then,
  \[
    f(\sum_{i=1}^{k+1}c_iX_i) =
    f(\sum_{i=1}^{k}c_iX_i) + f(c_{k+1}X_{k+1})= \sum_{i=1}^kc_if(X_i) +
    c_{k+1}f(X_{k+1}) = \sum_{i=1}^{k+1}c_if(X_i).
  \]

  The identity holds for all $n \in \Zg$.

  $(\impliedby)$ Take $n = 1$. Then, $f(c_1X_1) = c_1f(X_1)$. When $n = 2, c_1 =
  c_2 = 1$, $f(c_1X_1 + c_2X_2) = f(X_1 + X_2) = f(X_1) + f(X_2)$, and so $f$ is
  linear.
\end{proof}

\section*{Problem 2}

\begin{claim}
  For vector spaces $U, V$ over $F$, a linear map $f: U \rightarrow V$ is
  injective iff $\ker f = \{0\}$.
\end{claim}

\begin{proof}
  First we will show that $0 \in \ker f$ in every case of $f$ is linear, as
  $f(0) = f(0(0)) = 0f(0) = 0$.
  
  $(\implies)$ Since $f$ is injective, we have that only one member of the
  domain is mapped to $0$. Since $f(0) = 0$ always, $\ker f = \{0\}$.

  $(\impliedby)$ Suppose that $f$ is not injective. Then, we must have that
  $\exists x,y \mid f(x) = f(y), x \neq y$. Then, $f(x) - f(y) = f(x - y) = 0
  \implies x - y \in \ker f$, which means that $\ker f \neq \{0\}$, and $x - y
  \neq 0$. 
\end{proof}

\section*{Problem 4}

\begin{claim}
  If $f: U \rightarrow V$ and $g: V \rightarrow W$ are linear maps, then $g
  \circ f$ is also linear.
\end{claim}

\begin{proof}
  \begin{align*}
    (g \circ f)(cu) &= g(f(cu)) \\
                     &= g(cf(u)) \\
                     &= cg(f(u)) \\
                     &= c(g \circ f)(u) \\
    (g \circ f)(u_1 + u_2) &= g(f(u_1 + u_2) \\
                    &= g(f(u_1) + f(u_2)) \\
                    &= g(f(u_1)) + g(f(u_2)) \\
                    &= (g \circ f)(u_1) + (g \circ f)(u_2)
  \end{align*}
\end{proof}

\section*{Problem 5}

\subsection*{a}

\begin{claim}
  With $g: V \rightarrow W$ is a fixed linear map, then $L_g: \mathcal{L}(U, V)
  \rightarrow \mathcal{L}(V, W)$ is linear where $L_g(f) = g \circ f$.
\end{claim}

\begin{proof}
  \begin{align*}
    (L_g(cf))(u) &= (g \circ (cf))u) \\
                 &= g(cf(u)) \\
                 &= cg(f(u)) \\
                 &= (c(g \circ f))(u) \\
                 &= (cL_g(f))(u) \\
    (L_g(f_1 + f_2))(u) &= (g \circ (f_1 + f_2))(u) \\
                 &= g((f_1 + f_2)(u)) \\
                 &= g(f_1(u) + f_2(u)) \\
                 &= g(f_1(u)) + g(f_2(u)) \\
                 &= (L_g(f_1))(u) + (L_g(f_2))(u) \\
  \end{align*}

  We then have that $L_g(cf) = cL_g(f)$ and $L_g(f_1 + f_2) = L_g(f_1) + L_g(f_2)$.
\end{proof}

\subsection*{b}

\begin{claim}
  With $f: U \rightarrow V$ is a fixed linear map, then $R_g: \mathcal{L}(V, W)
  \rightarrow \mathcal{L}(U, W)$ is linear where $R_f(g) = g \circ f$.
\end{claim}

\begin{proof}
  \begin{align*}
    (R_f(cg))(u) &= ((cg) \circ f)u) \\
                 &= cg(f(u)) \\
                 &= (c(g \circ f))(u) \\
                 &= (cR_f(f))(u) \\
    (R_f(g_1 + g_2))(u) &= ((g_1 + g_2) \circ f)(u) \\
                 &= ((g_1 + g_2)(f(u)) \\
                 &= g_1(f(u)) + g_2(f(u)) \\
                 &= (R_f(g_1))(u) + (R_f(g_2))(u) \\
  \end{align*}

  We then have that $R_f(cg) = cR_f(g)$ and $R_f(g_1 + g_2) = R_f(g_1) + R_f(g_2)$.
\end{proof}

\end{document}

% LocalWords:  NetID fancyplain LocalWords colorlinks linkcolor linkbordercolor
