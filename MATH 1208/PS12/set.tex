\documentclass[12pt,letterpaper]{article}
\usepackage{fullpage}
\usepackage[top=2cm, bottom=4.5cm, left=2.5cm, right=2.5cm]{geometry}
\usepackage{amsmath,amsthm,amsfonts,amssymb,amscd}
\usepackage{lastpage}
\usepackage{enumerate}
\usepackage{fancyhdr}
\usepackage{mathrsfs}
\usepackage{xcolor}
\usepackage{graphicx}
\usepackage{listings}
\usepackage{hyperref}
\usepackage{tikz}
\usepackage{tikz-3dplot}
\usepackage{relsize}
\usepackage{fancyvrb}
\usetikzlibrary{shapes.geometric,fit}

\hypersetup{%
  colorlinks=true,
  linkcolor=blue,
  linkbordercolor={0 0 1}
}

\setlength{\parindent}{0.0in}
\setlength{\parskip}{0.05in}

\newcommand\course{MATH 1208}
\newcommand\hwnumber{12}
\newcommand\NetIDa{dc3451}
\newcommand\NetIDb{David Chen}

\theoremstyle{definition}
\newtheorem*{statement}{Statement}
\newtheorem*{claim}{Claim}
\newtheorem*{theorem}{Theorem}
\newtheorem*{lemma}{Lemma}

\newcommand{\contra}{\Rightarrow\!\Leftarrow}
\newcommand{\R}{\mathbb{R}}
\newcommand{\F}{\mathbb{F}}
\newcommand{\Z}{\mathbb{Z}}
\newcommand{\Zeq}{\mathbb{Z}_{\geq 0}}
\newcommand{\Zg}{\mathbb{Z}_{>0}}
\newcommand{\Req}{\mathbb{R}_{\geq 0}}
\newcommand{\Rg}{\mathbb{R}_{>0}}
\newcommand{\N}{\mathbb{N}}
\newcommand{\Q}{\mathbb{Q}}
\newcommand{\C}{\mathbb{C}}
\newcommand{\id}{\mathrm{Id}}
\newcommand{\im}{\mathrm{im}}
\newcommand{\tr}{\mathrm{tr}}
\newcommand{\diag}{\mathrm{diag}}
\newcommand{\rank}{\mathrm{rank}}
\newcommand{\spn}{\mathrm{span}}

\pagestyle{fancyplain}
\headheight 35pt
\lhead{\NetIDa}
\lhead{\NetIDa\\\NetIDb}
\chead{\textbf{\Large Homework \hwnumber}}
\rhead{\course \\ \today}
\lfoot{}
\cfoot{}
\rfoot{\small\thepage}
\headsep 1.5em

\begin{document}

\subsection*{Apostol p.385 no.1a}

Let $R = C \cup \text{int } C$.

\begin{align*}
  \oint_C y^2dx + xdy &= \int_R 1 - 2y \\
                                  &= \int_0^2\int_0^2(1-2y)dydx \\
                                  &= \int_0^2\left[ y-y^2 \right]\Big|_{y=0}^{y=2}dx \\
                                  &= \int_0^2 -2dx \\
                                  &= -4
\end{align*}

\subsection*{Apostol p.385 no.1d}

Let $R = C \cup \text{int } C$.

\begin{align*}
  \oint_C y^2dx + xdy &= \int_R 1 - 2y \\
                                  &= \int_{-2}^2\int_{-\sqrt{4-x^2}}^{\sqrt{4-x^2}}(1-2y)dydx \\
                                  &= \int_{-2}^2\left[ y-y^2 \right]\Big|_{y=-\sqrt{4-x^2}}^{y=\sqrt{4-x^2}}dx \\
                                  &= \int_{-2}^2 (2\sqrt{4-x^2}dx \\
                                  &= 4\pi
\end{align*}

Note that the last equality follows from the fact that
$\int_{-2}^2\sqrt{4-x^2}dx$ is the area of the semicericle with center at the
origin and radius two, and is thus $2\pi$.

\subsection*{Apostol p.386 no.5}

\begin{claim}
  If $f,g$ are continuously differentiable in an open connected set $S \subset \R^2$, then
  \[
    \oint_C f \nabla g \cdot d\alpha = -\oint_C g\nabla f\cdot d\alpha
  \]
  for every piecewise smooth Jordan curve $C$ in $S$.
\end{claim}

\begin{proof}
  Let $R = C \cup \text{int } C$.

  \begin{align*}
    \oint_Cf \nabla g \cdot d\alpha + \oint_Cg\nabla f\cdot d\alpha &= \oint_C \left(  f\frac{\partial g}{\partial x} + g\frac{\partial f}{\partial x}, f\frac{\partial g}{\partial y} + g\frac{\partial f}{\partial y}\right)\cdot d\alpha \\
                                                                    &= \oint_C \left( \frac{\partial}{\partial x} fg, \frac{\partial }{\partial y}fg \right) \\
                                                                    &= \int_R (\frac{\partial^2 fg}{\partial x \partial y}) - \frac{\partial^2 fg}{\partial y \partial x}dydx \\
                                                                    &= \int_R 0 dydx \\
                                                                    &= 0
  \end{align*}
  The last equalities follow from equality of mixed partials.
\end{proof}

\subsection*{Apostol p.386 no.8a}

\begin{align*}
  \oint_C \frac{\partial g}{\partial n} &= \oint_C \nabla g \cdot n ds\\
                                                 &= \oint_C \frac{1}{||(X'(t), Y'(t))||}\left( \frac{\partial g}{\partial x}(X(t),Y(t))Y'(t) - \frac{\partial g}{\partial y}(X(t),Y(t))X'(t) \right)ds \\
                                        &= \oint_C \left( -\frac{\partial g}{\partial y}(X(t),Y(t),  \frac{\partial g}{\partial x}(X(t),Y(t))) \cdot (X'(t), Y'(t))\right)dt \\
                                                 &= \oint_C (-\frac{\partial g}{\partial y}, \frac{\partial g}{\partial x})\cdot d\gamma \\
                                                 &= \int_R \left(\frac{\partial^2g}{\partial y^2} + \frac{\partial^2g}{\partial x^2}\right)dydx\\
                                                 &= \int_R \nabla^2gdxdy
\end{align*}

\subsection*{Apostol p.386 no.8b}

\begin{align*}
  \oint_C f\frac{\partial g}{\partial n} &= \oint_C f(\nabla g \cdot n)ds \\
                                         &= \oint_C \left( -f\frac{\partial g}{\partial y}, f\frac{\partial g}{\partial x} \right)\cdot d\gamma \\
                                         &= \int_R \left( \frac{\partial f}{\partial x}\frac{\partial g}{\partial x} + f\frac{\partial^2 g}{\partial x} + \frac{\partial f}{\partial x}\frac{\partial g}{\partial x} + f\frac{\partial^2 g}{\partial x} \right)dydx \\
                                         &= \int_R \left( f\nabla^2 g + \nabla f \cdot \nabla g \right) dydx
\end{align*}

\subsection*{Apostol p.400 no.9}

\begin{align*}
  \int_0^a\left(\int_0^{\sqrt{a^2-y^2}} (x^2+y^2)dx\right)dy &= \int_{0}^{\frac{\pi}{2}}\int_0^ar^3drd\theta \\
                                                              &= \int_{0}^{\frac{\pi}{2}}\frac{a^4}{4}d\theta \\
                                                              &= \frac{\pi}{8}a^4
\end{align*}

\subsection*{Apostol p.400 no.16a}

\begin{align*}
  I^2(r) &= \left(\int_{-r}^re^{-x^2}dx\right)\left( \int_{-r}^re^{-y^2}dy \right) \\
         &= \int_{-r}^r\left( \int_{-r}^re^{-x^2}dx\right)e^{-y^2}dy \\
         &= \int_{-r}^r\left( \int_{-r}^r e^{-x^2}e^{-y^2}dx \right)dy \\
         &= \int_R e^{-(x^2+y^2)}dxdy
\end{align*}

\subsection*{Apostol p.400 no.16b}

Consider the extensions $C_2 \rightarrow \R$
\begin{gather*}
  \alpha(x,y) =
  \begin{cases}
    e^{-(x^2+y^2)} & (x,y) \in R \\
    0 & \text{otherwise}
  \end{cases} \\
  \beta(x,y) =
  \begin{cases}
    e^{-(x^2+y^2)} & (x,y) \in C_1 \\
    0 & \text{otherwise}
  \end{cases} \\
  \gamma(x,y) = e^{-(x^2+y^2)}
\end{gather*}

Then we have that the claim is reduced to
\[
  \int_{C_2}\alpha < \int_{C_2} \beta < \int_{C_2} \gamma
\]

However, note that on $R \setminus C_1$, $\beta - \alpha = e^{-(x^2+y^2)} > 0$
and $\beta - \alpha = 0$ everywhere else, and thus $\int_{C_2}\beta - \alpha >
0$.

Similarly, on $C_2 \setminus R$, $\gamma - \beta = e^{-(x^2+y^2)} > 0$
and $\gamma - \beta = 0$ everywhere else, and thus $\int_{C_2}\gamma - \beta >
0$.


\subsection*{Apostol p.400 no.16c}

$C_1, C_2$ are respectively the circles of radius $r$ and $r\sqrt{2}$ centered
on the origin, respectively. Then,

\begin{align*}
  \int_{C_1}e^{-(x^2+y^2)}dxdy &= \int_0^{2\pi}\int_0^re^{-r^2}rdrd\theta \\
                               &= \int_0^{2\pi} \frac{1}{2}(1 -e^{-r^2}) d\theta \\
                               &= \pi(1-e^{-r^2})
\end{align*}

\begin{align*}
  \int_{C_1}e^{-(x^2+y^2)}dxdy &= \int_0^{2\pi}\int_0^{r\sqrt{2}}e^{-r^2}rdrd\theta \\
                               &= \int_0^{2\pi} \frac{1}{2}(1 -e^{-2r^2}) d\theta \\
                               &= \pi(1-e^{-2r^2})
\end{align*}

Then, we have that as $r\rightarrow \infty$,
\[
  \lim_{r\rightarrow \infty}\pi(1-e^{-r^2}) \leq \lim_{r\rightarrow \infty}I^2(r) \leq
  \lim_{r\rightarrow \infty}\pi(1-e^{-2r^2}) \implies \pi \leq
  \lim_{r\rightarrow \infty}I^2(r) \leq \pi
\]

Since we know that $I(r) > 0, as e^{-(x^2+y^2)} > 0$, we have that $I(r)
\rightarrow \sqrt{\pi}$ as $r \rightarrow \infty$.

\subsection*{Apostol p.437 no.5}

\begin{align*}
  \frac{\partial r}{\partial x} \times \frac{\partial r}{\partial y} &= \det
                                                                       \begin{bmatrix}
                                                                         i & j & k \\
                                                                         1 & 0 & \frac{\partial f}{\partial x} \\
                                                                         0 & 1 & \frac{\partial f}{\partial y} \\
                                                                       \end{bmatrix} \\
                                                                     &= -\frac{\partial f}{\partial x}j - \frac{\partial f}{\partial y}j + k \\
  \int_S F \cdot n\ dS &= \int_TF(r(x,y)) \cdot \left( \frac{\partial r}{\partial x} \times \frac{\partial r}{\partial y}  \right)dxdy \\
                       &= \int_T(-P\frac{\partial f}{\partial x} - Q\frac{\partial f}{\partial y} + R) dxdy
\end{align*}


\subsection*{Apostol p.437 no.7}

We are computing the integral $\int_SF \cdot dr^2$ where $F(x,y,z) = (xy, yz, x^2)$.

Consider that the sphere must be split into two halves in the parameterization,
namely $S_1 = (x,y,\sqrt{a^2-x^2-y^2})$ which is the positive hemisphere and
$S_2 = (x,y, -\sqrt{a^2-x^2-y^2})$ which is the negative hemisphere. The
integrals over these two hemispheres will be equal and opposite.

Consider some point $(x,y,z) \in S_1$ and some other point $(-x,-y,-z) \in S_2$.
Then, $F(x,y,z) = F(-x,-y,-z)$, but the normal vector must point in the opposite
direction, which inverts the sign of $F \cdot n$. Then,
\[
  \int_SF \cdot dr^2 = \int_{S_1}F \cdot dr^2 + \int_{S_2}F \cdot dr^2 = 0
\]

\section*{Problem 2}

Consider the parameterization of $z = \sqrt{f(x)^2 - y^2}$. Then, we have from
Apostol that the surface area is

\begin{align*}
  \int_a^b\int_{-f(x)}^{f(x)}\sqrt{1 + \left(\frac{\partial f}{\partial
  x}\right)^2 + \left(\frac{\partial f}{\partial
  y}\right)^2 }dydx &= \int_a^b\int_{-f(x)}^{f(x)} \sqrt{1 +
                      \frac{(f'(x)f(x))^2+y^2}{f(x)^2-y^2}}dydx \\
                    &= \int_a^b\int_{-f(x)}^{f(x)}f(x)\sqrt{\frac{1 + f'(x)^2}{f(x)^2-y^2}}dydx \\
                    &= \int_a^bf(x)\sqrt{1+f'(x)}\left(\int_{-f(x)}^{f(x)}(f(x)^2-y^2)^{-\frac{1}{2}}dy\right)dx \\
                    &= \int_a^bf(x)\sqrt{1+f'(x)}\left(\arcsin(\frac{y}{f(x)})\Big|_{y=-f(x)}^{y=f(x)}\right)dx \\
                    &= \int_a^bf(x)\sqrt{1+f'(x)}\left(\pi\right)dx \\
                    &= \pi\int_a^bf(x)\sqrt{1+f'(x)}dx
\end{align*}

The $\arcsin$ integral is in Apostol, as is the expression for surface area.

Since we have that this is only the surface area of the top half of the
revolution, we have that the total area is

\[
  2\pi \int_a^bf(x)\sqrt{1 + f'(x)}dx
\]

\section*{Problem 3}

\begin{claim}
  Let $r: T \rightarrow S \subset \R^3$ be a parametric surface and $\gamma:
  [a,b] \rightarrow T$ a path in the plane. The tangent vector to $r \circ
  \gamma$ at each $t$ is orthogonal to the outward normal vector of the surface.
\end{claim}

\begin{proof}
  The tangent vector to $r \circ \gamma$ at $t$ can be given by $(r \circ
  \gamma)'(t)$ (from Apostol). Suppose that $r = r(u,v)$, and $\gamma =
  \gamma(t)$. Then, computing, $(r \circ \gamma)'$ is

  \[
    \begin{bmatrix}
      D_1r_1 & D_2r_1 \\
      D_1r_2 & D_2r_2 \\
      D_1r_3 & D_2r_3 \\
    \end{bmatrix}
    \begin{bmatrix}
      D_1\gamma_1 \\
      D_1\gamma_2 \\
    \end{bmatrix}
    =
    \begin{bmatrix}
      D_1r_1D_1\gamma_1 + D_2r_1D_2\gamma_2 \\
      D_1r_2D_1\gamma_1 + D_2r_2D_2\gamma_2 \\
      D_1r_3D_1\gamma_1 + D_2r_3D_2\gamma_2 \\
    \end{bmatrix}
    =
    \frac{\partial r}{\partial u}\frac{\partial \gamma_1}{\partial t} + 
    \frac{\partial r}{\partial v}\frac{\partial \gamma_2}{\partial t}
  \]
  where
  \[
    \frac{\partial r}{\partial u} =
    \begin{bmatrix}
      D_1r_1 \\
      D_1r_2 \\
      D_1r_3
    \end{bmatrix},
    \frac{\partial r}{\partial v} =
    \begin{bmatrix}
      D_2r_1 \\
      D_2r_2 \\
      D_2r_3
    \end{bmatrix}
  \]
  Then, evaluating at $t$, we see that $(r \circ \gamma)'(t)$ is a linear
  combination of $\frac{\partial r}{\partial u}$ and $\frac{\partial r}{\partial
    v}$, as we have that $\frac{\partial \gamma_i}{\partial t}$ evaluated at $t$
  is in $\R$. Then, since the outward normal vector is defined to be
  $\frac{\partial r}{\partial u} \times \frac{\partial r}{\partial v}$, and that
  the linear combination of orthogonal vectors is still orthogonal, $(r \circ
  \gamma)'(t)$ is still orthogonal.
\end{proof}

\end{document}

% LocalWords:  NetID fancyplain LocalWords colorlinks linkcolor linkbordercolor
% LocalWords:  Apostol
  