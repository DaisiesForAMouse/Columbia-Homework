\documentclass[12pt,letterpaper]{article}
\usepackage{fullpage}
\usepackage[top=2cm, bottom=4.5cm, left=2.5cm, right=2.5cm]{geometry}
\usepackage{amsmath,amsthm,amsfonts,amssymb,amscd}
\usepackage{lastpage}
\usepackage{enumerate}
\usepackage{fancyhdr}
\usepackage{mathrsfs}
\usepackage{xcolor}
\usepackage{graphicx}
\usepackage{listings}
\usepackage{hyperref}
\usepackage{tikz}
\usepackage{xfrac}
\usepackage{nicefrac}
\usepackage{xcolor}

\usetikzlibrary{shapes.geometric,fit}

\hypersetup{
    colorlinks=true,
    linkcolor=blue,
    linkbordercolor={0 0 1}
}

\setlength{\parindent}{0.0in}
\setlength{\parskip}{0.05in}

\newcommand\course{ECON 3211}
\newcommand\hwnumber{5}
\newcommand\NetIDa{dc3451}
\newcommand\NetIDb{David Chen}          

\theoremstyle{definition}
\newtheorem*{statement}{Statement}
\newtheorem*{claim}{Claim}
\newtheorem*{theorem}{Theorem}

\newcommand{\contra}{\Rightarrow\!\Leftarrow}
\newcommand{\Lag}{\mathcal{L}}

\pagestyle{fancyplain}
\headheight 35pt
\lhead{\NetIDa}
\lhead{\NetIDa\\\NetIDb}
\chead{\textbf{\Large Problem Set \hwnumber}}
\rhead{\course \\ \today}
\lfoot{}
\cfoot{}
\rfoot{\small\thepage}
\headsep 1.5em

\begin{document}

\section*{Problem 2}

\subsection*{a)}

Increasing RTS (consider that $\gamma q(L,K) > q(\gamma L, \gamma K)$)

\subsection*{b)}

\[
  \frac{dq}{dL} = MP_L = 0.5L^{-0.5}K, \frac{dq}{dK} = L^{0.5}
\]

\subsubsection*{b.1}

\[
  \min_{[L,K]} 4000L + 8000K \text{ s.t. } q = L^{0.5}K = 1000
\]

\subsubsection*{b.2,b.3,b.4}

\begin{alignat*}{2}
  && \Lag &= 4000L + 8000K - \lambda[L^{0.5}K - 1000] \\
  && \frac{\delta \Lag}{\delta L} &= 4000 - 0.5\lambda L^{-0.5}K = 0 \\
  && \frac{\delta \Lag}{\delta K} &= 8000 - \lambda L^{0.5} =0 \\
  && \frac{\delta \Lag}{\delta \lambda} &=L^{0.5}K  - 1000 = 0\\
  &\implies& L^{-1}K &= 1 \\
  &\implies& L &= K \\
  &\implies& K &= 1000^{0.67} = 100 \\
  &\implies& L &= 1000^{0.67} = 100
\end{alignat*}

\subsubsection*{b.5}

To see 1000 patients, we have that the minimum cost is $4000(100) + 8000(100) = 1,200,000$.

\subsection*{c)}

\begin{center}
  \begin{tikzpicture}[
    dot/.style={shape=circle, inner sep=2pt, draw, node contents=},
    circ/.style={shape=circle, inner sep=2pt, draw, fill}]
    \draw[thick,->] (0,0) -- (6.5,0) node[anchor=north west] {$L$};
    \draw[thick,->] (0,0) -- (0,5.5) node[anchor=south east] {$K$};
    
    \draw (6cm,3pt) -- (6cm,-3pt) node[anchor=north] {$300$}; 
    \draw (3pt,3cm) -- (-3pt,3cm) node[anchor=east] {$150$};

    \draw (6,0) -- (0,3);
  \end{tikzpicture}
\end{center}

\subsection*{d)}

The share on labor is equal to

\[
  \frac{\text{total spent on labor}}{\text{total costs}} = \frac{wL}{TC} =
    \frac{4000(100)}{1200000} = \frac{1}{3}
\]

Similarly, the share on capital is equal to

\[
  \frac{\text{total spent on capital}}{\text{total costs}} = \frac{wL}{TC} =
  \frac{8000(100)}{1200000} = \frac{2}{3}
\]

We notice that $q(L,K) = L^{0.5}K$, and that the share spent on labor is
$\frac{0.5}{0.5+1} = \frac{1}{3}$, and that the share spent on capital is
$\frac{1}{0.5+1} = \frac{2}{3}$.

In general, for $q(L,K) = L^{\alpha}K^{\beta}$, we have that the share spent on
labor is $\frac{\alpha}{\alpha+\beta}$ and the share spent on capital is
$\frac{\beta}{\alpha+\beta}$.

\subsection*{e),f),g)}

These do not exist, for some reason.

\subsection*{h)}

We solve for cost minimizing $L,K$ and plug in the cost function.

\subsection*{i)}

\begin{align*}
  \frac{MP_L}{MP_K} &= \frac{w}{r} \\
  \frac{0.5L^{-0.5}K}{L^{0.5}} &= \frac{w}{r} \\
  \implies \frac{K}{2L} &= \frac{w}{r} \\
  \implies K &= \frac{2wL}{r} \\
  \implies q &= L^{0.5}\frac{2wL}{r} \\
                    &= L^{1.5}\frac{2wL}{r} \\
  \implies L&= (\frac{rq}{2w})^{\frac{2}{3}} \\
  \implies q &= (\frac{rq}{2w})^{\frac{1}{3}} K \\
  \implies K &= q(\frac{2w}{rq})^{\frac{1}{3}} \\
                    &= (\frac{2wq^2}{r})^{\frac{1}{3}}
\end{align*}

\subsection*{j)}

The clinic's total cost function is then $C(w,r,q) =
w(\frac{rq}{2w})^{\frac{2}{3}} + r(\frac{2wq^2}{r})^{\frac{1}{3}} =
(\frac{1}{2}rwq^{\frac{1}{2}})^{\frac{2}{3}} + (2wq^2r^2)^{\frac{1}{3}}$.

\subsection*{k)}

We have that

\begin{align*}
  AC &= \frac{C(w,r,q)}{q} \\
     &= \frac{1}{q} [(\frac{1}{2}rwq^{\frac{1}{2}})^{\frac{2}{3}} + (2wq^2r^2)^{\frac{1}{3}}] \\
     &= q^{-\frac{1}{3}}(\frac{1}{2}rw^{\frac{1}{2}})^{\frac{2}{3}} + q^{-\frac{1}{3}}(2wr^2)^{\frac{1}{3}} \\
     &= q^{-\frac{1}{3}}((\frac{1}{2}rw^{\frac{1}{2}})^{\frac{2}{3}} + (2wr^2)^{\frac{1}{3}}) \\
  \implies \frac{dAC}{dq} &= -\frac{1}{3}q^{-\frac{4}{3}}((\frac{1}{2}rw^{\frac{1}{2}})^{\frac{2}{3}} + (2wr^2)^{\frac{1}{3}}) < 0\\
\end{align*}

Thus, AC is decreasing with increased output, agreeing with our initial finding
of economies of scale.

\end{document}