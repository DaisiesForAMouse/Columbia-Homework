\documentclass[12pt,letterpaper]{article}
\usepackage{fullpage}
\usepackage[top=2cm, bottom=4.5cm, left=2.5cm, right=2.5cm]{geometry}
\usepackage{amsmath,amsthm,amsfonts,amssymb,amscd}
\usepackage{lastpage}
\usepackage{enumerate}
\usepackage{fancyhdr}
\usepackage{mathrsfs}
\usepackage{xcolor}
\usepackage{graphicx}
\usepackage{listings}
\usepackage{hyperref}
\usepackage{tikz}
\usepackage{xfrac}
\usepackage{nicefrac}
\usepackage{xcolor}

\usetikzlibrary{shapes.geometric,fit}

\hypersetup{
  colorlinks=true,
  linkcolor=blue,
  linkbordercolor={0 0 1}
}

\setlength{\parindent}{0.0in}
\setlength{\parskip}{0.05in}


\newcommand\course{ECON 3211}
\newcommand\hwnumber{3}
\newcommand\NetIDa{dc3451}
\newcommand\NetIDb{David Chen}          

\theoremstyle{definition}
\newtheorem*{statement}{Statement}
\newtheorem*{claim}{Claim}
\newtheorem*{theorem}{Theorem}

\newcommand{\contra}{\Rightarrow\!\Leftarrow}

\pagestyle{fancyplain}
\headheight 35pt
\lhead{\NetIDa}
\lhead{\NetIDa\\\NetIDb}
\chead{\textbf{\Large Problem Set \hwnumber}}
\rhead{\course \\ \today}
\lfoot{}
\cfoot{}
\rfoot{\small\thepage}
\headsep 1.5em

\begin{document}

\subsection*{Problem 1}
\subsubsection*{a)}

\begin{center}
    \begin{tikzpicture}[
            dot/.style={shape=circle, inner sep=2pt, draw, node contents=},
            circ/.style={shape=circle, inner sep=2pt, draw, fill}]
				\draw[thick,->] (0,0) -- (4.5,0) node[anchor=north west] {raspberries};
				\draw[thick,->] (0,0) -- (0,4.5) node[anchor=south east] {strawberries};
			
				\draw (4cm,3pt) -- (4cm,-3pt) node[anchor=north] {$8$}; 

				\draw (3pt,4cm) -- (-3pt,4cm) node[anchor=east] {$8$};
        
				\draw (0,4) -- node[circle,label=below:]{} (4,0);
				%\draw[thick] (2.5,4) -- (2.5,0.67) -- (4,0.67);
				%\draw (4,.67) node[label=right:IC]{};
				%\draw (2.5,0.67) node[circ,label=above right:{B$^*$, (5,500)}]{};
				
				%\draw[dashed] (2.5,0) -- (2.5,0.67);
				%\draw[dashed] (0,0.67) -- (2.5,0.67);
				%\draw[dashed] (0,0) -- (4,1.066);
 		\end{tikzpicture}
\end{center}

\subsubsection*{b)}

\begin{align*}
		MU_X &= \frac{\delta U}{\delta x} = 3 \\
		MU_Y &= \frac{\delta U}{\delta y} = 2 \\
		MRS &= -\frac{MU_X}{MU_Y} \\
		&= -\frac{3}{2} \\
\end{align*}

What this means is that, as $MRS$ is invariant in respect to both $x, y$, then for varying amounts of strawberries and raspberries $MRS$ is constant at $1.5$ for typical consumers.

\subsubsection*{c)}

As above, the marginal utilities for raspberries and strawberries are respectively $MU_Y=2$ and $MU_X = 3$. Now given that the prices of those two fruits are equal, then we have that an utility maximizing consumer buys only strawberries and no raspberries. This can be seen as $MRS = 1.5 > \frac{P_X}{P_Y} = 1$.















\subsection*{Problem 4}
\subsubsection*{a)}
\begin{align*}
		\max_{x,y}(z^2ln(x) + y), \text{ s.t. } p_xx + y = M
\end{align*}

\subsubsection*{b)}

\subsubsection*{c)}
Assume that for the rest of the problem that the solution is interior.

Neither normal nor inferior.

$x$ is not Giffen.

\subsubsection*{d)}
\subsubsection*{e)}

Note that goods whose consumption rises with price are called Veblen goods, or status symbol goods.
Demand increases with price because the good becomes a more effective means of converying status.

Note that here as $U$ varies with $z = p_x$, we have that demand and the effective preferences of consumers change with the price of a good.
However, in a Giffin good, we see that budget constraints change such that optimization changes.






















\end{document}