\documentclass[12pt,letterpaper]{article}
\usepackage{fullpage}
\usepackage[top=2cm, bottom=4.5cm, left=2.5cm, right=2.5cm]{geometry}
\usepackage{amsmath,amsthm,amsfonts,amssymb,amscd}
\usepackage{lastpage}
\usepackage{enumerate}
\usepackage{fancyhdr}
\usepackage{mathrsfs}
\usepackage{xcolor}
\usepackage{graphicx}
\usepackage{listings}
\usepackage{hyperref}
\usepackage{tikz}
\usepackage{relsize}
\usepackage{fancyvrb}
\usepackage{import}
\usepackage{float}
\usepackage{xifthen}
\usepackage{pdfpages}
\usepackage{transparent}
\usetikzlibrary{shapes.geometric,fit}

\hypersetup{%
  colorlinks=true,
  linkcolor=blue,
  linkbordercolor={0 0 1}
}

\setlength{\parindent}{0.0in}
\setlength{\parskip}{0.05in}

\theoremstyle{definition}
\newtheorem*{statement}{Statement}
\newtheorem*{claim}{Claim}
\newtheorem*{theorem}{Theorem}
\newtheorem*{lemma}{Lemma}

\newcommand{\contra}{\Rightarrow\!\Leftarrow}
\newcommand{\R}{\mathbb{R}}
\newcommand{\F}{\mathbb{F}}
\newcommand{\Z}{\mathbb{Z}}
\newcommand{\Zeq}{\mathbb{Z}_{\geq 0}}
\newcommand{\Zg}{\mathbb{Z}_{>0}}
\newcommand{\Req}{\mathbb{R}_{\geq 0}}
\newcommand{\Rg}{\mathbb{R}_{>0}}
\newcommand{\N}{\mathbb{N}}
\newcommand{\Q}{\mathbb{Q}}
\newcommand{\C}{\mathbb{C}}
\DeclareMathOperator{\ima}{im}
\DeclareMathOperator{\spn}{span}
\DeclareMathOperator{\rank}{rank}
\DeclareMathOperator{\real}{Re}
\DeclareMathOperator{\imag}{Im}
\DeclareMathOperator{\diver}{div}
\DeclareMathOperator{\curl}{curl}
\DeclareMathOperator{\id}{id}
\DeclareMathOperator{\inter}{int}
\DeclareMathOperator{\Dr}{Dr}
\DeclareMathOperator{\Jac}{Jac}

\newcommand{\incfig}[1]{\input{./figures/#1.pdf_tex}}
\graphicspath{ {./figures/} }

\title{MATH 4065 HW 1}
\author{David Chen, dc3451}
\date{\today}

\begin{document}

\maketitle

\subsection*{1a}

Note that if $z_1 = z_2 = z'$, any $z \in \C$ trivially satisfies the desired property $|z
- z'| = |z - z'|$.

When $z_1 \neq z_2$, the set of all $z$ satisfying the desired property
describes a line in the complex plane: in particular, it contains the point $(z_1
+ z_2) / 2$ and is perpendicular to the line containing both $z_1$ and $z_2$.

\begin{figure}[H]
  \centering
  \incfig{perp}
\end{figure}

\subsection*{1e}

We have that $\real(az + b) = \real(az) + \real(b) = \real(a)\real(z) -
\imag(a)\imag(z) + \real(b)$. Then, if $\real(z) \neq 0$ and $\imag(z) \neq 0$,
we want that $z$ is either above or below (depending on the sign of $\imag(a)$) the line
\[
  \imag(z) = \frac{\real(a)}{\imag(a)}\real(z) - \frac{\real(b)}{\imag(a)}
\]

In particular, this becomes more clear if we write $\imag(z) = y, \real(z) = x$,
such that if $\imag(a) < 0$, we want points in the plane that satisfy
\[
  y > \frac{\real(a)}{\imag(a)}x - \frac{\real(b)}{\imag(a)}
\]
Similarly, if we have $\imag(a) > 0$, we want points in the plane that satisfy
\[
  y < \frac{\real(a)}{\imag(a)}x - \frac{\real(b)}{\imag(a)}
\]

\begin{figure}[H]
  \centering
  \incfig{parte}
  \caption{The left is for some $a$ in the first quadrant. $z$ can be anywhere
  in the gray. The right is for some $a$ in the fourth quadrant.}
\end{figure}

Now, the expression is easier if $a$ is either real or imaginary; in the case
that $a$ is real and positive, we want $z$ such that $\real(z) >
-\frac{\real(b)}{\real(a)}$. In particular, this is anything to the right of the
vertical line $x = -\frac{\real(b)}{\real(a)}$. If $a$ is real and negative, $z$ can be
anything to the left of that line.

Similarly, if $a$ is imaginary with positive imaginary part, then we want $z$
such that $\imag(z) < \frac{\real(b)}{\imag(a)}$. This is anything below the
horizontal line $y = \frac{\real(b)}{\imag(a)}$. If $a$ is imaginary with
negative imaginary part, $z$ can be anything above that line.

If $a = 0$, the choice of $b$ will fix that either $z$ can be anything in $\C$
(if $\real(b) > 0$) or nothing.

Geometrically, this is reasonable as the transformation $az + b$ does the
following to the shaded region:

\begin{figure}[H]
  \centering
  \incfig{trans}
\end{figure}

\subsection*{11}

First, we'll do 10, which is that $4\frac{\partial }{\partial z}\frac{\partial
}{\partial \bar{z}} = \Delta$. This is true at least under the assumption that the partial
derivatives are continuous, which allows that $\frac{\partial }{\partial
x}\frac{\partial }{\partial y} = \frac{\partial }{\partial y}\frac{\partial
}{\partial x}$.

Then,

\begin{align*}
  \frac{\partial }{\partial z}\left(\frac{\partial }{\partial \bar{z}}\right) &= \frac{1}{2}\left(\frac{\partial }{\partial x}\left(\frac{\partial }{\partial \bar{z}}\right)\right) + \frac{1}{2i}\left(\frac{\partial }{\partial y}\left(\frac{\partial }{\partial \bar{z}}\right)\right) \\
                                                                   &= \frac{1}{2}\left(\frac{\partial }{\partial x}\left(\frac{1}{2}\frac{\partial }{\partial x} - \frac{1}{2i}\frac{\partial }{\partial y}\right)\right) + \frac{1}{2i}\left(\frac{\partial }{\partial y}\left(\frac{1}{2}\frac{\partial }{\partial x} - \frac{1}{2i}\frac{\partial }{\partial y}\right)\right) \\
                                                                   &= \frac{1}{4}\left(\frac{\partial ^2}{\partial x^2} + \frac{\partial ^2}{\partial y^2}\right) + \frac{1}{4i} \left(\frac{\partial }{\partial y}\frac{\partial }{\partial x} + \frac{\partial }{\partial x}\frac{\partial}{\partial y}\right) \\
                                                                   &= \frac{1}{4}\left(\frac{\partial ^2}{\partial x^2} - \frac{\partial ^2}{\partial y^2}\right) = \frac{1}{4} \Delta
\end{align*}

Now, we have that if $f$ is holomorphic, then $f$ obeys the Cauchy-Riemann
equations, and thus $\frac{\partial u}{\partial x} = \frac{\partial v}{\partial
y}$ and $\frac{\partial u}{\partial y} = -\frac{\partial v}{\partial x}$, where
$f(x + yi) = u(x,y) + v(x,y)i$.

However, this means that 

\begin{align*}
  \frac{\partial f}{\partial \bar{z}} &= \frac{1}{2}\left(\frac{\partial f}{\partial x} - \frac{1}{i}\frac{\partial f}{\partial y}\right) \\
                                      &= \frac{1}{2}\left(\left(\frac{\partial u}{\partial x} + i\frac{\partial v}{\partial x}\right) - \frac{1}{i}\left(\frac{\partial u}{\partial y} + i\frac{\partial v}{\partial y}\right)\right) \\
                                      &= \frac{1}{2}\left(\left(\frac{\partial u}{\partial x} - \frac{\partial v}{\partial y}\right) + i\left(\frac{\partial v}{\partial x} - \frac{1}{i^2} \frac{\partial u}{\partial y}\right)\right) \\
                                      &= \frac{1}{2}\left(\left(\frac{\partial u}{\partial x} - \frac{\partial v}{\partial y}\right) + i\left(\frac{\partial v}{\partial x} + \frac{\partial u}{\partial y}\right)\right) \\
                                      &= \frac{1}{2}(0 + 0) = 0
\end{align*}

Then, this gives us what we wanted: $\Delta f = \frac{\partial }{\partial z} \frac{\partial }{\partial \bar{z}} f = \frac{\partial }{\partial z}(0) = 0$.

\subsection*{13b}

Again, we have the Cauchy-Riemann equations, which give that $\frac{\partial u}{\partial x} = \frac{\partial v}{\partial y}$ and $\frac{\partial u}{\partial y} = -\frac{\partial v}{\partial x}$. However, if $f(x + yi) = u(x,y) + v(x,y)i$ has that $\imag(f)$ is constant, then $v(x,y)$ is constant which gives that $\frac{\partial v}{\partial x} = \frac{\partial v}{\partial y} = 0$. The Cauchy-Riemann equations then tell us that $\frac{\partial u}{\partial x} = \frac{\partial v}{\partial y} = 0$ and $\frac{\partial u}{\partial y} = -\frac{\partial v}{\partial x} = 0$. 

However, we have that since limit definition of the complex derivative allows us to approach the point from any direction, we can take that (as $f$ is holomorphic) $f'(x + yi) = \frac{\partial f}{\partial y} = \frac{\partial u}{\partial y} + i\frac{\partial v}{\partial y} = 0 + 0$. By a proposition proved in class, $f$ is constant on $\Omega$ (which is taken to be an open connected set).

\subsection*{14}

% We will induct on $N > M$ to show that
% \[
%   \sum_{n=M}^{N-1}(a_{n+1} - a_n)B_n = -a_M B_M + a_NB_{N-1}- \sum_{n=M+1}^{N-1}a_nb_n 
% \]

% In particular, we can check for a base case of $N = M + 1$ that
% % \[
% %   \sum_{n=M}^{M-1}(a_{n+1} - a_n)B_n =
% % \]
% % and
% % \begin{align*}
% %   0 &= -a_MB_M + a_MB_{M} \\
% %     &= -a_MB_M + a_M(B_{M-1}
% % \end{align*}
% \[
%   \sum_{n=M}^{M}(a_{n+1} - a_n)B_n = (a_{M+1} - a_M)B_M = -a_MB_M + a_{M+1}B_M = -a_MB_M + a_{N}B_{N-1}
% \]
% Since we take the convention that the empty sum is equal to 0, we have that
% \[
%   -a_MB_M + a_{N}B_{N-1} = -a_MB_M + a_{N}B_{N-1} + \sum_{M+1}^{M}a_nb_n = -a_MB_M + a_{N}B_{N-1} + \sum_{M+1}^{N-1}a_nb_n
% \]
% which was what we wanted.

% Then, we have that if the above holds for some $B = k

The relationship does not hold if $N < M$.

For the rest of this problem, the empty sum is 0, as it is in the book.

First, we check the degenerate case that $N = M$, in which case
\begin{align*}
  a_NB_N - a_MB_{M-1} - \sum_{n=M}^{N-1}(a_{n+1} - a_n)B_n &= a_MB_M - a_MB_{M-1} - \sum_{n=M}^{M-1}(a_{n+1} - a_n)B_n \\
                                                           &= a_M(B_M - B_{M-1}) \\
                                                           &= a_Mb_M \\
                                                           &= \sum_{n=M}^Ma_nb_n = \sum_{n=M}^Na_nb_n
\end{align*}

We have the following if $N > M$:

\begin{align*}
  \sum_{n=M}^{N-1}(a_{n+1}-a_n)B_n &= \sum_{n=M}^{N-1} a_{n+1}B_n - \sum_{n=M}^{N-1} a_nB_n \\
  \intertext{Reindexing,}
                                   &= \sum_{n=M+1}^N a_nB_{n-1} - \sum_{n=M}^{N-1} a_nB_n \\
  \intertext{This step is why we need $N > M$ to pull out the terms, otherwise both sums are empty:}
                                   &= a_NB_{N-1} - a_MB_M + \sum_{n=M+1}^{N-1} a_nB_{n-1} - \sum_{n=M+1}^{N-1} a_nB_n \\
                                   &= a_NB_{N-1} - a_MB_M - \sum_{n=M+1}^{N-1} a_n(B_{n} - B_{n-1}) \\
                                   &= a_NB_{N-1} - a_MB_M - \sum_{n=M+1}^{N-1} a_nb_n \\
\end{align*}

Then, we have that
\begin{align*}
  a_NB_N - a_MB_{M-1} - \sum_{n=M}^{N-1}(a_{n+1} - a_n)B_n &= a_NB_N - a_MB_{M-1} - a_NB_{N-1} + a_MB_M + \sum_{n=M+1}^{N-1} a_nb_n \\
                                                           &= a_N(B_N - B_{N-1}) + a_M(B_M - B_{M-1}) + \sum_{n=M+1}^{N-1} a_nb_n \\
                                                           &= a_Nb_N + a_Mb_M + \sum_{n=M+1}^{N-1} a_nb_n \\
                                                           &= \sum_{n=M}^{N} a_nb_n \\
\end{align*}
which was what we wanted.

\subsection*{16a}

We use the ratio test for radius of convergence, as given in class (and proved later in the problem set).

\begin{align*}
  \lim_{n \rightarrow \infty} \frac{|(\log(n+1))^2|}{|(\log(n))^2|} &= \lim_{n \rightarrow \infty} \left|\frac{(\log(n+1))^2}{(\log(n))^2}\right| \\
                                                                    &= \lim_{n \rightarrow \infty} \left|\left(\frac{\log(n+1)}{\log(n)}\right)^2\right| \\
                                                                    &= \lim_{n \rightarrow \infty} \left(\frac{\log(n+1)}{\log(n)}\right)^2 \\
                                                                    \intertext{From L'Hopital's, we get}
                                                                    &= \lim_{n \rightarrow \infty} \left(\frac{\frac{2\log(n+1)}{n+1}}{\frac{2\log(n)}{n}}\right) \\
                                                                    &= \lim_{n \rightarrow \infty} \left(\frac{\log(n+1)}{\log(n)}\right) \\
                                                                    &= \lim_{n \rightarrow \infty} \left(\frac{n}{n+1}\right) \\
                                                                    &= 1
\end{align*}

which gives us a radius of convergence of $1$.

\subsection*{16c}

\begin{align*}
  \lim_{n \rightarrow \infty} \frac{\left|\frac{(n+1)^2}{4^{n+1}+3(n+1)}\right|}{\left|\frac{n^2}{4^{n}+3n}\right|} &= \lim_{n \rightarrow \infty} \left| \frac{(n+1)^2}{n^2} \frac{4^n + 3n}{4^{n+1}+3(n+1)} \right| \\
                                                                                                                    &= \lim_{n \rightarrow \infty}  \frac{(n+1)^2}{n^2} \left|\frac{4^n + 3n}{4^{n+1}+3(n+1)} \right| \\
                                                                                                                    &= \lim_{n \rightarrow \infty} \left|\frac{4^n + 3n}{4^{n+1}+3(n+1)} \right| \\
                                                                                                                    \intertext{We can take $n$ large enough such that everything inside the modulus is positive, so} 
                                                                                                                    &= \lim_{n \rightarrow \infty} \frac{4^n + 3n}{4^{n+1}+3(n+1)} \\
                                                                                                                    \intertext{From L'Hopital,}
                                                                                                                    &= \lim_{n \rightarrow \infty} \frac{4^x\log(4) + 3}{4^{x+1}\log(4) + 3} \\
                                                                                                                    &= \lim_{n \rightarrow \infty} \frac{4^x\log(4)^2}{4^{x+1}\log(4)^2} \\
                                                                                                                    &= \lim_{n \rightarrow \infty} \frac{1}{4}\frac{4^x}{4^{x}} \\
                                                                                                                    &= \frac{1}{4}
\end{align*}

which gives us a radius of convergence of $4$.

\subsection*{16e}

The constant term of the series does not affect convergence. In particular, if we take the empty product to be 1, we have that $F$ can be reindexed to start from $n = 0$.

\begin{align*}
  \lim_{n \rightarrow \infty} \frac{\left|\frac{\alpha(\alpha+1)\cdots(\alpha+n)\beta(\beta+1)\cdots(\beta+n)}{(n+1)!\gamma(\gamma+1)\cdots(\gamma+n)}\right|}{\left| \frac{\alpha(\alpha+1)\cdots(\alpha+n-1)\beta(\beta+1)\cdots(\beta+n-1)}{n!\gamma(\gamma+1)\cdots(\gamma+n-1)}\right|} &= \lim_{n \rightarrow \infty} \left| \frac{(\alpha + n)(\beta + n)}{(n+1)(\gamma+n)}\right| = 1
\end{align*}

which gives us a radius of convergence of $1$. The limit follows from the numerator and the denominator both being monic quadratics.

\subsection*{17}

We have that $\lim_{n \rightarrow \infty} \frac{|a_{n+1}|}{|a_n|} = L$ yields some $N$ such that $\forall n \geq N$, $\left|\frac{|a_{n+1}|}{|a_n|} - L\right| < \epsilon$ for any positive $\epsilon \in \R$. Then, we have that $L - \epsilon < \frac{|a_{n+1}|}{|a_n|} < L + \epsilon$. Since this holds for any $n \geq N$, and $|a_{n+1}|, |a_n| > 0 \implies L > 0$ (we can then pick only $0 < \epsilon < L$, which is good enough) we have that 
\[
  (L - \epsilon)^{n-N} < \prod_{i=N}^{n-1}\frac{|a_{n+1}|}{|a_n|} = \frac{|a_n|}{|a_N|} < (L + \epsilon)^{n-N}
\]
for $n > N$.

Rearranging, we have that 
\[
  (L - \epsilon)^n(L - \epsilon)^{-N}|a_N| < |a_n| < (L + \epsilon)^{n}(L+\epsilon)^{-N}|a_N|
\]
which gives
\[
  (L - \epsilon)((L - \epsilon)^{-N}|a_N|)^{1 / n} < |a_n|^{1 / n} < (L + \epsilon)((L+\epsilon)^{-N}|a_N|)^{1 / n}
\]

Since we know that $\lim_{n \rightarrow \infty}b^{1 / n} = 1$ (for $b  > 0$), we have that $\exists N'$ such that for any $\epsilon'$, $1 - \epsilon' < b^{1 / n} < 1 + \epsilon'$. Applying this to $b = (L - \epsilon)^{-N}|a_N|$ and $\epsilon' = \frac{\epsilon}{L - \epsilon}$, we get that $(L - \epsilon)(1 - \epsilon') =  L - 2\epsilon < (L - \epsilon)((L - \epsilon)^{-N}|a_N|)^{1 / n}$, and similarly for $b = (L + \epsilon)^{-N}|a_N|$ and $\epsilon' = \frac{\epsilon}{L + \epsilon}$, we get $(L + \epsilon)((L + \epsilon)^{-N}|a_N|)^{1 / n} < (L + \epsilon)(1 + \epsilon') =  L + 2\epsilon$ for sufficiently large $n$, i.e. $n > \max{N, N'}$. 

This bounds
\[
  L - 2\epsilon < |a_n|^{1 / n} < L + 2\epsilon
\]
for $n > \max{N, N'}$. This then gives that $\lim_{n \rightarrow \infty} |a_n|^{1 / n} = L$, which was what we wanted.

\subsection*{7}

\subsection*{a}

Since $\left| \frac{w-z}{1 - \bar{w}z} \right| < 1 \iff \left| \frac{w-z}{1 - \bar{w}z} \right|^2 < 1^2$ and $\left| \frac{w-z}{1 - \bar{w}z} \right| = 1 \iff \left| \frac{w-z}{1 - \bar{w}z} \right|^2 = 1^2$ as the modulus is nonnegative, we can compute
\[
  \frac{w - z}{1 - \bar{w}z} \frac{\overline{w - z}}{\overline{1 - \bar{w}z}} = \frac{|w|^2 - \bar{z}w - z\bar{w} + |z|^2}{1 -  \bar{z}w - z\bar{w} + |wz|^2}
\]

This means that we want to show
\[
  |w|^2 - \bar{z}w - z\bar{w} + |z|^2 \leq 1 -  \bar{z}w - z\bar{w} + |wz|^2
\]
which reduces to $0 \leq 1 + |w|^2|z|^2 - |w|^2 - |z|^2 = (1 - |w|^2)(1-|z|^2)$. Since we have that if $|w|, |z| < 1$ that the right hand is positive, we have that $\left|\frac{w-z}{1-\bar{w}z}\right| < 1$ in that case. If either of $|w|, |z| = 1$, then the right hand vanishes, and $\left|\frac{w-z}{1-\bar{w}z}\right| = 0$.

\subsection*{b}
\subsubsection*{i}

We have that if $f, g$ are holomorphic, then $fg$, $f + g$, and $f / g$ are all holomorphic from class. This gives that $w - z$ is holomorphic, as is $1 - \bar{w}z$ since $w$ is fixed, and these are linear functions in $z$. Since $f / g$ is holomorphic, we have that $F = (w- z) / (1 - \bar{w}z)$ is itself holomorphic.

To show that it takes $\mathbb{D} \rightarrow \mathbb{D}$, consider that since $w$ is fixed in $\mathbb{D}$, we have that $|w| < 1$, and since $z \in \mathbb{D}$, that also $|z| < 1$. From part a, we have that $|F(z)| < 1 \implies F(z) \in \mathbb{D}$. 

To show that $F$ is actually a bijection, see part iv.

\subsubsection*{ii}

We can just compute this directly:
\[
  F(0) = \frac{w - 0}{1 - \bar{w}0} = \frac{w}{1} = w
\]
\[
  F(w) = \frac{w - w}{1 - \bar{w}w} = \frac{0}{1 - |w|^2} = 0
\]
We can take $\frac{0}{1 - |w|^2} = 0$ since we have that $w$ is in the unit disc, and thus satisfies that $|w| < 1 \implies |w|^2 < 1 \implies 1 - |w|^2 > 0$.

\subsubsection*{iii}

We have that $|F(z)| = \left| \frac{w-z}{1 - \overline{w}z} \right|$, which was shown to be 1 when $|z| = 1$ in part a.

\subsubsection*{iv}

We will take the book's hint and compute

\begin{align*}
  F(F(z)) &= \frac{w - \frac{w-z}{1 - \overline{w}z}}{1 - \overline{w}\frac{w-z}{1 - \overline{w}z}} \\
          &= \frac{\frac{w(1-\overline{w}z) - (w - z)}{1 - \overline{w}z}}{\frac{1 - \overline{w}z - \overline{w}(w - z)}{1 - \overline{w}z}}
          \intertext{Since we have that $w,z \in \mathbb{D}, |w|, |z| < 1 \implies |\overline{w}z| < 1 \implies 1 \neq \overline{w}z$, we can cancel:}
          &= \frac{w(1-\overline{w}z) - (w - z)}{1 - \overline{w}z - \overline{w}(w - z)} \\
          &= \frac{-|w|^2z + z}{1 - |w|^2} \\
          &= \frac{z(1 -|w|^2)}{1 - |w|^2} \\
          \intertext{Again, we have that $|w| < 1 \implies 1 - |w|^2 > 0$, so}
          &= z
\end{align*}

We can now find a preimage for any $z \in \mathbb{D}$, which is $F(z)$, as we have that $F(F(z)) = z$, so $F$ is surjective. Further, suppose we have $z_1, z_2 \in \mathbb{D}$ and $F(z_1) = F(z_2)$. Then, $F(z_1) = F(z_2) \implies F(F(z_1)) = F(F(z_2)) \implies z_1 = z_2$, so $F$ is also injective, and therefore an injection.

\end{document}

% LocalWords:  NetID fancyplain LocalWords colorlinks linkcolor linkbordercolor
