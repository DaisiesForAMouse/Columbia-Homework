\documentclass[12pt,letterpaper]{article}
\usepackage{fullpage}
\usepackage[top=2cm, bottom=4.5cm, left=2.5cm, right=2.5cm]{geometry}
\usepackage{amsmath,amsthm,amsfonts,amssymb,amscd}
\usepackage{lastpage}
\usepackage{enumerate}
\usepackage{fancyhdr}
\usepackage{mathrsfs}
\usepackage{xcolor}
\usepackage{graphicx}
\usepackage{listings}
\usepackage{hyperref}
\usepackage{tikz}
\usepackage{relsize}
\usepackage{fancyvrb}
\usepackage{import}
\usepackage{float}
\usepackage{xifthen}
\usepackage{pdfpages}
\usepackage{transparent}
\usetikzlibrary{shapes.geometric,fit}

\hypersetup{%
  colorlinks=true,
  linkcolor=blue,
  linkbordercolor={0 0 1}
}

\setlength{\parindent}{0.0in}
\setlength{\parskip}{0.05in}

\theoremstyle{definition}
\newtheorem*{statement}{Statement}
\newtheorem*{claim}{Claim}
\newtheorem*{theorem}{Theorem}
\newtheorem*{lemma}{Lemma}

\newcommand{\contra}{\Rightarrow\!\Leftarrow}
\newcommand{\R}{\mathbb{R}}
\newcommand{\D}{\mathbb{D}}
\newcommand{\F}{\mathbb{F}}
\newcommand{\Z}{\mathbb{Z}}
\newcommand{\Zeq}{\mathbb{Z}_{\geq 0}}
\newcommand{\Zg}{\mathbb{Z}_{>0}}
\newcommand{\Req}{\mathbb{R}_{\geq 0}}
\newcommand{\Rg}{\mathbb{R}_{>0}}
\newcommand{\N}{\mathbb{N}}
\newcommand{\Q}{\mathbb{Q}}
\newcommand{\C}{\mathbb{C}}
\DeclareMathOperator{\ima}{im}
\DeclareMathOperator{\spn}{span}
\DeclareMathOperator{\rank}{rank}
\DeclareMathOperator{\real}{Re}
\DeclareMathOperator{\imag}{Im}
\DeclareMathOperator{\diver}{div}
\DeclareMathOperator{\curl}{curl}
\DeclareMathOperator{\id}{id}
\DeclareMathOperator{\inter}{int}
\DeclareMathOperator{\Dr}{Dr}
\DeclareMathOperator{\Jac}{Jac}
\DeclareMathOperator{\res}{res}

\newcommand{\incfig}[1]{\input{./figures/#1.pdf_tex}}
\graphicspath{ {./figures/} }

\title{MATH 4065 HW 6}
\author{David Chen, dc3451}
\date{\today}

\begin{document}

\maketitle

\section*{12}

Taking $f(z) = \frac{\pi \cot(\pi z)}{(u + z)^{2}} = \frac{\pi \cos(\pi z)}{(u + z)^{2}\sin(\pi z)}$, we can see that this has poles where $(u + z)^{2} = 0$ and where $\sin(\pi z) = 0$. Since $\sin(\pi z) = \frac{e^{\pi i z} - e^{-\pi i z}}{2i} = 0 \implies e^{2\pi i z} = 1 \implies z = k$ for $k \in \Z$, we have that since $f(z) = (u+z)^{-2}\frac{\pi \cos(\pi z)}{\sin(\pi z)}$, and $\sin(\pi z) \neq 0$ on a neighborhood of $u$ (since $u$ is not an integer, and therefore is not a zero of $\sin(\pi z)$), the pole is of order two at $-u$. Computing the residue,
\begin{align*}
  \res_{-u} f &= \lim_{z \rightarrow -u} \frac{d}{dz}(z+u)^{2}f(z) \\
              &= \lim_{z \rightarrow -u} \frac{d}{dz}\frac{\pi \cos(\pi z)}{\sin(\pi z)} \\
              &= \lim_{z \rightarrow -u} \pi\frac{-\pi\sin^{2}(\pi z) - \pi\cos^{2}(\pi z)}{\sin^{2}(\pi z)} \\
              &= -\frac{\pi^{2}}{\sin^{2}(\pi z)}
\end{align*}

Similarly, around $z = k$ for $k \in \Z$, we have that (noting that from above, $\frac{d}{dz}\pi\cot(\pi z) = \frac{-\pi^{2}}{\sin^{2}(\pi z)}$), $1/f(z) = \frac{(u+z)^{2}}{\pi \cot(\pi z)}$, and
\[
  \frac{d}{dz}\frac{1}{f(z)} = \frac{2\pi \cot(\pi z)(u + z) + \pi^{2}\frac{(u + z)^{2}}{\sin^{2}(\pi z)}}{\pi^{2}\cot^{2}(\pi z)} = \frac{2\tan(\pi z)(u + z)}{\pi} + \frac{(u+z)^{2}}{\cos^{2}(\pi z)}
\]
which when evaluated at $z = k$, gives $\frac{d}{dz}\frac{1}{f(k)} = 0 + (u + k)^{2}$, and since $u$ is not an integer, $u + k \neq 0 \implies (u+k)^{2} \neq 0$, so the zero is simple at $k$ and the pole at $k$ of $f$ is thus also simple. Computing the residue,
\begin{align*}
  \res_{k}f &= \lim_{z \rightarrow k}(z - k)\frac{\pi \cos(\pi z)}{(u + z)^{2}\sin(\pi z)} \\
            &= \lim_{z \rightarrow k}\frac{\pi \cos(\pi z)}{(u + z)^{2}} \cdot \frac{z-k}{\sin(\pi z)} \\
            &= \lim_{z \rightarrow k}\frac{\cos(\pi z)}{(u + z)^{2}} \cdot \lim_{z \rightarrow k}\frac{\pi (z-k)}{\sin(\pi z)} \\
  \intertext{This holds when the limits exist, which is apparent for the left limit and will be shown immediately for the right limit via L'Hopital:}
            &= \frac{\cos(\pi k)}{(u + k)^{2}} \cdot \lim_{z \rightarrow k}\frac{\pi}{\pi\cos(\pi z)} \\
            &= \frac{\cos(\pi k)}{(u + k)^{2}} \cdot \frac{1}{\cos(\pi k)} = \frac{1}{(u+k)^{2}}
\end{align*}

Now, consider the integral $\int_{C_{R_{N}}}f(z)dz$, where $C_{R_{N}}$ is $\{z \mid |z| = R_{N} = N + \frac{1}{2}\}$ for $N$ a positive integer at least $|u|$. Then, we have that
\begin{align*}
  \int_{C_{R_{N}}}f(z)dz &\leq 2\pi \left(N + \frac{1}{2}\right) \sup_{z \in C_{R_{N}}}\left|f(z)\right| \\
                         &=  2\pi \left(N + \frac{1}{2}\right) \sup_{z \in C_{R_{N}}}\left|\frac{\pi \cos(\pi z)}{(u + z)^{2}\sin(\pi z)}\right| \\
                         &=  2\pi \left(N + \frac{1}{2}\right) \sup_{z \in C_{R_{N}}}\frac{\pi}{|u+z|^{2}}\left|\frac{\cos(\pi z)}{\sin(\pi z)}\right| \\
                         &\leq  2\pi \left(N + \frac{1}{2}\right) \sup_{z \in C_{R_{N}}}\frac{\pi}{||z|-|u||^{2}}\left|\frac{\cos(\pi z)}{\sin(\pi z)}\right| \\
                         &= 2\pi \left(N + \frac{1}{2}\right)\frac{\pi}{(N + \frac{1}{2} - |u|)^{2}} \sup_{z \in C_{R_{N}}}\left|\frac{\cos(\pi z)}{\sin(\pi z)}\right| \\
  \intertext{Note that with $z = \pm R_{N} = \pm (N + \frac{1}{2})$,}
  \left|\frac{\cos(\pm\pi(N + 1/2))}{\sin(\pm\pi(N+1/2))}\right| &= \frac{|\cos(\pi(N + 1/2))|}{|\pm \sin(\pi(N+1/2))|} \\
                         &= \frac{0}{1} = 0 \\
  \intertext{So we only really need to care about $z$ with nonzero imaginary part. With a bit of algebra,}
  \left|\frac{\cos(\pi z)}{\sin(\pi z)}\right| &= \left|\frac{e^{i \pi z} + \frac{1}{e^{i \pi z}}}{e^{i \pi z} - \frac{1}{e^{i \pi z}}}\right| \\
                         &= \left|\frac{e^{2i \pi z} + 1}{e^{2i \pi z} - 1}\right| \\
                         &= \frac{|1 + e^{2i \pi z}|}{|1 - e^{2i \pi z}|} \\
                         &\leq \frac{1 + e^{-2\pi \imag(z)}}{||1| - e^{-2\pi \imag(z)}|} \\
  \intertext{Put $z = R_{N}e^{i\theta}$, $\theta \neq 0, \pi$ since we already covered that case. We have two remaining cases: if $\imag(z) = R_{N}\sin(\theta) > 0$, then as $R_{N} \rightarrow \infty$, $\imag(z) \rightarrow \infty$, so $e^{-2\pi \imag(z)} \rightarrow 0$, and}
  \frac{1 + e^{-2\pi \imag(z)}}{||1| - e^{-2\pi \imag(z)}|} &\rightarrow \frac{1 + 0}{|1 - 0|} = 1 \\
  \intertext{If $\imag(z) = R_{N}\sin(\theta) < 0$, then $e^{-2\pi \imag(z)} > 1$, so}
  \frac{1 + e^{-2\pi \imag(z)}}{||1| - e^{-2\pi \imag(z)}|} &= \frac{e^{-2\pi R_{N}\sin(\theta)} + 1}{e^{-2\pi R_{N}\sin(\theta)} - 1} = \frac{e^{-2\pi (N + 1/2)\sin(\theta)} + 1}{e^{-2\pi (N + 1/2)\sin(\theta)} - 1}\\
  \intertext{Then, via L'Hopital,}
  \lim_{N \rightarrow \infty}\frac{e^{-2\pi (N + 1/2)\sin(\theta)} + 1}{e^{-2\pi (N + 1/2)\sin(\theta)} - 1} &= \lim_{N \rightarrow \infty}\frac{e^{-2\pi (N + 1/2)\sin(\theta)}}{e^{-2\pi (N + 1/2)\sin(\theta)}} = 1 \\
  \intertext{Finally, we get that in the limit $N \rightarrow \infty$, $\left|\frac{\cos(\pi z)}{\sin(\pi z)}\right| \leq 1$ (where $|z| = N + 1/2$) on the upper and lower half planes and vanishes on the real line. Finally we can bound the integral in the limit:}
  \lim_{N \rightarrow \infty}\int_{C_{R_{N}}}f(z)dz &\leq \lim_{N \rightarrow \infty} 2\pi \left(N + \frac{1}{2}\right)\frac{\pi}{(N + \frac{1}{2} - |u|)^{2}} \sup_{z \in C_{R_{N}}}\left|\frac{\cos(\pi z)}{\sin(\pi z)}\right| \\
  &\leq \lim_{N \rightarrow \infty} 2\pi \left(N + \frac{1}{2}\right)\frac{\pi}{(N + \frac{1}{2} - |u|)^{2}} \\
  &= 2\pi^{2} \lim_{N \rightarrow \infty} \frac{(N + \frac{1}{2})^{2}}{(N + \frac{1}{2} - |u|)^{2}} \\
  &= 2\pi^{2} (0)= 0\\
  \intertext{However, by the residue theorem, we have that}
  \int_{C_{R_{N}}}f(z)dz &= \res_{-u}f + \sum_{\substack{k \in \Z \\ -R_{N} < k < R_{N}}}\res_{k}f \\
                         &= -\frac{\pi^{2}}{\sin^{2}(\pi z)} + \sum_{\substack{k \in \Z \\ -R_{N} < k < R_{N}}} \frac{1}{(u+k)^{2}}\\
  \intertext{In the limit, as $N \rightarrow \infty$,}
  0 &= \lim_{N \rightarrow \infty}\int_{C_{R_{N}}}f(z)dz \\
                         &= -\frac{\pi^{2}}{\sin^{2}(\pi z)} + \lim_{N \rightarrow \infty}\sum_{\substack{k \in \Z \\ -R_{N} < k < R_{N}}} \frac{1}{(u+k)^{2}}\\
                         &= -\frac{\pi^{2}}{\sin^{2}(\pi z)} + \sum_{k=-\infty}^{\infty} \frac{1}{(u + k)^{2}}
\end{align*}
which finally gives us what we want:
\[
  \frac{\pi^{2}}{\sin^{2}(\pi z)} = \sum_{k=-\infty}^{\infty} \frac{1}{(u + k)^{2}}
\]


\section*{14}

Consider $g(z): \C \setminus \{0\} \rightarrow \C$, where $g(z) = f(1/z)$. There are three cases: either there is a removable singularity at $z = 0$, there is an essential singularity at $z = 0$, or there is a pole at $z = 0$.

In the first case, if the singularity is removable, then we have that $g$ is bounded in a neighborhood of the origin, say $D_{r}(0)$. In particular, this means that $|z| < r \implies |g(z)| < B$, which gives that $|z| < r \implies |f(1/z)| < B$, which in turn gives that $|z| > 1/r \implies |f(z)| < B$. Then, since $f$ is entire, we have that $f$ is continuous on $\overline{D}_{1/r}(0)$, which is a bounded and closed (and therefore compact) set, and thus $f$ is bounded on $\overline{D}_{1/r}(0)$, say by $B'$. Then, $f$ is bounded everywhere on the complex plane (for $|z| > 1/r$, we have that $|f(z)| < B$, and for $|z| \leq 1/r$, $|f(z)| < B'$), so $f$ is entire and bounded, and by Liouville is constant, so the singularity cannot be removable since constant functions cannot be injective.

In the second case that the singularity is essential, then Casorati-Weierstrass gives that the image under $g$ of $D_{1}(0) \setminus \{0\}$ is dense in the complex plane. Then, $g(z) = f(1/z)$, which is holomorphic everywhere in $\C \setminus \{0\}$, is then holomorphic on $D_{1}(2)$, which is a region that satisfies the open mapping theorem; thus, the image of $D_{1}(2)$ must be an open set in $\C$. However, since $g(D_{1}(0) \setminus \{0\})$ is dense, it must intersect $g(D_{1}(2))$ at some point; if it didn't, since $g(D_{1}(2))$ is open, we can pick any $w \in g(D_{1}(2))$ and note that there is some $r > 0$ such that $D_{r}(w) \subset g(D_{1}(2)) \implies g(D_{1}(0) \setminus \{0\}) \cap D_{r}(w) = \emptyset$, so $w$ cannot be contained in $g(D_{1}(0) \setminus \{0\})$, nor can it be a limit point of $g(D_{1}(0) \setminus \{0\})$, so $g(D_{1}(0) \setminus \{0\})$ cannot be dense in $\C$. $\contra$, so we have that there is some $x \in g(D_{1}(0) \setminus \{0\}) \cap g(D_{1}(2))$, so there are $z_{1} \in D_{1}(0) \setminus \{0\}$ and $z_{2} \in D_{1}(2)$ such that $g(z_{1}) = g(z_{2}) \implies f(1/z_{1}) = f(1/z_{2})$; further, since $D_{1}(0) \cap D_{1}(2) = \emptyset$, we have that $z_{1} \neq z_{2}$, so $f$ cannot be injective.

The last remaining case is that $g(z)$ contains a pole at $z = 0$. In this case, $f$ must be a polynomial: otherwise, we would have that
\[
  f(z) = \sum_{n=0}^{\infty}a_{n}z^{n} \implies g(z) = \sum_{n=0}^{\infty}\frac{a_{n}}{z^{n}}
\]
so for any nonnegative integer $n$, $g(z) = z^{-n}\left(\sum_{m=0}^{n}a_{n-m}z^{m} + \sum_{m=1}^{\infty}\frac{a_{m+n}}{z^{m}}\right)$, where $\sum_{m=0}^{\infty}\frac{a_{m+n}}{z^{m}}$ is clearly not holomorphic at $z = 0$ unless all of the $a_{k}$ vanish for $k > n$ for some nonegative integer $N$, so $g(z)$ cannot have a pole at $0$ of any order.

Then, if $f$ is a polynomial, then $f$ must be of degree 1; otherwise, if $f$ is of degree $n > 1$, then there are $n$ zeros of $f$ counting multiplicity. If there are two or more distinct zeros of $f$, then clearly $f$ is not injective, so $f(z) = c(z - w)^{n}$ for some root $w$ with multiplicity $n$. However, we have that if $f(z) = c(z - w)^{n}$, $f(w + e^{2\pi i / n}) = ce^{2\pi i} = c = f(w + 1)$, so $f$ cannot be injective. Thus, $f$ must be of degree 1 (if it were degree 0, it would be constant and thus not injective), so $f(z) = az + b$ for $a \neq 0$, as desired.


\section*{16}

\subsection*{a}

We begin with the fact that $f$ does not vanish on the circle $C = \{z \mid |z| = 1\}$, which gives that on $C$, $|f| > 0$. In particular, since $C$ is closed and bounded, and thus compact, $z \mapsto |f(z)|$ is continuous since $f$ is holomorphic on a region containing $C$ and is thus continuous on $C$, and so a minimum is attained on $C$; thus, on the unit circle, $|f| \geq N$ for some positive $N$. Similarly, $g$ being holomorphic in a region containing $C$ gives that $g$ is continuous on $C$, a compact set, and thus $g$ is bounded on $C$, say by $M$. Then, we have that $|f|/|g| \geq N/M$.

Thus, for $\epsilon < N/M$, we have that $|f|/|g| > \epsilon \implies |f| > \epsilon |g|$ on $C$, and so by Rouche's theorem we have that $f$ and $f_{\epsilon} = f + \epsilon g$ have the same amount of zeros, namely exactly 1.

\subsection*{b}

% Put $\eta: [0,r) \rightarrow \overline{D}_{1}(0)$ for the mapping $\epsilon \mapsto z_{\epsilon}$, where $z_{\epsilon}$ is the root of $f_{\epsilon}$. Then, suppose that $\eta$ is discontinuous at some $\epsilon$. Then, we have that there is some sequence $\{\epsilon_{n}\}_{n=1}^{\infty}$ such that $\epsilon_{n} \rightarrow \epsilon$ as $n \rightarrow \infty$, but $\eta(\epsilon_{n})$ does not converge to $\eta(\epsilon)$ as $n \rightarrow \infty$. However, the sequence $\eta(\epsilon_{n})$ is a sequence in a compact set, namely the closed unit disc, so there is some convergent subsequence, say $\eta(\epsilon_{n_{i}})$ which converges to some $w$ in the closed unit disc.

We'll use something that comes up in Conway's book, but not Stein, which is that under the assumptions of the argument principle and some holomorphic function $g$,
\[
  \frac{1}{2\pi i}\int_{C}g(z)\frac{f'(z)}{f(z)} = \sum_{i=1}^{n}g(z_{i})n_{i}
\]
where $z_{i}$ are all the zeros and poles of $f$, with order $n_{i}$ (poles will be written to have negative order, so a pole of order 1 at $z_{i}$ has $n_{i} = -1$).

To see this, note that we already have from the proof of the argument principle that
\[
  \frac{f(z)}{f'(z)} = \frac{n}{z - z_{0}} + G(z)
\]
for any pole or zero $z_{0}$ with order $n$. Then,
\[
  \res_{z_{0}}g(z)\frac{f(z)}{f'(z)} = \lim_{z \rightarrow z_{0}}(z-z_{0})\cdot \frac{g(z)n}{z - z_{0}} =  g(z_{0})n
\]
so the residue formula gives us what we want.

Then, in this case, we have that there are no poles in the closed unit disc for $f_{\epsilon}$, and there is only a simple zero in the closed unit disc. Then,
\begin{align*}
  z_{\epsilon} &= \frac{1}{2\pi i}\int_{C}z\frac{f'(z) + \epsilon g'(z)}{f(z) + \epsilon g(z)}dz \\
  &= \frac{1}{2\pi}\int_{0}^{2\pi}e^{it}\frac{f'(e^{i t}) + \epsilon g'(e^{i t})}{f(e^{i t}) + \epsilon g(e^{i t})}dt
\end{align*}
but the integrand is continuous in $\epsilon$, and therefore $z_{\epsilon}$ is continuous in $\epsilon$. In fact, it is differentiable in $\epsilon$ as well via Leibniz's rule.

\section*{17}

\subsection*{a}

Put the unit circle as $C$.

Note that for any $w_{0} \in D_{1}(0)$, we have that $\sup_{w_{0} \in C}|w_{0}| = |w_{0}| < 1$, and since $|f(z)| = 1$ on the unit circle, we have that on $C$, $1 = |f| > |-w_{0}| = |w_{0}|$ so Rouche gives that $f(z) - w_{0}$ has the same amount of zeros as $f$ in the unit disc; thus, if we can show that $f(z) = 0$ has a root, then $f(z) - w_{0} = 0$ also has a root, and so for every unit $w_{0}$ in the unit disc, $f(z) = w_{0}$ for some $z$ in the unit disc.

To see that $f(z)$ must have a zero, suppose that it vanishes nowhere in the unit disc. Then, $1/f(z)$ is holomorphic in the unit disc, and $|1/f(z)| = 1$ on $C$, the closure of the unit disc, so $|1/f(z)| \leq 1$ on the unit disc by the maximum modulus principle, and so $|f(z)| \geq 1$. However, by the maximum modulus principle, we also have that $|f(z)| = 1$ on $C$, so $|f(z)| \leq 1$ on the unit disc by the maximum modulus principle. Thus, $|f(z)| = 1$ on the unit disc, but this means that we map an open set, namely the unit disc, to a subset of $C$, which cannot be open, since for any $w \in C$, $w - \epsilon w / 2 \in D_{\epsilon}(w)$, since $|w - (w - \epsilon w/ 2)| = |w|\epsilon/2 = \epsilon/2 < \epsilon$, but $|w - \epsilon w/2| = 1 - \epsilon/2 < 1$, so $D_{\epsilon}(w) \not\subset C$ for any $\epsilon > 0$, so the image of an open set is not open, so $f$ cannot be nonconstant and holomorphic by the open mapping theorem. $\contra$, so $f(z)$ must have a zero, and by the earlier paragraph, we are done.

\subsection*{b}

Finally, a part b that's easier than part a...

Again, we only need to show that $f$ vanishes at some point in the unit disc. The following is copied with some slight modification from part a:

Note that for any $w_{0} \in D_{1}(0)$, we have that $\sup_{w_{0} \in C}|w_{0}| = |w_{0}| < 1$, and since $|f(z)| \geq 1$ on the unit circle, we have that on $C$, $|f| \geq 1 > |-w_{0}| = |w_{0}|$ so Rouche gives that $f(z) - w_{0}$ has the same amount of zeros as $f$ in the unit disc; thus, if we can show that $f(z) = 0$ has a root, then $f(z) - w_{0} = 0$ also has a root, and so for every unit $w_{0}$ in the unit disc, $f(z) = w_{0}$ for some $z$ in the unit disc.

Now suppose that $f(z)$ does not vanish on the closed unit disc. Note that $|f(z)| \geq 1 \implies |1/f(z)| \leq 1$ on $C$ so the maximum modulus principle gives that on the unit disc (since $f$ doesn't vanish, $1/f$ is holomorphic), $|1/f(z)| \leq \sup_{z \in C}|1/f(z)| \leq 1$. However, we already have a point $z_{0}$ in the unit disc that $|f(z_{0})| < 1 \implies |1/f(z_{0})| > 1$, so $\contra$, and $f(z)$ vanishes, and by Rouche we are done.

\section*{3}

We integrate over the following dotted contour $\gamma$:
\begin{figure}[H]
  \centering
  \incfig{drawing-6}
\end{figure}
where the corridors have width $\epsilon$, and the circular sections are $\epsilon$ away from $C_{r_{2}}, C_{r_{1}}$.

Note that the domain $\Omega = \{w \mid r_{1} \leq |w-z_{0}| \leq r_{2}\} \setminus \{w \mid w = z_{0} + \lambda(z - z_{0}), \lambda \in \R, \lambda > 0\}$ is a sector of an annulus (in particular, it is the annulus with the ray starting from $z_{0}$ and traveling through $z$ removed), which is simply connected, as seen in class. Now consider take $\frac{f(z)}{w - z}$ on $\Omega$: we have that $w \notin \Omega$, and $\Omega \subset \{w \mid r_{1} \leq |w-z_{0}| \leq r_{2}\}$, so $f(z)$ is holomorphic on $\Omega$, so $\frac{f(z)}{w-z}$ is holomorphic on $\Omega$. Since $\Omega$ is simply connected, we have that since $\gamma \subset \Omega$,
\begin{align*}
  \int_{\gamma}\frac{f(w)}{w-z}dw &= 0 \\
  \intertext{Taking $\epsilon \rightarrow 0$, we get the the corridors cancel, and}
  \int_{\gamma}\frac{f(w)}{w-z}dw &= \int_{C_{r_{2}}}\frac{f(w)}{w-z}dw - \int_{C_{z}}\frac{f(w)}{w-z}dw - \int_{C_{r_{1}}}\frac{f(w)}{w-z}dw = 0\\
  \intertext{Where $C_{z}$ is the circle around $z$, and the signs come from the orietation of the contour. But by Cauchy's integral formula,}
  \int_{C_{z}}\frac{f(w)}{w-z}dw &= 2\pi i f(z) \\
  \intertext{So, putting this together,}
  f(z) &= \frac{1}{2\pi i}\left(\int_{C_{r_{2}}}\frac{f(w)}{w-z}dw - \int_{C_{r_{1}}}\frac{f(w)}{w-z}dw\right)
\end{align*}

We proceed as in class, when we showed that a holomorphic function is infinitely differentiable. Then, on the interior of the annulus,
\begin{align*}
  \int_{C_{r_{2}}}\frac{f(w)}{w-z}dw &= \int_{C_{r_{2}}}\frac{f(w)}{(w-z_{0}) - (z-z_{0})}dw \\
               &= \int_{C_{r_{2}}}\frac{f(w)}{(w-z_{0})(1 - \frac{z-z_{0}}{w-z_{0}})}dw \\
  \intertext{Since $|w-z_{0}| = r_{2} > |z-z_{0}|$, the following series is absolutely convergent, which allows us to swap the sum and integral:}
                                       &= \int_{C_{r_{2}}}\frac{f(w)}{(w-z_{0})}\left(\sum_{n=0}^{\infty}\frac{(z-z_{0})^{n}}{(w-z_{0})^{n}}\right)dw \\
                                       &= \int_{C_{r_{2}}}\sum_{n=0}^{\infty}f(w)\frac{(z-z_{0})^{n}}{(w-z_{0})^{n+1}}dw \\
                                       &= \sum_{n=0}^{\infty}\int_{C_{r_{2}}}f(w)\frac{(z-z_{0})^{n}}{(w-z_{0})^{n+1}}dw \\
                                       &= \sum_{n=0}^{\infty}\left(\int_{C_{r_{2}}}\frac{f(w)}{(w-z_{0})^{n+1}}dw\right)(z-z_{0})^{n} \\
  \intertext{This sum converges whenever $r_{2} > |z - z_{0}|$, since this is the region where the first series expansion of $\frac{1}{1 - \frac{z-z_{0}}{w-z_{0}}}$ converged absolutely. Similarly,}
  -\int_{C_{r_{1}}}\frac{f(w)}{w-z}dw &= -\int_{C_{r_{1}}}\frac{f(w)}{(w-z_{0}) - (z-z_{0})}dw \\
               &= \int_{C_{r_{2}}}\frac{f(w)}{(z-z_{0})(1 - \frac{w-z_{0}}{z-z_{0}})}dw \\
  \intertext{Since $|w-z_{0}| = r_{1} < |z-z_{0}|$, the following series is absolutely convergent, which allows us to swap the sum and integral:}
                                       &= \int_{C_{r_{2}}}\frac{f(w)}{(z-z_{0})}\left(\sum_{n=0}^{\infty}\frac{(w-z_{0})^{n}}{(z-z_{0})^{n}}\right)dw \\
                                       &= \int_{C_{r_{2}}}\sum_{n=0}^{\infty}f(w)\frac{(w-z_{0})^{n}}{(z-z_{0})^{n+1}}dw \\
                                       &= \sum_{n=0}^{\infty}\int_{C_{r_{2}}}f(w)\frac{(w-z_{0})^{n}}{(z-z_{0})^{n+1}}dw \\
                                       &= \sum_{n=0}^{\infty}\left(\int_{C_{r_{2}}}\frac{f(w)}{(w-z_{0})^{-n}}dw\right)(z-z_{0})^{-(n+1)} \\
                                       &= \sum_{n=1}^{\infty}\left(\int_{C_{r_{2}}}\frac{f(w)}{(w-z_{0})^{-(n-1)}}dw\right)(z-z_{0})^{-n} \\
  \intertext{This sum converges whenever $r_{1} < |z - z_{0}|$, since this is the region where the first series expansion of $\frac{1}{1 - \frac{w-z_{0}}{z-z_{0}}}$ converged absolutely. Thus, both series are convergent on the interior of the annulus, allowing us to combine and reindex the sum as we like. Then, we have that, putting $a_{n} = \frac{1}{2\pi i}\int_{C_{r_{2}}}\frac{f(w)}{(w-z_{0})^{n+1}}dw$,}
  f(z) &= \frac{1}{2\pi i}\left(\int_{C_{r_{2}}}\frac{f(w)}{w-z}dw - \int_{C_{r_{1}}}\frac{f(w)}{w-z}dw\right) \\
                                     &= \sum_{n=0}^{\infty}a_{n}(z-z_{0})^{n} + \sum_{n=1}^{\infty}a_{-n}(z-z_{0})^{-n} \\
                                     &= \sum_{n=-\infty}^{\infty}a_{n}(z-z_{0})^{n}
\end{align*}
which gives us our desired Laurent expansion.

\end{document}

% LocalWords:  NetID fancyplain LocalWords colorlinks linkcolor linkbordercolor
% LocalWords:  holomorphic
