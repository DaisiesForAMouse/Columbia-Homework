\documentclass[12pt,letterpaper]{article}
\usepackage{fullpage}
\usepackage[top=2cm, bottom=4.5cm, left=2.5cm, right=2.5cm]{geometry}
\usepackage{amsmath,amsthm,amsfonts,amssymb,amscd}
\usepackage{lastpage}
\usepackage{enumerate}
\usepackage{fancyhdr}
\usepackage{mathrsfs}
\usepackage{xcolor}
\usepackage{graphicx}
\usepackage{listings}
\usepackage{hyperref}
\usepackage{tikz}
\usepackage{relsize}
\usepackage{fancyvrb}
\usetikzlibrary{shapes.geometric,fit}

\hypersetup{%
  colorlinks=true,
  linkcolor=blue,
  linkbordercolor={0 0 1}
}

\setlength{\parindent}{0.0in}
\setlength{\parskip}{0.05in}

\newcommand\course{MATH 1207}
\newcommand\hwnumber{9}
\newcommand\NetIDa{dc3451}
\newcommand\NetIDb{David Chen}

\theoremstyle{definition}
\newtheorem*{statement}{Statement}
\newtheorem*{claim}{Claim}
\newtheorem*{theorem}{Theorem}

\newcommand{\contra}{\Rightarrow\!\Leftarrow}
\newcommand{\R}{\mathbb{R}}
\newcommand{\F}{\mathbb{F}}
\newcommand{\Z}{\mathbb{Z}}
\newcommand{\Ze}{\mathbb{Z}_{\geq 0}}
\newcommand{\Zg}{\mathbb{Z}_{>0}}
\newcommand{\N}{\mathbb{N}}
\newcommand{\Q}{\mathbb{Q}}
\newcommand{\C}{\mathbb{C}}

\pagestyle{fancyplain}
\headheight 35pt
\lhead{\NetIDa}
\lhead{\NetIDa\\\NetIDb}
\chead{\textbf{\Large Homework \hwnumber}}
\rhead{\course \\ \today}
\lfoot{}
\cfoot{}
\rfoot{\small\thepage}
\headsep 1.5em

\begin{document}

\subsection*{Apostol p.236-237 no.17}

We proceed with integration by parts, which can be found in the Apostol reading,
to proceed.

\begin{alignat*}{2}
  &&u = \log^2(x),&\ du = \frac{2\log(x) dx}{x} \\
  &&v = x, &\ dv = dx \\
  &\implies& \int \log^2(x)dx &= x\log^2(x) - 2\int \log(x)dx \\
  &&u = \log(x), &\ du = \frac{dx}{x} \\
  &&v = x, &\ dv = dx \\
  &\implies& \int \log^2(x)dx &= x\log^2(x) - 2(x\log(x) - \int dx) \\
  && &= x\log^2(x) - 2x\log(x) + 2x + C
\end{alignat*}

\subsection*{Apostol p.236-237 no.19}

We again proceed with integration by parts:

\begin{alignat*}{2}
  &&u = \log^2(x), &\ du = \frac{2\log(x)dx}{x} \\
  &&v = \frac{1}{2}x^2, &\ dv = xdx \\
  &\implies& \int x\log^2(x)dx &= \frac{1}{2}x^2\log^2(x) - \int x\log(x)dx \\
  &&u = \log(x), &\ du = \frac{dx}{x} \\
  &&v = \frac{1}{2}x^2, &\ dv = xdx \\
  &\implies& \int x\log^2(x)dx &= \frac{1}{2}x^2\log^2(x) -
  (\frac{1}{2}x^2\log(x) - \frac{1}{2}\int xdx) \\
  && &= \frac{1}{2}x^2\log^2(x) - \frac{1}{2}x^2\log(x) + \frac{1}{4}x^2 + C \\
\end{alignat*}

\subsection*{Apostol p.236-237 no.30}

\begin{claim}
  \[ \forall r \in \Q, a > 0, \log(a^r) = r\log(a) \]
\end{claim}

\begin{proof}
  Let $r = \frac{m}{n}, m,n \in \Z$. It has been shown in the Apostol reading
  that for $n \in \Z, n \geq 1, \log(a^n) = n\log(a)$. Then, we have that 

  \begin{align*}
    \log(a^r) &= \log(a^{\frac{m}{n}}) \\
              &= \log((a^{\frac{1}{n}})^m) \\
              &= m\log(a^{\frac{1}{n}}) \\
  \end{align*}

  Further, we have that $n\log(a^{\frac{1}{n}}) = \log((a^{\frac{1}{n}})^n) = \log(a)$.
  Thus,
  \[
    \log(a^r) = m\log(a^{\frac{1}{n}}) = \frac{m}{n}n\log(a^{\frac{1}{n}}) =
    \frac{m}{n}\log(a) = r\log(a)
  \]
\end{proof}

\subsection*{Apostol p.250 no.39}

\begin{claim}
  Let $f$ be a function defined everywhere on the real axis, with a derivative
  $f'$ that satisfies the equation
  \[ \forall x, f'(x) = cf(x) \],
  where $c$ is a constant. There is a constant $K$ such that $f(x) = Ke^{cx}$
  for every $x$.
\end{claim}

\begin{proof}
  Consider $g(x) = f(x)e^{-cx}$. We will show that this is constant.
  \begin{align*}
    g'(x) &= f'(x)e^{-cx} + -cf(x)e^{-cx} \\
          &= cf(x)e^{-cx} - cf(x)e^{-cx} \\
          &= 0
  \end{align*}

  Thus, we have that $g(x) = K$ for some constant $K$, and then $K = f(x)e^{-cx}
  \implies f(x) = Ke^{cx}$.
\end{proof}

\subsection*{Apostol p.382 no.29}

\begin{claim}
  A series cannot converge to two different limits.
\end{claim}

\begin{proof}
  Suppose that a series $\{ a_n \}$ converges to $L, K$, and that $L \neq K$.
  Then, we have that for any $\frac{\epsilon}{2} > 0$, $\exists N_L \in \Zg \mid \forall n
  > N_L, |L - a_n| < \frac{\epsilon}{2}$. Similarly, $\exists N_K \in \Zg \mid \forall n >
  N_K, |K - a_n| < \frac{\epsilon}{2} \implies |L - K| = |(L - a_n) - (K - a_n)| \leq |L -
  a_n| + |K - a_n| < \epsilon$. Then, we have that since $\forall \epsilon, |L
  - K| < \epsilon$, $L = K$. (To see this last claim, $L \neq K \implies |L - K|
  \neq 0 \implies |L - K| > 0$. Then, take $\epsilon = \frac{|L-K|}{2}$, and $\contra$.)
\end{proof}

\subsection*{Apostol p.382 no.30}

\begin{claim}
  \[
    \lim_{n\rightarrow \infty} a_n = 0 \implies \lim_{n\rightarrow \infty} a_n^2 = 0
  \]
\end{claim}

\begin{proof}
  Since we have that $\lim_{n\rightarrow \infty} a_n = 0$, $\forall \epsilon
  \exists N \in \Zg \mid \forall n > N, |a_n| < \epsilon$. Further, $\exists N_1
  \mid n > N_1 \implies |a_n| < 1$. For any $\epsilon$, take now $\max(N, N_1)$,
  which means that both $|a_n| < \epsilon$ and $|a_n| < 1$, so $|a_n^2| < |a_n|
  < 1$, and also $|a_n^2| < |a_n| < \epsilon$ as well for $n > \max(N, N_1)$.
  Thus, we have that $\lim_{n\rightarrow \infty} a_n^2 = 0$.
\end{proof}

\section*{Problem 1}

\begin{claim}
  If $\lim_{n\rightarrow \infty} a_n = A, \lim_{n\rightarrow \infty} b_n = B$,
  and $a_n \leq b_n$ for all $n$, then $A \leq B$.
\end{claim}

\begin{proof}
  Suppose that $A > B$. Then, take $epsilon = \frac{A - B}{3}$. We have that
  $\exists N_A \in \Zg \mid n > N_A \implies |a_n - A| < \frac{A - B}{2}$, and also
  $\exists N_B \in \Zg \mid n > N_B \implies |b_n - B| < \frac{A - B}{2}$.
  Consider $N = \max(N_A, N_B)$. Then, $n > N \implies A - \frac{A - B}{2} < a_n
  < A + \frac{A - B}{2}, B - \frac{A - B}{2} < b_n < B + \frac{A - B}{2}
  \implies b_n < \frac{A + B}{2} < a_n$. $\contra$, so $A \leq B$.
\end{proof}

Note that we can't make a stronger claim then $A \leq B$ even if $a_n < b_n$.
Consider the sequences $a_n = \frac{1}{2^{n+1}}, b_n = \frac{1}{2^n}$. We have
that $b_n = 2a_n \implies b_n > a_n$. However, these both converge to $0$, as we
can take $N = \left\lceil \log_2(\frac{1}{\epsilon}) \right\rceil$ and $N =
\left\lceil \log_2(\frac{1}{\epsilon}) \right\rceil + 1$ respectively.


\section*{Problem 2}

\begin{claim}
  Let $\{a_n\}, \{b_n\}$ be sequences such that there exists $M \in \Zg$ such
  that $\forall n > M, a_n = b_n$. $a_n$ converges if and only if $b_n$
  converges, and if they converge they converge to the same value.
\end{claim}

\begin{proof}
  Suppose that $\{a_n\}$ converges to $L$. Note that proving this single direction is
  enough; the opposite direction can be shown by just swapping the names of
  $a_n$ and $b_n$ with each other. We have that $\forall \epsilon > 0$, $\exists
  N \in \Zg \mid n > N \implies |a_n - L| < \epsilon$. Then, consider $N' =
  \max(N, M)$. We then have that $\forall n > N', a_n = b_n \implies |a_n - L| =
  |b_n - L| < \epsilon$, and so $\{b_n\}$ converges to $L$ as well.
\end{proof}

\section*{Problem 4}

\begin{claim}
  If $\{x_n\}$ converges, then $\{|x_n|\}$ also converges.
\end{claim}

\begin{proof}
  Let $\{x_n\}$ converge to $L$. Then, $\forall \epsilon > 0, \exists N \in \Zg
  \mid n > N \implies |x_n - L| < \epsilon$. Further, we have that $||x_n| -
  |L|| \leq |x_n - L| < \epsilon$ as a conseqence of the previously proved reverse triangle
  inequality. Thus, $\{|x_n|\}$ converges to $|L|$.
\end{proof}

Note that the converse isn't true: consider that
\[
  x_n = \begin{cases}
    -1 & n \equiv 0 \mod 2 \\
    1 & n \equiv 1 \mod 2
  \end{cases}
\]

has $\{|x_n| \}$ converge to $1$, but $\{x_n\}$ diverges.

\end{document}

% LocalWords:  NetID fancyplain LocalWords colorlinks linkcolor linkbordercolor