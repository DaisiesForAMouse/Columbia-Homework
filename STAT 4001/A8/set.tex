\documentclass[12pt,letterpaper]{article}
\usepackage{fullpage}
\usepackage[top=2cm, bottom=4.5cm, left=2.5cm, right=2.5cm]{geometry}
\usepackage{amsmath,amsthm,amsfonts,amssymb,amscd}
\usepackage{lastpage}
\usepackage{enumerate}
\usepackage{fancyhdr}
\usepackage{mathrsfs}
\usepackage{xcolor}
\usepackage{graphicx}
\usepackage{listings}
\usepackage{hyperref}
\usepackage{tikz}
\usepackage{relsize}
\usepackage{fancyvrb}
\usetikzlibrary{shapes.geometric,fit}

\hypersetup{%
  colorlinks=true,
  linkcolor=blue,
  linkbordercolor={0 0 1}
}

\setlength{\parindent}{0.0in}
\setlength{\parskip}{0.05in}

\newcommand\course{STAT 4001}
\newcommand\hwnumber{8}
\newcommand\NetIDa{dc3451}
\newcommand\NetIDb{David Chen}

\theoremstyle{definition}
\newtheorem*{statement}{Statement}
\newtheorem*{claim}{Claim}
\newtheorem*{theorem}{Theorem}
\newtheorem*{lemma}{Lemma}

\newcommand{\contra}{\Rightarrow\!\Leftarrow}
\newcommand{\R}{\mathbb{R}}
\newcommand{\F}{\mathbb{F}}
\newcommand{\Z}{\mathbb{Z}}
\newcommand{\Zeq}{\mathbb{Z}_{\geq 0}}
\newcommand{\Zg}{\mathbb{Z}_{>0}}
\newcommand{\Req}{\mathbb{R}_{\geq 0}}
\newcommand{\Rg}{\mathbb{R}_{>0}}
\newcommand{\N}{\mathbb{N}}
\newcommand{\Q}{\mathbb{Q}}
\newcommand{\C}{\mathbb{C}}
\newcommand{\pr}[1]{\text{Pr}\left(#1\right)}
\newcommand{\var}[1]{\text{Var}\left(#1\right)}
\newcommand{\cov}[1]{\text{Cov}\left(#1\right)}

\pagestyle{fancyplain}
\headheight 35pt
\lhead{\NetIDa}
\lhead{\NetIDa\\\NetIDb}
\chead{\textbf{\Large Assignment \hwnumber}}
\rhead{\course \\ \today}
\lfoot{}
\cfoot{}
\rfoot{\small\thepage}
\headsep 1.5em

\begin{document}

\subsection*{9.1.1}

\subsubsection*{a}
\[
  \pi(\beta \mid \delta) = P(X \geq 1\mid \beta) = 1 - (1 - e^{-\beta}) = e^{-\beta}
\]

\subsubsection*{b}

The size of the test is
\[
  \sup_{\beta \geq 1}\left\{ e^{-\beta} \right\} = e^{-1}
\]

\subsection*{9.1.3}

\[
  \pi(p \mid \delta) = P(Y \geq 7) + P(Y \leq 1) =
  \sum_{i=7}^{20}\binom{20}{i}p^i(1-p)^{20-i}  + (1-p)^{20} + p(1-p)^{19}
\]

Computing, we arrive at

\begin{center}
\begin{tabular}{c|c}
  $p$ & $\pi(p \mid \delta)$ \\
  \hline
  0 & 1 \\
  0.1 & 0.394 \\
  0.2 & 0.158 \\
  0.3 & 0.399 \\
  0.4 & 0.751 \\
  0.5 & 0.942 \\
  0.6 & 0.993 \\
  0.7 & 0.999 \\
  0.8 & 0.999 \\
  0.9 & 0.999 \\
  1 & 1
\end{tabular}
\end{center}

\subsection*{9.1.20}

Let $\theta_0 \in \Omega$ be a random element of the sample space. Then,
$g(\theta_0) = g_0 \in \omega(x)$ if $\delta_{g_0}$ does not reject the null
$H_{0,g_0}: g(\theta) \leq g_0$ when $X = x$ is observed. Since the level of the test is $\alpha_0 = 1 -
\gamma$, the probability that $\delta_{g_0}$ does not reject is at least $1 -
\alpha_0 = \gamma$.

\subsection*{9.2.2}

\subsubsection*{a}

From the chapter, we ought to reject if $f_0(x) < 2f_1(x) \implies 1 < 2(2x)
\implies x > \frac{1}{4}$. Thus, the procedure is to reject the null if $x >
\frac{1}{4}$, and to not reject the null otherwise.

\subsubsection*{b}

The minimum value is

\[
  2\int_0^{\frac{1}{4}}f_1(x)dx + \int_{\frac{1}{4}}^1f_0(x)dx = \frac{1}{8} +
  \frac{3}{4} = \frac{7}{8}
\]

\subsection*{9.2.11}

We reject if $f_1(x) > f_0(x)$, where
\begin{gather*}
  f_0(x) = \prod_{i=1}^n\frac{1}{\sqrt{8\pi}}e^{-\frac{(x_i+1)^2}{8}} \\
  f_1(x) = \prod_{i=1}^n\frac{1}{\sqrt{8\pi}}e^{-\frac{(x_i-1)^2}{8}} \\
\end{gather*}

Then,
\[
  f_1(x) - f_0(x) = C(e^{-\sum_{i=1}^n\frac{(x_i+1)^2}{8}} -
  e^{-\sum_{i=1}^n\frac{(x_i-1)^2}{8}}) > 0 \implies \sum (x_i-1)^2 < \sum (x_i+1)^2
\]
for some constant $C$; since $\sum (x_i-1)^2 - \sum (x_i+1)^2 = -4\sum x_i$, we
have that we reject if $\overline{x_i} > 0$.

Further, $\overline{x_i}$ will be normally distributed,
such that $\sqrt{n}(\overline{x_i} + \mu)/2$ will be standard normal. Then,
$\alpha(\delta) = 1 - \Phi(\sqrt{n}/2)$, $\beta(\delta) = 1 - \Phi(\sqrt{n}/2)$,
such that
\[
  \alpha(\delta) + \beta(\delta) = 2(1 - \Phi(\sqrt{n}/2))
\]

Computing, we get the following:

\begin{center}
  \begin{tabular}{c|c}
    $n$ & $\alpha(\delta) + \beta(\delta)$ \\
    \hline 
    1 & 0.617 \\
    4 & 0.317 \\
    16 & 0.045 \\
    36 & 0.003
  \end{tabular}
\end{center}

\subsection*{9.3.1}

Let $y = \sum_{i=1}^nx_i$.
We have that
\[
  f_n(x \mid \lambda) = \prod_{i=1}^n\frac{\lambda^{x_i}e^{-\lambda}}{x_i!} =
  \frac{\lambda^ye^{-n\lambda}}{\prod x_i!}
\]

Then,
\[
  \frac{f_n(x \mid \lambda_2)}{f_n(x\mid\lambda_1)} =
  \frac{\lambda_2^ye^{-n\lambda_2}}{\lambda_1^ye^{-n\lambda_1}} = \left( \frac{\lambda_2}{\lambda_1} \right)^y\frac{e^{-n\lambda_2}}{e^{-n\lambda_1}}
\]
which is a stictly increasing function of $y$.

\subsection*{9.3.7}

The power is the likelihood of rejection; thus $\pi(\theta \mid \delta) = 0.05$.

\subsection*{9.3.11}

Let $y = \sum_{i=1}^nx_i$.
From the chapter, we have that the UMP test occurs when we have some $c$ such
that $P(y > c|\theta = 1) = 0.0143$, and we reject if $y > c$. In particular, we have that
$y$ has Poisson distribution with parameter $n\lambda = 10$. Computing, we have
that we need $c = 18$.

\subsection*{9.4.12}

\subsubsection*{a}

We must have that
\[
  P(x \leq c_1 \mid \beta = 1) + P(x \geq c_2 \mid \beta = 1) = 1 - e^{-c_1} + e^{-c_2} = \alpha_0
\]

\subsubsection*{b}

For any $c_1, c_2$, we can have $c_1 = -\log(1 - \alpha_0/2), c_2 =
-\log(\alpha_0/2)$, such that $c_1 = 0.051, c_2 = 3$.


\subsection*{9.5.1}

\subsubsection*{a}

The test statistic is
\[
  U = \sqrt{10}\frac{1.379 - 1.2}{0.3277} = 1.73
\]

Then, we reject $H_0$ at $\alpha_0 = 0.05$ if $U \geq 1.833$, which it is not.
Thus, we do not reject $H_0$ at level $0.05$.

\subsubsection*{b}

The $p$-value is, where $\Phi$ is the cdf of a $t-$distribution with $9$ degrees
of freedom,
\[
  p = 1-\Phi(1.73) = 0.0589
\]

% We wish to compute $U = \sqrt{n}\frac{\overline{X_n} - \mu_0}{\sigma'}$.
% \begin{gather*}
%   \overline{X_n} = \frac{-11.2}{8} = -1.4 \\
%   \sigma' = \left( \frac{\sum_i (X_i - \overline{X_n})^2}{n-1}
%   \right)^{\frac{1}{2}} = \left( \frac{\sum_i X_i^2  - 8X_i\overline{X_n} +
%       \overline{X_n}^2}{7} \right) = 2 \\
%   U = -1.98
% \end{gather*}

% Computing, we need $U \leq c_1 = -2.998, U \geq 1.4$ to reject; thus we do not reject.

\subsection*{9.5.2}

\subsubsection*{a}

First we compute $U = \sqrt{9}\frac{2}{3} = 2$. Then, we reject if $H_0 \geq 1.86$,
so we do reject.

\subsubsection*{b}

We reject if $|U| \geq 2.306$, so we do not reject.

\subsubsection*{c}

We have the confidence interval with bounds $22 \pm 2.306 = (19.694, 24.306)$.

\subsection*{9.5.18}

We first identify $\Omega_0 = \{(\mu, \sigma^2) \mid \mu \geq \mu_0\}, \Omega =
\{(\mu, \sigma^2)\}$. Then, if $\overline{x_n} \geq \mu_0$, $\Lambda(x) = 1$.
Otherwise, we have from the chapter that $\Lambda(x) = \left(
  \frac{\sigma^2}{\sigma_0^2} \right)^{\frac{n}{2}}$. Thus, $\Lambda$ is a
nondecreasing function of $u$, and for $k < 1$, we have that $\Lambda(x) \leq k \iff U
\leq c$ via the same algebra as the book.

\subsection*{9.6.2}

We have hypotheses:
\[
  H_0: \mu_1 \geq \mu_2 \\
  H_1: \mu_1 < \mu_2
\]
where the subscripts $1,2$ denote the mean concentrations of drugs A and B
respectively. Then, we compute
\[
  U = \frac{\sqrt{m+n-2}(\overline{x_n} - \overline{y_n})}{\sqrt{\frac{1}{m} +
      \frac{1}{n}}\sqrt{S_X^2 +  S_Y^2}} = -1.692
\]

We reject if $U < -1.356$; thus, we reject the null.

\subsection*{9.6.8}

We now have that $\psi = \left( \frac{1}{m} + \frac{1}{n} \right)^{-\frac{1}{2}}
= 2.108$. Computing, we get that
\[
  \pi(\mu_1, \mu_2, \sigma^2 \mid \delta) = T_{16}(-2.921) + 1 - T_{16}(2.921) = 0.247
\]

\subsection*{9.9.2}

% We have that when $\mu = 0$, $\sqrt{10000}\overline{X_n} = 100\overline{X_n}$ is
% the standard normal distribution. 

For some real value $\mu'$, $Z = 100(\overline{X_n} - \mu')$ has the standard normal
distribution, and $c$ must be the $97.5^{th}$ quintile of the normal
distribution, $1.96$, such that
\begin{align*}
  P(|\overline{X}| \geq c \mid \mu = \mu') &= P(100\overline{X} \leq -1.96, 1.96\leq 100 \overline{X} \mid \mu = \mu') \\
                                        &= P(Z \leq -1.96 - 100\mu', 1.96 - 100\mu' \leq Z) \\
                                           &= \Phi(-1.96 - 100\mu') + 1 - \Phi(1.96 - 100\mu')
\end{align*}

\subsubsection*{a}

Computing, we get the probability of rejection to be $0.17$.

\subsubsection*{b}

Computing, we get the probability of rejection to be $0.516$.

\end{document}

% LocalWords:  NetID fancyplain LocalWords colorlinks linkcolor linkbordercolor