\documentclass[12pt,letterpaper]{article}
\usepackage{fullpage}
\usepackage[top=2cm, bottom=4.5cm, left=2.5cm, right=2.5cm]{geometry}
\usepackage{amsmath,amsthm,amsfonts,amssymb,amscd}
\usepackage{lastpage}
\usepackage{enumerate}
\usepackage{fancyhdr}
\usepackage{mathrsfs}
\usepackage{xcolor}
\usepackage{graphicx}
\usepackage{listings}
\usepackage{hyperref}
\usepackage{tikz}
\usepackage{relsize}
\usepackage{fancyvrb}
\usetikzlibrary{shapes.geometric,fit}

\hypersetup{%
  colorlinks=true,
  linkcolor=blue,
  linkbordercolor={0 0 1}
}

\setlength{\parindent}{0.0in}
\setlength{\parskip}{0.05in}

\newcommand\course{STAT 4001}
\newcommand\hwnumber{2}
\newcommand\NetIDa{dc3451}
\newcommand\NetIDb{David Chen}

\theoremstyle{definition}
\newtheorem*{statement}{Statement}
\newtheorem*{claim}{Claim}
\newtheorem*{theorem}{Theorem}
\newtheorem*{lemma}{Lemma}

\newcommand{\contra}{\Rightarrow\!\Leftarrow}
\newcommand{\R}{\mathbb{R}}
\newcommand{\F}{\mathbb{F}}
\newcommand{\Z}{\mathbb{Z}}
\newcommand{\Zeq}{\mathbb{Z}_{\geq 0}}
\newcommand{\Zg}{\mathbb{Z}_{>0}}
\newcommand{\Req}{\mathbb{R}_{\geq 0}}
\newcommand{\Rg}{\mathbb{R}_{>0}}
\newcommand{\N}{\mathbb{N}}
\newcommand{\Q}{\mathbb{Q}}
\newcommand{\C}{\mathbb{C}}
\newcommand{\pr}[1]{\text{Pr}\left(#1\right)}

\pagestyle{fancyplain}
\headheight 35pt
\lhead{\NetIDa}
\lhead{\NetIDa\\\NetIDb}
\chead{\textbf{\Large Assignment \hwnumber}}
\rhead{\course \\ \today}
\lfoot{}
\cfoot{}
\rfoot{\small\thepage}
\headsep 1.5em

\begin{document}

\subsection*{2.1.1}

Since we have that $A \subset B$, $A \cap B = A$. Then,
\[
  \pr{A \mid B} = \frac{\pr{A \cap B}}{\pr{B}} = \frac{\pr{A}}{\pr{B}}
\]

\subsection*{2.1.4}

Odds are parenthetical. He must start with $A$ $(\frac{1}{2})$, then proceed to
stay $(\frac{1}{3})$, switch $(\frac{2}{3})$, and stay $(\frac{1}{3})$. This
leaves us with
\[
  \frac{1}{2} \cdot \frac{1}{3} \cdot \frac{2}{3} \cdot \frac{1}{3} = \frac{1}{27}
\]

\subsection*{2.1.11}

Note that since $A \cap B$ and $A^C \cap B$ are disjoint, as no $x$ can be both
in $A$ and $A^C$, $\pr{A \cap B} + \pr{A^C \cap B} = \pr{(A \cap B) \cup (A^C
  \cap B)} = \pr{B}$.

\[
  \pr{A^C \mid B} = \frac{\pr{A^C \cap B}}{\pr{B}} = \frac{\pr{B} - \pr{A \cap
    B}}{\pr{B}} = 1 - \frac{\pr{A \cap B}}{\pr{B}} = 1 - \pr{A \mid B}
\]

\subsection*{2.2.2}

We have that $\pr{A^C \cap B^C} = \pr{(A \cup B)^C} = 1 - \pr{A \cup B}$ from De
Morgan's, which expands to
\begin{align*}
  1 - (\pr{A} + \pr{B} - \pr{A \cap B}) &= 1 - \pr{A} - \pr{B} + \pr{A}\pr{B} \\
                                        &= (1 - \pr{A})(1 - \pr{B}) \\
                                        &= \pr{A^C}\pr{B^C} = \pr{A^C \cap B^C}
\end{align*}

This shows that $A^C, B^C$ are independent.

\subsection*{2.2.12a}

From before, we have that complements are also independent if the original
events were independent; similarly, we have that if $A, B$ are independent, as $A
\cap B$ and $A \cap B^C$ are disjoint we have $\pr{A \cap B^C} = \pr{A} - \pr{A
  \cap B} = \pr{A}(1 - \pr{B}) = \pr{A}\pr{B^C}$, so $A, B^C$ are also
independent.

The odds that none of the three occur:

\[
  (1 - \frac{1}{4})(1 - \frac{1}{3})(1 - \frac{1}{2}) = \frac{1}{4}
\]

\subsection*{2.2.12b}

The odds that exactly one occurs:

\[
  \frac{1}{4}(1 - \frac{1}{3})(1 - \frac{1}{2}) + \frac{1}{3}(1 - \frac{1}{4})(1 -
  \frac{1}{2}) + \frac{1}{2}(1 - \frac{1}{4})(1 - \frac{1}{3}) = \frac{1}{12} +
  \frac{1}{8} + \frac{1}{4} = \frac{11}{24}
\]

\subsection*{2.2.22}

\begin{claim}
  Suppose that $A_1, A_2, B$ are events such that $\pr{A_1 \cap B} > 0$. Then
  $A_1, A_2$ are conditionally independent given $B$ $\iff \pr{A_2 \mid A_1 \cap
  B} = \pr{A_2 | B}$. 
\end{claim}

\begin{proof}
  $(\implies)$ We already have that
  \[
    \pr{A_1 \cap A_2 \mid B} = \pr{A_1 \mid B}\pr{A_2 \mid B}
  \]
  as we assume that $A_1, A_2$ are conditionally independent.
  
  Expanding,

  \begin{alignat*}{2}
    &&\pr{A_1 \cap A_2 \mid B} &= \pr{A_1 \mid B}\pr{A_2 \mid B} \\
    &\implies& \frac{\pr{A_1 \cap A_2 \cap B}}{\pr{B}} &= \frac{\pr{A_1 \cap
        B}}{\pr{B}} \cdot \frac{\pr{A_2 \cap B}}{\pr{B}}  \\
    &\implies& \frac{\pr{A_1 \cap A_2 \cap B}}{\pr{A_1 \cap B}} &=
      \frac{\pr{A_2 \cap B}}{\pr{B}} \\
      &\implies& \pr{A_2 \mid A_1 \cap B} &= \pr{A_2 \mid B}
  \end{alignat*}

  $(\impliedby)$ We have that $\pr{A_2 \mid A_1 \cap B} = \pr{A_2 \mid B}
  \implies \frac{\pr{A_1 \cap A_2 \cap B}}{\pr{A_1 \cap B}} = \frac{\pr{A_2 \cap B}}{\pr{B}}$.

  \begin{align*}
    \pr{A_1 \mid B}\pr{A_2 \mid B} &= \frac{\pr{A_1 \cap B}}{\pr{B}} \cdot \frac{\pr{A_1 \cap B}}{\pr{B}}\\
                                   &= \frac{\pr{A_1 \cap B}}{\pr{B}} \cdot \frac{\pr{A_2 \cap A_2 \cap B}}{\pr{A_1 \cap B}} \\
                                   &= \frac{\pr{A_1 \cap A_2 \cap B}}{\pr{B}} \\
                                   &= \pr{A_1 \cap A_2 \mid B}
  \end{align*}
\end{proof}

\subsection*{2.3.4}

Let $E_1 = $ sick, $E_2 = $ not sick, and $A = $ a positive test.

\[
  \pr{E_1 \mid A} = \frac{\pr{E_1}\pr{A \mid E_1}}{\pr{E_1}\pr{A \mid E_1} +
    \pr{E_2}\pr{A \mid E_2}} = \frac{0.00001 \cdot 0.95}{0.99999 \cdot 0.05} = 0.00019
\]

\subsection*{2.3.8a}

The odds of arriving at that configuration with coin $n$ is $p_n(1-p_n)^3$, and
let the event of having coin $n$ have odds $\pr{n} = \frac{1}{5}$.
Then, Bayes' Theorem tells us that the posterior odds of having coin $n$ are

\[
  \frac{\pr{n}p_n(1-p_n)^3}{\sum_{i=1}^5\pr{n}p_i(1-p_i)^3} =
  \frac{p_n(1-p_n)^3}{\sum_{i=1}^5p_i(1 - p_i)^5}
\]

Actually computing, we have that the posterior odds are $\pr{1 \mid TTTH} = 0$,
$\pr{2 \mid TTTH} = 0.587$, $\pr{3 \mid TTTH} = 0.347$, $\pr{4 \mid TTTH} =
0.065$, and $\pr{5 \mid TTTH} = 0$.

\subsection*{2.3.8a}

This can be computed as $\sum_{i=1}^5\pr{i \mid TTTH}p_i(1-p_i)^2$, which comes
out to be 0.129.

\subsection*{2.5.1}

\begin{alignat*}{2}
  &&\frac{\pr{A \cap D}}{\pr{D}} &\geq \frac{\pr{B \cap D}}{\pr{D}} \\
  &\implies& \pr{A \cap D} &\geq \pr{B \cap D} \\
  &&\frac{\pr{A \cap D^C}}{\pr{D^C}} &\geq \frac{\pr{B \cap D^C}}{\pr{D^C}} \\
  &\implies& \pr{A \cap D^C} &\geq \pr{B \cap D^C} \\
  &\implies& \pr{A \cap D} + \pr{A \cap D^C} &\geq \pr{B \cap D} + \pr{B \cap
    D^C} \\
  &\implies& \pr{A} &\geq \pr{B}
\end{alignat*}

\subsection*{2.5.23}

Let $E_1 = $ statistician, $E_2 = $ economist, $A = $ shy.

\[
  \pr{E_1 \mid A} = \frac{\pr{E_1}\pr{A \mid E_1}}{\pr{E_1}\pr{A \mid E_1} +
    \pr{E_2}\pr{A \mid E_2}} = \frac{0.8 * 0.1}{0.8 * 0.1 + 0.15 * 0.9} = 0.372
\]

\subsection*{2.5.29a}

There are $\binom{52}{13}$ ways to choose 13 cards; $\binom{4}{n}\binom{48}{13 -
n}$ ways to choose 12 cards and $n$ aces.

\[
  1 - \frac{\binom{4}{1}\binom{48}{12}}{\binom{52}{13} - \binom{48}{13}} = 0.369
\]

\subsection*{2.5.29b}

Let $H = $ picking the ace of hearts, and $A =$ the number of aces. Then,
$\pr{H} = \frac{13}{52}$.

\begin{align*}
  \pr{A \geq 2 \mid H} &= \frac{\pr{A \geq 2 \cap H}}{\pr{H}} \\
                       &= \frac{\pr{H} - \pr{A < 2 \cap H}}{\pr{H}} \\
                       &= \frac{\pr{H} - \pr{A = 1 \cap H}}{\pr{H}} \\
                       &= \frac{\pr{H} - \frac{\binom{48}{12}}{\binom{52}{13}}}{\pr{H}} = 0.561
\end{align*}

\end{document}

% LocalWords:  NetID fancyplain LocalWords colorlinks linkcolor linkbordercolor
