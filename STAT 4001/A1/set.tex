\documentclass[12pt,letterpaper]{article}
\usepackage{fullpage}
\usepackage[top=2cm, bottom=4.5cm, left=2.5cm, right=2.5cm]{geometry}
\usepackage{amsmath,amsthm,amsfonts,amssymb,amscd}
\usepackage{lastpage}
\usepackage{enumerate}
\usepackage{fancyhdr}
\usepackage{mathrsfs}
\usepackage{xcolor}
\usepackage{graphicx}
\usepackage{listings}
\usepackage{hyperref}
\usepackage{tikz}
\usepackage{relsize}
\usepackage{fancyvrb}
\usetikzlibrary{shapes.geometric,fit}

\hypersetup{%
  colorlinks=true,
  linkcolor=blue,
  linkbordercolor={0 0 1}
}

\setlength{\parindent}{0.0in}
\setlength{\parskip}{0.05in}

\newcommand\course{STAT 4001}
\newcommand\hwnumber{1}
\newcommand\NetIDa{dc3451}
\newcommand\NetIDb{David Chen}

\theoremstyle{definition}
\newtheorem*{statement}{Statement}
\newtheorem*{claim}{Claim}
\newtheorem*{theorem}{Theorem}
\newtheorem*{lemma}{Lemma}

\newcommand{\contra}{\Rightarrow\!\Leftarrow}
\newcommand{\R}{\mathbb{R}}
\newcommand{\F}{\mathbb{F}}
\newcommand{\Z}{\mathbb{Z}}
\newcommand{\Zeq}{\mathbb{Z}_{\geq 0}}
\newcommand{\Zg}{\mathbb{Z}_{>0}}
\newcommand{\Req}{\mathbb{R}_{\geq 0}}
\newcommand{\Rg}{\mathbb{R}_{>0}}
\newcommand{\N}{\mathbb{N}}
\newcommand{\Q}{\mathbb{Q}}
\newcommand{\C}{\mathbb{C}}
\newcommand{\pr}[1]{\text{Pr}\left(#1\right)}

\pagestyle{fancyplain}
\headheight 35pt
\lhead{\NetIDa}
\lhead{\NetIDa\\\NetIDb}
\chead{\textbf{\Large Assignment \hwnumber}}
\rhead{\course \\ \today}
\lfoot{}
\cfoot{}
\rfoot{\small\thepage}
\headsep 1.5em

\begin{document}

\subsection*{1.4.1}
Consider $x \in B^C$. Assume that $x \notin A^C$. Then, we have that $x \notin
A^C \implies x \in A \implies x \in B$. $\contra$, so $x \in A^C$.

\subsection*{1.4.8}

\begin{center}
  \begin{tabular}{c|c}
    Blood type & Set representation \\
    \hline
    A & $A \cap B^C$ \\
    B & $B \cap A^C$ \\
    AB & $A \cap B$ \\
    O & $A^C \cap B^C$
  \end{tabular}
\end{center}

\subsection*{1.5.12}

Note that the $B_i$ are all disjoint; any $B_i$ contains elements in $A_i$ that
are not seen before as elements of $A_1, A_2, ..., A_{i-1}$.

More formally, suppose that  for $i \neq j, x \in B_i, B_j$. Without loss of generality, take
$j > i$. Since $x \in B_j, x \in A_i^C$, and since $x \in B_i, x \in A_i$.
$\contra$, so $B_i, B_j$ are disjoint.

Further, we have that for any $n$, $\bigcup_{i=1}^nA_i = \bigcup_{i=1}^nB_i$.
For any $x \in \bigcup_{i=1}^nA_i$, let $k$ be the least index such that $x \in
A_k$. Then, $x \in B_k$ as well; similarly, since $B_k \subset A_k$, we have
that $\bigcup_{i=1}^nA_i = \bigcup_{i=1}^nB_i$.

We have now that
\[
  \sum_{i=1}^n\pr{B_i} = \pr{\bigcup_{i=1}^n B_i} = \pr{\bigcup_{i=1}^n A_i}
\]

\subsection*{1.5.14.a}

Since there are only four possibilities, we have that $\pr{\text{AB}} = 1 - (0.5
+ 0.12 + 0.34) = 0.04$.

The events below are disjoint, as one can only have a single blood type.

\begin{center}
  \begin{tabular}{c|c}
    Antigen & Probability \\
    \hline
    anti-A & $\pr{\text{A or AB}} = 0.34 + 0.04 = 0.38$\\
    anti-B & $\pr{\text{B or AB}} = 0.12 + 0.04$
  \end{tabular}
\end{center}

\subsection*{1.5.14.b}

This probability is exactly $\pr{AB} = 0.04$.

\subsection*{1.6.6}

Each outcome of faces is equally likely by assumption, where there are $2^3$ total
possibilities. There are only two outcomes where all are the same (all heads, all
tails), so the probability is $\frac{1}{4}$.

\subsection*{1.7.3}

This is simply just $5! = 120$ total ways.

\subsection*{1.7.6}

There are in total $6!$ ways to order $1,2,3,4,5,6$; there are $6^6$ total ways
to get a sequence of rolls. The odds are $\frac{6!}{6^6} = \frac{5}{324} \approx 0.0154$.

\subsection*{1.7.10.abc}

In any given ordering of the balls, the odds that any given position is occupied
by a red ball is $\frac{r}{100}$. Thus, the odds for all three parts are all $\frac{r}{100}$.

\subsection*{1.7.11}

\[
  P_{n,k} =
  \frac{(2\pi)^{\frac{1}{2}}n^{n+\frac{1}{2}}e^{-n}}{(2\pi)^{\frac{1}{2}}(n-k)^{n-k
      + \frac{1}{2}}e^{k-n}}
  = \frac{n^{n+\frac{1}{2}}}{e^k(n-k)^{n-k+\frac{1}{2}}}
\]

\subsection*{1.8.15.a}

\[
  \sum_{i=0}^n\binom{n}{i}
\]

is simply the total amount of subsets of any size of a size of set $n$. This can
be counted by considering that either an element is in a subset or not, leading
to a total of $2^n$.

Alternatively, consider that for $f(x, y) = (x + y)^n =
\sum_{i=0}^n\binom{n}{i}x^iy^{n-i}$ has that $f(1, 1) = 2^n = \sum_{i=0}^n\binom{n}{i}1^i1^{n-i}$.

\subsection*{1.8.15.b}

Take the same function $f(x, y) = (x + y)^n$ again; we see that $f(-1, 1) = 0^n
= 0 = \sum_{i=0}^n\binom{n}{i}(-1)^i = 0$.

\subsection*{1.8.18}

There are $\binom{20}{2}$ ways to choose the two recipients for each class;
there are $\binom{100}{10}$ ways to choose the ten recipients.

The probability is then

\[
  \frac{\binom{20}{2}^5}{\binom{100}{10}} \approx 0.014
\]

\subsection*{1.9.4}

There is only one correct spelling of statistics; the total amount of ways to do
such an ordering is

\[
  \binom{10}{3,3,2,1,1} = \frac{10!}{3!3!2!1!1!} = 50400
\]

The probability of an ordering spelling statistics is then

\[
  \frac{1}{\binom{10}{3,3,2,1,1}} = \frac{1}{50400} \approx 0.0000198
\]

\subsection*{1.10.6}

Put $R, W, B$ as the odds of picking no red, white, and blue balls,
respectively. We then need to compute $\pr{R \cup W \cup B} = \pr{R} + \pr{W} +
\pr{B} - \pr{R, W} - \pr{R, B} - \pr{W, B}$.

We have that $\pr{R} = \pr{W} = \pr{B} = \frac{\binom{60}{10}}{\binom{90}{10}}$,
and that $\pr{R, W} = \pr{R, B} = \pr{W, B} =
\frac{\binom{30}{10}}{\binom{90}{10}}$, so that

\[
  \pr{R \cup W \cup B} = 3\frac{\binom{60}{10} - \binom{30}{10}}{\binom{90}{10}}
\]


\subsection*{1.10.10}

The probability of exactly the first envelope being correct is $\frac{1}{6!}$,
as there is only one such ordering that places only the first envelope correctly
out of six total orderings; this is disjoint with only the second and third
envelopes each being correct, so the total probability is $3\frac{1}{6} = \frac{1}{2}$.

\end{document}

% LocalWords:  NetID fancyplain LocalWords colorlinks linkcolor linkbordercolor
