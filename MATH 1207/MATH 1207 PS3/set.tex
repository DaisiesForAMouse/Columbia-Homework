\documentclass[12pt,letterpaper]{article}
\usepackage{fullpage}
\usepackage[top=2cm, bottom=4.5cm, left=2.5cm, right=2.5cm]{geometry}
\usepackage{amsmath,amsthm,amsfonts,amssymb,amscd}
\usepackage{lastpage}
\usepackage{enumerate}
\usepackage{fancyhdr}
\usepackage{mathrsfs}
\usepackage{xcolor}
\usepackage{graphicx}
\usepackage{listings}
\usepackage{hyperref}
\usepackage{tikz}
\usepackage{relsize}
\usepackage{fancyvrb}
\usetikzlibrary{shapes.geometric,fit}

\hypersetup{%
  colorlinks=true,
  linkcolor=blue,
  linkbordercolor={0 0 1}
}

\setlength{\parindent}{0.0in}
\setlength{\parskip}{0.05in}

% Edit these as appropriate
\newcommand\course{MATH 1207}
\newcommand\hwnumber{3}                  % <-- homework number
\newcommand\NetIDa{dc3451}           % <-- NetID of person #1
\newcommand\NetIDb{David Chen}           % <-- NetID of person #2 (Comment this line out for problem sets)

\theoremstyle{definition}
\newtheorem*{statement}{Statement}
\newtheorem*{claim}{Claim}
\newtheorem*{theorem}{Theorem}

\newcommand{\contra}{\Rightarrow\!\Leftarrow}
\newcommand{\R}{\mathbb{R}}
\newcommand{\F}{\mathbb{F}}
\newcommand{\Z}{\mathbb{Z}}
\newcommand{\N}{\mathbb{N}}
\newcommand{\Q}{\mathbb{Q}}
\newcommand{\C}{\mathbb{C}}

\pagestyle{fancyplain}
\headheight 35pt
\lhead{\NetIDa}
\lhead{\NetIDa\\\NetIDb}                 % <-- Comment this line out for problem sets (make sure you are person #1)
\chead{\textbf{\Large Homework \hwnumber}}
\rhead{\course \\ \today}
\lfoot{}
\cfoot{}
\rfoot{\small\thepage}
\headsep 1.5em

\begin{document}

\subsection*{Apostol p.19 no.4}
\begin{claim}
    $0$ has no reciprocal.
\end{claim}

\begin{proof}
    We prove that $\forall x \in \F, 0 \cdot x = 0$, where $\F$ is a field.
    \begin{alignat*}{2}
        &&0x &= (0 + 0)x \\
        && &= 0x + 0x \\
        &\implies& 0x + -(0x) &= (0x + 0x) + -(0x) \\
        &\implies& 0x + -(0x) &= 0x + (0x + -(0x)) \\
        &\implies& 0 &= 0x + 0 \\
        && &= 0x
    \end{alignat*}
    Now suppose that an inverse $0^{-1} \in \F$ exists, such that $0(0^{-1}) = 1$.
    This contradicts the above. $\contra$
\end{proof}

\subsection*{Apostol p.21 no.1, thm.1.24}

\begin{claim}
    $ab > 0 \implies a,b \in \R^+$ or $a,b \notin \R^+$.
\end{claim}

\begin{proof}
    Suppose that $b \in \R^+ \iff b > 0$.
    It has already been shown in Apostol that for $c > 0$, $a > b \implies ac > bc$.
    \begin{alignat*}{2}
        && ab &> 0 \\
        &\implies& (ab)b^{-1} &> 0b^{-1} \\
        &\implies& a(bb^{-1}) &> 0 \\
        &\implies& a(1) &> 0 \\
        &\implies& a &> 0 \\
        &\implies& a &\in \R^+
    \end{alignat*}
    Suppose that $b \notin \R^+, b \neq 0 \iff b < 0$.
    It has already been shown in Apostol that for $c < 0$, $a > b \implies ac < bc$.
    \begin{alignat*}{2}
        && ab &> 0 \\
        &\implies& (ab)b^{-1} &< 0b^{-1} \\
        &\implies& a(bb^{-1}) &< 0 \\
        &\implies& a(1) &< 0 \\
        &\implies& a &< 0 \\
        &\implies& a &\notin \R^+
    \end{alignat*}
    Now $b = 0 \implies ab = 0$, and so $ab > 0$ is impossible. 
    We have covered all the cases for $b$, and the claim follows.
\end{proof}

\subsection*{Apostol p.21 no.10}

\begin{claim}
    $\forall h \in \R^+, 0 \leq x < h \implies x = 0$.
\end{claim}

\begin{proof}
    We have three cases: $x < 0, x = 0, x > 0$.
    Clearly as $0 \leq x$ it must be one of the latter two.
    Suppose that $0 < x$.
    Then the premise has that $x < 1 \implies x(x) < 1(x) \implies x^2 < x$.
    However, we also have that as $x^2 \in \R^+$, $x < x^2$ by the premise.
    Having both $x^2 < x$ and $x < x^2$ violates trichotomy, which was proved earlier
    in Apostol. This means that $x = 0$, as it is the only remaining case, if it exists.
    Clearly $\forall h \in \R^+, 0 \leq 0 < h \implies x = 0$.
\end{proof}

\subsection*{Apostol p.35 no.1c}

\begin{claim}
    $\sum_{i=1}^{n} i^3 = (\sum_{i=1}^{n} i)^2$
\end{claim}

\begin{proof}
    We proceed with induction. The base case is that $1^3 = 1^2 = 1$.

    Suppose that the claim holds for $n = k$.
    Note that it has been proved earlier in Apostol that 
    $\sum_{i=1}^{n} i = \frac{n(n+1)}{2}$.
    Then we have that 
    \begin{align*}
        \sum_{i=1}^{k+1} i^3 &= \sum_{i=1}^{k} i^3 + (k+1)^3 \\
        &= (\sum_{i=1}^{k} i)^2 + (k+1)^3 \\
        &= (\frac{k(k+1)}{2})^2 + (k+1)^3 \\
        &= (k+1)^2(\frac{k^2 + 4(k+1)}{4}) \\
        &= (k+1)^2(\frac{(k+2)}{2})^2 \\
        &= (\frac{(k+1)(k+2)}{2})^2 \\
        &= (\sum_{i=1}^{k+1} i)^2
    \end{align*}

    Thus we have that the claim holds for $n = k+1$, and by the principle of induction,
    the claim is true for all $n \in \Z_+$. 
\end{proof}

\subsection*{Apostol p.36 no.4}

\begin{claim}
    $\prod_{i=2}^{n}(1 - i^{-1}) = n^{-1}$.
\end{claim}

\begin{proof}
    The base case of $i = 2$ holds, as $1 - 2^{-1} = 2^{-1}$.
    
    Suppose that the claim holds for $n = k$.
    Then we have that
    \begin{align*}
        \prod_{i=2}^{k+1}(1 - i^{-1}) &= (\prod_{i=2}^{k}(1 - i^{-1}))(1 - (k+1)^{-1}) \\
        &= (k^{-1})((k+1)(k+1)^{-1} - (k+1)^{-1}) \\
        &= (k^{-1})((k+1-1)(k+1)^{-1}) \\
        &= (k^{-1})(k(k+1)^{-1}) \\
        &= (k+1)^{-1}
    \end{align*}

    Note that we have omitted all steps regarding commutativity and associativity.
    Thus we have that the claim holds for $n = k+1$, and by the principle of induction,
    the claim is true for all $n \in \Z_+$. 
\end{proof}

\subsection*{Apostol p.43 no.1j}

\begin{claim}
    $||x| - |y|| \leq |x - y|$
\end{claim}

\begin{proof}
    First we prove that for any $x \in \R$, we have that $|x| = |-x|$.
    Specifically, if $x = 0$, then $|x| = |-x| = 0$, and if $x > 0, x(-1) = -x < 0$ then 
    $|-x| = -(-x) = x = |x|$, and if $x < 0, x(-1) = -x > 0$ then
    $|x| = -x = |-x|$.

    Should $|x| = |y|$, then we have that both sides are $0$, so the claim holds.

    Consider the following for the other two cases:
    the triangle inequality, which was proved earlier in Apostol,
    has that $\forall y,z \in \R, |y|+|z| \geq |y+z|$.

    \begin{alignat*}{2}
        && |w+z| &\leq |w| + |z| \\
        &\implies& |w+z| - |w| &\leq |w| + |z| - |w| \\
        &\implies& |w+z| - |w| &\leq |z|\\
    \end{alignat*}


    For $|x| > |y|$, we have that $||x| - |y|| = |x| - |y|$.
    If we take $z = x - y, w = y$, then the claim follows:
    $|y + x - y| - |y| = |x| - |y| \leq |x - y|$.

    For $|x| < |y|$, we have that $||x| - |y|| = -(|x| - |y|) = |y| - |x|$.
    If we take $z = y - x, w = x$, then we have 
    $|y - x + x|-|x| = |y| - |x| \leq |y - x| = |-(y - x)| = |x - y|$.

    By trichotomy, we have covered all cases relating $|x|$ and $|y|$, so the claim follows.
\end{proof}

\section*{Problem 1}
\subsection*{a}

We put $\N$ for the nonegative integers $\Z_{\geq 0}$, mostly because
\verb#\Z_{\geq 0}# is hard to type compared to \verb#\N#.

\begin{claim}
    For any subset $S \subseteq \N$, the set $\{\ell \in \N \mid \ell \leq k\}$
    either contains no elements if $S$ or a least element of $S$.
\end{claim}

\begin{proof}
    Note that as $\N$ is the smallest inductive set containing $0$, 
    we have that $0$ is the minimal element in $\N$.

    Thus, the base case of $k = 0$ is true; $\{\ell \in \N \mid \ell \leq 0\} = \{0\}$, and
    either $0 \in S$ and is the minimal element, or else $0 \notin S$.

    Suppose that the claim is true for $k = n$. 
    Then the set $\{\ell \in \N \mid \ell \leq n+1\} = \{\ell \in \N \mid \ell \leq n\} \cup \{n+1\}$ 
    either does not contain any elements of $S$, or contains at least one element of $S$.

    If if it the following, then the claim follows for $k = n+1$. 
    If it is the latter, then consider that by the inductive hypothesis, we have that
    the set $\{\ell \in \N \mid \ell \leq n\}$ either contains the smallest element of $S$,
    or it contains no element of $S$. 

    In the first case, as $\{\ell \in \N \mid \ell \leq n\} \subseteq \{\ell \in \N \mid \ell \leq n+1\}$,
    we are done. In the second case, as $\{\ell \in \N \mid \ell \leq n+1\} \setminus \{\ell \in \N \mid \ell \leq n\} = \{n+1\}$,
    so we have that $n+1$ must be an element of $S$.

    However, since $\{\ell \in \N \mid \ell \leq n\}$ contains no elements of $S$,
    no $s \in \N$ exists such that $s < n+1$ and $s \in S$, or else it would be in both sets.
    By trichotomy, we have that all other elemnents of $S$ must be greater than $n+1$,
    and so $n+1$ is the minimal element of $S$, fulfilling the inductive step.

    We have that the claim holds for $k = n+1$, and by the principle of induction, 
    the claim is true for any $k \in \N$.
\end{proof}

\subsection*{b}

\begin{claim}
    $\N$ is well ordered.
\end{claim}

\begin{proof}
    Consider that if $S \subseteq \N$ is nonempty,
    we have $s \in S$ such that $\{\ell \in \N \mid \ell \leq s\}$ contains an element in $S$.
    In particular, the choice of $s$ does not matter.

    From part a, this set, a subset of $\N$, contains the least element of $S$, implying that any nonempty
    subset of the naturals has a least element.
\end{proof}

\section*{Problem 2}
\begin{claim}
    The Cartesian product of two finite sets $S,T$ is finite.
\end{claim}

\begin{proof}
    We proceed by showing proving some properties of finite sets.

    First, all subsets of finite sets are finite themselves.
    
    Suppose we have a finite set $A$.
    Since we have $A$ finite, $\exists n$ such that we have a bijection 
    $f: A \rightarrow \{i \in \N \mid 0 < i \leq n\}$.
    
    We induct on $n$.

    The base case is that $A = \{a\}$ has one object, $a$, with a bijection $f: a \mapsto 1$. 
    The bijection taking $\emptyset \rightarrow \emptyset$ shows $\emptyset$ is finite, so
    we see that the subsets of $A$, namely $A, \emptyset$, are both obviously finite.

    Now suppose that the claim holds for $n = k$

    Let $A$ be a finite set with a bijection $f: A \rightarrow \{i \in \N \mid 0 < i \leq k+1\}$
    We can find an element $a \in A$ such that $f^{-1}(k+1) = a$.
    Note that this element is not special; we can rearrange the bijection such that
    if we have another element $f^{-1}(k') = a'$, then just take $f'$ such that $f'^{-1}(k+1) = a'$
    and $f'^{-1}(k') = a$ and we have another bijection $f'^{-1}$ where $f'^{-1}(k+1) = a'$.

    Now, for any subset $B$, either $B = A$, in which case $B$ is finite, or $\exists a \mid a \in A, a \notin B$. Then
    the set $A \setminus \{a\}$ is finite, with the bijection simply $f$ restricted to
    the domain of $\{i \in \N \mid 0 < i < k\}$. 
    Then, we have that $B \subseteq A \setminus \{a\}$, but the latter set has only finite
    subsets by the inductive hypothesis.
    Thus, all subsets of finite subsets are finite themselves.

    We now proceed by showing that the union of two sets is itself finite.
    Note that we need only handle the case in which they are disjoint, as $A \cup B = A \cup (B \setminus A)$,
    which are two disjoint sets, and $(B \setminus A) \subseteq B$ is finite if $B$ is finite.

    We have then that for two disjoint finite sets $A, B$, with bijections
    $f_A: A \rightarrow \{i \in \N \mid 0 < i \leq n\}$ and $f_B: B \rightarrow \{i \in \N \mid 0 < i \leq m\}$.
    Their union has a bijection $f: A\cup B \rightarrow \{i \in \N \mid 0 < i \leq n+m\}$
    defined by

    \[
    f(x) = \begin{cases}
        f_A(x) & x \in A \\
        f_B(x) + n & x \in B \\
    \end{cases}
    \]

    This is injective as we have that if two elements map to the same element,
    they must be in the same set as $f_A(x) \leq n < f_B(x) + n$. Now,
    because $f_A, f_B$ are bijections, $f$ is then injective.

    Surjectivity is shown as $f(f_A^{-1}(k)) = k$ if $0 < k \leq n$ and $f(f_B^{-1}(k-n)) = k$ if $n < k \leq n+m$.

    We finally prove the original claim.

    We induct on $n$, with $T$ having a bijection $f_T: T \rightarrow \{i \in \N \mid 0 < i \leq n\}$.
    Let $S$ have a bijection $f_S: S \rightarrow \{i \in \N \mid 0 < i \leq m\}$.

    The base case is $S \times \{t\}$, which has a bijection $f_1: S\times \{t\} \rightarrow  \{i \in \N \mid 0 < i \leq n\}$
    given by $f_1: (s,t) \mapsto f_S(s)$.

    Now suppose that the claim holds for $n = k$.

    We have that $S \times (T \cup \{t\}) = (S \times T) \cup (S \times \{t\})$.
    This comes from 
    \begin{align*}
        (x,y) \in S \times (T \cup \{t\}) &\iff (x \in S) \land (y \in T \cup \{t\}) \\
        &\iff (x \in S) \land ((y \in T) \lor (y \in \{t\}) \\
        &\iff ((x \in S) \land (y \in T)) \lor ((x\in S) \land (y \in \{t\})) \\
        &\iff (x,y) \in (S \times T) \cup (S \times \{t\})
    \end{align*}

    However, we have that, taking $t \notin T$, $T$ has a bijection to 
    $\{i \in \N \mid 0 < i \leq n\}$, $S \times T$ is finite from the inductive hypothesis and that
    $S \times \{t\}$ is finite from the base case, and the union of two finite sets
    is itself finite, so the claim holds for $n = k+1$.

    By the principle of induction, this is true for any $n \geq 1 \in \N$.

    Note that if either of the sets are empty, $S \times T = \emptyset$, which is finite.
\end{proof}

\section*{Problem 3}

This is actually solved above, as the intersection of two sets is a subset of both sets.

\section*{Problem 4}

\begin{claim}
    $\forall x, \epsilon > 0 \in \R, |x| < \epsilon \iff -\epsilon < x < \epsilon$.
\end{claim}

\begin{proof}
    It is proved in Apostol that $a > b, c < 0 \implies ac < bc$.
    Further, that $x(-1) = -x$ follows from the uniqueness of the additive inverse,
    proved in Apostol, and the fact that $x(-1) + x = x(-1 + 1) = x(0) = 0$.

    We have not shown that $-1 < 0$, but note $\forall x,y > 0 \implies xy > 0$ implies that
    $x^2 > 0$. $1^2 = 1 > 0 \implies 1 + (-1) > 0 + (-1) \implies -1 < 0$.

    ($\implies$) By trichotomy, we have three cases.
    
    If $x < 0$, then $|x| = -x < \epsilon \implies -x(-1) > \epsilon(-1) \implies x > -\epsilon$.
    Further, $x < 0 < \epsilon \implies -\epsilon < x < \epsilon$.

    If $x = 0$, then $|x| = 0 < \epsilon \implies 0(-1) > \epsilon(-1) \implies 0 > -\epsilon \implies -\epsilon < 0 < \epsilon$.

    If $x > 0$, then $|x| = x < \epsilon, -\epsilon < 0 < x \implies -\epsilon < x < \epsilon$.

    ($\impliedby$) By trichotomy, we have three cases.

    If $x < 0$, then we have that $|x| = -x$. 
    Further, $-\epsilon < x \implies -\epsilon(-1) > x(-1) \implies -x = |x| < \epsilon$.

    If $x < 0$, then $|x| = 0 < \epsilon$.

    If $x > 0$, then $|x| = x < \epsilon$.
\end{proof}

\end{document}