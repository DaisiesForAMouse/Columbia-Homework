\documentclass[12pt,letterpaper]{article}
\usepackage{fullpage}
\usepackage[top=2cm, bottom=4.5cm, left=2.5cm, right=2.5cm]{geometry}
\usepackage{amsmath,amsthm,amsfonts,amssymb,amscd}
\usepackage{lastpage}
\usepackage{enumerate}
\usepackage{fancyhdr}
\usepackage{mathrsfs}
\usepackage{xcolor}
\usepackage{graphicx}
\usepackage{listings}
\usepackage{hyperref}
\usepackage{tikz}
\usepackage{relsize}
\usepackage{fancyvrb}
\usetikzlibrary{shapes.geometric,fit}

\hypersetup{%
  colorlinks=true,
  linkcolor=blue,
  linkbordercolor={0 0 1}
}

\setlength{\parindent}{0.0in}
\setlength{\parskip}{0.05in}

% Edit these as appropriate
\newcommand\course{MATH 1207}
\newcommand\hwnumber{5}                  % <-- homework number
\newcommand\NetIDa{dc3451}           % <-- NetID of person #1
\newcommand\NetIDb{David Chen}           % <-- NetID of person #2 (Comment this line out for problem sets)

\theoremstyle{definition}
\newtheorem*{statement}{Statement}
\newtheorem*{claim}{Claim}
\newtheorem*{theorem}{Theorem}

\newcommand{\contra}{\Rightarrow\!\Leftarrow}
\newcommand{\R}{\mathbb{R}}
\newcommand{\F}{\mathbb{F}}
\newcommand{\Z}{\mathbb{Z}}
\newcommand{\Ze}{\mathbb{Z}_{\geq 0}}
\newcommand{\Zg}{\mathbb{Z}_{>0}}
\newcommand{\N}{\mathbb{N}}
\newcommand{\Q}{\mathbb{Q}}
\newcommand{\C}{\mathbb{C}}

\pagestyle{fancyplain}
\headheight 35pt
\lhead{\NetIDa}
\lhead{\NetIDa\\\NetIDb}                 % <-- Comment this line out for problem sets (make sure you are person #1)
\chead{\textbf{\Large Homework \hwnumber}}
\rhead{\course \\ \today}
\lfoot{}
\cfoot{}
\rfoot{\small\thepage}
\headsep 1.5em

\begin{document}

\subsection*{Apostol p.70-71 no.3}

\begin{claim}
  $$\int_a^b [x]dx  + \int_a^b[-x]dx = a - b$$
\end{claim}

\begin{proof}
  % Consider that for any $n \in \Z, x \in \R$, we have that $n < x < n+1 \implies
  % [x] = n$. Now, this means that $f:[a,b] \rightarrow \R : x \mapsto [x]$ is equivalent to a step
  % function with partitions $\{a, [a] + 1 ...

  We first show that $\forall x \in \R \setminus \Z, [x] + [-x] = -1$. If for $n \in
  \Z, n < x < n + 1 \implies -(n + 1) < -x < -n \implies [x] = n, [-x] = -(n +
  1) \implies [x] + [-x] = n + -(n + 1) = -1$. Should  $x \in \Z$, we
  have that $[x] = x, [-x] = -x \implies [x] + [-x] = 0$.

  We also will eventually need that for $\{x_0, x_1, ..., x_n \mid x_i \in \R\}, \sum_{i=1}^n-(x_i - x_{i-1})
  = x_0 - x_n$.

  We induct on $n$, and the base case $n = 1$, and  $\sum_{i=1}^1-(x_i - x_{i-1})
  = x_0 - x_1$. For the inductive case, assume the statement holds for $n = k$.
  Then,
  \begin{align*}
    \sum_{i=1}^{k+1} -(x_i - x_{i-1}) &= \sum_{i=1}^k -(x_i - x_{i-1}) + -(x_{k+1} - x_k) \\
                                      &= x_0 - x_k - x_{k+1} + x_k \\
                                      &= x_0 -x_{k+1}
  \end{align*}
  and we are done.

  We have proved additivty of integrals in class, so we have that $\int_a^b[x]dx
  + \int_a^b[-x]dx = \int_a^b([x] + [-x])dx$. However, consider that $\forall x
  \in \R \setminus \Z, [x] + [-x] = 1$, so $[x] + [-x]$, restricted to $[a,b]$,
  is then a step function with partition $\{a,b\} \cup \{n \in \Z \mid a < n <
  b\} = \{a = x_0, x_1, x_2,...,x_k = b\}$ and constant values all equal to $-1$
  (On any interval $(x_{i-1},x_i)$, we have that there are no integers and thus
  must that $f|_{(x_{i-1},x_i)} = -1$).
  This then leaves us with
  \begin{align*}
    \int_a^b([x]+ [-x])dx = \sum_{i = 1}^k (-1)(x_i - x_{i-1}) = \sum_{i=1}^k-(x_1 - x_{i-1}) = x_0 - x_k = a - b
  \end{align*}
\end{proof}

\subsection*{Apostol p.70-71 no.5a}

\begin{claim}
  $$\int_0^2[t^2]dt = 5 - \sqrt{2} - \sqrt{3}$$
\end{claim}

\begin{proof}
  First, note that $0 \leq a < b \implies a(a) < b(a), a(b) < b(b) \implies a^2 < b^2$,
  so that $a < t < b \implies a^2 < t^2 < b^2$.

  It is easy to see that $[t^2]$ on the open subintervals of the partition $P = \{0, 1, \sqrt{2},
  \sqrt{3}, 2\}$, $[t^2]$ is constant. To be clear,
  \begin{align*}
    0 < t < 1 \implies 0 < t^2 < 1 \implies [t^2] = 0 \\
    1 < t < \sqrt{2} \implies 1 < t^2 < 2 \implies [t^2] = 1 \\
    \sqrt{2} < t < \sqrt{3} \implies 2 < t^2 < 3 \implies [t^2] = 2 \\
    \sqrt{3} < t < 2 \implies 3 < t^2 < 4 \implies [t^2] = 3
  \end{align*}

  Then, we have that
  \begin{align*}
    \int_0^2[t^2]dt &= \sum_{i=1}^4 c_i(x_i - x_{i-1}) \\
                    &= 0(1 - 0) + 1(\sqrt{2} - 1) + 2(\sqrt{3} - \sqrt{2}) + 3(2 - \sqrt{3}) \\
                    &= 5 - \sqrt{2} - \sqrt{3} \\
  \end{align*}
\end{proof}

\subsection*{Apostol p.70-71 no.5b}

Note that $[t^2] = [(-t)^2]$, so by problem 1 we have that $\int_{-3}^3 [t^2]dt
= \int_{-3}^{0}[t^2]dt + \int_0^3[t^2]dt = 2\int_0^3[t^2]dt = 2(\int_0^2[t^2]dt
+ \int_2^3[t^2]dt) = 2((5 - \sqrt{2} -\sqrt{3}) + \int_2^3[t^2]dt)$.

We can then check that we have the following:

\begin{align*}
  2 < t < \sqrt{5} \implies 4 < t^2 < 5 \implies [t^2] = 4 \\
  \sqrt{5} < t < \sqrt{6} \implies 5 < t^2 < 6 \implies [t^2] = 5 \\
  \sqrt{6} < t < \sqrt{7} \implies 6 < t^2 < 7 \implies [t^2] = 6 \\
  \sqrt{7} < t < \sqrt{8} \implies 7 < t^2 < 8 \implies [t^2] = 7 \\
  \sqrt{8} < t < 3 \implies 8 < t^2 < 9 \implies [t^2] = 8 \\
\end{align*}

Thus, we have that
\[
  \int_2^3[t^2] = 4(\sqrt{5} - 2) + 5(\sqrt{6} - \sqrt{5}) + 6(\sqrt{7} - \sqrt{6}) + 7(\sqrt{8} - \sqrt{7}) + 8(3 - \sqrt{8})) \\
\]

Then, we finally get that $\int_{-3}^3[t^2]dt = 2(5 - \sqrt{2} - \sqrt{3} +
(4(\sqrt{5} - 2) + 5(\sqrt{6} - \sqrt{5}) + 6(\sqrt{7} - \sqrt{6}) + 7(\sqrt{8}
- \sqrt{7}) + 8(3 - \sqrt{8}))) =  2(21 - \sqrt{2} - \sqrt{3} - \sqrt{5} - \sqrt{6} - \sqrt{7} - \sqrt{8})$.

Generally, $\int_0^n [t^2] = (n^2 - 1)n - \sum_{i=1}^{n^2 - 1} \sqrt{i}$.

\subsection*{Apostol p.70-71 no.11}

Unless specified otherwise, all functions are assumed step functions.

\subsection*{Apostol p.70-71 no.11a}

\begin{claim}
  $$ \int_a^b s + \int_b^c s = \int_a^c s$$
\end{claim}

\begin{proof}
  True; we have that $P = P_1 \cup P_2 = \{x_0,...,x_{n+m}\}$, where $P_1 =
  \{x_0,...,x_n\}, P_2 = \{x_n,...,x_m\}$ are partitions of $s$
  on $[a,b]$ and $[b,c]$, is then a partition of $s$ on $[a,c]$.
  \[
    \int_a^b s + \int_b^cs = \sum_{i=n}^ms_i^3(x_i - x_{i-1}) +
    \sum_{i=1}^ns_i^3(x_i - x_{i-1})  = \sum_{i=1}^{n+m}s^3(x_i - x_{i-1}) = \int_a^cs
  \]
\end{proof}

\subsection*{Apostol p.70-71 no.11b}

\begin{claim}
  $$\int_a^b (s+t) = \int_a^bs + \int_a^bt$$
\end{claim}

This is false, consider that $\int_a^b 2 = 2^3(b-a) \neq \int_a^b 1 + \int_a^b 1
= 1^3(b-a)  +1^3(b-a) = 2(b-a)$.

\subsection*{Apostol p.70-71 no.11c}

\begin{claim}
  $$c\int_a^bs = \int_a^b c \cdot s $$
\end{claim}

This is also false, consider again $\int_a^b2 = 2^3(b-a) \neq 2\int_a^b 1 = 2(b-a)$.

\subsection*{Apostol p.70-71 no.11d}

\begin{claim}
  $$\int_{a+c}^{b+c} s(x)dx = \int_a^b s(x+c)dx $$
\end{claim}

\begin{proof}
  True; we have that for a partition $P = \{x_0, x_1,...,x_n\}$ of $s(x)$ over $[a+c,b+c]$, then $P_c =
  \{x_0 - c, x_1 - c,...,x_n - c\}$ is a partition of $s(x + c)$ over $[a+c,b+c]$.
  Then, we have that
  \[
    \int_{a+c}^{b+c}s(x)dx = \sum_{i=1}^nc_i^3(x_i - x_{i-1}) =
    \sum_{i=1}^nc_n^3((x_i - c) - (x_{i-1} - c)) = \int_a^bs(x + c)dx
  \]
\end{proof}

\subsection*{Apostol p.70-71 no.11e}

\begin{claim}
  $\forall x \in [a,b], s(x) < t(x) \implies \int_a^bs < \int_a^bt$.
\end{claim}

\begin{proof}
  Note that for $a,b \in \R, a < b \implies a^3 < b^3$. This follows from
  casework on $a,b$: if $a = 0$, then $ab^2 < b^3 \implies 0 = a^3 < b^3$. If $b
  = 0$, then $a^3 < ba^2 = 0 = b^3$. If $a, b < 0$, then $a < b \implies a^2 >
  b^2 \implies a^3 < b^3$. If $a < 0, b > 0$, then $a < b \implies a^3 < 0 <
  b^3$. If $a>0, b>0$, then $a<b \implies a^2 < b^2 \implies a^3 < b^3$.
    
  Suppose that $s,t$ have partitions $P_s, P_t$. Then $P = P_s \cup P_t =
  \{x_0,...x_n\}$ is a partition for both, where $s$ has constant values $s_i$
  and $t$ has constant values $t_i$, where $i$ ranges from $1$ to $n$.
  Now, $\int_a^b s = \sum_{i=1}^n s_i^3(x_i - x_{i-1}) < \sum_{i=}^n t_i^3(x_i -
  x_{i-1}) = \int_a^b t$. We can show that the inequality holds by inducting on $n$. 
  
  In general, we wish to show that if for $i = 1,2,...,n$, we have that $a_i <
  b_i$, then $\sum_{i=1}^na_i < \sum_{i=1}^nb_i$. We induct on $n$. If $n = 1$, then we have that $a_1 < b_1$, which is true by the premise. For the inductive step, suppose that the inequality holds for $n = k$. Then,
  $\sum_{i=1}^{k+1}a_i = \sum_{i=1}^k a_i + a_{k+1} < \sum_{i=1}^k b_i + a_{k+1} < \sum_{i=1}^k b_i + b_{k+1} < \sum_{i=1}^{k+1} b_i$.
  
  Now, we know that as $s(x) < t(x)$ for $x \in [a,b]$, we have that $s_i < t_i \implies s_i^3 < t_i^3$. Applying this to the sum, we see that the inequality holds.
  
\end{proof}

\subsection*{Apostol p.70-71 no.15}

\begin{claim}
  For step functions $s,t$, we have that $\forall x \in [a,b], s(x) < t(x) \implies \int_a^b s < \int_a^b t$.
\end{claim}

\begin{proof}
  From above, we have that if for $i = 1,2,...,n$, we have that if $a_i < b_i$, then $\sum_{i=1}^na_i < \sum_{i=1}^nb_i$.
  
  Suppose that $s,t$ have partitions $P_s, P_t$. Then $P = P_s \cup P_t = \{x_0,...x_n\}$ is a partition for both, where $s$ has constant values $s_i$ and $t$ has constant values $t_i$, where $i$ ranges from $1$ to $n$.
  Now, $\int_a^b s = \sum_{i=1}^n s_i(x_i - x_{i-1}) < \sum_{i=}^n t_i(x_i - x_{i-1}) = \int_a^b t$.
  
  The middle inequality holds as we have that for any interval $(x_{i-1},x_i)$, we have that $x \in (x_{i-1},x_i) \implies s(x) = s_i < t(x) = t_i \implies s_i(x_i - x_{i-1}) < t_i(x_i - x_{i-1})$ as we have that $x_i > x_{i-1} \implies x_i - x_{i-1} > 0$.
  
\end{proof}

\subsection*{Apostol p.83 no.25a}

\begin{claim}
  If $f:[a,b] \rightarrow \R$ is an even function, $\int_{-b}^b f = 2\int_0^b f$.
\end{claim}

\begin{proof}
  From problem 1, we have that $\int_{-b}^b f = \int_{-b}^0f(x)dx  + \int_0^b f(x)dx = \int_0^b f(-x) + \int_0^b f(x)  = \int_0^b f(x) + \int_0^b f(x) = 2\int_0^b f $
\end{proof}

\subsection*{Apostol p.83 no.25b}

\begin{claim}
  If $f:[a,b] \rightarrow \R$ is an odd function, $\int_{-b}^b f = 0$.
\end{claim}

\begin{proof}
  From problem 1, we have that $\int_{-b}^b f = \int_{-b}^0f(x)dx  + \int_0^b f(x)dx = \int_0^b f(-x) + \int_0^b f(x)  = -\int_0^b f(x) + \int_0^b f(x) = 0 $
\end{proof}

\section*{Problem 1}

This does not rely on any Apostol problem which relies on this.

\begin{claim}
  If $f$ is integrable on $[a,b]$, then
  \[
    \int_{-b}^{-a}f(-x)dx = \int_a^b f(x)dx
  \]
\end{claim}

\begin{proof}
  We will first show this for step functions. If $P = \{x_0,...,x_n\}$ is a partition for $f(x)$ over $[a,b]$ with constant values $\{c_1,...,c_n\}$, then we have that $-P = \{-x_n,...,-x_0\}$ is a partition for $g(x):[-b,-a] \rightarrow \R: x \mapsto f(-x)$ over $[-b, -a]$ with the same constant values (though in reverse order). This can be seen as we have that for any open subinterval of $-P$, $(-x_i, -x_{i-1})$, $\forall x \in \R -x_{i-1} < x < -x_i \implies x_{i-1} < -x < x_i \implies g(x) = f(-x) = c_i$.
  
  Thus, we have that $\int_{-b}^{-a}g(x)dx = \int_{-b}^{-a}f(-x)dx = \sum_{i=1}^n c_i(-x_{i-1} - (-x_i)) = \sum_{i=1}^n c_i(x_i - x_{i-1}) = \int_a^b f(x)dx$.
  
  Now to show the general case, we will show that for $f,g$ we have that $\underline{I}(f) = \underline{I}(g)$ and $\overline{I}(f) = \overline{I}(g)$. 

  Suppose that $\int_a^b s(x)dx \in \underline{I}(f)$. Then we have that $s(x) <
  f(x) \implies s(-x) < f(-x) =g(x) \implies \int_{-b}^{-a}s(-x)dx =
  \int_a^bs(x)dx \in \underline{I}(g)$.

  Suppose that $\int_a^b s(x)dx \in \underline{I}(g)$. Then we have that $s(x) <
  g(x) \implies s(-x) < g(-x) = f(x) \implies \int_{-b}^{-a}s(-x)dx = \int_a^bs(x)dx \in
  \underline{I}(f)$.
  
  Suppose that $\int_a^b s(x)dx \in \overline{I}(f)$. Then we have that $s(x) >
  f(x) \implies s(-x) > f(-x) =g(x) \implies \int_{-b}^{-a}s(-x)dx =
  \int_a^bs(x)dx \in \overline{I}(g)$.

  Suppose that $\int_a^b s(x)dx \in \overline{I}(g)$. Then we have that $s(x) >
  g(x) \implies s(-x) > g(-x) = f(x) \implies \int_{-b}^{-a}s(-x)dx = \int_a^bs(x)dx \in
  \overline{I}(f)$.
  
  Thus, we have that $\underline{I}(f) = \underline{I}(g), \overline{I}(f) =
  \overline{I}(g)$ and so $\int_a^b f(x)dx = \int_{-b}^{-a}g(x)dx = \int_{-b}^{-a}f(-x)dx$. 
\end{proof}

\section*{Problem 2}

\begin{claim}
  $$ \left| \int_a^b f(x)dx \right| \leq \int_a^b|f(x)|dx$$
\end{claim}

\begin{proof}
  We first need to show that for any integrable functions $f,g$ on $[a,b]$, we
  have that $f \leq g \implies \int_a^b f \leq \int_a^b g$.

  In the case that $f < g$, we have that $\int_a^b f < \int_a^b g$ holds for
  step functions (as shown in Apostol pg.70-71 no.15). The same proof holds for
  $\leq$ instead of $<$ as well.

  In the general case, we have that $f \leq g \implies \underline{I}(f)
  \subseteq \underline{I}(g), \overline{I}(g) \subseteq \overline{I}(f)$, as $s
  < f \implies s < g$, and $s > g \implies s > f$. Thus, we have that
  $\sup(\underline{I}(f)) \leq \sup(\underline{I}(g)), \inf(\overline{I}(g))
  \geq \inf(\overline{I}(g))$ as properties of $\inf, \sup$. These lead to the
  conclusion that $\sup(\underline{I}(f)) = \inf(\overline{I}(f)) \leq
  \inf(\overline{I}(g)) = \sup(\overline{I}(g)) \implies \int_a^bf \leq \int_a^b
  g$.
  
  Note that we have that $-|f(x)| \leq f(x) \leq |f(x)|$ as a general property of the absolute
  value. Further, we have then $\int_a^b -|f(x)|dx = -\int_a^b|f(x)|dx \leq \int_a^bf(x)dx \leq \int_a^b |f(x)|dx$.
  The linearity of the integral was proved in class.

  In general, we have that $-a \leq x \leq a \implies |x| \leq a$, as we have
  three cases: $x = 0$, which follows as $-a \leq 0 \leq a \implies 0 \leq a$,
  $x < 0$, which follows as $-a \leq x \implies |x| = -x \leq -(-a) = a$, and $x
  > 0$, which follows as $x \leq a \implies |x| = x \leq a$. 
  
  We can now conclude that $|\int_a^bf(x)dx| \leq \int_a^b|f(x)|dx$.
\end{proof}

\section*{Problem 3}

The function
\[
  f(x) = \begin{cases}
    1 & x \in \Q \\
    -1 & x \in \R \setminus \Q
  \end{cases}
\]
is not, as we showed in class, integrable on any interval $[a,b], a \neq b$. However, we have that 
\begin{align*}
  |f(x)| &= \begin{cases}
    |1| & x \in \Q \\
    |-1| & x \in \R \setminus \Q
  \end{cases} \\
  &= \begin{cases}
    1 & x \in \Q \\
    1 & x \in \R \setminus \Q
  \end{cases} \\
  &= 1
\end{align*}
which is integrable over $[a,b]$. Namely, since it is equivalent to a step function with partition $\{a,b\}$, it evaluates to $b - a$.


\section*{Problem 4}

\subsection*{a}

\begin{claim}
  The only function both odd and even is $f(x) = 0$.
\end{claim}

\begin{proof}
  $f(x) = f(-x) = -f(x)$. In general, $a \in \R, a = -a \implies a = 0$. This is
  because $a = -a \implies a + a = 0 \implies 2a = 0 \implies a = (2^{-1})0 =
  0$. Thus, $f(x) = -f(x) \implies f(x) = 0$.
\end{proof}

\subsection*{b}

\begin{claim}
  Let $f$ be integrable on every closed interval $[a,b]$, and let $g(x) =
  \int_0^xf(t)dt$. If $f$ is odd, then $g$ is even, and if $f$ is even, then $g$
  is odd.
\end{claim}

\begin{proof}
  Consider that we have that $\int_a^bf = -\int_b^a f$, as we have that
  $\int_a^b f + \int_b^c f = \int_a^c f$. Taking $c =a$, we see that $\int_a^bf
  + \int_b^a f = \int_a^a f = 0 \implies \int_a^b f = -\int_b^a f$.

  Further, we have that $\int_0^x g(x)dx = \int_{-x}^0g(-x)dx$ by problem 1.

  We will first show that $f$ odd $\implies g$ even.

  \begin{align*}
    g(-x) &= \int_0^{-x}f(x)dx \\
          &= \int_x^0 f(-x)dx \\
          &= \int_x^0 -f(x)dx \\
          &= -\int_x^0 f(x)dx \\
          &= \int_0^x f(x)dx = g(x)
  \end{align*}

  Now, we handle that $f$ even $\implies g$ odd.

  \begin{align*}
    g(-x) &= \int_0^{-x}f(x)dx \\
          &= \int_x^0 f(-x)dx \\
          &= \int_x^0 f(x)dx \\
          &= \int_x^0 f(x)dx \\
          &= -\int_0^x f(x)dx = -g(x)
  \end{align*}
\end{proof}

\section*{Problem 5}

\begin{claim}
  $f: [a,b] \rightarrow \R$ is integrable iff $\forall \epsilon > 0, \exists
  s,t$ step functions such that $s \leq f \leq t$ and $\int_a^b(t - s)dx < \epsilon$.
\end{claim}

\begin{proof}
  $(\implies)$ $f$ is integrable $\implies \sup(\underline{I}(f)) =
  \inf(\overline{I}(f))$. This means that the approximation theorem furnishes $x =
  \int_a^b s \in \underline{I}(f) \mid \sup(\underline{I}(f)) -
  \frac{\epsilon}{2} < x \implies -\int_a^bs = \int_a^b-s < -\sup(\underline{I}(f)) +
  \frac{\epsilon}{2}$ for any $\epsilon > 0$. Similarly, we can get $y = \int_a^b t \in
  \overline{I}(f) \mid \inf(\underline{I}(f)) + \frac{\epsilon}{2} > y =
  \int_a^b t$. Adding the two inequalities, we have that

  \[
    \int_a^b t + \int_a^b-s = \int_a^b (t - s) < \inf(\overline{I}(f)) -
    \sup(\underline{I}(f)) + \epsilon = \epsilon
  \]

  $(\impliedby)$
  We have that $\forall \epsilon > 0, \exists s,t \mid s \leq f \leq t,
  \int_a^b(t-s) < \epsilon \implies \int_a^b t - \int_a^bs < \epsilon \implies
  \int_a^bt < \int_a^bs + \epsilon$.

  We have in general that if for any $a \in A, b \in B, \forall \epsilon > 0, b < a +
  \epsilon \implies \sup(A) \geq \inf(B)$. To see this, consider that if
  $\sup(A) < \inf(B)$, we would have $\epsilon = \frac{\inf(B) - \sup(A)}{2}$
  such that $a + \epsilon \leq \sup{A} + \epsilon = \frac{\inf(B) + \sup(A)}{2}
  < \inf(B) \leq b$. $\contra$.

  Also, we have that if $\forall a \in A, \forall b \in B, a \leq b \implies
  \sup(A) \leq \inf(B)$. To see this, consider that if $\sup(A) > \inf(B),
  \exists \epsilon > 0, a \in A, b \in B \mid \sup(A) - \epsilon < a, \inf(B) +
  \epsilon > b$ If we take $\epsilon = \frac{\sup(A) - \inf(B)}{2}$, then we
  have that $b < \frac{\sup(A) + \inf(B)}{2} < a$. $\contra$.

  The premise then gives us the fact that $\sup(\underline{I}(f)) \geq
  \inf(\overline{U}(f))$, as well as that since $s \leq t \implies \int_a^b s
  \leq \int_a^bt$, $\sup(\underline{I}(f)) \leq \inf(\overline{I}(f))$. Thus, to
  not violate trichotomy, we must have $\sup(\underline{I}(f)) =
  \inf(\overline{I}(f))$, and thus $f$ is integrable.
\end{proof}

\end{document}

% LocalWords:  NetID fancyplain LocalWords colorlinks linkcolor linkbordercolor
