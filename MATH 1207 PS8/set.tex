\documentclass[12pt,letterpaper]{article}
\usepackage{fullpage}
\usepackage[top=2cm, bottom=4.5cm, left=2.5cm, right=2.5cm]{geometry}
\usepackage{amsmath,amsthm,amsfonts,amssymb,amscd}
\usepackage{lastpage}
\usepackage{enumerate}
\usepackage{fancyhdr}
\usepackage{mathrsfs}
\usepackage{xcolor}
\usepackage{graphicx}
\usepackage{listings}
\usepackage{hyperref}
\usepackage{tikz}
\usepackage{relsize}
\usepackage{fancyvrb}
\usetikzlibrary{shapes.geometric,fit}

\hypersetup{%
  colorlinks=true,
  linkcolor=blue,
  linkbordercolor={0 0 1}
}

\setlength{\parindent}{0.0in}
\setlength{\parskip}{0.05in}

\newcommand\course{MATH 1207}
\newcommand\hwnumber{8}
\newcommand\NetIDa{dc3451}
\newcommand\NetIDb{David Chen}

\theoremstyle{definition}
\newtheorem*{statement}{Statement}
\newtheorem*{claim}{Claim}
\newtheorem*{theorem}{Theorem}

\newcommand{\contra}{\Rightarrow\!\Leftarrow}
\newcommand{\R}{\mathbb{R}}
\newcommand{\F}{\mathbb{F}}
\newcommand{\Z}{\mathbb{Z}}
\newcommand{\Ze}{\mathbb{Z}_{\geq 0}}
\newcommand{\Zg}{\mathbb{Z}_{>0}}
\newcommand{\N}{\mathbb{N}}
\newcommand{\Q}{\mathbb{Q}}
\newcommand{\C}{\mathbb{C}}

\pagestyle{fancyplain}
\headheight 35pt
\lhead{\NetIDa}
\lhead{\NetIDa\\\NetIDb}
\chead{\textbf{\Large Homework \hwnumber}}
\rhead{\course \\ \today}
\lfoot{}
\cfoot{}
\rfoot{\small\thepage}
\headsep 1.5em

\begin{document}

\subsection*{Apostol p.155 no.8}

\begin{claim}
  If $f$ is continuous on $[a,b]$, and $\int_a^bf(x)g(x)dx = 0$ for every
  function $g$ continuous on $[a,b]$, then $f(x) = 0$.
\end{claim}

\begin{proof}
  Take $g(x) = f(x)$. We have that $\int_a^b(f(x))^2dx = 0$. However, we have
  that $f(x)^2 \geq 0$. Now, suppose that $f(y)^2 > 0$. Then, we have that, as
  $f$ is continuous, that $\exists \delta > 0 \mid 0 < |x - y| < \delta \implies
  |f(x)^2 - f(y)^2| < \frac{1}{2}f(y)^2 \implies f(x)^2 > \frac{1}{2}f(y)^2 >
  0$. Thus,

  \begin{align*}
    \int_a^bf(x)^2dx &= \int_a^{y - \delta}f(x)^2dx + \int_{y - \delta}^{y+\delta}f(x)^2dx + \int_{y+\delta}^bf(x)^2dx \\
                     & \geq \int_{y - \delta}^{y+\delta}f(x)^2dx \\
                     & \geq \int_{y - \delta}^{y+\delta}f(y)^2dx \\
                     &= 2\delta f(y)^2> 0
  \end{align*}
  
  $\contra$, so $f(x)^2 = 0 \implies \forall x \in [a,b], f(x) = 0$.
\end{proof}

\subsection*{Apostol p.168 no.22}

We first show the power rule for rational exponents. We already have that power
rule for integral exponents, and for $f(x) = x^{\frac{p}{q}}$, $f(x)^q = x^p$.
Further, we have that the chain rule yields $(f(x)^q)' = (qf(x)^{q-1})(f'(x))$. Thus,
$(qf(x)^{q-1})(f'(x)) = px^{p-1} \implies f'(x) = \frac{px^{p-1}}{qf(x)^{q-1}} =
\frac{p}{q}\frac{x^{p-1}}{x^{\frac{p(q-1)}{q}}} = \frac{p}{q}x^{p- 1 - p +
  \frac{p}{q}} = \frac{p}{q}x^{\frac{p}{q} - 1}$.

Then, we have that $(\sqrt{x})' = (x^{\frac{1}{2}})' = \frac{1}{2\sqrt{x}}$.
Further, $(1 + x)' = (1)' + (x)' = 0 + 1 = 1$.

The quotient rule then yields that
\[
  f'(x) = \frac{\frac{1}{2\sqrt{x}}(1+x) - \sqrt{x}}{(1+x)^2}  =
  \frac{1}{2\sqrt{x}(1+x)} - \frac{\sqrt{x}}{(1+x)^2} = \frac{1-x}{2\sqrt{x}(1+x)^2}
\]

\subsection*{Apostol p.168 no.24}

\begin{claim}
  \[
    g' = f'_1(f_2f_3...f_n) + f'_2(f_1f_3...f_n) + ... +
    f'_n(f_1f_2...f_{n-1}) = \sum_{i=1}^n (f'_i\prod_{j = 1}^{i-1} f_j \prod_{j=i+1}^nf_j)
  \]
\end{claim}

\begin{proof}
  Take the base case of $n = 1$, or $g = f_1$. Then,
  $g' = f_1'$, as given by the formula, as the prodcuts are empty.

  Assume that the formula holds for $n = k$, and put $g_k = \prod_{i=1}^kf_i$. Then,
  \begin{align*}
    g'_{k+1} &= g_k'f_{k+1} + f_{k+1}'g_k \\
             &= (\sum_{i=1}^k (f'_i\prod_{j = 1}^{i-1} f_j \prod_{j=i+1}^kf_j))f_{k+1} + f_{k+1}'(\prod_{i=1}^kf_i) \\
             &= \sum_{i=1}^k (f'_i\prod_{j = 1}^{i-1} f_j \prod_{j=i+1}^{k+1}f_j) + f_{k+1}'(\prod_{i=1}^kf_i) \\
             &= \sum_{i=1}^{k+1} (f'_i\prod_{j = 1}^{i-1} f_j \prod_{j=i+1}^{k+1}f_j)
  \end{align*}
\end{proof}

\begin{claim}
  \[
    \frac{g'}{g} = \sum_{i=1}^n\frac{f_i'}{f_i} \\
  \]
\end{claim}

\begin{proof}
  \begin{align*}
    g' &= \sum_{i=1}^n (f'_i\prod_{j = 1}^{i-1} f_j \prod_{j=i+1}^nf_j) \\
       &= \sum_{i=1}^n \frac{f'_i\prod_{j = 1}^{n} f_j}{f_i} \\
    \implies \frac{g'}{g} &= \frac{\sum_{i=1}^n \frac{f'_i\prod_{j = 1}^{n} f_j}{f_i}}{\prod_{j=1}^nf_j} \\
       &= \frac{\sum_{i=1}^nf'_i}{f_i} = \sum_{i=1}^n \frac{f'_i}{f_i}
  \end{align*}
\end{proof}

\subsection*{Apostol p.174 no.15}

\subsection*{a}

\begin{claim}
  \[
    f'(a) = \lim_{h \rightarrow 0}\frac{f(h) - f(a)}{h - a}
  \]
\end{claim}

\begin{proofof}
  Put $h = k + a$.
  \begin{align*}
    f'(a) &= \lim_{k \rightarrow 0}\frac{f(a + k)-  f(a)}{k} \\
          &= \lim_{h \rightarrow a}\frac{f(a + (h-a)) - f(a)}{h-a} \\
          &= \lim_{h \rightarrow a}\frac{f(h) - f(a)}{h-a}
  \end{align*}
\end{proof}

\subsection*{b}

\begin{claim}
  \[
    f'(a) = \lim_{h \rightarrow 0}\frac{f(a) - f(a-h)}{h}
  \]
\end{claim}

\begin{proof}
  Put $h = -k$.
  \begin{align*}
    f'(a) &= \lim_{k \rightarrow 0}\frac{f(a + k)-  f(a)}{k} \\
          &= \lim_{h \rightarrow 0}\frac{f(a - h) - f(a)}{-h} \\
          &= \lim_{h \rightarrow 0}\frac{f(a) - f(a-h)}{h}
  \end{align*}
\end{proof}

\subsection*{c}

False. Consider $f(x) = x$. Then, $f'(a) = 1$. However,
\[
  \lim_{t\rightarrow 0}\frac{f(a + 2t) - f(a)}{t} = \lim_{t\rightarrow
    0}\frac{2t}{t} = 2
\]

In general, we have that since $\lim_{t \rightarrow 0} f = \lim_{ct \rightarrow
  0}f, c \neq 0$,
\[
  \lim_{t\rightarrow 0}\frac{f(a + 2t) - f(a)}{t} =
  2\lim_{2t\rightarrow 0}\frac{f(a+2t) -f(a)}{2t} = 2\lim_{t\rightarrow
  0}\frac{f(a+t)-f(a)}{t} =2f'(a)
\]

\subsection*{d}

False. Consider $f(x) = x$. Then, $f'(a) = 1$. However,
\[
  \lim_{t\rightarrow 0}\frac{f(a + 2t) - f(a+t)}{2t} = \lim_{t\rightarrow
    0}\frac{t}{2t} = \frac{1}{2}
\]

In general, we have that
\begin{align*}
  \lim_{t\rightarrow 0}\frac{f(a + 2t) - f(a+t)}{2t} &= \frac{1}{2}(\lim_{t\rightarrow 0}\frac{f(a + 2t) - f(a)}{t} - \lim_{t\rightarrow 0}\frac{f(a + t) - f(a)}{t}) \\
                                                     &= \frac{1}{2}(2f'(a) - f'(a)) \\
                                                     &= \frac{1}{2}f'(a)
\end{align*}

\section*{Problem 1}

\begin{claim}
  Let $f: [a,b] \rightarrow \R$ be integrable. Then, $\exists c \in [a,b]$ such
  that
  \[
    \int_a^cf(x)dx = \frac{1}{2}\int_a^cf(x)dx
  \].
\end{claim}

\begin{proof}

  First, we extend the Intermediate Value Theorem to be slightly stronger. The
  original statement is that for
  $f: [a,b] \rightarrow \R$ continuous, if $f(a) < K < f(b)$ then $\exists c \in
  [a,b] \mid f(c) = K$. Further, we will show that if $f(b) < K < f(a)$ then
  $\exists c \in [a,b] \mid f(c) = K$. To see this, consider $g(x) -f(x)$. We
  have that $g(a) < -K < g(b)$, so $\exists c \in [a,b] \mid g(c) = -K \implies
  f(c) = K$.
  
  Consider $g(x) : [a,b] \rightarrow \R, g(x) = \int_a^xg(t)dt$. Then, we have
  that $g(a) = \int_a^ag(t)dt = 0, g(b) = \int_a^bg(t)dt$. Further, $\frac{1}{2}
  \int_a^bf(x)dx = \frac{g(b)}{2} = \frac{g(a) + g(b)}{2}$, and if $g(b) > 0 = g(a)$, then $g(a) <
  \frac{g(a) + g(b)}{2} < g(b)$, and if $g(b) < 0 = g(a)$, then $g(b) <
  \frac{g(a) + g(b)}{2} < g(a)$, and so by the Intermediate Value Theorem,
  $\exists c \in [a,b] \mid g(c) = \frac{g(a) + g(b)}{2} \implies \int_a^cf(x)dx
    = \frac{1}{2}\int_a^bf(x)dx$.
\end{proof}

\section*{Problem 2}

\begin{claim}
  $f$ is continuous on $[0,1]$, and has $f(0) = f(1)$. $\forall n \in \Zg,
  \exists x \in [0,1 - 1/n] \mid f(x) = f(x + 1/n).$
\end{claim}

\begin{proof}
  Consider $g(x) = f(x) - f(x + 1/n)$. Suppose that $g > 0$. Then, we have that
  $f(1) > f(0)$. To see this, consider the set $\{ g(k/n) \mid k \in \Zg, k \leq
  n\}.$ We then have that $k > 0 \implies f(k/n) > f(0)$, as we can induct on
  $k$. If $k = 1$, then $g(1/n) > 0 \implies f(1/n) - f(0) > 0 \implies f(1/n) >
  f(0)$. Assume that the hypothesis holds for $k < n$. Then, $g(k/n) > 0
  \implies f(k/n) - f((k+1)/n) > 0 \implies f((k+1)/n > f(k/n) > f(0)$. This
  shows that $f(k/n) > f(0)$ for all $k \in \Zg, k \leq n$. Critically, this
  then means that $f(1) > f(0)$. $\contra$

  Now suppose that $g < 0$. Then, we have that
  $f(1) < f(0)$. To see this, consider the set $\{ g(k/n) \mid k \in \Zg, k \leq
  n\}.$ We then have that $k > 0 \implies f(k/n) < f(0)$, as we can induct on
  $k$. If $k = 1$, then $g(1/n) < 0 \implies f(1/n) - f(0) < 0 \implies f(1/n) <
  f(0)$. Assume that the hypothesis holds for $k < n$. Then, $g(k/n) < 0
  \implies f(k/n) - f((k+1)/n) < 0 \implies f((k+1)/n < f(k/n) < f(0)$. This
  shows that $f(k/n) < f(0)$ for all $k \in \Zg, k \leq n$. Critically, this
  then means that $f(1) < f(0)$. $\contra$.

  Thus, we must have that $g$ cannot be positive nor negative everywhere,
  meaning that $\exists x,y \in [0, 1-1/n] \mid g(x) > 0, g(y) < 0$. By the
  Intermediate Value Theorem, we have that $\exists z \in [0, 1-1/n] \mid g(z) =
  0 \implies \exists z \in [0,1-1/n] \mid f(z) - f(z + 1/n) = 0 \implies f(z) =
  f(z + 1/n)$.
\end{proof}

\section*{Problem 4}

\subsection*{a)}

Consider the counter example of

\[
  f(x) = \begin{cases}
    1 & x \in \Q \\
    0 & x \notin \Q \\
  \end{cases}
  \ \ \ \ 
  g(x) = \begin{cases}
    0 & x \in \Q \\
    1 & x \notin \Q \\
  \end{cases}
\]

We have that $f + g = 1$, which is differentiable everywhere $(f + g)' = 0$.
However, we have that $f,g$ are nowhere continuous and thus nowhere
differentiable (the contrapositive, differentiable $\implies$ continuous was
proved in class).

In general, take any function $f$ not differentiable at $x$. Then, $f + (-f) =
0$ is differentiable at $x$, but neither $f, -f$ are.

\subsection*{b)}

\begin{claim}
  If $f(x) \neq 0$, then $g$ is differentiable at $x$.
\end{claim}

\begin{proof}
  We have that the quotient rule states for functions $s,t$ differentiable at
  $x$, then if $t(x) \neq 0$, $(\frac{s}{t})' = \frac{s't - st'}{t^2}$ at $x$.
  Taking $s = fg, t = f$, we have that $f(x) \neq 0 \implies g'(x)$ exists by
  the quotient rule.
\end{proof}

\section*{Problem 5}

\subsection*{a)}

\begin{claim}
  $f(x) = xg(x)$, $g$ continuous at 0 $\implies f$ is differentiable at $0$.
\end{claim}

\begin{proof}
  \[
    \lim_{h\rightarrow 0}\frac{f(0 + h) - f(0)}{h} = \lim_{h\rightarrow 0}\frac{f(h)}{h} 
  \]

  Consider $\lim_{h\rightarrow 0}(\frac{f(h)}{h} - g(h))$. For any $\epsilon$,
  take arbitrary $\delta > 0$. We then have that $0 < |x| < \delta \implies x
  \neq 0 \implies \frac{f(x)}{x} = g(x) \implies |\frac{f(x)}{x} - g(x)| = 0 < \epsilon$.

  Thus, we have that $\lim_{h\rightarrow 0} (\frac{f(h)}{h} - g(h)) = 0 \implies
  \lim_{h\rightarrow 0} \frac{f(h)}{h} = \lim_{h\rightarrow 0} g(h) = g(0)$, as
  $g$ is continuous.
\end{proof}

\subsection*{b)}

\begin{claim}
  Suppose that $f$ is differentiable at $0$ and $f(0) = 0$. Then, $\exists g(x)
  \mid f(x) = xg(x), g$ continuous at $0$.
\end{claim}

\begin{proof}
  Consider
  \[
    g(x) = \begin{cases}
      f'(0) & x = 0 \\
      \frac{f(x)}{x} & x \neq 0
    \end{cases}
  \]
  which is well defined as we have that $f$ is differentiable at 0.
  
  Then, we have that
  \[
    xg(x) = \begin{cases}
      0 & x = 0 \\
      f(x) & x \neq 0
    \end{cases}
  \]

  This is equal to $f(x)$ everywhere.

  Now, to prove that $g(x)$ is continuous, note first that we have that
  $\lim_{h\rightarrow 0} \frac{f(h)}{h} = \lim_{h\rightarrow 0}\frac{f(0 + h) -
    f(0)}{h} = f'(0)$, as we have that $f$ is differentiable at 0. Further,
  $\lim_{h\rightarrow 0} (g(h) - \frac{f(h)}{h} = 0$, as for any $\epsilon > 0$,
  take arbitrary $\delta > 0 \mid 0 < |x| < \delta \implies x \neq 0 \implies
  g(x) = \frac{f(x)}{x} \implies |g(x) - \frac{f(x)}{x} - 0| = 0 < \epsilon$.
  Finally, we have that $\lim_{h\rightarrow 0}g(h) = \lim_{h\rightarrow
    0}\frac{f(h)}{h} = f'(0) = g(0)$, so $g$ is continuous.
\end{proof}

\end{document}

% LocalWords:  NetID fancyplain LocalWords colorlinks linkcolor linkbordercolor