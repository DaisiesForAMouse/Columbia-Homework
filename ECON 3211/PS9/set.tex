\documentclass[12pt,letterpaper]{article}
\usepackage{fullpage}
\usepackage[top=2cm, bottom=4.5cm, left=2.5cm, right=2.5cm]{geometry}
\usepackage{amsmath,amsthm,amsfonts,amssymb,amscd}
\usepackage{lastpage}
\usepackage{enumerate}
\usepackage{fancyhdr}
\usepackage{mathrsfs}
\usepackage{xcolor}
\usepackage{graphicx}
\usepackage{listings}
\usepackage{hyperref}
\usepackage{tikz}
\usepackage{xfrac}
\usepackage{nicefrac}
\usepackage{xcolor}
\allowdisplaybreaks

\usetikzlibrary{shapes.geometric,fit}
\usetikzlibrary{patterns}

\hypersetup{
  colorlinks=true,
  linkcolor=blue,
  linkbordercolor={0 0 1}
}

\setlength{\parindent}{0.0in}
\setlength{\parskip}{0.05in}

\newcommand\course{ECON 3211}
\newcommand\hwnumber{9}
\newcommand\NetIDa{dc3451}
\newcommand\NetIDb{David Chen}

\newcommand\R{\mathbb{R}}

\theoremstyle{definition}
\newtheorem*{statement}{Statement}
\newtheorem*{claim}{Claim}
\newtheorem*{theorem}{Theorem}

\newcommand{\contra}{\Rightarrow\!\Leftarrow}
\newcommand{\Lag}{\mathcal{L}}

\pagestyle{fancyplain}
\headheight 35pt
\lhead{\NetIDa}
\lhead{\NetIDa\\\NetIDb}
\chead{\textbf{\Large Problem Set \hwnumber}}
\rhead{\course \\ \today}
\lfoot{}
\cfoot{}
\rfoot{\small\thepage}
\headsep 1.5em

\begin{document}

\section*{Problem 1}

\subsection*{a)}

The payoffs are written on the bottom in the format Firm 1:Firm 2 in
millions, where Firm 1 is on the side of the matrix and Firm 2 on the top (as in
the notes).

\tikzstyle{block} = [rectangle, draw, fill=blue!20, 
text width=4em, text centered, rounded corners, minimum height=3em]
\tikzstyle{nodecirc} = [circle, draw, fill=blue!20]
\tikzstyle{line} = [draw, ->]
\tikzstyle{arrow} = [->,>=stealth]
\begin{center}
  \begin{tikzpicture}[node distance = 1.5cm, auto]
    \node[nodecirc] (p1) {};
    \node[right of=p1] {Firm 1};
    \node[nodecirc,below of=p1] (p1b) {};
    \node[right of=p1b] (s1) {};
    \node[right of=s1] (s2) {};
    \node[nodecirc,right of=s2] (p1c) {};
    \node[left of=p1b] (s1) {};
    \node[left of=s1] (s2) {};
    \node[nodecirc,left of=s2] (p1a) {};
    \node[nodecirc,below of=p1b] (p1bp2b) {};
    \node[nodecirc,right of=p1bp2b] (p1cp2c) {};
    \node[nodecirc,left of=p1bp2b] (p1ap2a) {};
    
    \node[right of=p1c] {Firm 2};
    
    \node[nodecirc,below of=p1b] (p1bp2b) {3:2};
    \node[nodecirc,right of=p1bp2b] (p1bp2c) {2:1};
    \node[nodecirc,left of=p1bp2b] (p1bp2a) {2:3};
    
    \node[nodecirc,right of=p1bp2c] (p1cp2a) {5:1};
    \node[nodecirc,right of=p1cp2a] (p1cp2b) {2:2};
    \node[nodecirc,right of=p1cp2b] (p1cp2c) {1:5};

    \node[nodecirc,left of=p1bp2a] (p1ap2c) {2:4};
    \node[nodecirc,left of=p1ap2c] (p1ap2b) {0:3};
    \node[nodecirc,left of=p1ap2b] (p1ap2a) {1:5};

    \draw[arrow] (p1) -- node[anchor=center] {A} (p1a);
    \draw[arrow] (p1) -- node[anchor=center] {B} (p1b);
    \draw[arrow] (p1) -- node[anchor=center] {C} (p1c);
    \draw[arrow] (p1a) -- node[anchor=center] {A} (p1ap2a);
    \draw[arrow] (p1a) -- node[anchor=center] {B} (p1ap2b);
    \draw[arrow] (p1a) -- node[anchor=center] {C} (p1ap2c);
    \draw[arrow] (p1b) -- node[anchor=center] {A} (p1bp2a);
    \draw[arrow] (p1b) -- node[anchor=center] {B} (p1bp2b);
    \draw[arrow] (p1b) -- node[anchor=center] {C} (p1bp2c);
    \draw[arrow] (p1c) -- node[anchor=center] {A} (p1cp2a);
    \draw[arrow] (p1c) -- node[anchor=center] {B} (p1cp2b);
    \draw[arrow] (p1c) -- node[anchor=center] {C} (p1cp2c);
  \end{tikzpicture}
\end{center}

\subsection*{b)}

\[
  BR_1(s_2) = \begin{cases}
    C & s_2 = A \\
    B & s_2 = B \\
    A \text{ or } B & s_2 = C
  \end{cases}
\]

\subsection*{c)}

Firm 1 does not have a dominant or weakly dominant strategy; however, Version A
is weakly dominated by Version B.

\subsection*{d)}

\[
  BR_2(s_1) = \begin{cases}
    A & s_1 = A \\
    A & s_1 = B \\
    C & s_1 = C
  \end{cases}
\]

\subsection*{e)}

There are no pure strategy Nash equilibria. 

Put probability $\theta_X$ for the probability that Firm 1 chooses
Version $X$, and $\rho_X$ the probability that Firm 2 chooses Version $X$. 

Now, consider that we see that Version B is strongly dominated by a mixture of
Version A and Version C (for example, $\theta_A = 0.5, \theta_C = 0.5$) for
Firm 2. Eliminating this, we will check the resulting $2\times3$ game for mixed
Nash equilibria. Seeing that Version A is weakly dominated by Version B for
Firm 1, we easily check that no Nash equilibrium occurs when mixing over $\{A, B, C\}$
for Firm 2. We arrive at the game

\begin{center}
  \begin{tabular}{c|c|c}
    & A & C \\
    \hline
    B & 2,3 & 2,1\\
    \hline
    C & 5,1 & 1,5
  \end{tabular}
\end{center}

Computing the Nash equilibrium, we see that it ought to be $3\theta_B + 1\theta_C =
\theta_B + 5\theta_C \implies \theta_B = \frac{2}{3}, \theta_C = \frac{1}{3}$
and $2\rho_A + 2 \rho_B = 5\rho_A + \rho_B \implies \rho_A = \frac{1}{4}, \rho_C
= \frac{3}{4}$.

Therefore, the Nash equilibrium is then for Firm 1 $(0,\frac{2}{3},\frac{1}{3})$ and Firm
2 $(\frac{1}{4},0,\frac{3}{4})$.


\section*{Problem 2}

\subsection*{a),b)}

The pure cooperation game is an example of what we want. Take payoffs $P_i:
\{A,B\}\times\{A,B\} \rightarrow \mathbb{R}$:

\[
  P_1(s_1,s_2) = \begin{cases}
    1 & s_1 = s_2 \\
    0 & s_1 \neq s_2
  \end{cases}
\]

\[
  P_2(s_1,s_2) = \begin{cases}
    1 & s_1 = s_2 \\
    0 & s_1 \neq s_2
  \end{cases}
\]

\begin{center}
  \begin{tabular}{c|c|c}
    & A & B \\
    \hline
    A & 1,1 & 0,0\\
    \hline
    B & 0,0 & 1,1
  \end{tabular}
\end{center}

There are two pure Nash equilibria: $(A,A)$ and $(B,B)$.

\subsection*{c)}

Let $\theta$, $\rho$ be the probabilities of player 1 (row player) and player 2
(column) player choosing strategy A, respectively.
The mixed strategy Nash equilibrium then satisfies $\theta = 1 - \theta, \rho = 1 - \rho
\implies \theta = \frac{1}{2}, \rho = \frac{1}{2}$.

\section*{Problem 3}


\subsection*{a)}

The profit maximizing quantity satisfies that $MR = 70 - 2Q = 10 = MC \implies Q
= 30*, P = 40*$.

\subsection*{b)}

Consumer surplus in this market can be computed to be $\frac{1}{2}(70-40)(30) =
450$. The monopolist's surplus is then $(40-10)(30) = 900$. 

\subsection*{c)}

The game in normal form would have the set of players $S = \{1, 2\}$, picking
actions $q_1, q_2 \in \R_{\geq 0}$ representing their quantity produced. We
define payoff functions then as follows: $P_i: \R_{\geq 0} \times \R_{\geq 0} \rightarrow \R$ is
equal to $\pi_1(q_1,q_2) = (70 - 10(q_1+q_2))q_1 - 10q_1$, $\pi_2(q_1,q_2) = (70 - 10(q_1+q_2))q_2 - 10q_2$.

\subsection*{d),f)}

We have that

\begin{align*}
  BR_2(\bar{q_2}) &= \max_{q_1}(70 - (q_1+\bar{q}_2)q_1 - 10q_1 - 450) \\
  \frac{\partial BR_1}{\partial q_1} &= 70 - 2q_1 - \bar{q_2} - 10 = 0 \\
  q^*_1 &= \frac{60-\bar{q_2}}{2} \\
  BR_2(\bar{q_1}) &= \max_{q_2}(70 - (q_2+\bar{q}_1)q_2 - 10q_2 - 450) \\
  \frac{\partial BR_2}{\partial q_2} &= 70 - 2q_2 - \bar{q_1} - 10 = 0\\
  q^*_2 &= \frac{60-\bar{q_1}}{2}\\
                                     &= \frac{60 - \frac{60-\bar{q_2}}{2}}{2} \\
                                     &= \frac{60+q^*_2}{4} \\
                                     &= \frac{q^*_2}{4} + 15 \\
  q^*_2 &= \frac{4}{3}15 = 20 \\
  q^*_1 &= \frac{4}{3}15 = 20
\end{align*}

\subsection*{e),f)}

\begin{center}
  \begin{tikzpicture}[
    dot/.style={shape=circle, inner sep=2pt, draw, node contents=},
    circ/.style={shape=circle, inner sep=2pt, draw, fill}]
    \draw[thick,->] (0,0) -- (7,0) node[anchor=north west] {$Q_1$};
    \draw[thick,->] (0,0) -- (0,7) node[anchor=south east] {$Q_2$};
    \draw (3pt,2cm) -- (-3pt,2cm) node[anchor=east] {$20$}; 
    \draw (2cm,3pt) -- (2cm,-3pt) node[anchor=north] {$20$};
    \draw[domain=0:60,scale=0.1,smooth,variable=\x] plot ({\x},{30-\x/2}) node[label=above:{$BR_2$}]{};
    \draw[domain=0:60,scale=0.1,smooth,variable=\x] plot ({30-\x/2},{\x}) node[label=right:{$BR_1$}]{};
    \draw (2,2) node[circ,label=above right:{$(q_1^*,q_2^*)$}]{};
    \draw[dashed] (0,2) -- (2,2) -- (2,0);
  \end{tikzpicture}
\end{center}


\subsection*{g)}

The total quantity is 40, at a price of 30. This is more output for a lower
price than the sole monopolist, and leads to a closer to socially optimal
outcome (although not actually optimal).

\subsection*{h)}

There is now a consumer surplus of $\frac{1}{2}(70-30)(40) = 800$ and a producer
surplus of $(30 - 10)(20) = 400$ each, for a total producer surplus of $800$ and
a total surplus of $1600$. Producer surplus fell, consumer surplus rose, the
total surplus rose ($1350 \rightarrow 1600$).

\subsection*{i)}

The firms make in profit $\pi = 30(20) - 10(20) - 450 = -50$ each, so they are
losing money; previously, a monopoly would make $\pi = 40(30) - 10(30) -450 =
450$. A regulator should not push for a duopoly unless absolutely convinced for
external reasons that they will stay in business while taking a loss
economically (or they don't care that no output will be produced, such as if
this market is untenable socially in the monopoly case).


\end{document}
% LocalWords:  nodecirc
